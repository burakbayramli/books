\section{Projection Method}
%
In this volume we treat a class of linear programming methods where one
computes at first a vertex of the feasible domain and then proceeds from one
vertex to the next until an optimum vertex is found.  However we follow here
an idea proposed first by \cite{BeRi} and consider for basic problem a problem
where no equalities occur:
%
\begin{equation} \label{e24}
\fbox{$
\max\{ax, \; Bx \leq c\}, \; B \in \Bbb{R}^m{}_n, \; \rank (B) = n.
$}
\end{equation}
%
This way has some striking advantages in the development of more specific
techniques for linear and nonlinear programming problems.  By Theorem \ref{s3},
a point $x \in \Omega$ is a vertex if and only if in this point the gradients
of
the active side conditions span the entire row space $\Bbb{R}_n$.  By Theorem
\ref{s2}, a vertex $x^*$ is optimum if in this point the gradient $a$ of the
objective function is contained in the positive cone of the gradients of the
active side conditions.  This means that all edges {\bf pointing to} $x$ form
with $a$ an angle $0 \leq \phi \leq \pi/2$.  If the vertex is not optimum then
one has to leave this vertex in direction of an edge to an other vertex point
such that the objective function is improved or at least is not deteriorated.
By this way one remains always in the feasible domain or --- more exactly ---
on its boundary.  In marching from one vertex to the other, the optimum
direction to be chosen can be read directly from the {\sc Lagrange} multipliers
in the Main Theorem \ref{s2} which are calculated also in the
numerical procedure.
\par
So, let $x$ be an arbitrary vertex point of the primal problem (\ref{e24}).
Then the associated index vector $\A(x)$ contains a collection of $n$ indices
of active side conditions with linear independent gradients and the other
indices are assembled in the index vector $\N(x)$.  Let henceforth
\[ \ba{.}{l}
\A(x) = \{\rho _1, \ldots, \rho _n\}, \;
\N (x) = \{ \sigma _1, \ldots, \sigma _{m-n}\},\\
\A(x) \cap \N(x) = \emptyset, \; \A(x) \cup \N(x) = \{1,\ldots,m\}.
\ea{.}
\]
in a fixed order which can be chosen at the beginning and let
\[
\fbox{$
B^{\A} = \ba{[}{l} b^{\rho _1}\\ \vdots\\ b^{\rho _n} \ea{]}, \; b^{\rho _j}
\; \mbox{rows}, \quad A := [B^{\A}]^{-1} =: [a_1, \ldots, a_n], \; a_k \;
\mbox{columns}.
$}
\]
Then the matrix $B^{\A}$ is regular by assumption and because
$B^{\A}A = I$ we have
%
\beqn
b^{\rho _j}a_k = \delta ^j{}_k \quad (\mbox{{\sc Kronecker} symbol}).
\eeqn
%
Accordingly, a fixed column vector $a_k$ is contained in
$n - 1$ hyperplanes
%
\begin{equation} \label{e25}
b^{\rho _j}x = 0, \; \rho _j \in \A(x), \; \rho _j \neq \rho _k,
\end{equation}
%
which lie parallel to the corresponding boundary planes $b^{\rho _j}x = \gamma
^{\rho _j}$ of the feasible domain.  Hence every fixed $a_k$ is parallel
to exactly one edge of $\Omega$:
\par
\[
\fbox{$
\mbox{The columns of $A$ are the edges of $\Omega$ pointing to $x$}
$}
\]
\par
and the matrix $A$ is called {\bf edge matrix}.  By assumption, the vectors
$b^{\rho _j}, j = 1, \ldots n $, form a {\bf basis} of the row space
$\Bbb{R}_n$ and $B^{\A}$ is called therefore {\bf basis matrix} (in vertex
$x$).  There exists a $w \in \Bbb{R}_n$ such that
%
%
\begin{equation} \label{e26}
a = \sum_{k=1}^n w_kb^{\rho _k} \Longleftrightarrow a = wB^{\A}
\Longleftrightarrow w = aA.
\end{equation}
%
%
The vector $w_ka_k/\|a_k\|^2$ is the orthogonal projection of the gradient $a$
of the objective function on the edge $a_k$ hence the present method is called
projection method.  The vector $y \in \Bbb{R}_m$ with
\[
\fbox{$
y_{\rho _k} = \ba{\{}{ll} w_k, &k= 1, \ldots, n,\\
                          0    & \mbox{else},       \ea{.}
$}
\]
i.e., the vector
\[
y \in \Bbb{R}_m, \; y_{\A} = w, \; y_{\N} = 0,
\]
satisfies the optimality condition of Theorem \ref{s2}, $yB = a, \; y \geq 0,$
if and only if $w \geq 0$ does hold.  Hence, by Corollary \ref{f4}, this $y$ is
solution of the dual row problem
%
\[
\min \{yc, \; yB = a, \; y \geq 0\}.
\]
Summing up, the {\bf optimality condition} found from (\ref{e26})  has
the following form
%
\begin{equation} \label{e27}
\fbox{$
y_{\rho _k} \equiv w_k := aa_k \geq 0, \; k = 1, \ldots, n.
$}
\end{equation}
%
\par
If the vertex $x$ is {\bf not optimum} then the optimality condition of the
Main Theorem is violated.  We now introduce two {\bf index functions}, $\phi$
and $\psi$.  The first determines the side condition with the absolute number
$\rho _j$ whose gradient is removed from the row basis $\{b^{\rho _k}, \; k =
1, \ldots, n\}$, and the second determines the side condition with the absolute
number $\sigma _i$ whose gradient is taken into the row basis instead.
Geometrically, this means that we exchange one vertex of the feasible domain
for another in a way that the objective function is improved and the entire
process is called {\bf swapping} below.  In order to have a unique value, we
choose
%
\begin{equation} \label{e28}
\fbox{$
j = \min \arg_k \min \{\phi (k) := aa_k, \; k \in \{1, \ldots, n\}\},
$}
\end{equation}
i.e., we choose the smallest index $j$ for which $\phi (k), \; k = 1, \ldots,
n$, becomes minimum.  Then $aa_j < 0$ as the optimality conditions is
assumed to be violated , and minimum, therefore the angle between the gradient
$a$ of the objective function and $a_j$ is maximum for all $k = 1, \ldots ,n$.
Consequently, $- a \, a_j > 0$ is maximum and $- a_j$ is the {\bf optimum
search direction} in vertex $x$.  For the new point
\[
\wi{x} := x - \tau a_j, \; \tau  > 0,
\]
we obtain by this construction
%
\begin{equation} \label{e29}
a\tilde{x} = a(x - \tau a_j) \geq a \, x.
\end{equation}
In order to compute the {\bf optimum step width} $\tau $, we substitute
$\wi{x}$ into the inactive side conditions and determine the maximum $\tau
$ such that all of these conditions remain still valid:
%
\[
b^{\sigma _k}(x - \tau a_j) \leq \gamma ^{\sigma _k}, \; \sigma _k \in \N(x),
\]
i.e.,
\begin{equation} \label{e30}
b^{\sigma _k}x - \gamma ^{\sigma _k} \leq \tau b^{\sigma _k} a_j, \;
\sigma _k \in \N(x).
\end{equation}
%
\par
%
{\bf Case 1.} $\forall \; \sigma _k \in \N (x):  \; b^{\sigma _k}a_j
\geq 0$.  Then (\ref{e30}) holds for all $\tau > 0$.  In this case the
objective function $ax$ is unbounded from above in the feasible domain $\Omega$
and the problem has no solution.
\par
{\bf Case 2.} $\exists \; \sigma _k \in \N(x): \; b^{\sigma _k}a_j < 0$.
Then we choose
%
\begin{equation} \label{e31}
\fbox{$ \dis
i := \min \arg_k \min \{\psi(k) := \frac{b^{\sigma _k}x - \gamma
^{\sigma _k}}{b^{\sigma _k}a_j}, \; b^{\sigma _k}a_j < 0, \; \sigma _k \in
\N(x)\}.
$}
\end{equation}
%
It follows that $\tau ^* := \psi(i) \geq 0$ holds because $b^{\sigma _k}x -
\gamma ^{\sigma _k} \leq 0, \; \sigma _k \in \N(x)$.  But in a degenerated
vertex we obtain $\tau ^* = 0$ because $b^{\sigma _k}x - \gamma ^{\sigma _k} =
0$ holds also for some $\sigma _k \in \N(x)$.  Then the basis is exchanged in
the vertex $x$ but the vertex itself is not changed.  This situation can lead
to a cyclic altering of basis elements yielding no remove from the vertex $x$.
If the vertex is not degenerate then we always have $\tau ^* > 0$.  Degenerate
vertices appear frequently but cycles are very seldom nevertheless they must be
prevented by suitable arrangements in order that the algorithm always comes to
an end (`` {\sc Bland}'s rule'').
\par
By construction we now have $a\wi{x} \geq ax$ and, for $i$ and $j$
fixed,
\beqn \ba{.}{lll}
b^{\rho _k}\tilde{x} = \gamma ^{\rho _k}, & \rho _k \in \A(x), \; \rho _k \neq
\rho _j, \; \rho _k = \sigma _i \in \N(x) ,\\
%
b^{\sigma _k}\tilde{x} \leq \gamma ^{\sigma _k}, & \sigma _k \in \N(x), \;
\sigma _k \neq \sigma _i, \\
%
b^{\rho _j}\tilde{x} = b^{\rho _j}x - \tau ^*b^{\rho _j}a_j &= \gamma ^{\rho
_j} - \tau ^* \cdot 1 \leq \gamma ^{\rho _j}.
 \ea{.} \eeqn
Therefore there are at least $n$ side conditions active in $\wi{x}$ again and
the new point $\wi{x}$ is a vertex with
\[
\A(\wi{x}) = (\A(x) \backslash \{\rho _j\}) \cup \{\sigma _i\}, \quad
\N(\wi{x}) = (\N(x) \backslash \{\sigma _i\}) \cup \{\rho _j\},
\]
if the vectors $b^k, \; k \in \A (\wi{x})$, are also linear independent.
\par
For instance, let
%
\begin{equation} \label{e32}
\A(x) = \{1, \ldots,n\}, \; \N(x) = \{n+1, \ldots, m\},
\end{equation}
and let $\rho _j = j$ then
\[
\A(\wi{x}) = \{1, \ldots, j-1,i,j+1, \ldots,n\}, \quad
\N(\wi{x}) = \{n+1, \ldots, i-1, j, i+1, \ldots, m\}.
\]
%
After exchange, the $j$-th row of $B^{\A}$ has changed and is now the $i$-th
row of $B^{\N}$ whereas the former i-th row of $B^{\N}$ has become the $j$-th
row of $B^{\A}$.
\par
In order to prove that $\wi{x}$ is a vertex again, it suffices to consider the
model case (\ref{e32}).  Then we have to show that the gradients
\beqn
b^k, \; k = 1, \ldots ,n, \; k \neq j, \; k = i > n,
\eeqn
are linear independent.  The vectors $b^k, \; k = 1, \ldots ,n$ are independent
because $x$ has been a vertex.  Moreover, we have $a_j \neq 0$, and $b^ka_j =
0, \; k = 1, \ldots ,n, \; k \neq j$, as well as $b^ia_j \neq 0 $ by
(\ref{e31}).  Without loss of generality, let $j = n, \; b^i = b, \; a_j = u$.
%
%
\begin{lemma} \label{l7}
Let $b^1,\ldots, b^{n-1}, b \in \Bbb{R}_n, \; 0 \neq u \in \Bbb{R}^n$,
and $b^1, \ldots b^{n-1}$ linear independent then
\beqn
b^1, \ldots , b^{n-1}, b \; \mbox{linear independent} \;
\Longleftrightarrow (b^ku = 0, \; k = 1, \ldots, n - 1, \Longrightarrow bu
\neq 0).
\eeqn
\end{lemma}
%
%
Proof.  The assertion is obviously true because $n$ independent vectors
in $\Bbb{R}^n$ cannot be contained in a hyperplane $xu = 0$.
\par
\vspace{2mm}
It is unnecessary to update the index vectors $\A(x)$ and $\N(x)$ in the
projection method if only the solution shall be computed since a change of
basis implies only a permutation of side conditions here. But in all other
cases these index vectors must be actualized, in particular,
%
\begin{itemize}
\item if the {\sc Lagrange} multiplier $y$ or a solution of the dual
problem shall be computed with the same algorithm,
%
\item if {\sc Bland}'s rule comes to application to avoid cycling,
%
\item if one works with reduced  ``tableaus'' as e.g. in the simplex method.
\end{itemize}
%
We always compute the {\sc Lagrange} multipliers in the procedures
developed below hence also the index sets are always updated.
\par
%
%-------------------------------------------------------------------------
In the projection method for the general problem (\ref{e1}),
\[
\fbox{$
\ba{.}{l}
\max\{ax, \; B^{\P}x = c^{\P}, \; B^{\Q}x \leq c^{\Q}\},\\
B \in \Bbb{R}^m{}_n, \; \rank (B^{\P}) = p, \;
\rank (B) = n,
\ea{.}
$}
\]
we
compute an initial vertex point $x$ which satisfies
$B^{\P}x = c^{\P}$ and afterwards exclude the first $p$ equations from
inactivation. The index vectors now are
\[ \ba{.}{rcl}
\A(x) &=& \{1, \ldots, p, \rho _{p+1}, \ldots, \rho _n\}, \;
\A(x) \backslash \P = \{\rho _{p+1}, \ldots, \rho _n\},\\
\N(x) &=& \{\sigma _1, \ldots , \sigma _{m-n}\}
\ea{.}
\]
and the first $p$ rows of $B$ belong to the basis of every vertex $x$.  The
optimality conditions modify slightly,
\begin{equation} \label{e34}
\fbox{$
y_{\rho _k} \equiv w_k := aa_k \geq 0,\; k = p+1, \ldots, n.
$}
\end{equation}
If a vertex is not optimum then we choose the row $j$ to be inactivated such
that
\[
y_{\rho _j} = aa_j \leq aa_k, \; k = p+1, \ldots, n.
\]
The remaining part of the procedure is the same as above, in particular, the
pivot rule for the row to be taken into the basis remains unaltered.
\par
In times of mechanic computing machines, it has been of advantage to write
all data into a large matrix called {\bf tableau} during calculation.  We
join this practice by optical reasons although it has losen somewhat of its
significance today. Recalling that
\[
y_{\P} = [y_1, \ldots, y_p], \;
y_{\A \backslash \P} = [y_{\rho _{p+1}}, \ldots, y_{\rho _n}],
\]
and writing
\[
A_{1:p} := [a_1, \ldots,a_p], \;
A_{p+1:n} := [a_{p+1}, \ldots, a_n]
\]
in the edge matrix $A$, we use the following three forms of the {\bf full}
tableau being simple and more explicit from left to right
%
%
\begin{equation} \label{e36}
\fbox{$
\ba{.}{rcl}
{\bf P(x,y)} &:=&
\ba{[}{cc} A & x\\ D & r\\w & f
\ea{]}
:=
\ba{[}{cc} A & Ac^{\A}\\B^{\N}A & B^{\N}Ac^{\A} - c^{\N}\\aA &
aAc^{\A} \ea{]}\\[7mm]
&=&
\ba{[}{ccc} A_{1:p} &  A_{p+1:n}           & Ac^{\A}\\
      B^{\N}A_{1:p} & B^{\N}A_{p+1:n}      & B^{\N}Ac^{\A} - c^{\N}\\
       y_{\P}       & y_{\A \backslash \P} & aAc^{\A} \ea{]}.
\ea{.}
$}
\end{equation}
The vector $x = Ac^{\A}$ is the actual vertex point, $r := B^{\N}x - c^{\N}$
contains the nontrivial i.e.  nonzero part of the residuum $Bx - c$, and $f =
aAc^{\A}$ is the actual value of the objective function.
Let $P := P(x)$ be the {\bf row permutation matrix} defined by
\[
P\ba{[}{c}1\\ \vdots\\m \ea{]} = \ba{[}{c}\A(x) \\ \N(x) \ea{]}
\]
and let $x^*$ be a solution of the primal problem (\ref{e24}) then the vector
$y^* = [y^*_{\A}, y^*_{\N}]P$ with $y^*_{\N} = 0 \in \Bbb{R}_{m-n}$ is a {\sc
Lagrange} multiplier and hence a solution of the dual problem (\ref{e8}) by
Theorem \ref{f5}.  But observe that the first $p$ components of $y_{\A}$ belong
to the equality restrictions and thus the optimality condition reads
\[
y_{\A \backslash \P} \geq 0.
\]
\par
In the {\bf small} tableau, the first block row of (\ref{e36}) is omitted and
the solution $x^*$ is computed afterwards using the index vector $\A(x)$,
\[
B^{\A}x^* = c^{\A}.
\]
This way of procedure has some advantages in large problems because here the
vertex $x$ is computed only once at the end of calculation.
\par
With respect to the entries in the left form of the tableau (\ref{e36}), the
pivot rules (\ref{e28}) and (\ref{e31}) have a somewhat simpler form,
namely
%
\begin{equation} \label{e36a}
\fbox{$
\ba{.}{rcll}
j &=& \min \arg_k \min \{w_k, & k \in \{p+1, \ldots, n\}\},\\
i &=& \dis \min \arg_k \min \{\frac{r^k}
{d^k{}_j}, \; d^k{}_j < 0, & k \in \{1, \ldots, m -n \}\},\\
(n+i,j) &=& \; \mbox{pivot position in {\bf P}}.
\ea{.}
$}
\end{equation}
Naturally, in the {\bf small} tableau the element at position $(i,j)$ is the
pivot element.
