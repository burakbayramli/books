\section{Characterization of Solutions}
In algorithmic design of linear programming problems by ``vertex methods'',
optimality or possible improvement is checked by a geometric property of the
gradient $a$ of the objective function $x \mapsto ax$.  Because $a$ is an
element of the dual space $\Bbb{R}_n$ to the column space $\Bbb{R}^n$, the
optimum is characterized in the following {\bf main theorem of linear
programming } problem (\ref{e1}) by ``dual'' properties of a solution.
In contrast to many authors, a {\bf solution} of a problem is always an
optimum here and not only a feasible point. Recall that
\[
\P = \{1, \ldots, p\}, \quad \Q = \{p+1, \ldots,m\}.
\]
%
\begin{theorem} \label{s2} ({\sc Farkas}, Characterization Theorem)
Let
$
a \in \Bbb{R}_n, \; B \in \Bbb{R}^m{}_n, \; c \in \Bbb{R}^m.
$
Then $x^* \in \Bbb{R}^n$ is a solution of
%
\begin{equation} \label{e4}
 \max\{ax, \; B^{\P}x = c^{\P}, \; B^{\Q}x \leq c^{\Q}\}
\end{equation}
%
if and only if there exists a pair $(x^*,y^*) \in \Bbb{R}^n \times
\Bbb{R}_m $ with the following three properties
%
\begin{equation} \label{e5}
\fbox{$
\ba{.}{lll}
(a) & B^{\P}x^* = c^{\P}, \; B^{\Q}x^* \leq c^{\Q},        & \mbox{primal
feasibility},\\
(b) & y^*B = a, \; y^*_{\Q} \geq 0,  & \mbox{dual feasibility},\\
(c) & y^*_{\Q}(B^{\Q}x^* - c^{\Q}) = 0,               & \mbox{complement
condition}. \ea{.}
$}
\end{equation}
\end{theorem}
%
%
A vector $y^*$ with the above properties is called {\sc
Lagrange} multiplier of the problem.
\par
%
Proof. If $(x^* ,y^*)$ is a pair with the above properties then
$x^*$ is feasible by (a) and we have for all feasible $x$
%
\beqn
\ba{.}{rcl}
ax^* - ax & \stackrel{(b)}{=} & y^*Bx^* - y^*Bx\\
& \stackrel{(a)}{=}&
y^*_{\P}c^{\P} + y^*_{\Q}B^{\Q}x^* - y^*_{\P}c^{\P} - y^*_{\Q}B^{\Q}x\\
& \stackrel{(c)}{=}&
y^*_{\Q}c^{\Q} - y^*_{\Q}B^{\Q}x = y^*_{\Q}(c^{\Q} - B^{\Q}x)  \geq 0
\ea{.}
\eeqn
hence $x^*$ is optimum.
\par
On the other side, if $x^*$ is a solution of the problem (\ref{e4})
then $x^*$ is feasible and thus (a) is fulfilled. Let
\[
A := \ba{[}{r} B^{\P}\\ -B^{\P} \\B^{\Q} \ea{]}, \; g := \ba{[}{r} c^{\P}\\
-c^{\P}\\  c^{\Q} \ea{]}, \; q := g - Ax^* = \ba{[}{c}0\\ 0 \\ c^{\Q} -
B^{\Q}x^* \ea{]} \geq 0 \]
and let
%
\beqn \ba{.}{lll}
\L &:= \{(x,r) \in \Bbb{R}^{n+1}, & [-A,q]\ba{[}{l}x\\r\ea{]} \geq 0 \},\\
\M &:= \{(x,r) \in \Bbb{R}^{n+1}, & [-a,0]\ba{[}{l}x\\r\ea{]} \geq 0 \}.
\ea{.} \eeqn
%
Both these sets are convex cones.  If $(x,r) \in \L$ and $0 < \alpha , \;
\alpha r \leq 1$, then
\[
q - \alpha Ax = q(1 - \alpha r) + \alpha (- Ax + qr) \geq 0,
\]
because $q \geq 0$, $1 - \alpha r \geq 0$ and we have $- Ax + qr \geq 0$.
Hence, by definition of $q$,
\[
0 \leq g - A(x^* + \alpha x).
\]
Therefore $x^* + \alpha x$ is also feasible and thus
\[
a(x^* + \alpha x) \leq ax^*,
\]
because $x^*$ was optimum.  It follows that $- a \, x \geq 0$, i.e., $(x,r) \in
\M$.  By this way the inclusion $\L \subset \M$ is shown which means that
\[
[-A,q]\ba{[}{l}x\\r\ea{]} \geq 0  \Longrightarrow
[-a,0]\ba{[}{l}x\\r\ea{]} \geq 0.
\]
By the corollary to the Lemma of {\sc Farkas} there exists a $0 \leq u^* =
(v^*_1, v^*_2, w^*) \in \Bbb{R}_p \times \Bbb{R}_p \times \Bbb{R}_{m-p}$ with
\[
u^*[- A, \; q] = [-a, \; 0].
\]
This equation is equivalent to
\beqn \ba{.}{rl}
- v_1^*B^{\P} + v_2^*B^{\P} - w^*B^{\Q}  &= - a,\\
w^*(c^{\Q} - B^{\Q}x^*)             & = 0.
\ea{.}
\eeqn
Because $w^* \geq 0$ the second part of the assertion now follows for
$y^*_{\P} = v^*_1 - v^*_2, \; y^*_{\Q} = w$.\\
%$\Box$ \hfill\\
%
%
%
By simple transposition we derive from Theorem  \ref{s2}
%
\begin{corollary} \label{f3}
Let
$
a \in \Bbb{R}^n, \; B \in \Bbb{R}^n{}_m, \; c \in \Bbb{R}_m.
$
Then $x^* \in \Bbb{R}_n$ is a solution of
%
\begin{equation} \label{e6}
\max\{xa, \; xB_{\P} = c_{\P}, \; xB_{\Q} \leq c_{\Q}\}
\end{equation}
%
if and only if there exists a pair $(x^*,y^*) \in (\Bbb{R}_n
\times \Bbb{R}^m)$ with the following three properties
%
\begin{equation} \label{e7}
\ba{.}{lll}
(a) & x^*B_{\P} = c_{\P}, \; x^*B_{\Q} \leq c_{\Q},       & \mbox{primal
feasibility},\\
(b) & By^* = a, \; y^{*\Q} \geq 0,    & \mbox{dual feasibility},\\
(c) & (x^*B_{\Q} - c_{\Q})y^{*\Q} = 0,              & \mbox{complement
condition}.  \ea{.}
\end{equation}
\end{corollary}
%
The next result corresponds to Theorem \ref{s2} and characterizes a solution
$y^*$ of the dual problem.
\par
\begin{corollary} \label{f4}
Let
$
a \in \Bbb{R}_n, \; B \in \Bbb{R}^m{}_n, \; c \in \Bbb{R}^m.
$
Then $y^* \in \Bbb{R}_m$ is a solution of
%
\begin{equation} \label{e8}
\min\{yc, \; yB = a, \; y_{\Q} \geq 0\}
\end{equation}
%
if and only if there exists a pair $(y^*,z^*) \in \Bbb{R}_m \times
\Bbb{R}^{n+m-p}$ with the following three
properties where $z^* = (u^*, v^*), \; u^* \in \Bbb{R}^n, \; v^* \in
\Bbb{R}^{m-p}$,
%
\begin{equation} \label{e9}
\fbox{$
\ba{.}{lll}
(a) & y^*B = a, \; y^*_{\Q} \geq 0, &\mbox{primal feasibility},\\
(b) & B^{\P}u^* + c^{\P} = 0, \; B^{\Q}u^* + c^{\Q} = v^* \geq 0,
                                      & \mbox{dual feasibility},\\
(c) & y^*_{\Q}v^* = 0,              & \mbox{complement condition}.
\ea{.}
$}
\end{equation}
\end{corollary}
%
%
Proof. We apply Corollary \ref{f3} and substitute for this
\[
\wi{x} = y, \; \wi{a} = - c,\;
\wi{B} = \ba{[}{cc} B^{\P} & O\\ B^{\Q} & - I_{m-p} \ea{]}, \;
\wi{c} = a, \;  \wi{d} = 0,
\]
into the equivalence conditions of Corollary \ref{f3}.  In order to avoid
confusions, we denote the {\sc Lagrange} multiplicators of the dual problem
(\ref{e8}) always by $z = (u,v) \in \Bbb{R}^{n+m-p}$.  Then the following
optimality conditions follow immediately from Corollary \ref{f3} for (\ref{e8})
\[
\ba{.}{l}
\wi{x}^*\wi{B}_{\P} = \wi{c}_{\P}, \; \wi{x}^*\wi{B}_{\Q} \leq \wi{c}_{\Q}
\Longleftrightarrow  y^*B = a, \; y^*_{\Q} \geq 0;\\[2mm]
\wi{B}\ba{[}{c}u^*\\v^*\ea{]} = \wi{a}, \; v^* \geq 0\\
\Longleftrightarrow \ba{[}{l} B^{\P}\\ B^{\Q}\ea{]}u^*
+ \ba{[}{c} O \\ -I_{m-p} \ea{]}v^* = - \ba{[}{l}c^{\P}\\ c^{\Q} \ea{]}, \; v^*
\geq 0\\ \Longleftrightarrow
B^{\P}u^* + c^{\P} = 0, \; B^{\Q}u^* + c^{\Q} = v^* \geq 0,\\[2mm]
y^*\ba{[}{c} 0 \\ -I_{m-p}\ea{]}v^* = 0
\Longleftrightarrow y^*_{\Q}v^* = 0.
\ea{.}
\]
%$\Box$ \hfill \\
Now, writing $u^* = - x^*$, the three conditions (\ref{e9}) coincide with the
three conditions (\ref{e5}) of the Main Theorem \ref{s2}. Moreover, we have
by (a), (b) and (c)
\[
ax^* = y^*Bx^*
     = y^*B^{\P}x^* + y^*B^{\Q}x^*
     = y^*c^{\P} + y^*c^{\Q} = y^*c.
\]
Therefore the following important result is a direct conclusion from the Main
Theorem \ref{s2} and Corollary \ref{f4} and justifies geometrically the
notation {\bf primal problem} and {\bf dual problem}:
%
\begin{theorem} \label{f5} (Duality Theorem)\\
(a) The problem (\ref{e4}),
\[
\max \{ax, \; x \in \Omega\}, \; \Omega = \{x \in \Bbb{R}^n, \; B^{\P}x =
c^{\P}, \; B^{\Q}x \leq c^{\Q}\},
\]
has a solution $x^*$ if and only if the problem (\ref{e8})
\[
\min \{yc, \; y \in \Omega'\}, \;
\Omega' = \{y \in \Bbb{R}_m, \; yB = a, \; y_{\Q} \geq 0\},
\]
has a solution $y^*$.\\
(b)
\begin{equation} \label{e10}
\max \{ax, \; x \in \Omega\} = ax^* = y^*c
= \min \{yc, \; y \in \Omega'\}.
\end{equation}
(c) A {\sc Lagrange} multiplier $y^*$ of (\ref{e4}) is
a solution of (\ref{e8}).\\
(d) If $z^* = (u^*,v^*)$ is a {\sc Lagrange} multipliers of
(\ref{e8}) then $- u^*$ is a solution of (\ref{e4}).\\
(e) A solution $x^*$ of (\ref{e4}) and a {\sc Lagrange}
multiplier $z^* = (u^*,v^*)$ (\ref{e8}) satisfy the relation
\begin{equation} \label{e11}
\fbox{$
x^* = - u^*, \; r^* := B^{\Q}x^* - c^{\Q} = - v^*.
$}
\end{equation}
(f) If $x^*$ is feasible for (\ref{e4}) and $y^*$ is feasible for
(\ref{e8}) then $x^*$ is a solution of (\ref{e4}) and $y^*$ is a solution
of (\ref{e8}) if and only if the complement condition is satisfied,
\[
y^*_{\Q}(B^{\Q}x^* - c^{\Q}) = 0.
\]
\end{theorem}
%
The dual of a dual problem is always a primal problem.  However, in some
special problems, the dual problem has a slightly different form to that given
in Theorem \ref{f5}, because trivial parts of a {\sc Lagrange} multiplier
are omitted.  If, for instance, the primal problem has the form
\[
\max\{ax, \; Bx \leq c, \; x \geq 0\},
\]
then the dual problem can be written in the simpler form
\[
\min \{yc, \; yB \geq a, \; y \geq 0\},
\]
as is verified easily.
\par
In the next definition, $b^k$ denotes again a row of $ B = [b^k]_{k=1}^m$ and
$\gamma ^k$ an element of the vector $c$.  Recall that an {\bf index vector}
$\A$ is an {\bf index set} where the position of the entries is fixed.  The
length of $\A$ is denoted by $|\A|$.
%
\begin{definition} \label{d2}
(a) Let $\C \subset \Bbb{R}^n$ be convex then  $v \in \C$ is an
{\bf extreme point} if
\beqn
\forall \: x, \, y \in \C: \; v \in [x,y] \Longrightarrow v = x \;
\vee \; v = y.
\eeqn
(b) An extreme point $x$ of the polyeder $\Omega := \{ x \in \Bbb{R}^n, \;
B^{\P}x = c^{\P}, \; B^{\Q}x \leq c^{\Q}\}$ is a {\bf vertex} of $\Omega$.\\
(c) For $x \in \Omega := \{ x \in \Bbb{R}^n, \; Bx \leq c \}, \; B \in
\Bbb{R}^m{}_n$, let
%
\beqn \ba{.}{lll}
\A^*(x)   &:= \{k \in \{1,\ldots ,m\}, \; b^kx = \gamma ^k\},\\
\N^*(x)   &:= \{1,\ldots,m\} \backslash \A^*(x),
\ea{.} \eeqn
be the index vector of the active side conditions and of the inactive side
conditions in $x$, respectively.
\end{definition}
%
\par
For the remaining part of this section, it suffice to consider the somewhat
simpler primal problem
%
\begin{equation} \label{e12}
\fbox{$
\max\{a x, \; Bx \leq c\}, \; B \in \Bbb{R}^m{}_n.
$}
\end{equation}
%
\begin{theorem} \label{s3} (Characterization of a vertex)
Let $\rank (B) = n$.  Then $x \in \Omega = \{x \in \Bbb{R}^n, \:
Bx \leq c\}$ is a vertex if and only if there exists an index vector of
length $n$,
\[
\A (x) = \{\rho _1, \ldots, \rho _n\}, \; \rho _k \in \A^*(x),
\]
such that the matrix
\[
B^{\A} := \ba{[}{c}b^{\rho _1} \\ \vdots \\ b^{\rho _n} \ea{]}, \;
\A := \A(x),
\]
is regular.
\end{theorem}
%
%
Proof. (a) If there exists an index vector $\A(x)$ of length $n$ with the
mentioned property then the system of equations
\beqn
b^{\rho _k}x = \gamma  ^{\rho _k}, \; \rho _k \in {\A}(x),
\eeqn
i.e.  the system $B^{\A}x = c^{\A}$, has a unique solution $x$.  Let $x + \tau
\, v \in \Omega$, i.e., $B(x + \tau \, v) \leq c$ then $\tau \, b^{\rho _k}v
\leq 0$ must hold for all $\rho _k \in {\A}(x)$ hence we have $\tau \geq 0$ or
$ \tau \leq 0$.  Then $x$ cannot be a genuine convex combination and therefore
must be a vertex.
\par
(b) If $\rank (B^{\A}) < n$ holds for all index vectors $\A(x)$ which --- as
index sets --- satisfy ${\A}(x) \subset \A^*(x)$ then let $\A(x) = \{1, \ldots,
j\}, \; j < n$ without loss of generality.  Then there exists a $0 \neq v \in
\Bbb{R}^n$ such that $b^kv = 0, \; k = 1, \dots ,j$, and we obtain
%
\beqn \ba{.}{ll}
b^k(x + \tau v) = \gamma ^k, \; \tau \in \Bbb{R}, &k \in
{\cal A}(x), \\
\dis b^k(x + \tau v)  = (\sum^j_{i=1}\kappa ^k{}_ib^i)(x + \tau v)
= \sum^j_{i=1}\kappa ^k{}_ib^ix = b^kx = \gamma ^k, & k \in {\cal A}^*(x)
\backslash {\cal A}(x).
\ea{.} \eeqn
%
Consequently $x + \tau v \in \Omega$ for all sufficient small $\tau \in
\Bbb{R}$ (because of the inactive side conditions) therefore $x$ cannot be a
vertex.\\
%$\Box$ \hfill \\
%
In other words, $x$ is a vertex of the feasible domain $\Omega = \{x \in
\Bbb{R}^n, \; Bx \leq c\}$ if and only if there exists an index set $\A$
such that the following representation holds for $x$:
\[
x = [B^{\A}]^{-1}c^{\A}.
\]
At least $n$ side conditions $b^kx \leq \gamma ^k$ are active in a vertex
$x$ and exactly $n$ gradients of these side conditions are linear independent.
A vertex $x$ is called {\bf degenerate} if $|\A^*(x)| > |\A (x)| = n$, i.e.,
if more than $n$ side conditions are active in $x$ with the maximum
number $n$ of independent gradients.  In this case the index set $\A(x)$ of
Theorem \ref{s3} is not determined uniquely.
\par
If $x$ is a vertex and $\rho _k \in \A(x)$ then we say that $b^{\rho _k}$ is
``in the basis of $x$'' or briefly that $\rho _k$ is in the basis of $x$
because the vectors $b^{\rho _k}$ with $\rho _k \in \A(x)$ form a basis of the
row space $\Bbb{R}_n$.
\par
%
Equality restrictions
\[
B^{\P}x = c^{\P}, \;  \rank (B^{\P}) = p
\]
can be incorporated into (\ref{e12}) easily if we require that for all index
vectors defined by Theorem \ref{s3} holds
%
\begin{equation} \label{e14}
\A(x) = \{1, \ldots , p, \rho _{p+1}, \ldots, \rho _n\},
\end{equation}
%
i.e., if we require that the first $p$ restrictions remain always active.
\par
%
The next result shows that the set of solutions of the primal and of the dual
problem is a convex set.
%
\begin{theorem} \label{s4} Let $x_1, \ldots x_k$ be solutions of the problem
\[
\max \{ ax, \; B^{\P}x = c^{\P}, \; B^{\Q}x \leq c^{\Q}\}.
\]
Then also every {\bf convex combination}
\[
x = \sum_{i=1}^k\lambda ^ix_i, \; 0 \leq \lambda ^i \leq 1, \;
\sum_{i=1}\lambda ^i = 1,
\]
is a solution.
\end{theorem}
%
Proof. We have $x \in \Omega$ because $\Omega$ is a convex set and
$ax_1 = \ldots = ax_k$ by assumption. Consequently,
\[
ax = \lambda ^1ax_1 + \ldots + \lambda ^kax_k
= (\lambda ^1 + \ldots + \lambda ^k)ax_1 = ax_1.
\]
%$\Box$ \hfill\\
%
The following theorem now shows that it suffices to consider the vertices of
the feasible domain in solving a linear programming problem, and it
also explains the way to find a vertex starting from a feasible point.
%
\begin{theorem} \label{s5}
(Vertex Theorem) Let $\rank (B) = n$, let $u \in \Omega = \{x \in \Bbb{R}^n, \;
Bx \leq c\}, \; B \in \Bbb{R}^m{}_n$, and let the objective function $x \mapsto
ax$ be bounded on $\Omega$.  Then there exists a vertex $v \in \Omega$ with $av
\geq au$.
\end{theorem}
%
%
Beweis. We consider a sequence of corrections
\[
v = u + \sigma s, \quad \sigma  > 0,
\]
with search direction $s$ and feasible points
$u$ and $v$. If $u$ is an interior point of
$\Omega$ then we choose $s = a^T$, i.e., that direction in which the
objective function grows in the strongest way. In the other cases let
without loss of generality
\begin{equation} \label{e15}
\ba{.}{ll}
b^ku = \gamma^k \; \mbox{and} \; b^k \; \mbox{linear independent}, \quad
k = 1, \ldots ,j \; ( 0 \leq j \; \mbox{maximum}),\\
b^ku = \gamma^k, \quad k = 1, \ldots ,l,\\
b^ku < \gamma  ^k, \quad k = l+1, \ldots ,m.
\ea{.}
\end{equation}
If $j = n$ then $v = u$ is a vertex by Theorem \ref{s3}.
If $j < n$ then there exists a search direction
$s \neq 0$ with
%
\[
b^ks = 0, \quad k = 1, \dots ,j.
\]
Because of the linear dependence of the vectors $b^{j+1}, \ldots, b^l$ from
$b^1, \ldots, b^j$ we also have
\[
b^ks = 0, \quad k = 1, \ldots, l.
\]
Let without loss of generality $as \geq 0$ else replace $s$
by $-s$. Then
%
\begin{equation} \label{e16}
\forall \: \sigma  > 0: \; a(u + \sigma s) \geq a \, u
\end{equation}
(for all $\sigma  \in \Bbb{R}$ if $as = 0$). Substitution of $u +
\sigma s$ in the side conditions yields
\beqn
b^k(u + \sigma s) = b^ku = \gamma ^k, \; k = 1, \ldots, l,
\eeqn
and the conditions for feasibility
\beqn
b^k(u + \sigma s) &= b^ku + \sigma c^ks \leq \gamma ^k, \; k = l+1, \ldots, m.
\eeqn
These inequalities imply that
%
\begin{equation} \label{e17}
\sigma b^ks \leq \gamma ^k - b^ku \; (\geq 0), \quad k = l+1, \ldots ,m,
\end{equation}
are to be satisfied.
\par
{\bf Case 1.}
$\exists \;  k \in \{l+1, \ldots, m\}: \; b^ks > 0$. Then let
\beqn \ba{.}{rcl}
i  &=& \arg_k \min \{\psi (k) := \dis \frac{b^ku - \gamma ^k}{- b^ks}, \;
b^ks > 0, \; k = l+1, \dots ,m\}\\
\Longrightarrow \sigma  &:=& \psi(i) > 0.
\ea{.} \eeqn
Then $i$ is that index for which the index function $\psi:  k \mapsto \psi(k)$
attains its minimum under the given conditions.  Without loss of generality
let $i = l+1$ then it follows that
\beqn
b^{l+1}(u + \sigma s) &\dis = b^{l+1}u + \frac{\gamma ^{l+1} -
b^{l+1}u}{b^{l+1}s} b^{l+1}s
\dis = b^{l+1}u + \gamma ^{l+1} - b^{l+1}u = \gamma ^{l+1}.
\eeqn
Moreover, we have
%
\beqn \ba{.}{lll}
b^k(u + \sigma s) &=    \gamma ^k, &k = 1, \ldots ,l,\\
b^k(u + \sigma s) &\leq \gamma ^k, &k = l+2, \ldots ,m.
\ea{.} \eeqn
%
Thus at least $l+1$ side conditions are active in $v = u + \sigma s \in \Omega$
and $av \geq au$ because $as \geq 0$ by (\ref{e16}).  Besides, $b^{l+1}$ is
linear independent from $b^1, \ldots , b^l$ because
\beqn
b^{l+1}s > 0 \; \mbox{und} \; b^ks = 0, \; k = 1, \ldots ,l.
\eeqn
%
%
{\bf Case 2.} $\forall \; k \in \{l+1, \ldots, m\}:  \; b^ks \leq 0$.  By
assumption we have $b^ks = 0, \; k = 1, \ldots ,l < n$ therefore we obtain
because (\ref{e15}) that
\[
\ba{.}{lll}
b^k(u + \sigma s) &= \gamma ^k, &\forall \: \sigma  > 0, \; k = 1, \ldots
,l,\\
b^k(u + \sigma s) &< \gamma ^k, &\forall \: \sigma  > 0, \; k = l+1,
\ldots ,m.
\ea{.}
\]
Hence $u + \sigma s \in \Omega $ holds for all $\sigma > 0$ and this second
case can appear only if the feasible set $\Omega$ is not bounded.  Because $as
\geq 0$ we have $a(u + \sigma s) \geq au$.  If $as > 0$ then the objective
function is not bounded on $\Omega$ and the problem has no solution.  If $as =
0$ then the search direction is perpendicular to the gradient of the objective
function.
\par
If now $b^ks = 0, \; k = 1, \ldots ,m \geq n$ then all $b^k$ are contained in
the hyperplane defined by $s^Tx = 0$ which is a contradiction to the assumption
that $\rank(B) = n$.  Hence there exists a $k \in \{l+1, \ldots ,m\}$ with
$b^ks < 0$.  Choose
\beqn \ba{.}{rcl}
i  &:=& \dis \arg_k \min \{\psi(k) := \frac{b^ku - \gamma ^k}{b^ks}, \; b^ks
< 0, \; k = l+1, \ldots ,m\}\\
\Longrightarrow \sigma  &:=& \psi(i) > 0.
\ea{.} \eeqn
Without loss of generality let $i = l+1$ then we obtain in the same way as
above
%
%
\beqn \ba{.}{ll}
b^k(u - \sigma s) &= \gamma ^k, \quad k = 1, \ldots ,l, \; \mbox{since} \;
b^ks = 0, \\
b^{l+1}(u - \sigma s) &\dis = b^{l+1}u - \frac{\gamma ^{l+1} -
b^{l+1}u}{- b^{l+1}s} b^{l+1}s = \gamma ^{l+1},\\
b^k(u - \sigma s) &\leq \gamma ^k, \quad k = l+2, \ldots ,m.\\
 \ea{.} \eeqn
%
%
Now at least $l+1$ side conditions are active in $v = u - \sigma s$ and we have
$av = au$ because of $as = 0$.  In the present case, too, $c^{l+1}$ is linear
independent from $b^k, \; k = 1, \ldots ,l$, because
\beqn
b^{l+1} s < 0 \; \mbox{and} \; b^ks= 0, \; k = 1, \ldots ,l.
\eeqn
%
A comparison shows that the improvement of Case 2 is also attained if one
replaces here $s$ by $-s$ and then returns to Case 1.
\par
The continuation of this process yields finally a $v \in \Omega$ with $n$
active side conditions of which the gradients are linear independent hence a
vertex $v$.\\
%$\Box$ \hfill\\
%
Naturally, the Vertex Theorem \ref{s5} holds also for inequalities in the row
form $xB \leq c \in \Bbb{R}_m$, for the full primal problem (\ref{e1}), and for
the dual problem problem (\ref{e8}).  It shows that the feasible domain has
vertices if the rank condition is fulfilled and their number is bounded because
$m$ side conditions allow at most ${m \choose n}$ different basis sets.  For
every solution there exists a vertex being also solution.
%
\par
%
If $\rank (B) < n$ then the feasible domain has no vertices but the boundary
consists of edges and of linear varieties of higher dimension. Let again
$\A(x)$ denote an index vector of active side conditions in $x$ with linear
independent gradients.
%
\begin{definition} \label{d3}
Let $\rank (B) < n$ and let
\[
\E := \{x \in \Omega , \; {\cal A}(x) = {\cal A}(\E) \; \mbox{konstant}, \;
|{\cal A}(\E)| = \rank (B)\} \neq \emptyset,
\]
Then $\E$ is a {\bf extreme set} of $\Omega$.
\end{definition}
%
Thus the set ${\cal E}$ is characterized by the fixed index set
 ${\cal A}({\cal E})$.
\par
Let $l = \rank (B)$, let $\E$ be an extreme set of
$\Omega$ and let without loss of generality
\[
b^kx = \gamma ^k, \; k \in {\cal A}(\E) = \{1,\ldots, l\}, \; b^kx \leq \gamma
 ^k, \; k = l+1, \ldots, m.
\]
If $u \in \E$ and $v$ satisfies
%
\begin{equation} \label{e18}
b^kv = 0, \; k = l + 1,\ldots , m,
\end{equation}
%
then also $u + v \in \E$.  However, the system (\ref{e18}) has a nontrivial
solution therefore every extreme set contains an infinite number of elements.
Moreover, we have
\[
u, v \in \E \Longrightarrow \; \forall \; \lambda: \; u + \lambda (v - u) \in
\E .
\]
If the objective function is bounded on $\Omega$ then $ a(v - u) = 0 $ must be
fulfilled, i.e.,
\[
\forall \; u, v \in \E: \; au = av.
\]
Consequently the objective function is constant on $\E$.  By definition, two
different extreme sets cannot have the same set of active side conditions,
hence there exists only a finite number of extreme sets.  A suitable
modifications of the proof of Theorem \ref{s5} shows that for every $x \in
\Omega$ there exists an extreme set $\E$ such that $ax \leq av, \; \forall \; v
\in \E$.  Thus we obtain the following result:
%
\begin{theorem} \label{s6}
If $\rank (B) < n$ then the feasible domain $\Omega = \{x \in \Bbb{R}^n, \; Bx
\leq c\}$ has a finite number of extreme sets.  If the problem has a solution
then there exists an extreme set of solutions.
\end{theorem}
%
