Let $B \in \Bbb{R}^m{}_n$ denote a matrix with $m$ rows and $n$ columns, let
$\Bbb{R}^n := \Bbb{R}^n{}_1, \; \Bbb{R}_n := \Bbb{R}^1{}_n$, and let $a \in
\Bbb{R}_n, \; c \in \Bbb{R}^m$.  Using the {\sc Matlab} notation we write
$B^{\A} := B(\A,\, : \,)$ and $B_{\A} := B(\, : \, ,\A)$ with a suitable
index set $\A$ and e.g. $1:p = \{1,\ldots,p\}$. We consider the pair of
linear problems
\[
\ba{.}{lll}
 (P) & \Max\{ax, \; B^{1:p}x = c^{1:p}, \; B^{p+1:m}x \leq c^{p+1:m}\}, & x
\in \Bbb{R}^n,\\

 (D) & \Min\{yc,   \; yB = a, \; y_{p+1:m} \geq 0\}, & y \in \Bbb{R}_m,
\ea{.}
\]
and compare the tableaus of both problems without using slack variables.
However, following \cite{BeRi}, the problem (P) is here considered for primal
problem and the simplex problem (D) has a slightly modified form in comparison
to standard textbooks; cf. e.g. \cite{BeRi}, \cite{Las}, \cite{Naz},
\cite{Pad}, \cite{Schr}.
\par
An application of the Lemma of {\sc Farkas}, cf. \cite{Spe}, \cite{BeRi},
yields the following characterization of a solution of (P) and (D):
%
\begin{theorem} \label{t1} (a)
$x^{\ast} \in \Bbb{R}^n$ is a solution of (P) iff there exists a pair
$(x^*,y^*) \in \Bbb{R}^n \times \Bbb{R}_m$ satisfying
%
\[
\ba{.}{lll}
(i) & B^{1:p}x^{\ast} = c^{1:p}, \; B^{p+1:m}x^* \leq c^{p+1:m}  &
\mbox{primal feasibility},\\
(ii) & y^*B = a, \; y^*_{p+1:m} \geq 0        & \mbox{dual feasibility},\\
(iii) & y^*_{p+1:m}(B^{p+1:m}x^* - c^{p+1:m}) = 0 &
\mbox{complementary slackness}.
 \ea{.}
\]
(b) $y^* \in \Bbb{R}_m$ is a solution of (D) iff there
exists a pair $(y^*,z^*) \in \Bbb{R}_m \times \Bbb{R}^{n+m-p}$
with $z^* = (u^*,v^*), \; u^* \in \Bbb{R}^n, \;
v^* \in \Bbb{R}^{m-p}$, and the following properties
\[
\ba{.}{lll}
(i)   & y^*B = a, \; y^*_{p+1:m} \geq 0  & \mbox{primal feasibility},\\
(ii)  & B^{1:p}u^* + c^{1:p} = 0, \; B^{p+1:m}u^* + c^{p+1:m} = v^* \geq 0
& \mbox{dual feasibility},\\
(iii) & y^*_{p+1:m}v^* = 0       &
\mbox{complementary slackness}. \ea{.}
\]
\end{theorem}
%
$y^*$ is the {\sc Lagrange} multiplier of the solution $x^*$ of (P) and $z^*$
is the {\sc Lagrange} multiplier of the solution $y^*$ of (D).  Writing
%
\begin{equation} \label{1}
x^* = - u^*, \; B^{p+1:m}x^* - c^{p+1:m} = - v^*,
\end{equation}
we see that the conditions in (a) and in (b) are identical and
%
\begin{equation} \label{2}
ax^* = y^*Bx^* = y^*_{1:p}B^{1:p}x^* + y^*_{p+1:m}B^{p+1:m}x^* =
y^*_{1:p}c^{1:p} + y^*_{p+1:m}c^{p+1:m} = y^*c.
\end{equation}
Therefore we have the following corollary to Theorem \ref{t1}:
%
\begin{corollary} \label{c1}
The problem (P) has a solution $x^*$ iff the problem (D) has a solution
$y^*$ and then (\ref{2}) holds.
\end{corollary}
%
%
In the sequel we suppose that the following rank condition is fulfilled:
%
\begin{equation} \label{3}
\mbox{rank} (B^{1:p}) = p  \;\: \mbox{and} \;\: \mbox{rank} (B) = n.
\end{equation}
%
Let then $x$ be an extreme point of (P), let
$\A(x)$ be (the index set of) a basis of $x$ and $\N(x)$ the corresponding
non-basis,
%
\[ \ba{.}{rcl}
\A(x) &=& \{\rho _1, \ldots, \rho _n\} = \{1, \ldots,p,\rho _{p+1}, \ldots,
\rho _n\},
\\
\N(x) &=& \{\sigma _1, \ldots,\sigma _{m-n}\} = \{1, \ldots, m\} \backslash
\A(x), \ea{.}
\]
and let
\[
B^{\A} = \ba{[}{l} b^{\rho _1} \\ \vdots \\ b^{\rho _n}\ea{]}, \;
%
A = [B^{\A}]^{-1} = [a_1, \ldots, a_n],
\]
where row vectors are denoted by upper indices and column vectors by
lower
indices.  The columns of $A$ are the edges of the feasible set in the extreme
point $x$ with direction pointing to $x$.  Supposing that the problem (P) is
solvable and omitting {\sc Bland}'s rule we choose with $c = [\gamma ^1,
\ldots, \gamma ^m]^T$ for pivot position $(i,j)$
\[ \ba{.}{rcl}
j &=& \min \arg_k \min \{\phi (k) := aa_k, \; k \in \{ p+1, \ldots, n\}\},\\

i &=& \dis  \min \arg_k \min \{\psi(k) := \frac{b^{\sigma _k}x - \gamma
^{\sigma _k}}{b^{\sigma _k}a_j}, \; b^{\sigma _k}a_j < 0, \; k \in
\{1, \ldots, m - n\}\}. \ea{.}
\]
Then a better --- at least not worse --- extreme point $\widehat{x}$ is found
from $x$ by $ \widehat{x} = x - \psi(i)a_j $.  This means that the row vector
$b^{\rho _j}$ is removed from and $b^{\sigma _i}$ is taken into the row basis
of $x$ yielding the basis of $\widehat{x}$.  The tableau of the primal problem
(P) has the following form in the extreme point $x$:
%
\begin{equation} \label{5}
{\bf P}(x,y) = [p^k{}_l] :=
\ba{[}{ll} A & x\\B^{\N}A & r^{\N}\\w & f \ea{]}
= \ba{[}{lll} A_{1:p} & A_{p+1:n} & x\\
                     B^{\N}A_{1:p} & B^{\N}A_{p+1:n} & r^{\N}\\
                     y_{1:p} & y_{\A\backslash {1:p}} & f
           \ea{]},
\end{equation}
where
\[
x = Ac^{\A}, \; r^{\N} = B^{\N}x - c^{\N}, \; y = aA, \; f = aAc^{\A}.
\]
The tableau ${\bf P}(\widehat{x},\widehat{y}) = [q^k{}_l]$ of the extreme point
$\widehat{x}$ is obtained from ${\bf P}(x,y)$ by a global {\sc Gauss-Jordan}
step %
\beqn
\ba{.}{lll}
q^i{}_j = 1/p^i{}_j & \mbox{(pivot element)}\,,\; &
q^k{}_j = p^k{}_j/p^i{}_j \,,\;k \neq i  \quad \mbox{(pivot column)}\,,\\
q^i{}_l = - p^i{}_l/p^i{}_j\,,\: l \neq j
& \mbox{(pivot row)}\,,\; & q^k{}_l =  p^k{}_l - p^k{}_jp^i{}_l/p^i{}_j
\quad\mbox{(others)}.
\ea{.}
\eeqn
%--------------------------------------------------------------------------
\par
Let us now turn to the dual problem (D) which is is written in row form here
for convenience.  This problem has $m$ variables and $n + m - p$ side
conditions being written in the form
%
\begin{equation} \label{6}
y\ba{[}{rr}  B^{1:p} & 0 \\B^{p+1:m} & - I_{m-p} \ea{]} \leq [a, 0] \in
\Bbb{R}_{n+m-p}
\end{equation}
%
where the first $n$ conditions are always active, $yB = a$.  The matrix of the
side conditions (\ref{6}) has the full rank $m$ by assumption (\ref{3}).  A row
vector $y$ is an extreme point iff besides the $n$ active restrictions $yB = a
\in \Bbb{R}_n$ at least further $m - n$ conditions $y^i \leq 0$ are active.
Let $y$ be an extreme point of (D).  Let $\A(y)$ and
$\N(y)$ be index vectors of the same form as above,
\[
\ba{.}{rcl}
\A(y) &=& \{1, \ldots,p,\rho _{p+1},\ldots, \rho _n\},\\
\N(y) &=& \{\sigma _1, \ldots, ,\sigma _{m-n}\} = \{1, \ldots, m\} \backslash
\A(y),
\ea{.}
\]
but with {\bf converse} meaning for the problem (D):
\[ \ba{.}{lll}
\mbox{primal problem} &\; b^ix < \gamma ^i &\Longrightarrow i \in \N(x),\\
\mbox{dual  problem}  &\; y^i > 0          &\Longrightarrow i \in \A(y).
\ea{.}
\]
The gradient matrix $B^{\A}$ and the matrix $B^{\N}$ of the primal problem
have here analogues in the following matrices:
%
\[
\wi{B}_{\wi{\A}} = P^T\ba{[}{cc}  B^{\A} & O \\ B^{\N} & - I_{m-n}\ea{]},
\; \wi{B}_{\wi{\N}} = P^T\ba{[}{c}O \\-I_{n-p} \\ O \ea{]},
\]
%
where $B^{\A}$ and $B^{\N}$ have the same meaning as above and again $A =
[B^{\A}]^{-1} \in \Bbb{R}^n{}_n$.  The row permutation with the matrix $P$ must
be regarded here because of the left-multiplication with the row vector $y,
\; y = [y_{\A},y_{\N}]P$. The edge matrix $\wi{A}$ of the dual problem in
a vertex $y$ is
%
\begin{equation} \label{8}
\wi{A} \equiv \ba{[}{l}\wi{a}^1\\\vdots\\ \wi{a}^m\ea{]}
:= [\wi{B}_{\wi{\A}}]^{-1} =
\ba{[}{cc} A & O\\B^{\N}A & -I_{m-n}\ea{]}P.
\end{equation}
%
Now we recall the dual feasibility condition (ii) of Theorem \ref{t1}(b) and
the relation (\ref{1}) between a solution $x^*$ of the primal problem and the
{\sc Lagrange} multiplier $z^* = (u^*,v^*)$ of the dual problem.  Because
$B^{\A}x^* = c^{\A}$ hence $x^* = Ac^{\A}$, a substitution into (\ref{1})
yields the {\bf dual optimality condition}
%
\[
r^{\N} = - v^{\N} = B^{\N}Ac^{\A} - c^{\N} \leq 0.
\]
\par
If $y$ is not optimum then this condition is violated and,
following the pattern of the primal method,  we have to remove a column of the
gradient matrix $\wi{B}_{\wi{\A}}$ from the basis of $y$, namely
\[
\mbox{a {\bf column} $e_i$ of}  \; P^T\ba{[}{c}O\\ O\\- I_{m-n}\ea{]}.
\]
Furthermore, we have to take a column of $\wi{B}_{\wi{\N}}$ into the basis,
namely
\[
\mbox{a column $e_j$ of} \; P^T\ba{[}{c} O\\ - I_{n-p}\\ O \ea{]}.
\]
In the gradient matrix $\wi{B}_{\wi{\A}}$, this exchange means that a row
$b^{\sigma _i}$ of $B^{\N}$ is taken into the row basis, i.e., into the rows of
$B^{\A}$ and a row $b^{\rho _j}$ of $B^{\A}$ is removed from the basis into
the non-basis, i.e., into the rows of $B^{\N}$.  We choose
\[
i = \max \arg_k \max \{\phi (k) := - v^{\sigma _k}, \; k \in \{ 1, \ldots, m -
n\}\}.
\]
%
Observe that $- v^{\N} = \wi{A}^{(n+1):m}c$ hence $- v^{\sigma _i} =
\wi{a}^{n+i}c > 0$ is maximum.  The selected search direction $\wi{a}^{n+i}$
yields a proper descend for the new point $\widehat{y} = y - \tau \wi{a}^{n+i}$
in the case where $\tau > 0$:
\[
[y - \tau \wi{a}^{n+i}]c < yc.
\]
\par
The maximum feasible step width $\tau$ determines the vector $e_j$ which is
taken into the column basis of $y$ corresponding to the remove of a row of
$B^{\A}$.  For the computation of $\tau $ we subtitute again $\widehat{y}$ into
the inactive side conditions in the same way as in the primal method and then
enlarge $\tau$ until at least one of these conditions becomes `active'.
However, the inactive side conditions are purely sign conditions here and read
%
\[
\widehat{y}_{p+1:m} = [y - \tau \wi{a}^{n+i}]_{p+1:m} \geq 0.
\]
With the above introduced permutation matrix $P$ we obtain
\[
[y - \tau \wi{a}^{n+i}]P^T = [y_{\A}, y_{\N}] - \tau [b^{\sigma _i}A, -e_i]
= [y_{\A} - \tau b^{\sigma _i}A, \; y_{\N} + \tau e_i]
\]
where $e_i \in \Bbb{R}_{m-n}$ is the $i$-th row identity vector and $y_{\N} =
0$ by definition of the index set $\N (y)$. Therefore the conditions for
feasibility are
%
\[
y_{\rho _k} - \tau b^{\sigma _i}a_k \geq 0, \; k = p+1, \ldots ,n.
\]
%
Thus we choose in problem (D) for pivot position $(i,j)$
\begin{equation} \label{6a}
 \ba{.}{rcl}
i &=& \max \arg_k \max \{\phi (k) := - v^{\sigma _k}, \; k \in \{ 1, \ldots, m
- n\}\},\\
j &=& \dis \min \arg_k \min \{\psi(k) := \frac{y_{\rho
_k}}
{b^{\sigma _i}a_k}, \;
%
b^{\sigma _i}a_k > 0, \;  k \in \{p+1, \ldots, n\}\}.
\ea{.}
\end{equation}
%
The new point $\widehat{y}$ is again a vertex point.
\par
For the construction of the tableau, we write the present {\bf row problem} as
maximum problem, $\max\{\wi{x}\wi{a}, \; \wi{x}\wi{B} \leq \wi{c}\}$, then we
obtain in complete analogy to (\ref{5})
%
\begin{equation} \label{14}
\wi{{\bf P}}(x,y) :=
\ba{[}{ccc} \wi{A} & \wi{A}\wi{B}_{\wi{\N}} & \wi{w}\\
\wi{x} & \wi{r}_{\wi{\N}} &  \wi{f} \ea{]}.
\end{equation}
%
Here, $y = \wi{x}P$ is the actual vertex point, $\wi{w}$
contains the nontrivial part of the multipliers $(u,v)$ of the dual problem,
and the vector $\wi{r}$ consists of the nontrivial part of the residuum
again,
\[
y = [y_{\A}, \; y_{\N}]P, \;
\wi{w} = \ba{[}{l}u \\v^{\N}\ea{]}, \;
\wi{r}_{\wi{\N}} = - y_{\A \backslash \{1:p\}}, \;
\wi{f} = - yc,
\]
and
\[
\wi{A}\wi{B}_{\wi{\N}} =
\ba{[}{ccc} A_{1:p} & A_{p+1:n} & O\\ B^{\N}A_{1:p} & B^{\N}A_{p+1:n} &
- I_{m-n} \ea{]}\ba{[}{c} O \\ - I_{n-p} \\ O \ea{]}
%
= \ba{[}{c}- A_{p+1:n} \\ - B^{\N}A_{p+1:n} \ea{]}.
\]
The last column of (\ref{14}) can be multiplied with $-1$ because it
becomes never a pivot column.  Recalling (\ref{8}), the complete tableau
of the simplex method reads in a vertex $y$
%
\[
{\bf \wi {P}}(x,y)
=
\ba{[}{ccccc}
A_{1:p}       & A_{p+1:n}       & O           & -A_{p+1:n}            &
-u \\ B^{\N}A_{1:p} & B^{\N}A_{p+1:n} & - I_{m-n} & -B^{\N}A_{p+1:n}
&    -v^{\N}\\
y_{1:p}  & y_{\A \backslash \{1:p\}}    & y_{\N}      & -y_{\A \backslash
\{1:p\}} &    -\wi{f} \ea{]}.
\]
In this tableau, the second block column appears in the fourth column again
with negative sign hence the fourth block column can be omitted entirely.  Also
the third block column can be deleted because the index vector $\A(y)$ must be
updated and because $y_{\N} = 0$ holds by definition of the index vector
$\N(y)$.  If we regard once more the relation (\ref{1}) between the
solution of the primal problem and the {\sc Lagrange} multipliers $z = (u,v)$
of the dual problem then it turns out that the tableau of the primal problem
and the reduced tableau of the dual problem coincide in the respective optimum
$(x^*,y^*)$,
%
\[
{\bf \wi{P}}_{red}(x^*,y^*) \equiv
{\bf P}(x^*,y^*)
=
\ba{[}{ccc}
A_{1:p}       & A_{p+1:n}                        &   x^* \\
B^{\N}A_{1:p} & B^{\N}A_{p+1:n}                  &   r^{*\N}\\
y^*_{1:p}      & y^*_{\A \backslash \{1:p\}}     &   f^*
\ea{]},
\]
%
if the solution $x^*$ of the primal problem is not degenerate and unique.  (In
this case also the solution $y^*$ of the dual problem is not degenerate and
unique.) Therefore the computational amount of work is the same for
both problems in the mean.
\par
With the pivot elements defined by (\ref{6a}), the global {\sc Gauss-Jordan}
step in problem (D) is the same as in the problem (P).
\par
If $x$ is a non-optimum vertex of the primal problem then the {\sc
Lagrange} multiplier $y$ in the primal tableau is an unfeasible point of
the dual problem because the sign conditions fail.  Conversely, if
$y$ is a non-optimum vertex of the dual problem then the point $x$ being
associated to the {\sc Lagrange} multiplier by  (\ref{1}) is an
unfeasible point of the primal problem.  In solving the primal problem in the
dual form, the solution is thus approximated from the unfeasible domain and the
same fact holds if the dual problem is solved in the primal form.
