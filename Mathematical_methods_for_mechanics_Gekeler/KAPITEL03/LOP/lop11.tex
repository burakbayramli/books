\section{Pivot Rule in Degeneration}
It is well known that a linear programming method built on vertex selection can
fail to be finite because of the possibility of cycling.  There have been
several propositions to overcome this drawback by choosing lexikographic order
rules, cf.  e.g.  \cite{Dan} and \cite{AzDi} but the rule of {\sc Bland}
\cite{Bla} has revealed to be the simplest device and the easiest to implement.
For its development, it suffices to consider the standard primal problem
%
\[
\max \{ax, \; Bx \leq c \}, \; B \in \Bbb{R}^m{}_n,
\]
where $\rank (B) = n$ and the rows of $B$ are denoted by $b^i$ again.  We
suppose that the vertex $x$ is degenerate then the index vectors associated to
$x$, namely
\[
\ba{.}{rcl}
\A(x) &=& \{\rho _1, \ldots, \rho _n\}, \; b^{\rho _k}x =
\gamma ^{\rho _k}, \; b^{\rho _k} \; \mbox{linear independent},\\
\N(x) &=& \{\sigma _1,\ldots,\sigma _{m-n}\} = \{1, \ldots, m\} \backslash
\A(x),
\ea{.}
\]
are not determined uniquely and some side conditions $b^ix \leq \gamma ^i,
\; i \in \N(x)$, are active also.  We choose an arbitrary pair of both index
vectors for further consideration, recall the notations
\[
B^{\A} = [b^i]_{i\in \A}, \; B^{\N} = [b^k]_{k \in \N}, \;
A =[a_1, \ldots, a_n] := [B^{\A}]^{-1},
\]
and introduce the following abbreviations:
\[
w = y_{\A}, \; r = B^{\N}x^{\N} - c^{\N}, \; d_j = [B^{\N}A]_j \equiv
B^{\N}a_j. \]
The otherwise fast pivot ruling pair (\ref{e36a}), namely
\[
\ba{.}{rcl}
j &=& \min \arg_k \min \{w_k, \; k \in \{p+1, \ldots, n\}\},\\
i &=& \dis \min \arg_k \min \{\frac{r^k}
{b^{\sigma _k}a_j}, \; b^{\sigma _k}a_j < 0, \; k \in \{1, \ldots, m -n \}\}.
\ea{.}
\]
is replaced in degeneration by the following rule of {\sc Bland}:\\
(a) Choose the basis element $b^{\rho _j}$ to be removed from the row basis of
$x$, i.e., the pivot column $j$ in the tableau {\bf P} such that
%
\[
  \rho _j = \min\{\rho _k \in \A(x), \; y^{\rho _k} < 0\},
\]
i.e.,
\begin{equation} \label{bl1}
      j = \arg_k\min\{\rho _k \in \A(x), \; y^{\rho _k} < 0\}.
\end{equation}
(b) Choose the vector $b^{\sigma _i}$ to be taken into the basis, i.e., the
pivot row $i$ in the small tableau {\bf P} such that
%
\[
\sigma _i = \dis\min\{\sigma _k \in \N(x), \;
d^k{}_j < 0 \; \mbox{and} \; \frac{r^k}{d^k{}_j} = \min\{\frac{r^l}{d^l{}_j},
\; d^l{}_j < 0\}\},
\]
i.e.,
\begin{equation} \label{bl2}
i =  \arg_k\min\{\sigma _k \in \N(x), \;
d^k{}_j < 0 \; \mbox{and} \; \frac{r^k}{d^k{}_j} = \min\{\frac{r^l}{d^l{}_j},
\; d^l{}_j < 0\}\}.
\end{equation}
Note that all $r^l \leq 0$ and some $r^l = 0$ in a degenerate vertex.  If all
$d^k{}_j > 0$ for the index $j$ given by (\ref{bl1}) then the problem has no
solution.
%
\begin{theorem} \label{s9}
The projection method with pivot rules (\ref{bl1}) and (\ref{bl2}) is finite.
\end{theorem}
%
Proof. In changing a basis in the projection method, either optimality is
stated or it is stated that the problem has no solution or the above
pivot rule fixes a new basis in a unique way. We prove the assertion by
contradiction and suppose that cycling occurs in a vertex $x$. Let
\[
\ba{.}{rcl}
\R(x) &=& \{k \in \{1, \ldots, m\}, \; \forall \; \A(x): \; k \in \A(x)\},\\
{\cal S}(x) &=& \{k \in \{1, \ldots, m\}, \; \forall \; \N(x): \; k \in
\N(x)\},\\
\T(x) &=& \{1, \ldots, m\} \backslash (\R(x) \cup {\cal S}(x))\},
\ea{.}
\]
then the index set
$\T(x)$ consists of all indices entering or leaving a basis during the cycle.
\par
Let
\[
q = \max\{k, \; k \in \T(x)\},
\]
and let
\[
{\bf P} = \ba{[}{ccc} &\A(x) & 0 \\\N(x) & B^{\N}A & r\\ 0 & w &
f\ea{]}, \quad w = y_{\A},
\]
be a tableau in which the pivot column $j$ and the pivot row $i$ are chosen by
(\ref{bl1}) and (\ref{bl2}) such that
%
\[
\rho _j = q, \; i = s,
\]
then, by construction,
\begin{equation} \label{vv}
y_q < 0 \; \mbox{and} \; y_{\rho} \geq 0 \; \mbox{for} \;  \rho < q, \; \rho
\in \A(x).
\end{equation}
Secondly, let
\[
\wi{{\bf P}}   = \ba{[}{ccc} & \wi{\A}(x) & 0  \\
\wi{\N}(x) & B^{\wi{\N}}\wi{A} & \wi{r}\\
0 & \wi{w} & \wi{f}\ea{]},
\; \wi{A} = [\wi{a}_1, \ldots, \wi{a}_n] = [B^{\wi{\A}}]^{-1},
\; \wi{w} = \wi{y}_{\wi{\A}},
\]
be a tableau in which the pivot column $j$ and the pivot row  $i$ are chosen by
(\ref{bl1}) and (\ref{bl2}) such that
%
\begin{equation} \label{uu}
j = t, \; \sigma _i = q,
\end{equation}
%
then, by construction,
%
\[
[B^{\wi{\N}}\wi{a}_t]^i < 0 \; \mbox{and} \;
\forall \; \sigma_k \in \wi{\N}(x): \; \sigma _k < \sigma _i
\Longrightarrow [B^{\wi{\N}}\wi{a}_t]^k \geq 0
\]
which is equivalent to
\begin{equation} \label{ww}
v^q := b^q\wi{a}_t < 0 \; \mbox{and} \; \forall \; \rho  \in \wi{\N}(x): \;
\rho  < q  \Longrightarrow v^{\rho } := b^{\rho }\wi{a}_t \geq 0 .
\end{equation}
\par
The indices $s$ and $t$ are determined by the pivot rule.  By this exchange,
the row $b^q$ with the absolute index $q$ is removed from the basis in
tableau {\bf P} and comes into basis again in tableau $\wi{{\bf P}}$.  By
definition, we have $y_{\A} = aA$, and we write
%
\begin{equation} \label{zz}
\wi{v} := B\wi{a}_t \in \Bbb{R}^m \; \mbox{then} \;
\wi{v}^{\wi{\A}} = B^{\wi{\A}}\wi{a}_t = e_t \equiv [\delta ^k{}_t]_{k=1}^n,
\; \wi{v}^{\wi{\N}} = B^{\wi{\N}}\wi{a}_t.
\end{equation}
If now $P$ denotes the permutation matrix with
\[
[1, \ldots ,m] = [\A(x), \; \N(x)]P,
\]
then
\[
P\wi{v} = \ba{[}{c} v_{\A}\\ v_{\N} \ea{]}
\]
and, by the pivot rule (\ref{uu}),
\[
y_{\A}v^{\A} = aAB_{\A}^{-1}\wi{a}_t = a \, \wi{a}_t = \wi{w}_t < 0.
\]
\par
Thus there exists a $\rho \in \A(x)$ with $y_{\rho }v^{\rho } < 0$ hence
$v^{\rho } \neq 0$ and, by (\ref{zz}), it follows that $\rho \in \wi{\N}(x)
\cup \{\sigma _t\}$.  By this way, we obtain
%
\begin{equation} \label{bl9}
\ba{.}{c} \rho  \in \A(x)\\ \rho  \in \wi{\N}(x) \cup \{\sigma_t\}
\ea{\}} \Longrightarrow \rho  \in \T(x).
\end{equation}
%
Because $y_q < 0$ by (\ref{bl1}) and $v^q < 0$ by (\ref{ww}) but $y_{\rho
}v^{\rho } < 0$ we find that $\rho \neq q$.  As $q \in\T (x)$ was maximum, it
follows by (\ref{bl9}) that $\rho < q$.  But then $y_{\rho} > 0$ by (\ref{vv})
and thus $v^{\rho } < 0$.  This is a contraciction to the choice of $\sigma _i
= q$ in (\ref{uu}) because of (\ref{ww}).
\par
The proof can be modified easily to the case where the first $p$ side
conditions are equations and hence always active.
