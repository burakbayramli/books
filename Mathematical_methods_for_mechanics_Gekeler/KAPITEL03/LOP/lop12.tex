\section{Multiple Solutions and Degeneration}
Let again $B \in \Bbb{R}^m{}_n$ with rows $b^i$ and $c \in \Bbb{R}^m$ with
elements $\gamma ^i$. In this section we consider the general primal-dual pair
of problems (\ref{e1}),
%
\begin{equation} \label{ss1}
\ba{.}{llll}
(P) &\max \{ax, & B^{\P}x = c^{\P}, & B^{\Q}x \leq c^{\Q}\},\\
(D) &\min \{yc, & yB = a,           & y_{\Q} \geq 0\},
\ea{.}
\end{equation}
%
with respective solutions $x^*$ and $y^*$. The complement condition
of Theorem (\ref{s2}) then reads
%
\begin{equation} \label{ss2}
\sum_{i=p+1}^my^*_i(b^ix^* - \gamma ^i) = 0
\end{equation}
%
and obviously, for $i = p+1, \ldots,m$,
\[
\ba{.}{rcl}
y^*_i > 0 &\Longrightarrow& b^ix^* - \gamma ^i = 0,\\
b^ix^* - \gamma ^i < 0 &\Longrightarrow& y^*_i = 0
\ea{.}
\]
because of the side conditions. Recall that
\begin{equation} \label{ss10}
\ba{.}{rcll}
\A(x^*) &=& \{1,\ldots, p,\rho _{p+1}, \ldots,\rho _n\} = \;
\mbox{index vector of a row basis of $x^*$},\\
\N(x^*) &=& \{\sigma _1,\ldots, \sigma _{m-n}\} = \;
 \I_m \backslash \A(x^*) \supset \{i \in \I_m, \; b^ix^* - \gamma ^i
< 0\},\\
\A(y^*) &=& \; \mbox{index vector of a column basis of $y^*$}, \\
 &&\A(y^*) \supset \{i \in \I_m, \; y_i > 0\},\\
\N(y^*) &=& \I_m \backslash \A(y^*),
\ea{.}
\end{equation}
and that
\[
B^{\A} = [b^{\rho _i}]_{\rho _i\in \A(x^*)}, \;
[B^{\A}]^{-1} =: A = [A_{1:p}, A_{p+1:n}] = [[a_1\ldots,
a_p][a_{p+1},\ldots,a_n]]
\]
are the basis matrix and the edge matrix of the primal problem.  If both $x^*$
and $y^*$ are unique then the tableau of the primal problem and the reduced
tableau of the dual problem coincide in allentries in the respective optimum.
If $x^*$ or $y^*$ are not unique then a solution of the respective dual problem
can still be read from data of the tableau:
%
\begin{equation} \label{ss1a}
{\bf P} :=
\ba{[}{cc} A & x\\B^{\N}A & r\\w & f \ea{]}
%
= \ba{[}{ccc} A_{1:p}        & A_{p+1:n}              & x\\
              B^{\N}A_{1:p}  & B^{\N}A_{p+1:n}        & r\\
                      y_{\P} & y_{\A \backslash \P}   & f
           \ea{]},
\end{equation}
\[
x = Ac^{\A}, \; r = B^{\N}x - c^{\N}, \;
y_{\N} = 0, \; f = aAc^{\A}.
\]
Also, the tableau yields the relation
\begin{equation} \label{ss1b}
\A(x^*) = \A(y^*), \; \N(x^*) = \N(y^*)
\end{equation}
which however does not make sense in a non-optimum because then either $x$ or
$y$ is not feasible.
\par
In the present section we deal with the case where $x^*$ or $y^*$ are not
unique or may be degenerate. Then (\ref{ss1b}) does not fix the index
vectors of the respective dual problem in a unique way.
%
 \begin{definition}
\label{ss3} (a) The solution $x^*$ of (\ref{ss1})(P) satisfies the strong
complement condition if
\[
\forall \; i > p:  b^ix^* =  \gamma ^i \Longleftrightarrow y^*_i > 0.
\]
(b) The solution $y^*$ of (\ref{ss1})(D) satisfies the strong complement
condition if
\[
\forall \; i > p: y^*_i = 0 \Longleftrightarrow b^ix^* - \gamma ^i < 0.
\]
\end{definition}
%
\begin{theorem} \label{ss4}
Let $x^*$ be a solution of (\ref{ss1})(P) and a not-degenerate vertex. Then
$x^*$ is the unique solution if and only if it satisfies  the strong
complement condition.
\end{theorem}
%
Proof.
Without loss of generality, let $\A(x^*) = \{1, \ldots, n\}$ then $B^{\A}x^* =
c^{\A}$.\\
(a) Substitution of $x^* - \tau a_k, \; k = p+1, \ldots ,n,$ into the
side conditions yields for sufficiently small $\tau > 0$
%
\[ \ba{.}{lll}
b^k(x^* - \tau a_k) &=  b^kx^* - \tau \cdot 1  < \gamma ^k,\\
b^i(x^* - \tau a_k) &=  b^ix^*  = \gamma ^i, & i = 1, \ldots , n , \; i
\neq k,\\
b^i(x^* - \tau a_k) &< \gamma ^i, & i = n+1, \ldots ,m.
\ea{.}
\]
Therefore $x^* - \tau a_k \in \Omega$ for sufficiently small
$\tau > 0$ and $k = p + 1, \ldots, n$.  We then obtain from the uniqueness of
$x^*$ and the dual feasibility of $y^*$ that
%
\[ \ba{.}{l}
a(x^* - \tau a_k) = a \, x^* - \tau aAe_k
= a \, x^* - \tau y^*_k < a^Tx^*,
\ea{.}
\]
hence $y^*_k > 0, \; k = p + 1, \ldots , n$.\\
(b) On the other side, if $x^*$ satisfies the strong complement condition
and if $x \neq x^*$ is a feasible point then we can write
\[
x = x^* +  Au
\]
with a suitable vector $u \neq 0$ because $A$ is regular. Substitution
into the active side conditions yields
\[
u^i = 0, \; i = 1, \ldots, p, \quad u^i \leq 0, \; i = p + 1, \ldots, n.
\]
The dual feasibility says that $a = y^*_{\A}B^{\A}$ and it follows that
\[
ax = ax^* + aAu = ax^* + y^*_{\A}B^{\A}Au
= ax^* + y^*_{\A}u = ax^* + \sum_{i=p+1}^ny^*_iu^i.
\]
Because $y^*_i > 0, \; i = p+1, \ldots, n$, and $0 \neq u \leq 0$ we obtain $ax
< ax^*$ hence $x^*$ is unique.\\
%$\Box$ \hfill\\
%
%
By application of this result to the dual problem (\ref{ss1})(D) in the form
\begin{equation} \label{ss3c}
\min \{yc, \; yB = a,  y\ba{[}{c}  O  \\-I_{m-p}\ea{]} \leq 0\}
\end{equation}
we obtain
%
\begin{theorem} \label{ss5}
Let $y^*$ be a solution of (\ref{ss1})(D) and a not-degenerate vertex. Then
$y^*$ is the unique solution if and only if it satisfies  the strong
complement condition.
\end{theorem}
%
For further investigation we now introduce the following index sets:
\[
\ba{.}{rclll}
\R(x^*) &=& \{i \in \{n+1, \ldots, m \}, &\sigma _i  \in \N(x^*), &r^i =
b^{\sigma _i}x - \gamma ^{\sigma _i} = 0\},\\
{\cal S} (y^*) &=& \{j \in \{ p+1, \ldots, n \}, &\rho _j  \in \A(y^*),
&y^{\rho _j} = aa_j = 0\},
\ea{.}
\]
then we may redefine degeneracy:
\par
\[
\fbox{$
\ba{.}{l}
\mbox{The solution $x^*$ of (\ref{ss1})(P) is degenerate if and only if
$\R(x^*) \neq \emptyset$}.\\
\mbox{The solution $y^*$ of (\ref{ss1})(D) is degenerate if and only if ${\cal
S} (y^*) \neq \emptyset$}.
\ea{.}
$}
\]
\par
%
With these notations and the matrix
\[
A_{{\cal S} } = [a_i]_{i\in {\cal S} (y^*)}, \; A = [a_1, \ldots, a_n],
\]
the next result describes the set of solutions of the primal problem
(\ref{ss1})(P).
%
\begin{theorem} \label{ss7} Let $x^*$ be a solution of (\ref{ss1})(P) with
basis index vector $\A(x^*)$ and $\N(x^*) = \I_m \backslash
\A(x^*)$. Let $\A(y^*) = \A(x^*), \; s := |{\cal S} (y^*)| > 0$ and let
\[
{\cal H} = \{h \in \Bbb{R}^s, \; h \geq 0, \; B^{\N}(x^* - A_{{\cal S} }h)
\leq c^{\N}\}.
\]
Then $\{x^* - A_{{\cal S} }h, \; h \in {\cal H}\}$ is the set of solutions
of (\ref{ss1})(P).
\end{theorem}
%
If ${\cal H}$ is not empty we may introduce a new objective function $x
\mapsto bx$ and solve the secondary linear programming problem
\[
\max\{b(x^*- A_{{\cal S} }h), \; h \in {\cal H}\}.
\]
%
\ Proof. Let again
\[
\A(x^*) = \{1, \ldots , n\}, \; \N(x^*) = \{n+1, \ldots , m\}
\]
without loss of generality.
\par
(a) Let $x \neq x^*$ be optimum then we can write
\[
x = x^* - Ag,
\]
with a suitable vector $g \in \Bbb{R}^n$ and, by the feasibility of $x$, we
have
%
\begin{equation} \label{ss8}
\ba{.}{lll}
b^i(x^* - Ag) &= \gamma ^i, &i = 1, \ldots , p,\\
b^i(x^* - Ag) &\leq \gamma ^i, & i = p+1, \ldots, m.
\ea{.}
\end{equation}
We write $\dis g = \ba{[}{c}g^{1:p}\\ g^{p+1:n} \ea{]}$
because the first $p$ side conditions are always active. From
%
\[
B^{\A} = \ba{[}{c}B^{1:p}\\B^{p+1:n}\ea{]}, \quad A = [A_{1:p}, A_{p+1:n}]
\]
we obtain
\[ \ba{.}{ll}
B^{1:p} \cdot A_{1:p} = I_p, & \; B^{1:p} \cdot A_{(p+1):n} = 0,\\
B^{(p+1):n} \cdot A_{1:p} = 0,   & \; B^{(p+1):n} \cdot A_{(p+1):n} = I_{n-p}.
\ea{.}
\]
Substution into (\ref{ss8}) yields
\[ \ba{.}{ll}
B^{1:p}A_{1:p}g^{1:p} = g^{1:p} &= 0,\\
B^{(p+1):n}A_{(p+1):n}g^{(p+1):n} = g^{(p+1):n} &\geq 0,\\
B^{\N}(x^* - Ag)&\leq c^{\N}.
\ea{.}
\]
%
Because $x$ and $x^*$ are optimum by assumption, we have $a(x^* - Ag) = ax^*$
hence $a A_{(p+1):n}g_{(p+1):n} = 0$.  Because $ y^*_{\A} = aA = [aA_{1:p},
aA_{(p+1):n}] \geq 0, $ and $g_{(p+1):n} \geq 0$, we find that
%
\[
\forall \; i \in \{p+1, \ldots, n\} \backslash {\cal S} (y^*): \; y^*_{\rho i}
> 0 \Longrightarrow g^i = 0.
\]
Hence, writing $g_{{\cal S} } = h$, the first part of the assertion is
established, \[
x = x^* - A_{{\cal S} }h, \; h \in {\cal H}.
\]
\par
(b) By definition of the index set ${\cal S} $ we have $y^*_{{\cal S} } =
aA_{{\cal S} } = 0$ hence it follows that every $x = x - A_{{\cal S} }h$ is
optimum, \[
ax = ax^* - a A_{{\cal S} }h = ax^* - y^*_{{\cal S} }h = ax^*.
\]
Moreover, $x$ is feasible for $h \geq 0$ because
\[
B^{1:p}(x^* - A_{{\cal S} }h) = B^{1:p}x^* + 0 = c^{1:p},
\]
and
\[
\ba{[}{c}B^{(p+1):n}\\B^{(n+1):m}\ea{]}(x^* - A_{{\cal S} }h)
= \ba{[}{c}c^{(p+1):n}\\B^{(n+1):m}x^* \ea{]} - \ba{[}{c}I_{{\cal S}
}\\B^{(n+1):m}A_{{\cal S} }\ea{]}h \leq
\ba{[}{c}c^{(p+1):n}\\c^{(n+1):m}\ea{]}
\]
where the lower inequality follows by assumption.
%
\par
%
\begin{theorem} \label{sss10}
Let $y^*$ be a solution of (\ref{ss1})(D) with basis index vector $\A(y^*)$ and
$\N(y^*) = \I_m \backslash \A(y^*)$.  Let $\A(x^*) = \A(y^*), \; q := |\R(x^*)|
> 0$ and let
\[
{\cal H} = \{h \in \Bbb{R}_q, \; h \geq 0, \; [y^* - h[B^{\N}A]^{\R}]_{\A
\backslash \{1:p\}} \geq 0\}.
\]
Then $\{y \in \Bbb{R}_m, \; y_{\A} = [y^* - h[B^{\N}A]^{\R}]_{\A}, \; y_{\N}
= h[I_{m-n}]^{\R}, \; h \in {\cal H}\}$ is the set of
solutions of (\ref{ss1})(D).
\end{theorem}
%
If ${\cal H}$ is not empty we may introduce a new objective function $y
\mapsto yb$ again and solve the secondary programming problem
\[
\min \{([y^* - h[B^{\N}A]^{\R}]_{\A}, \; h[I_{m-n}]^{\R})Pb, \; h \in
{\cal H}\}.
\]
Proof. Recall that $y = [y_{\A}, \; y_{\N}]P, \; y^*_{\A \backslash \P} \geq 0,
\; y^*_{\N} = 0$. We consider the dual problem (\ref{ss3c}) for primal
problem in row form as in Section 2.2 and obtain for basis matrix and non-basis
matrix as in  (\ref{bab})
\[
\wi{A}^{-1} = \wi{B}_{\wi{\A}} = P^T\ba{[}{cc} B^{\A} & 0\\B^{\N} & -
I_{m-n}\ea{]}, \;
\wi{B}_{\wi{\N}} = P^T \ba{[}{c} O \\- I_{n-p}\\ O \ea{]}.
\]
(a) Let $y \neq y^*$ be optimum then we can write
\begin{equation} \label{ss11}
y = y^* - g\wi{A},
\end{equation}
and by the feasibility of $y$ we have
\begin{equation} \label{hhh}
(y^* - g\wi{A})B = a, \; - (y^* - g\wi{A})_{p+1:m} \leq 0.
\end{equation}
We split again $g = [g_{1:n}, \, g_{(n+1):m}]$ and
in the same way
\[
\wi{B}_{\wi{\A}} = [\wi{B}_{1:n}, \; \wi{B}_{(n+1):m}], \; \wi{A} =
\ba{[}{c}\wi{A}^{1:n}\\
\wi{A}^{(n+1):m}\ea{]}
\]
then
\[ \ba{.}{ll}
\wi{A}^{1:n} \cdot \wi{B}_{1:n} = I_n, &
\wi{A}^{1:n} \cdot \wi{B}_{(n+1):m} = 0,\\
\wi{A}^{(n+1):m} \cdot \wi{B}_{1:n} = 0, &
\wi{A}^{(n+1):m} \cdot \wi{B}_{(n+1):m} = I_{m-n}.
\ea{.}
\]
Substitution of (\ref{ss11}) into the first equation of (\ref{hhh}) yields
\[
a = (y^* - g\wi{A})B = a - g\wi{A}\wi{B}_{1:n} = a - g_{1:n}
\Longrightarrow g_{1:n} = 0.
\]
Substitution of (\ref{ss11}) into the second equation of (\ref{hhh}) yields
because $\N \subset \{p+1, \ldots, m\}$
\[
- (y^* - g\wi{A})_{\N} = - y^*_{\N} + g\wi{A}P^T\ba{[}{c}O \\ -I_{m-n}\ea{]}
= -g\wi{A}\wi{B}_{(n+1):m} = - g_{(n+1):m} \leq 0
\]
hence $g_{(n+1):m} \geq 0$.
Because $y$ and $y^*$ are optimum by assumption, we have
\[
(y^* - g\wi{A})c = y^*c
\]
hence
\begin{equation} \label{ss12}
g_{(n+1):m}\wi{A}^{(n+1):m}c = 0.
\end{equation}
The nontrival part of the {\sc Lagrange} multiplier of the present problem has
the form
\[
z^{*\wi{\A}} = \wi{A}c = \ba{[}{c}u\\ v^{\N}\ea{]}
 = \ba{[}{c}\wi{A}^{1:n}c \\  \wi{A}^{(n+1):m}c\ea{]}
= - \ba{[}{c} Ac^{\A}\\ B^{\N}Ac^{\A} - c^{\N}\ea{]},
\; v^{\N} \geq 0.
\]
Hence we obtain from (\ref{ss12}) and
$g_{(n+1):m} \geq 0$ that
\[
\forall \; i \in \{n+1,\ldots,m\} \backslash \R(y^*): \; v^{\sigma _i} > 0
\Longrightarrow g_i = 0.
\]
Writing $g_{\R} = h$, the assertion now follows,
\[
y = y^* - h\wi{A}^{\R}
 = [y^* - h[B^{\N}A]^{\R}]_{\A}, \; h[I_{m-n}]^{\R}]P, \; h \in {\cal H}.
\]
(b) By definition of the index set $\R$ we have $z^{*\R} = -
[B^{\N}Ac^{\A} - c^{\N}]^{\R} = 0$ hence every $y = y^*
- h\wi{A}^{\R}$ is optimum,
\[
yc = y^*c - h\wi{A}^{\R}c = y^*c - hz^{*\R} = y^*c.
\]
Also, $y$ is feasible for $h \geq 0$ because $\R \subset \{n+1,\ldots,m\}$
and thus
%
\[ \ba{.}{rcl}
(y^* - h\wi{A}^{\R})B &=&
y^*B - h\wi{A}^{\R}\wi{B}_{1:n} = y^*B + 0 = a,\\
(y^* - h\wi{A}^{\R})_{\N} &=&  0 - h\wi{A}^{\R}P^TI_{m-n}
= h\wi{A}^{\R}\wi{B}_{(n+1):m} = hI^{\R}_{m-n} \geq 0, \\
(y^* - h\wi{A}^{\R})_{\A \backslash \P} &\geq& 0
\ea{.}
\]
where the last inequality follows by assumption.
%$\Box$ \hfill
\par
The results being derived above can also be represented in a somewhat condensed
form.  For this we say that a problem (P)/(D) is degenerate/unique if the
respective solution has this property.
%
\begin{corollary} \label{cs1} (P) unique and not-degenerate if and only if (D)
unique and not-degenerate.
\end{corollary}
%
Proof. If (P) unique and not-degenerate then (D) unique by Theorem \ref{sss10}.
But (P) satisfies the strong complement condition by Theorem \ref{ss4} hence
$y^{*i} > 0$ for all $i \in \A(y^*)$ hence (D) is not degenerate. The other
direction follows in the same way.
\par
\begin{corollary} \label{cs2} \hfill\\
(a) (P) and (D) not degenerate then (P) and (D) unique.\\
(b) (P) degenerate and (D) not then (P) unique and (D) not.\\
(c) (D) degenerate and (P) not then (D) unique and (P) not.\\
(d) (P) and (D) degenerate then both (P) and (D) unique or not.
\end{corollary}
%
Proof. Cf. \cite{Bau}. (a) follows from Theorem \ref{ss7} and \ref{sss10}.\\
(b) It follows from Theorem \ref{ss7} that (P) is unique.  But (D) cannot
fulfill the strong complement condition because $r^i = 0$ for some $i \in
\N(x^*)$ and $y^{*i} = 0$ for all $i \in \N(x^*)$.  Hence (D) is not unique by
Theorem \ref{ss5}.\\
(c) Same proof as in (b) with roles interchanged.\\
(d) For every of the four cases an example can be found.\\
%$\Box$ \hfill


