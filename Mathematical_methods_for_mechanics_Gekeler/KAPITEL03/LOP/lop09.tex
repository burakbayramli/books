\section{Phase 1 -- Phase 2 in Simplex Method}
For the treatment of phase 1 we write the dual problem
in the following form:
%
\begin{equation} \label{uuu}
\min \{yc, \; yB = a, \; y\ba{[}{cc}0 \\ - I_{m-p}\ea{]} \leq 0\},
\; B \in \Bbb{R}^m{}_n, \; a \geq 0,
\end{equation}
%
with identity matrix $I_{m-p}$ and rank condition
%
\[
\rank (B^{1:p}) = p, \; \rank (B) = n.
\]
%
Naturally, this problem can be viewed for special problem with bounded
variables, $l \leq y, \;  l^i = - \infty, \; i = 1, \ldots p, \; l^i
= 0, \; i = p+1, \ldots m, \; y \leq u, \; u^i = \infty, \; i = 1, \ldots, m$.
But here we prefer to deal with the above form because of the significance
of the simplex method.
\par
Without loss of generality, we suppose that $a \geq 0$ but possible sign
changes must be regarded if a solution $x^*$ of the primal problem is read from
the final tableau.  The associated phase-1--problem (\ref{e20}) has also a row
form here, namely
%
\begin{equation} \label{kvk}
\max \{[\xi, \; y]\ba{[}{c}-1 \\b\ea{]}, \;
[\xi , \; y]\ba{[}{ccc} -1 & 0 & - e\\ O & B^{1:p}  & O\\
O & B^{(p+1):m} & -I_{m-p} \ea{]} \leq [0, \; a, \; d]
\end{equation}
where
\[
d = 0 \in \Bbb{R}_{m-p}, \; b = \sum_{i=1}^nb_i
\]
with columns $b_i$ of $B$. Here we have
\[
\delta  = \sum_{i=1}^n\alpha ^i, \;
\xi ^0 = \max\{0, - d_1, \ldots, - d_{m-p}\} = 0
\]
with the components $d_k = 0$ of the vector $d$ and thus the point $(\xi ,y) =
(0,0) \in \Bbb{R}_{m+1}$ is feasible for (\ref{kvk}).  If $(\xi
^{\star},y^{\star})$ is a solution of (\ref{kvk}) then $y^{\star}$ is feasible
for (\ref{uuu}) if $\xi ^{\star} = 0$ or if $y^{\star}b = \delta $ by Corollary
(\ref{f6}) and Lemma (\ref{l6}).  Therefore we may cancel the additional
variable $\xi $ entirely and ask whether a solution $y^{\star}$ satisfies
$y^{\star}b = \delta$.  So we consider the following phase 1 problem
%
\begin{equation} \label{sv1}
\max \{yb, \; y\wi{B} \leq \wi{a} \}
\end{equation}
with
\begin{equation} \label{sv2}
\wi{B} =  \ba{[}{cc} B^{1:p} & O\\ B^{p+1:m} & -I_{m-p}\ea{]}, \;
\wi{a} =  [ a, \; 0] \in \Bbb{R}_{n+m-p},
\end{equation}
having the same dimension as (\ref{uuu}) with $n+m-p$ side conditions and $m$
variables.  Regarding the particular form of the design matrix, we split up the
column and the row index set in the following way into disjunct subsets:
%
\begin{equation} \label{ph3}
\ba{.}{rcll}
\{1, \ldots,m\} &\simeq& \{\W, \; p + \L, \; p + \U\},\\
\W &=& \{1, \ldots, p\},\\
\L &=& \{i \in \{1, \ldots, m - p\}, \; y_{p+i} \geq 0 \}, & |\L| =: n_l,\\
\U &=& \{i \in \{1, \ldots, m - p\}, \; y_{p+i} = 0\}, & |\U| =: n_u,\\[2mm]
\{1, \ldots,n+m-p\} &\simeq& \{\C, \; \D, \; n + \L, \; n + \U\},\\
\C &=& \{i \in \{1, \ldots, n\}, \; yb_i = \alpha _i\}, &
|\C| =: n_c\},\\
 \D &=& \{i \in \{1, \ldots, n\}, \; i \notin \C \}, &
|\D| = n_d = n - n_c. \ea{.}
\end{equation}
In particular, this splitting means that in a vertex $y$ the index
vector $\L$ shall contain at least all $i$ with $y^{p+i} > 0$ and the index
vector $\C$ contains only $i$ with $yb_i = \alpha _i$.
Let $P$ be the permutation matrix defined
by \[
yP = [y_{\W}, \; y_{\L}, \; y_{\U}].
\]
As in the standard simplex method, the row permutation in $\wi{B}$ must be
regarded here because the vertex $y$ is also written in row form.
If $y$ is a vertex of the problem (\ref{sv1}) considered for {\bf
primal problem} in row form then it must have  an basis index vector of the
form
\[\ba{.}{rcll}
\A(y) &=& \C(y) \cup (n + \U(y)), & n_c + n_u = m,\\
\N(y) &=& \D(y) \cup (n + \L(y)), & n_d + n_l = n-p.
\ea{.}
\]
and a regular matrix of the (column) gradients,
%
\[
P\wi{B}_{\A} = \ba{[}{cc}
         B^{\W}{}_{\C} & O\\ B^{\L}{}_{\C} & O\\ B^{\U}{}_{\C} & -I_{n_u}
                  \ea{]}.
\]
The regularity is fulfilled if and only if the matrix
%
\begin{equation} \label{hjh}
A^{-1} := \ba{[}{c}B^{\W}{}_{\C} \\ B^{\L}{}_{\C} \ea{]}, \;
n_c = p + n_l,
\end{equation}
is regular. Then the edge matrix of the present problem is
%
\[
\wi{A} = [\wi{B}^{\A}]^{-1} = \ba{[}{cc} A & O\\ B^{\U}{}_{\C}A & -
I_{n_u}\ea{]}P  \in \Bbb{R}^m{}_m,
\]
and the matrix of the gradients of the remaining side conditions is
 %
\[ P\wi{B}_{\N} = \ba{[}{ccc}
B^{\W}{}_{\D} & O\\ B^{\L}{}_{\D} & - I_{n_l}\\ B^{\U}{}_{\D} & O
\ea{]} \in \Bbb{R}^m{}_{n-p}.
\]
We obtain
\[
\wi{A}\wi{B}_{\N} =
\ba{[}{cl} \hspace*{-5mm}\!
 A\ba{[}{c}B^{\W}{}_{\D} \\ B^{\L}{}_{\D}\ea{]} & -A_{(p+1):n_c}\\[4mm]
B^{\U}{}_{\C}A \ba{[}{c}

B^{\W}{}_{\D} \\ B^{\L}{}_{\D}\ea{]}

 - B^{\U}{}_{\D} &
- B^{\U}{}_{\C}A_{(p+1):n_c}
\ea{]}
\]
and the full tableau of the problem (\ref{sv1}),
%
\[
\wi{{\bf P}} = \ba{[}{ccc} \wi{A} & \wi{A}\wi{B}_{\N} & \wi{w} \\
                           \wi{y} & \wi{r}            &  \wi{f}
               \ea{]},
\]
has the form
\[
\wi{{\bf P}} = \ba{[}{ccccc}
 A  & O & \hspace{-5mm} \!A\ba{[}{c}B^{\W}{}_{\D} \\ B^{\L}{}_{\D}\ea{]} &
 - A_{(p+1):n_c} & \wi{w}^{1:n_c}\\[4mm]
%
B^{\U}{}_{\C}A & -I_{n_u} &
B^{\U}{}_{\C}A\ba{[}{c}B^{\W}{}_{\D} \\
B^{\L}{}_{\D} \ea{]} - B^{\U}{}_{\D} &
- B^{\U}{}_{\C}A_{(p+1):n_c}  &
 \wi{w}^{(n_c+1):m}\\
%
\wi{y}_{1:n_c} &
\wi{y}_{(n_c+1):m} &
\wi{r}_{1:(n-n_c)} &
\wi{r}_{(n-n_c+1):(n-p)} &
\wi{f}
\ea{]}
\]
where
\[ \ba{.}{lll}
\wi{y} = yP, & \wi{w} = \wi{A}b, & \wi{f} = yb,\\
\wi{r}_{1:(n-n_c)} = yB_{\D} - a_{\D}, &
\wi{r}_{(n-n_c+1):(n-p)} = - y_{p+\L}.
\ea{.}
\]
No information is lost here if the columnms $p+1, \ldots, n$ are cancelled,
and the reduced tableau has the form
%
\begin{equation} \label{svs7}
{\bf P} = \ba{[}{cccc}
 A_{1:p} & \hspace{-5mm} \! A\ba{[}{c}B^{\W}{}_{\D} \\ B^{\L}{}_{\D} \ea{]} &
- A_{(p+1):n_c}  & \wi{w}^{1:n_c}\\[4mm]
%
B^{\U}{}_{\C}A_{1:p} &
B^{\U}{}_{\C}A\ba{[}{c}B^{\W}{}_{\D} \\
B^{\L}{}_{\D}\ea{]} - B^{\U}{}_{\D} &
 - B^{\U}{}_{\C}A_{(p+1):n_c}
 & \wi{w}^{(n_c+1):m}\\
%
y_{1:p} &  \wi{r}_{1:(n-n_c)} & -y_{p+\L}  & \wi{f}
\ea{]}
\end{equation}
with $m+1$ rows and $n+1$ columns.
\par
In phase 1 we start with $n_c = p$ and $\L = \emptyset$ until $n_c = n$ and $\D
= \emptyset$. If this is not possible then the feasible domain is empty. In a
swapping step we have to interchange a column $j > n_c$ of $P\wi{B}_{\A}$ with
absolute index $\rho _j$ for a column $i$ of $P\wi{B}_{\N}$ with absolute index
$\sigma _i$. After swapping, the shape of both $P\wi{B}_{\A}$ and
$P\wi{B}_{\N}$ must be regained if necessary. According to the form of
$P\wi{B}_{\N}$ we have here two cases:
\par
{\bf Case 1.} $i \in \{1, \ldots, n - n_c\}$. Then
swapping yields with $\rho _j \in \U$ and $\sigma _i \in \D$
\[
\ba{.}{rcl}
\U &=& \{\rho _{n_c+1}, \ldots, \rho _{j-1}, \sigma _i,\rho _{j+1}, \ldots,
\rho _{n_c+n_u} \},\\
\D &=& \{\sigma _1, \ldots, \sigma _{i-1}, \rho _j, \sigma _{i+1}, \ldots,
\sigma _{n-n_c}\},\\
\ea{.}
\]
In order to regain the shape of the tableau, we have to make the following
transformations:
\[
\ba{.}{rcll}
\U &\longrightarrow& \{\rho _{n_c+1}, \ldots, \rho _{j-1},\rho _{j+1}, \ldots,
\rho _{n_c+n_u} \}, &n_u := n_u - 1,\\
\C &\longrightarrow& \{\rho _1, \ldots, \rho _{n_c},\sigma _i\}, &n_c := n_c +
1,\\
\D &\longrightarrow& \{\sigma _1, \ldots, \sigma _{i-1}, \sigma _{i+1},
\ldots, \sigma _{n-n_c}\}, &n_d := n_d - 1,\\
\L &\longrightarrow& \{\sigma _{n-n_c+1},\ldots ,\sigma _{n-p},\rho
_j\}, &n_l := n_l + 1.\ea{.}
\]
An interchange of columns in the basis matrix implies an interchange of rows in
the edge matrix.  Therefore, row $j$ is to be moved at positition $n_c
+ 1$ and column $p + i$ in reduced tableau is to be moved at position $n$.  The
other rows and column are to be shifted in a corresponding way.
\par
{\bf Case 2.} $i \in \{n-n_c+1, \ldots, n - n_c + n_l\}$. Then
swapping yields
\[
\ba{.}{rcl}
\U &=& \{\rho _{n_c+1}, \ldots, \rho _{j-1},\sigma _i,\rho _{j+1},
\ldots, \rho _{n_c+n_u}\},\\
\L &=& \{\sigma _1, \ldots, \sigma _{i-1}, \rho _j, \sigma _{i+1}, \ldots,
\sigma _{n-n_c}\}.
\ea{.}
\]
Here the shape of the tableau is preserved.
\par
The problem (\ref{sv1}) has the feasible point $x = 0 \in \Bbb{R}^n$ with
\[
\ba{.}{rcl}
\W(x) &=& \{1, \ldots, p\},\\
\L(x) &=& \emptyset,\\
\U(x) &=& \{p+1, \ldots, m\}
\ea{.}
\]
So it remains to add some trivial side conditions such that $y = 0$ becomes a
vertex of the second auxiliary problem.  We add the conditions
\[
y \ba{[}{c}-I_p \\ 0 \ea{]} \leq 0
\]
then $y = 0$ satisfies $-yI_m = 0$ and thus is a vertex of the augmented
auxiliary problem
%
\begin{equation} \label{sv4}
\max\{yb, \; y\ba{[}{ccc} -I_p & B^{1:p} & O \\O  & B^{(p+1):m} & -I_{m-p}
\ea{]} \leq [0, \; a, \; 0] \}
\end{equation}
with $n+m$ side conditions. A basis index vector of the vertex $y = 0$ of
(\ref{sv4}) is
\[
\A(y) = \{1, \ldots,p,n+p,\ldots,n+m\}.
\]
But in the algorithm we write
%
\[
\A(y) = \{0, \ldots,0,n+p,\ldots,n+m\}
\]
where the first $p$ zeros indicate the spurious side conditions which have to
be inactivated at first.  The reduced first tableau of the second auxiliary
problem has the form
\[
{\bf P} = \ba{[}{ccc} -I_p & -B^{1:p}  & \wi{w}^{1:p}\\
      O & -B^{p+1:m} & \wi{w}^{(p+1):m}\\
       y_{1:p} & -a & \wi{f}
       \ea{]}
\]
with $ y_{1:p} = 0, \; \wi{w} = -b$ and $\wi{f} = 0$.  The regular matrix
$-I_p$ plays here the role of the key matrix $A^{-1}$.  The last column becomes
never pivot column hence the entire tableau can be multplied by $-1$ with a
suitable modification of the pivot rule as in the original simplex problem.  If
$y^{\star}b - \delta = 0$ then the solution $y^{\star}$ of the auxiliary
problem (\ref{sv1}) is a feasible point and also a vertex of the original
problem (\ref{uuu}).  Then $y^{\star}B = a$ holds but it may happen that not
all $b^i, \; i = 1,\ldots,n$ are in the basis of $y^{\star}$.  Then the above
Case 1 must be repeated until $n_c = n$.  Afterwards the basis index vector of
$y^{\star}$ has the desired form
\[
\A(y^{\star}) = \{\rho _1, \ldots, \rho _m\}
\]
where
\[
\rho _k \in \ba{\{}{cc} \{1, \ldots, n\}, & 1 \leq k \leq n,\\
                        \U(y^{\star}), & n < k.
            \ea{.}
\]
But note that not necessarily $\rho _k = k, \; k = 1, \ldots, n$.
\par
At the end of phase 1, we have $\D(y^{\star}) = \emptyset$ hence the reduced
tableau (\ref{svs7}) has the form
%
\begin{equation} \label{svs8}
{\bf P} = \ba{[}{ccc}
 A_{1:p}                 & - A_{(p+1):n} & \wi{w}^{1:n}\\
 B^{\U}{}A_{1:p}    & - B^{\U}A_{(p+1):n}
 & \wi{w}^{(n+1):m}\\
y_{1:p} & -y_{p+\L} & \wi{f}
\ea{]}.
\end{equation}
Hence this tableau can be used for start tableau in phase 2 with $\N = \U$ if
the sign is changed in the second block column and if the last column is
replaced by
\[
\ba{[}{c}w \\ f \ea{]} = \ba{[}{c} \wi{A}P^Tc \\ yc \ea{]}.
\]
An basis index vector of $y^{\star}$, considered for vertex of the dual
problem, is
\[
\A(y^{\star}) = \{1, \ldots, p, p+\L\}.
\]
because the matrix $A$ defined by (\ref{hjh}) is regular.
