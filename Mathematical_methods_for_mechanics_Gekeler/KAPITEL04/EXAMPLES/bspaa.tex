\documentclass[12pt,a4paper,twoside,leqno]{article}
\input aaformat
\newcommand{\Release}{25/07/02} %%%%%%%%%%%%%%

\begin{document}
\addtolength{\abovedisplayshortskip}{-0.5ex}
\setlength{\fboxsep}{1ex}
\parskip1ex

{\large\bf Kontrolltheorie: Beispiele 1--8 \hfill E.Gekeler}
\par
\vspace{-0.5ex}
\hfill{\footnotesize\Release\ }
\par
%\vspace{-0.5ex}
\hrule
\par
Vgl. {\sc Hartl et al}.: SIAM REVIEW 37 (1995), pp. 181-218.
\par
%%%%%%%%%%%%%%%%%%%%%%%%%%%%%%%%%%%%%%%%%%%%%%%%%%%%%%%%%%%%%%
{\bf Beispiel 1}
Vgl. {\sc Bryson-Ho}: Applied Optimal Control, S. 59.\\
In einem $(x_1,x_2)$-Koordinatensystem wird ein Raumschiff von der Masse $m$
und der Schubkraft $ma(t)$ in Richtung seiner K"orperachse beschleunigt.
\[
\ba{.}{ll}
(x_1(t),x_2(t))  & \mbox{Position des Raumschiffes}\\
x_3(t)           & \mbox{Geschw. in $x_1$-Richtung}\\
x_4(t)           & \mbox{Geschw. in $x_2$-Richtung}\\
u(t)  & \mbox{Winkel der Schiffsachse zur $x_1$-Richtung (Kontrolle)}.
\ea{.}
\]
Das Raumschiff soll in einer vorgegebenen Zeit $T$ auf eine Flugbahn parallel
zur $x_1$-Achse in H"ohe $H$ gebracht werden. Dabei soll die Geschwindigkeit
$x_3(T)$ maximal sein. Es sei $x(0) = {0}$.

Problem:
\[
\ba{.}{c}
J(x,u) = x_3(T)  = \Max !,\\[3mm]
\dot{x}_1 = x_3(t), \quad
\dot{x}_2 = x_4(t), \quad
\dot{x}_3 = a(t) \cos(u(t)), \quad
\dot{x}_4 = a(t) \sin(u(t))\\[3mm]
x_2(T) = H\,, \quad x_4(T) = 0\,.
\ea{.}
\]
\par \vspace{1ex} \hrule \par \vspace{2ex}
%%%%%%%%%%%%%%%%%%%%%%%%%%%%%%%%%%%%%%%%%%%%%%%%%%%%%%%%%%%%%%%
{\bf Beispiel 2}
Vgl. {\sc Bryson-Ho}: Applied Optimal Control, S. 66.\\
Ein Raumschiff soll in einer vorgegebenen Zeit $T$ von einem kreisf"ormigen
Orbit mit Radius $r_0$ den h"ochstm"oglichen kreisf"ormigen Orbit erreichen.  Der
Einfluss weiterer Himmelsk"orper wird vernachl"assigt.  Das Raumschiff fliegt im
Gegenuhrzeigersinn.  In Polarkoordinaten $(r,\phi )$ mit dem (punktf"ormigen)
Zentralk"orper als Zentrum sei
\[
\ba{.}{ll}
r(t)               & \mbox{radiale Entfernung des Raumschiffes}\\
u = \dot{r}        & \mbox{radiale Geschwindigkeit}\\
v = r\dot{\phi }   & \mbox{Geschwindigkeit senkrecht zum Radius}\\
m(t)               & \mbox{Masse des Raumschiffes}\\
|\dot{m}(t)|       & \mbox{{\em konstante} Treibstoffverbrauchsrate}\\
S = |\dot{m}(t)|a  & \mbox{Schubkraft}\\
a                  & \mbox{Maschinenkonstante}\; [m/s] \\
\alpha (t)         & \mbox{Schubwinkel}\\
\gamma             & \mbox{Gravitationskonstante}\\
M                  & \mbox{Masse des Zentralk"orpers}\\
G                  & = \gamma\cdot M\\
\rho               & = |\dot{m}(t)|/m(0) \; [1/s]\\
S                  & = \rho \cdot a \cdot m(0)
\ea{.}
\]
Die Kontrolle $\alpha (t)$  ist der Winkel zwischen der Tangente des
kreisf"ormigen Orbits zum Zeitpunkt $t$ und der Schiffsachse in Flugrichtung
gesehen (lokales Koordinatensystem).

Bewegungsgleichungen:
\[
\ba{.}{rclcl}
\dot{r} &=& u(t),\\[5mm]
\dot{u} &=& \dis \frac{v(t)^2}{r(t)} - \frac{G}{r(t)^2}
+ \frac{S(t)\sin (\alpha(t) )}{m(0)(1 - \rho \, t)}
&=&
\dis \frac{v(t)^2}{r(t)} - \frac{G}{r(t)^2}
+ \frac{\rho \cdot a \cdot \sin (\alpha(t) )}{1 - \rho \, t},\\[5mm]
\dot{v} &=& \dis - \frac{u(t)\cdot v(t)}{{\bf 2}\cdot r(t)}
 + \frac{S(t)\cos (\alpha(t) )}{m(0)(1 - \rho \, t)}.
&=& \dis - \frac{u(t)\cdot v(t)}{{\bf 2}\cdot r(t)}
 + \frac{\rho \cdot a \cdot \cos (\alpha(t) )}{1 - \rho \, t}.
\ea{.}
\]

Der Faktor $1/2$ fehlt bei {\sc Bryson-Ho} und {\sc Dyer-McReynolds}\;(?).\\
Randbedingungen:
\[
r(0) = r_0, \; u(0) = 0, \; v(0) = \sqrt{G/r_0}, \;
u(T) = 0, \; v(T) = \sqrt{G/r(T)}.
\]

Reduktion auf dimensionsloses System:\\
Der Radius $r$ und die Zeit $t$ m"ussen durch dimensionslose Gr"ossen ersetzt
werden:
\[
s = \frac{G^{1/2}}{r_0^{3/2}}\, t, \quad R(s) = \frac{1}{r_0}\, r(t).
\]
Die zweite und dritte Bewegungsgleichung m"ussen dann mit $r^2_0/G$
multipliziert werden. F"ur $U(s) = R'(s)$ und $V(s) = R(s)\phi '(s)$
folgt dann mit der dimensionlosen Konstanten
\[
\kappa = \rho  \cdot a \cdot r^2_0 /G
\]

\[
R' = U, \quad
U' =  \frac{V^2}{R} - \frac{1}{R^2} + \kappa\,\frac{\sin (\alpha )}{1 - \rho\,
t}, \quad
V' =  - \frac{UV}{2R} + \kappa\,\frac{\cos (\alpha )}{1 - \rho\,  t}.
\]

Wir schreiben wieder $t$ statt $s$, $T$ statt $s_f = TG^{1/2}/r_0^{3/2}$, $u$
statt $\alpha $ und
\[
x = [x_1,x_2,x_3]^T = [R,U,V]^T\,.
\]

Transformiertes Problem:
\[
\ba{.}{c}
J(x,u) = x_1(T) = \Max !\\[5mm]
\dot{x}_1 = x_2(t),\\[5mm]
\dot{x}_2 = \dis \frac{x_3^2(t)}{x_1(t)} - \frac{1}{x_1^2(t)}
+ \kappa \,\frac{\sin (u(t))}{1 - \rho \, t},\\[5mm]
\dot{x}_3 = \dis - \frac{x_2(t)x_3(t)}{2 \cdot x_1(t)}
 + \kappa \,\frac{\cos (u(t))}{1 - \rho \,t}\\[5mm]
x_1(0) = 1\,, \quad x_2(0) = 0\,,\quad x_3(0) = 1\\[5mm]
x_2(T) = 0\,,\quad x_3(T)^2x_1(T) - 1 = 0\,.
\ea{.}
\]
\par \vspace{1ex} \hrule \par \vspace{2ex}
%%%%%%%%%%%%%%%%%%%%%%%%%%%%%%%%%%%%%%%%%%%%%%%%%%%%%%%%%%%
{\bf Beispiel 3}
Vgl. {\sc Bryson-Ho}: Applied Optimal control, S. 67.\\
({\sc Zermelo}sches Problem) In einem
$(x_1,x_2)$-Koordinatensystem soll ein Schiff mit konstanter Geschwindigkeit
$S$ relativ zum Wasser in m"oglichst kurzer Zeit $T$ von dem Punkt $A =
(a_1,a_2)$ zu dem Punkt $B = (b_1,b_2)$ fahren ($x = (x_1,x_2)$).
\[
\ba{.}{ll}
(x_1(t),x_2(t))  & \mbox{Position des Schiffes}\\
v_1(x) & \mbox{Str"omungsgeschwindigkeit in $x_1$-Richtung}\\
v_2(x) & \mbox{Str"omungsgeschwindigkeit in $x_2$-Richtung}\\
u & \mbox{Winkel der Schiffsachse zur $x_1$-Richtung (Kontrolle)}.
\ea{.}
\]

Bewegungsgleichungen:
%
\[
\dot{x}_1 = S \cos(u(t)) + v_1(x(t)),\quad
\dot{x}_2 = S \sin(u(t)) + v_2(x(t))
\]

{\sc Goh-Teo}-Transformation:
Vgl.\ {\sc Craven}: Control and Optimization, \S 6.4.3.\\
Es wird die Substitution
\[
t = sT, \quad 0 < s < 1
\]
\par
\vspace{1mm}

gew"ahlt und die unbekannte Zeit $T$ als neue Ver"anderliche
eingef"uhrt:
\[
X_1(s) = x_1(sT), \quad
X_2(s) = x_2(sT), \quad
U(s) = u(sT)\,.
\]
\par
Transformiertes Problem:
\[
\ba{.}{c}
J(T) = T = \Min!\\[3mm]
X'_1(s) = [S \cos(U(s)) + v_1(X(s))]T,\\[3mm]
X'_2(s) = [S \sin(U(s)) + v_2(X(s))]T, \\[3mm]
X_1(0) = a_1\,, \quad X_2(0) = a_2\,,\\[3mm]
X_1(1) = a_1\,, \quad X_2(1) = a_2\,.
\ea{.}
\]
\par\vspace{1ex}\hrule\par\vspace{2ex}
%%%%%%%%%%%%%%%%%%%%%%%%%%%%%%%%%%%%%%%%%%%%%%%%%%%%%5
{\bf Beispiel 4}
Servo-Problem. Vgl. {\sc Burges--Graham}, S. 281 ff.\\
Man l"ose das folgende Problem ($a \geq 0$)
\[
 \ba{.}{l}
\Min \{T, \; \ddot{x} + a\dot{x} + \omega ^2x = u, \; 0 < t < T, \;\\
\qquad x(0) = a_1, \; x'(0) = a_2, \; x(T) = x'(T) = 0, \; |u(t)| \leq 1\}\,.
\ea{.}
\]
Der Einfachheit halber betrachten wir den Fall $\omega ^2 = 1$ und $a = 0$
(keine D"ampfung.)\\
Zustandsgleichungen als System 1. Ordnung:
\[
\dot{x}_1 = x_2\,, \quad \dot{x}_2 = u - x_1\,.
\]

{\sc Goh-Teo}-Transformation:
Es wird die Substitution
\[
t = sT, \quad 0 < s < 1
\]
\par
\vspace{1mm}

gew"ahlt und die unbekannte Zeit $T$ als Ver"anderliche eingef"uhrt:
\[
X_1(s) = x_1(sT), \quad
X_2(s) = x_2(sT), \quad
U(s) = u(sT)\,.
\]
\par
Tranformiertes Problem
\[
\ba{.}{c}
J(T) = T = \Min!\\[3mm]
X'_1(s) = T\cdot X_2(s)\,, \quad
X'_2(s) = T \cdot (U(s) - X_1(s))\,, \\[3mm]
X_1(0) = a_1\,, \quad X_2(0) = a_2\,,\quad
X_1(1) = 0\,, \quad X_2(1) = 0\,,\\[3mm]
1 - U(s) \geq 0\,, \quad U(s) - 1 \geq 0\,.
\ea{.}
\]
\par\vspace{1ex}\par\vspace{2ex}
%%%%%%%%%%%%%%%%%%%%%%%%%%%%%%%%%%%%%%%%%%%%%%%%%%%%%%%%%%%%%5
{\bf Beispiel 5} {\sc Hartl et al}, S. 204.
\par
\[
\ba{.}{c}
\dis J(x,u) = \int_0^3x\, dt = \Min !,\\[3mm]
\dot{x} = u, \; x(0) = 1,\quad x(3) = 1, \quad
-1 \leq u \leq 1, \; 0 \leq x.
\ea{.}
\]
L"osung:
\[
x^* = \ba{\{}{c} 1 - t\\ 0\\ t - 2 \ea{.}, \quad
%
u^* =  \ba{\{}{c}   - 1\\ 0    \\ 1     \ea{.} \;  \mbox{f"ur} \;
t  \in \ba{\{}{c} [0,1)\\[0mm] [1,2]\\ (2,3] \ea{.}\; .
\]

\[
H := x + y \cdot u\,, \quad
L := H + v_1(1 + u) + v_2(1 - u) + wx.
\]

Notwendige Bedingungen:

\[
\ba{.}{c}
L_u = y + v_1 - v_2 = 0\\[2mm]
\dot{y} = - L_x = - 1 - w\,, \quad y(3) = z, z \in \mb{R} \\[2mm]
v_1 \geq 0, \; v_2 \geq 0, \; v_1(1 + u) = v_2(1 - u) = 0, \;
w \geq 0, \; w\, x = 0\\[2mm]
\ea{.}
\]

Beginnend mit dem Intervall $(1,2)$ erhalten wir
die folgende Tafel der Multiplikatoren:
\par
\begin{center}
\begin{tabular}{|c|c|c|c|c|}\hline
Intervall  & $y$     & $v_1$   & $v_2$   & $w$\\ \hline
$[0,1)$   & $t - 1$ & $1 - t$ & 0       & 0\\
$[1,2]$   & 0       & 0       & 0       & 1\\
$(2,3]$   & $t - 2$ & 0       & $t - 2$ & 0\\        \hline
\end{tabular}
\end{center}
\par \vspace{1mm}\hrule\par\vspace{2mm}
% ------------------------------------------
Gel"ost wird das Problem

\[
\ba{.}{c}
\dis J(x,u)
= \dis
\int_0^3x\, dt = \Min !,\\[3mm]
\dis x(t) -
1 - \int_0^tu(\tau)\, d\tau = 0\,, \; 0 \leq t\leq 3\,,
\quad x(3) - 1 = 0\,,\\[3mm]
u + 1 \geq 0\,, \quad 1 - u \geq 0\,, \quad x \geq 0\, .
\ea{.}
\]
\par\vspace{1ex}\hrule\par\vspace{2ex}
%%%%%%%%%%%%%%%%%%%%%%%%%%%%%%%%%%%%%%%%%%%%%%%%%%%%%%%%%%
{\bf Beispiel 6} {\sc Hartl et al.}, S. 207.

\[
\ba{.}{c}
\dis  J(x,u) = \int_0^3e^{-rt}u\,dt = \Min !, \; r \geq 0,\\[3mm]
\dot{x} = u, \; x(0) = 0,\quad
0 \leq u \leq 3, \; x -1 + (t - 2)^2 \geq 0.
\ea{.}
\]

L"osung:
\[
x^* = \ba{\{}{c} 0\\ 1 - (t - 2)^2\\ 1 \ea{.}, \;
%
u^* =  \ba{\{}{c}   0\\ 2(2 - t)  \\ 0     \ea{.}, \;  \mbox{f�r} \;
t  \in \ba{\{}{l} [0,1)\\[0mm] [1,2]\\ (2,3] \ea{.}.
\]

\[ \ba{.}{rcl}
H &=& - e^{-rt}u + yu,\\
L &=& H + v_1u + v_2(3 - u) + w[x - 1 + (t - 2)^2].
\ea{.}
\]

Notwendige Bedingungen:

\[
\ba{.}{c}
L_u = - e^{-rt} + y + v_1 - v_2 = 0\\[2mm]
\dot{y} = - L_x = - w,\\[2mm]
v_1 \geq 0, \; v_2 \geq 0, \; v_1u = v_2(3 - u) = 0,\\[2mm]
w \geq 0, \; w[x - 1 + (t - 2)^2] = 0,\\[2mm]
y(3) = 0, \; y(2-) = y(2+) - c, \; c \geq 0.
\ea{.}
\]

Beginnend mit dem Intervall $(2,3)$ erhalten wir
\par
%
Tafel der Multiplikatoren:
\par
\begin{center}
\begin{tabular}{|c|c|c|c|c|}\hline
Intervall  & $y$       & $v_1$              & $v_2$    & $w$\\ \hline
$[0,1)$   & $e^{-r}$  & $e^{-rt} - e^{-r}$ & 0        & 0\\
$[1,2]$   & $e^{-rt}$ & 0       & 0       & $re^{-rt}$\\
$(2,3]$   & $0$       & $e^{^-rt}$    & 0 & 0\\        \hline
\end{tabular}
\end{center}

\par \vspace{2mm} \hrule \par \vspace{4mm}
Gel"ost wird das Problem

\[
\ba{.}{c}
\dis  (x,u)J = \int_0^3e^{-rt}u\,dt = \Min !, \; r \geq 0,\\[3mm]
\dis x(t) - \int_0^tu(\tau )\, d\tau  = 0\\[3mm]
u \geq 0\,, \quad 3 - u \geq 0\,, \quad x - 1 + (t - 2)^2 \geq 0\,.
\ea{.}
\]
\par\vspace{1ex}\hrule\par\vspace{2ex}
% ---------------------------------------------
{\bf Beispiel 7} {\sc Hartl et al.}, S. 208. $x = (x_1\,,\; x_2)$.

\[
\ba{.}{c}
\dis  J(x,u) = \int_0^32x_1\,dt = \Min !,\\[3mm]
\dot{x}  = \ba{(}{c} x_2\\u\ea{)}\,,\quad x(0) = \ba{(}{c}2\\0\ea{)}\\[3mm]
-2 \leq u \leq 3, \quad x_1 \geq \alpha \,, \quad
\alpha \in \mb{R}\,, \quad \alpha  \leq 0\, .
\ea{.}
\]

{\bf 1. Fall:} $\alpha  \leq - 7$.

L"osung:
\[
x^* = \ba{\{}{c} 2 - t^2\\ -2t\ea{.}, \;
%
u^* =  -2, \;  \mbox{f"ur} \; 0 \leq t \leq 3\,.
\]

{\bf 2.  Fall:} $- 7 < \alpha  \leq 2.5$.

Hier gibt es einen Schaltzeitpunkt

\[
\sigma  = 3 - \frac14(56 + 8 \alpha )^{1/2}\, .
\]

L"osung:
\[
x_1^* = \ba{\{}{c} 2 - t^2\\2 + t^2 + 2\sigma ^2 - 4\sigma t  \ea{.}, \quad
%
x_2^* = \ba{\{}{c} - 2t\\ 2(t -2\sigma ) \ea{.}, \quad
%
u^* =  \ba{\{}{c}   - 2\\ 2\ea{.} \;  \mbox{f�r} \;
t  \in \ba{\{}{c} [0,\sigma )\\[0mm] [\sigma ,3] \ea{.}\; .
\]
%
\par \vspace{2mm} \hrule \par \vspace{4mm}
% -----------------------------------------------
{\bf 3. Fall:} $-2.5 < \alpha  \leq 0$.

Es gibt einen Schaltzeitpunkt (switching time) $\sigma$ und
einen Kontaktzeitpunkt (junction time) $\rho $ mit $0 < \sigma  < \rho  < 3$.
$\rho $ ist auch ein Eintrittspunt (entry point) in den Randbogen.
Es gilt
\[
\rho  = 2\sigma  = (4 - 2\alpha )^{1/2}\, .
\]

L"osung:
\[
x_1^* = \ba{\{}{c} 2 - t^2\\2 + t^2 + 2\sigma ^2 - 4\sigma t  \ea{.}, \quad
%
x_2^* = \ba{\{}{c} - 2t\\ 2(t -2\sigma ) \ea{.}, \quad
%
u^* =  \ba{\{}{c}   - 2\\ 2\ea{.} \;  \mbox{f�r} \;
t  \in \ba{\{}{c} [0,\sigma )\\[0mm] [\sigma ,\rho ] \ea{.}\; .
\]

Im Intervall $(\rho \,,\; 3)$ gilt
\[
x_1 = \alpha \,, \quad x_2 = 0\,, u = 0\, .
\]

\par \vspace{2mm} \hrule \par \vspace{4mm}
Gel"ost wird f"ur verschiedene $\alpha $ das Problem

\[
\ba{.}{c}
\dis  J(x,u) = \int_0^32x_1\,dt = \Min !,\\[3mm]
\dis x_1(t) - 2 - \int_0^tx_2(\tau )\, d\tau  = 0\,,\\[3mm]
\dis x_2(t) - \int_0^tu(\tau )\, d\tau  = 0\,,\\[3mm]
u + 2 \geq 0\,, \quad 3 - u \geq 0\,, \quad x_1 \geq \alpha \,.
\ea{.}
\]
\par\vspace{1ex}\hrule\par\vspace{2ex}
%%%%%%%%%%%%%%%%%%%%%%%%%%%%%%%%%%%%%%%%%%%%%%%%%%%%
{\bf Beispiel 8} {sc Hartl et al.}, S. 210.
\[
\ba{.}{c}
\dis \int_0^1\left[10x^2 - u^2\right]\, dt = \Max!\,,\\[3mm]
\dot{x} = x^2 - u\,\quad x(0) = x(1) = 1\,,\\[3mm]
x(t) \leq  1.5\,.
\ea{.}
\]

Der Zustand $x$ w"achst monoton, trifft den Rand $x = 1.5$ zum
Zeitpunkt $t_1 = 0.345037$ und verl"asst ihn wieder zum Zeitpunkt
$t_2 = 1 - t_1$. Die Kontrolle $u$ ist stetig und an den Punkten $t_1$ und
$t_2$ `tangential', weil das Problem regul"ar ist.
\par\vspace{1ex}\hrule\par\vspace{2ex}
\end{document}
