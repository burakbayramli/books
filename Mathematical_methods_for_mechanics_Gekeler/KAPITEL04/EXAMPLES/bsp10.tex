\documentclass[12pt,a4paper,twoside,leqno]{article}
\input aaformat
\newcommand{\Release}{10/05/05} %%%%%%%%%%%%%%
\setlength{\fboxsep}{1ex}
\parskip1ex
\parindent0ex

\begin{document}
{\large\bf Kontrolltheorie: Beispiel 10 \hfill E.Gekeler}
\par
\vspace{-0.5ex}
\hfill{\footnotesize\Release\ }
\par\hrule\par\vspace{2ex}
% ---------------------------------------------------------
Reentry Problem nach [Stoer].
Ein Raumschiff soll beim Wiedereintritt in die Erdatmosph"are einer m"oglichst
geringen W"armebelastung ausgesetzt sein.
(1 [ft] = 0.3048 [m].)
\par
{\em Bezeichnungen:}
$x_1(t)$ Geschwindigkeit des Raumschiffs,
$x_2(t)$ Flugwinkel,
$x_3(t)$ normalisierte H"ohe,
$u(t)$ Kontrollfunktion (Angriffswinkel),
$w(t)$ zur"uckgelegter Wege relativ zur Erdoberfl"ache,
$g$ Gravitationsbeschleunigung.
\par
{\em Materialkonstanten} f"ur ein Raumschiff vom {\sc Apollo}-Typ;
vgl. [Stoer]
%
\[
\ba{.}{lllll}
h & \text{H"ohe "uber Erdoberfl"ache},& \beta &=& 4.26\\[0.2ex]
%
m & \text{Masse des Raumschiffs},& \rho_0 &=& 2.70\cdot 10^{-3} \\[0.2ex]
%
R_E & \text{Erdradius}, &      R_E &=& 209 \ (\, = 209\cdot 10^5\  [ft])\\[0.2ex]
S & \text{Frontfl"ache des Raumschiffs},&
S/(2m)  &=& 26 600 \\[0.2ex]
%
T & \text{Zeitraum f"ur das Man"over}, &  g &=& 3.2172\cdot 10^{-4}\ [10^5\, ft/sec^2]
\ea{.}
\]
\[
\ba{.}{lcll}
C_D(u) &=& 1.174 - 0.9\cos u &
\text{aerodynamischer Widerstandskoeffizient}\\[0.2ex]
C_L(u) &=& 0.6 \sin u &
\text{aerodynamischer Auftriebskoeffizient}\\[0.2ex]
\rho(x_3(t)) &=& \rho_0\exp (-\beta R_E x_3(t)) &
 \text{atmosph"arische Dichte}\,.
\ea{.}
\]
\par
{\em Kontrollproblem}
%
\[
J(x) = \dis 10\int_0^Tx^3_1 \rho(x_3)^{1/2}\, dt = \Min !
\]
\par\vspace{1ex}
\[
\fbox{$
\ba{.}{llllll}
\dot{x}_1(t) &=& f_1(x,u) &=&
\dis  - \frac{S\rho\, x_1^2}{2 m}C_D(u)
     - \frac{g \, \sin x_2}{(1 + x_3)^2}\\[2ex]
%
\dot{x}_2(t) &=& f_2(x,u) &=&
\dis \frac{S \rho x_1}{2 m}C_L(u)
+ \frac{x_1 \cos x_2}{R_E(1 + x_3)}
- \frac{g \, \cos x_2}{x_1(1 + x_3)^2}\\[2ex]
%
\dot{x}_3(t) &=& f_3(x) &=&
\dis \frac{x_1 \sin x_2}{R_E}
\ea{.}
$}
\]
Randbedingungen:
\[
\fbox{$
\ba{.}{llllll}
x_1(0) &=& 0.36  \; ( = 0.36\cdot 10^5  \text{ft/sec}), \quad
&
x_1(T) &=& 0.27 \; ( = 0.27\cdot 10^5  \text{ft/sec})\\
%
x_2(0) &=& \dis - 8.1^{\circ}\frac{\pi}{180^{\circ}}, \quad
&
x_2(T) &=&  0\\
%
x_3(0) &=& \dis \frac{4}{R_E}\,, \; [h(0) = 4\cdot 10^5  \text{ft}], \quad
&
x_3(T) &=& \dis \frac{2.5}{R_E}\,, \; [h(T) = 2.5 \cdot 10^5  \text{ft}]
\ea{.}
$}\;.
\]
%
Die abh"angige Variable $w(t)$ tritt im Problem nicht auf.
Wenn der ``Planungszeitraum'' $T$ frei ist, muss das Problem in ein
parameterabh"angiges Problem transformiert werden. Der Einfachheit halber
wird $T = 230$\  [sec] fest gew"ahlt.
\par
{\em Besonderheit:}
Durch Aufl"osen von $H_u = 0$ nach $u$ l"asst sich die Kontrolle eliminieren,
dazu ben"otigt man aber die L"osungen der Kozustandsgleichungen:
\par
{\sc Hamilton}-Funktion:
\[
H(x,y,u) = 10x^3_1\rho(x_3)^{1/2} + y_1f_1(x,u) + y_2f_2(x,u) + y_3f_3(x)
\]
\[
\dot{y}_1(t) = - \frac{\da H}{\da x_1}\,,\quad
\dot{y}_2(t) = - \frac{\da H}{\da x_2}\,, \quad
\dot{y}_3(t) = - \frac{\da H}{\da x_3}\,.
\]
Optimale Kontrolle implizit:
\[
\sin u = - 0.6 \,\frac{y_2}{v^{1/2}}\,,
\quad
\cos u = - 0.9\, \frac{x_1 y_1}{v^{1/2}}\,,
\quad
v(t) = 0.36 y_2^2 + 0.81x_1^2y_1^2\,.
\]
%
Nach dem Einsetzen von $u$ ergeben die Zustands- und
Kozustandsgleichungen zusammen ein System von sechs Differentialgleichungen
mit sechs Randbedingungen. Wir setzen $y_1 = x_4, \; y_2 = x_5, \; y_3 = x_6$
sowie $D_i = \da/\da x_i$ und erhalten
\[
\ba{.}{rcl}
\dot{x}_1 = f_1(x) &=&
\dis  - \frac{S\rho}{2 m}
        \left[1.174x_1^2 + 0.81 \frac{x_1^3x_4}{v^{1/2}}\right]
     - \frac{g \, \sin x_2}{(1 + x_3)^2}\\[2ex]
%
\dot{x}_2 = f_2(x) &=&
\dis - 0.36 \frac{S \rho x_1x_5}{2 m v^{1/2}}
+ \frac{x_1 \cos x_2}{R_E(1 + x_3)}
- \frac{g \, \cos x_2}{x_1(1 + x_3)^2}\\[2ex]
%
\dot{x}_3 = f_3(x)&=&
\dis \frac{x_1 \sin x_2}{R_E}\\[2ex]
%
\dot{x}_4 = - D_1H &=& \dis
- \left[30\, x^2_1\rho^{1/2} + x_4D_1f_1 + x_5 D_1f_2 + x_6D_1f_3\right]
\\[2ex]
%
\dot{x}_5 = - D_2H &=& \dis
- \left[x_4D_2f_1 + x_5D_2f_2 + x_6D_2f_3\right]
\\[2ex]
%
\dot{x}_6 = - D_3H &=& \dis
- \left[-5\, x^3_1 \beta R \rho^{1/2}
+ x_4 D_3f_1 + x_5D_3f_2 + x_6D_3f_3\right]
\ea{.}
\]
mit sechs Randbedingungen.
\par
Gel"ost wird das Problem direkt ohne Verwendung der Kozustandsgleichungen.
Starttrajektorie ist die gerade Verbindung zwischen den Randbedingungen,
und Nominalkontrolle ist wie bei \cite{Stoer}
\[
u(t) = p_1 \text{erf}(p_2(p_3 + t))\,, \;
\text{erf}(u) = \frac{2}{\sqrt{\pi}}\int_0^u e^{-s^2}\, ds\,, \;
p_1 =  1.6\,\; p_2 = 4\,, \; p_3 = 0.5\,.
\]
Die Berechnung von Starttrajektorien f"ur die drei Kozustandsver"anderlichen
entf"allt aber; im "Ubrigen tritt dabei eine Singularit"at auf. Man beachte auch,
dass das feste Zeitintervall $[0,\,T]$ bei der Rechnung auf das
Einheitsintervall transformiert wird. Das unten angegebene Resultat stimmt
bis auf die Kontrolle mit den Ergebnissen von \cite{Stoer},  S.\ 496/497 "uberein,
was physikalisch erkl"art werden kann.
\par\vspace{0.5ex}
\bc
\begin{minipage}[t]{5.5cm}
% Breite = 470, Hoehe = 371, H/B = 0.8
%\epsfig{file=bld040409e.eps,height=4.4cm,width=5.5cm}\\
Abb. 76 -- Beispiel 10, Skizze
\end{minipage}
\ec
\bc
\begin{minipage}[t]{5.5cm}
% Breite = 482, Hoehe = 377, H/B = 0.8
%\epsfig{file=bld040409a.eps,height=4.4cm,width=5.5cm}\\
Abb. 77 -- Beispiel 10, Geschwindigkeit
\end{minipage}
\;\;
\begin{minipage}[t]{5.5cm}
% Breite = 479, Hoehe = 377, H/B = 0.8
%\epsfig{file=bld040409b.eps,height=4.4cm,width=5.5cm}\\
Abb. 78 -- Beispiel 10, Flugwinkel
\end{minipage}
\ec
\bc
\begin{minipage}[t]{5.5cm}
% Breite = 476, Hoehe = 377, H/B = 0.8
%\epsfig{file=bld040409c.eps,height=4.4cm,width=5.5cm}\\
Abb. 79 -- Beispiel 10, Hoehe ($\times 10^5$ ft)
\end{minipage}
\;\;
\begin{minipage}[t]{5.5cm}
% Breite = 482, Hoehe = 377, H/B = 0.8
%\epsfig{file=bld040409d.eps,height=4.4cm,width=5.5cm}\\
Abb. 80 -- Beispiel 10, Kontrolle ($\times \pi$)
\end{minipage}
\ec
\end{document}
