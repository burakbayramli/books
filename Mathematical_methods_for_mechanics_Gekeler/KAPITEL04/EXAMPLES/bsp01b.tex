\documentclass[12pt,a4paper,twoside,leqno]{article}
\input aaformat
\newcommand{\Release}{10/05/05} %%%%%%%%%%%%%%
\begin{document}
{\large\bf Kontrolltheorie: Beispiel 1a \hfill E.Gekeler}
\par
\vspace{-0.5ex}
\hfill{\footnotesize\Release\ }
\par\hrule\par\vspace{2ex}
% ---------------------------------------------------------
{\bf Thrust-Problem}
Vgl. {\sc Bryson-Ho}: Applied Optimal Control, S. 59.\\
Wie {\bf Beispiel 1}, aber die Kontrolle wird nicht eliminiert, sondern
das Gradientenverfahren verwendet.

Zustandsgleichungen (Bewegungsgleichungen) $\dot{x} = [\nabla_y H]^T$\,:
\begin{equation} \label{1}
\fbox{$
\dot{x}_1 = x_3(t), \quad
\dot{x}_2 = x_4(t), \quad
\dot{x}_3 = a(t) \cos(u(t)), \quad
\dot{x}_4 = a(t) \sin(u(t))
$}\; .
\end{equation}

{\bf (a)} Zielfunktion und {\sc Hamilton}-Funktion:
\begin{equation} \label{2}
\ba{.}{rcl}
J(x) &:=& p(x(T)) + \int_0^TL(x(t),u(t),t)dt = -x_3(T)  = \Min !,\\
H(x,y,u,t) &=&  y^Tf = y_1x_3 + y_2x_4 + y_3a(t)\cos(u) + y_4a(t)\sin(u),\\
\nabla_x H &=& (0,0,y_1,y_2)\quad \mbox{(Zeilenvektor)}.
\ea{.}
\end{equation}

Kozustandsgleichungen ({\sc Euler-Lagrange}-Gleichungen) $\dot{y} = -
[\nabla_x H]^T$:
\begin{equation} \label{3}
\fbox{$
\dot{y}_1 = 0,\quad
\dot{y}_2 = 0,\quad
\dot{y}_3 = - y_1,\quad
\dot{y}_4 = - y_2
$}\; .
\end{equation}
L"osung von (\ref{3})
\[
y(t) = [c_1,c_2,-c_1t + c_3,-c_2t + c_4]^T.
\]
\par
\vspace{1mm}
Anfangsbedingungen f"ur den Zustand $x$:
\[
x(0) = a = 0\,.
\]
Endbedingungen f"ur den Kozustand $y$:
\[
y(T) = [\nabla_xp]^T(T) = \ba{[}{c}0\\0\\-1\\0\ea{]}\, .
\]
Zus"atzliche Endbedingungen
\[
\ba{.}{rcl}
K = q(x(T)) &=& \ba{[}{c}x_2(T) - H\\x_4(T)\ea{]} = 0\\[6mm]
[\nabla_x]^Tq(T) &=& \ba{[}{cc} 0 & 0\\1 & 0\\0 & 0\\0 & 1\ea{]}\, .
\ea{.}
\]
F"ur dieses Problem versagt das Gradientenverfahren!
\par
\vspace{1mm}
% ---------------------------------------------------------------------
{\bf (b)} Einbindung der zus"atzlichen Endbedingungen in die Zielfunktion:
\[
J(x) := p(x(T))= -x_3(T) + (x_2(T) - H)^2 + x_4(T)^2 = \Min !
\]
Dann ist wieder die Endbedingung f"ur $y$ gegeben durch
$y(T) = [\nabla_xp]^T(x(T))$.

Bei diesem Problem ist das Optimum als station"arer Punkt vermutlich keine
lokale Extremalstelle von $u$. Das Gradientenverfahren versagt hier ebenfalls.
\par
\vspace{1mm}
{\bf (c)} Die Wahl der Zielfunktion
\[
J(x) := p(x(T))= (x_2(T) - H)^2 + x_4(T)^2 = \Min !
\]
liefert das gleiche Ergebnis wie CNTRL01, Beispiel 2. Daher hat es den
Anschein, als sei die urspr"ungliche Bedingung $x_3(T) = \Max!$ unter Wahl der
zus"atzlichen Endbedingungen $x_2(T) - H = 0$ und $x_4(T) = 0$
automatisch erf"ullt.


%\newpage

%\bc
%\begin{minipage}{10.cm}
%   \epsfxsize=9.9cm
%   \epsffile{fig1.eps}
%\end{minipage}
%\ec
%\centerline{\epsfig{file=stab.eps,height=6cm},bbllx=-6cm,bblly=5mm}
\end{document}
