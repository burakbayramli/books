\documentclass[12pt,a4paper,twoside,leqno]{article}
\input aaformat
\newcommand{\Release}{10/05/05} %%%%%%%%%%%%%%
\begin{document}
{\large\bf Kontrolltheorie: Beispiel 3a \hfill E.Gekeler}
\par
\vspace{-0.5ex}
\hfill{\footnotesize\Release\ }
\par\hrule\par\vspace{2ex}
%%%%%%%%%%%%%%%%%%%%%%%%%%%%%%%%%%%%%%%%%%%%%%%%%%%%%%%%%%%
{\large\bf Kontrolltheorie: CNTRL02, Beispiel 1/2 (Dyer) \hfill E.Gekeler}
Vgl. {\sc Dyer--McReynolds}, S. 127/128.

{\bf Beispiel 1:}\\
Zielfunktion
\[
J(x,u) = \int_0^1(x_1(t)^2 + u(t)^2)\, dt = \Min\, !
\]
Bewegungsgleichungen
\[
\fbox{$
\dot{x}_1 = x_2 + u\,, \quad \dot{x}_2 = - u
$}\; .
\]
Anfangsbedingungen f"ur den Zustand $x$
\[
x(0) = a = \ba{[}{c}2\\-1\ea{]}\, .
\]
{\sc Hamilton}-Funktion
\[
\ba{.}{rcl}
H(x,u) &=& x_1^2 + u^2 + y_1(x_2 + u) + y_2(-u)\,,\\[2mm]
\nabla_x H(x,u) &=& [2x_1\,, \; y_1]\,,\\[2mm]
H_u(x,u) &=& 2u + y_1 -y_2\, .
\ea{.}
\]
Kozustandsgleichungen
\[
\fbox{$
\dot{y}_1 = -2x_1\,, \quad \dot{y}_2 = -y_1
$}\; .
\]
Endbedingungen f"ur den Kozustand $y$
\[
y(1) = \ba{[}{c}0\\0\ea{]}\,.
\]
Nominalkontrolle $u = 0$.
\par\vspace{0.5ex}
%%%%%%%%%%%%%%%%%%%%%%%%%%%%%%%%%%%%%%%%%%%%%%%%%%
{\bf Beispiel 2:} Problem der Brachistochrone.\\
Zielfunktion
\[
J(x,u) =
\int_0^1\ell(x,u)\, dt
= \int_0^1\frac{[1 + u(t)^2]^{1/2}}{[1 - x(t)]^{1/2}}\, dt = \Min \, !
\]

Bewegungsgleichung mit Anfangsbedingung

\[
\fbox{$
\dot{x} = u\,, \quad x(0) = 0
$}\; .
\]
{\sc Hamilton}-Funktion
\[
\ba{.}{rcl}
H(x,u) &=& \dis
\frac{[1 + u(t)^2]^{1/2}}{[1 - x(t)]^{1/2}} + yu\,,\\[5mm]
H_x(x,u) &=& \dis
\frac12\cdot \frac{(1+u^2)^{1/2}}{(1 - x)^{3/2}}\,,\\[5mm]
H_u(x,u) &=& \dis
\frac{u}{(1+u^2)^{1/2}(1 - x)^{1/2}} + y\,.
\ea{.}
\]
Kozustandsgleichungen mit Endbedingung

\[
\fbox{$
\dot{y}= \dis -
\frac12\cdot \frac{(1+u^2)^{1/2}}{(1 - x)^{3/2}}\,, \quad y(1) = 0
$}\; .
\]
Zus"atzliche Endbedingung
\[
\ba{.}{rcl}
K = q(x(T)) &=& \dis x(1) + \frac12 = 0\,,\\[2mm]
q_x(x(T))   &=&  1\,.
\ea{.}
\]
Nominalkontrolle $u(t) = 0\,, \; 0 \leq t < 0.5\,, \quad u(t) = - 1\,,\;
0.5 \leq t \leq 1$.

Zus"atzliches System f"ur $K$
\[
H_1 = y_1f\,, \quad \dot{y}_1 = -(H_1)_x = 0\,,
\quad y_1(T) = q_x(x(T)) = 1\,.
\]

\[
\ba{.}{rcl}
\delta u &=& \dis \epsilon [(H_u + z (H_1)_u]\\[4mm]
\da K &=& \dis
\epsilon \left[\int_0^T (H_1)_u \cdot H_u\, dt
+ z \int_0^T [(H_1)_u]^2\, dt\right] =: \epsilon [b + Qz]
\ea{.}
\]

{\em W"ahle}  $\da K = K$ (Residuum), dann folgt f"ur $z$
\[
z = Q^{-1}[K/\epsilon  - b]\, .
\]
%\newpage

%\bc
%\begin{minipage}{10.cm}
%   \epsfxsize=9.9cm
%   \epsffile{fig1.eps}
%\end{minipage}
%\ec
%\centerline{\epsfig{file=stab.eps,height=6cm},bbllx=-6cm,bblly=5mm}
\end{document}
