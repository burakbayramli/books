\documentclass[12pt,a4paper,twoside,leqno]{article}
\input aaformat
\newcommand{\Release}{10/05/05} %%%%%%%%%%%%%%
\begin{document}
{\large\bf Kontrolltheorie: Beispiel 3 \hfill E.Gekeler}
\par
\vspace{-0.5ex}
\hfill{\footnotesize\Release\ }
\par\hrule\par\vspace{2ex}
% ---------------------------------------------------------
{\bf Zermelosches Problem}
Vgl. {\sc Bryson-Ho}: Applied Optimal control, S. 67.\\
In einem
$(x_1,x_2)$-Koordinatensystem soll ein Schiff mit konstanter Geschwindigkeit
$S$ relativ zum Wasser in m"oglichst kurzer Zeit $T$ von dem Punkt $A =
(a_1,a_2)$ zu dem Punkt $B = (b_1,b_2)$ fahren ($x = (x_1,x_2)$).
\[
\ba{.}{ll}
(x_1(t),x_2(t))  & \mbox{Position des Schiffes}\\
v_1(x) & \mbox{Str"omungsgeschwindigkeit in $x_1$-Richtung}\\
v_2(x) & \mbox{Str"omungsgeschwindigkeit in $x_2$-Richtung}\\
u & \mbox{Winkel der Schiffsachse zur $x_1$-Richtung (Kontrolle)}.
\ea{.}
\]
Bei diesem Problem werden die {\sc Lagrange}-Multiplikatoren eliminiert und
$T$ als neue abh"angige Ver"anderliche eingef"uhrt.

Hinweis: Der Gradient einer Abbildung $f:\mb{R}^m \to \mb{R}^n$ ist eine
(n,m)-Matrix.

Zustandsgleichungen (Bewegungsgleichungen) $\dot{x} = [\nabla_y H]^T$:
\begin{equation} \label{1}
\fbox{$
\dot{x}_1 = S \cos(u(t)) + v_1(x(t)),\quad
\dot{x}_2 = S \sin(u(t)) + v_2(x(t))
$}\; .
\end{equation}

Zielfunktion und {\sc Hamilton}-Funktion:
\begin{equation} \label{2}
\ba{.}{rcl}
J(T) &=& \int_0^T1\, dt = \Min!\\
H(x,y,u) &=& y_1[S \cos(u) + v_1(x)] + y_2[S \sin(u) + v_2(x)] + 1.
\ea{.}
\end{equation}

Kozustandsgleichungen ({\sc Euler-Lagrange}-Gleichungen)
$\dot{y} =  - [\nabla_x H]^T$:
\begin{equation} \label{3}
\fbox{$
\dot{y}_1 = -y_1(v_1)_{x_1} - y_2(v_2)_{x_1}, \quad
\dot{y}_2 = -y_1(v_1)_{x_2} - y_2(v_2)_{x_2}
$}\; .
\end{equation}
$H$ und die rechte Seite von (\ref{1}) h"angen  nicht explizit von der Zeit $t$
ab. Daraus folgt die Bedingung $H(x(T),y(T),u(T)) = 0$. Es gilt also
im Optimum
\[
H(x(t),y(t),u(t)) = \dot{H}(x(t),y(t),u(t)) = 0 \qquad \forall
\; t.
\]
Weil die Kontrolle nicht beschr"ankt ist, gilt $H_u = 0$, und es folgt
aus (\ref{2})
\[
\ba{.}{rcl}
y_1(S \cos(u) + v_1) + y_2(S \sin(u) + v_2 &=& - 1,\\
- y_1 \sin(u)  + y_2 \cos(u)              &=& 0.
\ea{.}
\]
Also
\begin{equation} \label{4}
y_1 = \frac{- \cos(u)}{S + v_1 \cos(u) + v_2 \sin(u)}, \quad
y_2 = \frac{- \sin(u)}{S + v_1 \cos(u) + v_2 \sin(u)}.
\end{equation}
Durch Einsetzen von (\ref{4}) in eine der Gleichungen von (\ref{3})
erh"alt man eine Differentialgleichung f"ur die Kontrolle $u$
\par
\begin{equation} \label{5}
\fbox{$
\dot{u} = \sin^2(u)(v_2)_{x_1} + \sin(u) \cos(u)[(v_1)_{x_1} - (v_2)_{x_2}]
- \cos^2(u)(v_1)_{x_2}.
$}
\end{equation}
\par
{\sc Goh-Teo}-Transformation, vgl.\ {\sc Craven}: Control and Optimization,
\S 6.4.3: Es wird die Substitution
\[
t = sT, \quad 0 < s < 1
\]
gew"ahlt und die unbekannte Zeit $T$ als neue (abh"angige) Ver"anderliche
eingef"uhrt:
\[
\ba{.}{rclrcl}
X_3(s) &=& T, &
X_1(s) &=& x_1(sT) = x_1(sX_3(s)), \\
X_2(s) &=& x_2(sT) = x_2(sX_3(s)), &
U(s) &=& u(sX_3(s))\,.
\ea{.}
\]
Man erh"alt dann ein Differentialsystem der Ordnung 4:
\[
\ba{.}{rclrcl}
X_3'(s) &=& 0, &
X_1'(s) &=& \dot{x}_1(sX_3)\cdot X_3(s), \\
X_2'(s) &=& \dot{x}_2(sX_3)\cdot X_3(s), &
U'(s) &=& \dot{u}(sX_3)\cdot X_3(s).
\ea{.}
\]
Der Einfachheit halber sei $y(s) = [X_1,X_2,X_3,U]$, dann ergibt sich das
folgende Randwertproblem:
\[
\ba{.}{rcl}
y'_1 &=& [S \cos(y_4) + v_1(y_1,y_2)]y_3,\\
y'_2 &=& [S \sin(y_4) + v_2(y_1,y_2)]y_3,\\
y'_3 &=& 0,\\
y'_4 &=& [\sin^2(y_4)(v_2)_{y_1} + \sin(y_4)\cos(y_4)
((v_1)_{y_1} - (v_2)_{y_2})
- \cos^2(y_4)(v_1)_{y_2}]y_3.
\ea{.}
\]
mit den Randbedingungen
\[
y(0) = (a_1,a_2,0,0), \quad y(1) = (b_1,b_2,0,0).
\]
Es sei $\dot{y} = f(y)$ und $\grad v = [v_{ik}]$. Dann gilt
\[
\grad f(y) = \ba{[}{cccc}
v_{11}y_3 & v_{12}y_3 & S\cos(y_4) + v_1& - S \sin(y_4)y_3\\
v_{21}y_3 & v_{22}y_3 & S\sin(y_4) + v_2&  S \cos(y_4)y_3\\
0 & 0 & 0 & 0\\
p_1 & p_2 & p_3 & p_4
\ea{]}
\]
mit
\[
\ba{.}{rcl}
p_1 &=& y_3\big(\sin^2(y_4)v_{211} + \sin(y_4)\cos(y_4)(v_{111} - v_{221})
- \cos^2(y_4)v_{121}\big)\\[1mm]
p_2 &=& y_3\big(\sin^2(y_4)v_{212} + \sin(y_4)\cos(y_4)(v_{112} - v_{222})
- \cos^2(y_4)v_{122}\big)\\[1mm]
p_3 &=& \sin^2(y_4)v_{21} + \sin(y_4)\cos(y_4)
(v_{11} - v_{22}) - \cos^2(y_4)v_{12},\\[1mm]
p_4 &=& y_3\big(2\sin(y_4)\cos(y_4)v_{21} + (\cos^2(y_4) - \sin^2(y_4))(v_{11}
- v_{22}) + 2\cos(y_4)\sin(y_4)v_{12}\big)
\ea{.}
\]
Man braucht also zus"atzlich die Matrix
\[
D = \ba{[}{cc} v_{111}  & v_{221}\\ v_{112} & v_{222}\ea{]}
\]
\par
% ----------------------------------------------------------
Das nichtlineare Randwertproblem wird mit dem Box-Schema und dem
{\sc Newton}-Verfahren gel"ost. Als Startwert wird die direkte Fahrt von $A$
nach $B$ gew"ahlt. Im folgenden einfachen Beispiel ist dies eine Strecke.

{\bf Beispiel}: Fahrt  von $A = (a_1,a_2)$ nach $B = (0,0)$ mit $v = (- S/2,0)$.
Winkel $\psi$ zwischen $\ol{AB}$ und $x_1$-Achse:
\[
\cos \psi = - \frac{a_1}{a^2_1 + a^2_2}, \;
\sin \psi = - \frac{a_2}{a^2_1 + a^2_2}.
\]
Kosinussatz:
\[
S^2 =  d^2 + \frac{S^2}{4} - dS \cos(\psi), \qquad
d = \dis \frac{S}{2}(\cos \psi + (3 + \cos^2  \psi)^{1/2})\,.
\]
Winkel $\phi $ zwischen Schiffsachse und Achse $\ol{AB}$:
\[
\frac{S^2}{4} = d^2 + S^2 - 2dS \cos \phi
\quad \Longrightarrow \quad
\cos \phi  = \dis \frac{d^2 + 3S^2/4}{2dS}.
\]
\end{document}
