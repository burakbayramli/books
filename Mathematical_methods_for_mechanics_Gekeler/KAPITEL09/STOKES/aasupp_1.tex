\documentclass[12pt,a4paper,USenglish,twoside]{book}
\input aaformat_e
%%%%%%%%%%%% Springer-Format %%%%%%%%%%
%\documentclass[envcountchap,USenglish,natbib]{svmono}
%%%%%%%%%%%%%%%%%%%%%%%%%%%%%
\usepackage{float}
\usepackage{multicol}
\usepackage{graphics}
%\input aaformat_s
\smartqed
\begin{document}
\setlength{\fboxsep}{1ex}
\newcommand{\Release}{11/10/06}
\addtolength{\abovedisplayshortskip}{-1ex}
\setlength{\fboxsep}{1.5ex}
\parskip0.5ex
\parindent0ex
\mainmatter
%
{\large\bf Supplements 1 to Chapter IX\hfill E.\ Gekeler
}
\par
\vspace{-1mm}
\hfill{\footnotesize\Release\ }
\par
\vspace{-2mm}
\rule{\textwidth}{1pt}
\par\vspace{2ex}
%%%%%%%%%%%%%%%%%%%%%%%%%%%%%%%%%%%%%%%%%%%%%%%%%%%%%%%%%%%
{\bf Case Study: Taylor-Hood element}
\par
%
Complete quadratic ``ansatz'' for each of the velocity components and linear ansatz for 
the pressure component in unit triangle $S$:
\[
\ba{.}{lll}
v(\xi,\eta) &=& a_1 + a_2\xi + a_3\eta + a_4\xi^2 + a_5\xi\eta + a_6\eta^2\\
q(\xi,\eta) &=& b_1 + b_2\xi + b_3\eta
\ea{.}
\]
{\bf (a)} 
\[
\ba{.}{llll}
\Theta(\xi,\eta) &=& [1,\ \xi,\ \eta,\ \xi^2 ,\ \xi\eta, \ \eta^2]^T
& \text{{\em column} vector of algebraic basis}\\
\Psi(\xi,\eta) &=& [\psi_1(\xi,\eta)\,,\ldots\,, \psi_6(\xi\eta)]^T
& \text{{\em column} vector of shape functions in unit triangle}
\ea{.}
\]
Then $v(\xi,\eta) = \Theta(\xi,\eta)^T\a$\,. The connection between both bases
is given by  the design matrix $B$\,:
\begin{equation}\label{e1}
\Psi(\xi,\eta)^T = \Theta(\xi,\eta)^TB \quad \text{(row vector)}\,.
\end{equation}

For the 6 node points $(0,0), (1,0), (0,1), (1/2,0), (1/2,1/2), (0,1/2)$
in unit triangle we find
\[
\ba{.}{lll}
u_1 = v(0,0) &=& a_1\\
u_1 = v(1,0) &=& a_1 + a_2 + a_4\\
u_3 = v(0,1) &=& a_1 + a_3 + a_6\\
u_4 = v(1/2,0) &=& a_1 + a_2/2 + a_4/4\\
u_5 = v(1/2,1/2) &=& a_1 + a_2/2 + a_3/2 + a_4/4 + a_5/4 + a_6/4\\
u_6 = v(0,1/2) &=& a_1 + a_3/2 + a_6/4
\ea{.}
\]
It follows that $\u = A\a, \; \a = B\u$ where
   \[
A = \ \ba{[}{cccccc}
     1 &   0 &   0 &   0 &   0 &   0\\
     1 &   1 &   0 &   1 &   0 &   0\\
     1 &   0 &   1 &   0 &   0 &   1\\
     1 & 1/2 &   0 & 1/4 &   0 &   0\\
     1 & 1/2 & 1/2 & 1/4 & 1/4 & 1/4\\
     1 &   0 & 1/2 &   0 &   0 & 1/4 \ea{]}\,,\;
B = A^{-1} = \ \ba{[}{cccccc}
      1 &    0 &    0 &    0 &    0&     0\\
     -3 &   -1 &    0 &    4 &    0 &    0\\
     -3 &    0 &   -1 &    0 &    0 &    4\\
      2 &    2 &    0 &   -4 &    0 &    0\\
      4 &    0 &    0 &   -4 &    4 &   -4\\
      2 &    0 &    2 &    0 &    0 &   -4\ea{]}\,.
\]
and then, by (\ref{e1}),
\[
\ba{.}{rclcl}
\psi_1 &=& (1-\xi -\eta )(1-2\xi -2\eta )
&=& \zeta _1(2\zeta _1 - 1)\\
\psi_2 &=& \xi (2\xi   - 1) &=& \zeta _2(2\zeta _2 - 1)\\
\psi_3 &=& \eta (2\eta  - 1) &=& \zeta _3(2\zeta _3 - 1)\\
\psi_4 &=& 4\xi (1-\xi -\eta ) &=& 4\zeta _1\zeta _2\\
\psi_5 &=& 4\xi \eta  &=& 4\zeta _2\zeta _3\\
\psi_6 &=& 4\eta (1-\xi -\eta ) &=& 4\zeta _1\zeta _3
\ea{.}
\]
($\zeta_1\,, \zeta_2\,, \zeta_3$ are the barycentric coordinates.)
Now we calculate the following matrices in unit triangle
\[
\ba{.}{llllll}
S_1 &=& \dis \int_S\Psi_{\xi}\Psi_{\xi}^T\, d\xi d\eta\,, &
S_2 &=& \dis \int_S[\Psi_{\xi}\Psi_{\eta}^T + \Psi_{\eta}\Psi_{\xi}^T]\, d\xi d\eta\\
S_3 &=& \dis \int_S\Psi_{\eta}\Psi_{\eta}^T\, d\xi d\eta\,,  &
S_4 &=& \dis \int_S\Psi\Psi^T\, d\xi d\eta
\ea{.}
\]
then
\[
S_1 = \ \frac16\,\ba{[}{rrrrrr}
      3 & 1 & 0 & -4 & 0 & 0\\
     1 & 3 & 0 & -4 & 0 & 0\\
     0 & 0 & 0 & 0 & 0 & 0\\
     -4 & -4 & 0 & 8 & 0 & 0\\
     0 & 0 & 0 & 0 & 8 & -8\\
     0 & 0 & 0 & 0 & -8 & 8\ea{]}\,, \;\;
%
S_2 = \ \frac16\,\ba{[}{rrrrrr}
      6 & 1 & 1 & -4 & 0 &-4\\
     1 & 0 &-1 & -4 & 4 & 0\\
     1 &-1 & 0 & 0 & 4 &-4\\
     -4 & -4 & 0 & 8 &-8 & 8\\
     0 & 4 & 4 &-8 & 8 & -8\\
    -4 & 0 &-4 & 8 & -8 & 8\ea{]}\,,
\]
\[
S_3 = \ \frac16\,\ba{[}{rrrrrr}
      3 & 0 & 1 & 0 & 0 & -4\\
     0 & 0 & 0 &  0 & 0 & 0\\
     1 & 0 & 3 & 0 & 0 &-4\\
     0 & 0 & 0 & 8 &-8 & 0\\
     0 & 0 & 0 &-8 & 8 &  0\\
    -4 & 0 &-4 & 0 &0 & 8\ea{]}\,, \;\;
%
S_4 = \ \frac{1}{360}\,\ba{[}{rrrrrr}
      6 &-1 &-1 &  0 &-4 & 0\\
    -1 & 6 &-1 &  0 & 0 &-4\\
    -1 &-1 & 6 &-4 & 0 & 0\\
     0 &  0 &-4 &32 &16 &16\\
    -4 & 0 & 0 &16 &32 & 16\\
     0 &-4 & 0 &16 & 16 &32\ea{]}\,,
\]
Likewise one calculates the shape functions for linear approximation in unit triangle:
\[
\wi{\Psi} = [1 - \xi -\eta\,, \ \xi\,, \eta]^T
\]
and the both integrals
 \[
C_1 = \int_S\Psi_{\xi}\wi{\Psi}^T\, d\xi d\eta
= \frac16 \, \ba{[}{rrr}
 -1 & 0&  0\\
  0 & 1&  0\\
  0 & 0&  0\\
  1 &-1&  0\\
  1 & 1&  2\\
 -1 &-1& -2\ea{]}\,,\;\;
%
C_2 = \int_S\Psi_{\eta}\wi{\Psi}^T\, d\xi d\eta
= \frac16 \, \ba{[}{rrr}
 -1&  0&  0\\
  0&  0&  0\\
  0&  0&  1\\
 -1& -2& -1\\
  1&  2&  1\\
  1&  0& -1\ea{]}
\]
and the mass matrix for linear elements
\[
S_5 = \int_S\wi{\Psi}\wi{\Psi}^T\, d\xi d\eta =
\frac{1}{24}\,\ba{[}{ccc} 2 & 1 & 1\\ 1 & 2 & 1\\1 & 1 & 2\ea{]}\,.
\]
Stationary {\sc Navier-Stokes} equations with convection term:
\[
\ba{.}{rclcll}
a(\v,\u) + c(\v,\u,\u) &-&(\div \v,p) &=& (\v,\f)
+ (\v,\ul{\sigma}_n(\u,p))_{\Gamma}\,, &\v \in \V \\[1ex]
-(\div \v,q) && &=& 0\,,     & q \in \Q\\[1ex]
\ul{\sigma}_n(\u,p) && &=& (2\nu \eps(\u) - p)\n
\ea{.}
\]
In arbitrary triangle $T(x,y)$ we have
\[
\ba{.}{lll}
a(\u,\v) &=& \nu \dis \int_T\grad \u :\grad \v \, dxdy\\[1ex]
&=& \dis \nu \int_T[u_{1,x}v_{1,x} + u_{1,y}v_{1,y}]\, dxdy
+ \nu \int_T[u_{2,x}v_{2,x} + u_{2,y}v_{2,y}]\, dxdy\\[1ex]
&=& \wi{a}(u_1,v_1) + \wi{a}(u_2,v_2)\\[1ex]
b(\u,q) &=& \dis \int_T\div \u \cdot q\, dxdy
= \int_T(u_{1,x} + u_{2,y})\,q \, dxdy\\[1ex]
c(\u,\v,\w) &=& \dis \int_T\u \cdot (\grad \v)\w\, dxdy
\ea{.}
\]
For transformation on the unit triangle we have e.g.
\[
\ba{.}{lll}
b(\u,q)
&=& \dis
\int_T(u_{1,\xi}\xi_x + u_{1,\eta}\eta_x)\,q\,dxdy
+ \int_T(u_{2,\xi}\xi_y + u_{2,\eta}\eta_y)\,q\,dxdy\\[1ex]
&=& \dis
J \int_S(u_{1,\xi}\xi_x + u_{1,\eta}\eta_x)\,q\,d\xi d\eta
+ J \int_S(u_{2,\xi}\xi_y + u_{2,\eta}\eta_y)\,q\,d\xi d\eta\\[1ex]
&=& \dis
\frac{J}{J}U_1^T(C_1y_{31} - C_2y_{21})Q
+ \frac{J}{J}U_2^T(C_2 x_{21} -C_1x_{31})Q\,,
\ea{.}
\]
where $U_1,U_2 \in \mb{R}^6\,,\; Q \in \mb{R}^3$ are the local vectors of the node
variables in the triangle
\par
{\bf (b) Trilinear Form} with shape functions.\\
$
P(\u,\v,\w) = \u^T[\grad \v]\w
$
\[
\u \simeq \ \ba{[}{r}\Phi(x,y)^TU_1\\ \Phi(x,y)^TU_2\ea{]}\,,\;
\]
\par\vspace{1ex}
\[
N(\u,\v,\w) \simeq \
[U^T_1,U^T_2]\ \ba{[}{cc}
(\Phi_x^TV_1)\Phi\,\Phi^T &
(\Phi_y^TV_1)\Phi\,\Phi^T \\
(\Phi_x^TV_2)\Phi\,\Phi^T &
(\Phi_y^TV_2)\Phi\,\Phi^T \ea{]}\,
\ba{[}{r}W_1\\ W_2\ea{]}\,,\;
\]
\par
\[
\Phi_x = \Psi_{\xi}\xi_x + \Psi_{\eta}\eta_x\,,\;\;
\Phi_y = \Psi_{\xi}\xi_y + \Psi_{\eta}\eta_y\,,
\]
\par\vspace{1ex}
\[
\int_TN(\u,\v,\w) \simeq
[U_1^T,U^T_2]\, \ba{[}{rr} A(V_1) & B(V_1)\\ C(V_2) & D(V_2)\ea{]}
\ba{[}{r}W_1\\ W_2\ea{]}\,,\;
\]
where
\begin{equation}\label{e2}
\ba{.}{lll}
A(V_1) &=& \dis J\int_S\big((\Psi_{\xi}\xi_x + \Psi_{\eta}\eta_x)^TV_1\big)
\Psi\,\Psi^T\, d\xi d\eta\,, \\[1ex]
B(V_1) &=& \dis J\int_S\big((\Psi_{\xi}\xi_y + \Psi_{\eta}\eta_y)^TV_1\big)
\Psi\,\Psi^T\, d\xi d\eta\\[1ex]
C(V_2) &=& \dis J\int_S\big((\Psi_{\xi}\xi_x + \Psi_{\eta}\eta_x)^TV_2\big)
\Psi\,\Psi^T\, d\xi d\eta \,,\\[1ex]
D(V_2) &=& \dis J\int_S\big((\Psi_{\xi}\xi_y + \Psi_{\eta}\eta_y)^TV_2\big)
\Psi\,\Psi^T\, d\xi d\eta
\ea{.}
\end{equation}
Now we have the representation
\[
\Psi_{\xi} = \ \ba{[}{rrr}
 -3 & 4 & 4\\
 -1 & 4 & 0\\
 0 & 0 & 0\\
 4 & -8 & -4\\
 0 & 0 & 4\\
0 & 0 & -4\ea{]}
\ba{[}{r}1\\ \xi \\ \eta\ea{]}
 =: [A_1\,,\, A_2\,,\, A_3]\ \ba{[}{r}1\\ \xi \\ \eta\ea{]}
\]
\[
\Psi_{\eta} = \ \ba{[}{rrr}
 -3 & 4 & 4\\
  0 & 0 & 0\\
 -1 & 0 & 4\\
 0 & -4 & 0\\
 0 & 4 & 0\\
 4 & -4 & -8\ea{]}\
\ba{[}{r}1\\ \xi \\ \eta\ea{]}
 =: [B_1\,,\, B_2\,,\, B_3]\ \ba{[}{r}1\\ \xi \\ \eta\ea{]}
\]
Accordingly, only the both matrices $P\,,Q$ are to be calculated:
\[
M := S_4 = \int_S \Psi\,\Psi^T\, d\xi d\eta\,, \;\;
P = \int_S\xi \Psi\,\Psi^T\, d\xi d\eta\,, \;\;
Q = \int_S\eta \Psi\,\Psi^T\, d\xi d\eta\,,
\]
Then we obtain
\[
\ba{.}{lll}
\dis \int_S(\Psi_{\xi}V)\Psi\Psi^T \,d\xi d\eta
&=& (A_1^TV)M + (A_2^TV)P + (A_3^TV)Q\\[1ex]
\dis \int_S(\Psi_{\eta}V)\Psi\Psi^T \,d\xi d\eta
&=& (B_1^TV)M + (B_2^TV)P + (B_3^TV)Q\\[1ex]
\ea{.}
\]
and the integrals (\ref{e2}) are linear combinations of these both integrals with the 
corresponding arguments $V_1$ resp.\ $V_2$
\[
\ba{.}{lll}
A(V_1) &=& [(A^T_1y_{31} + B^T_1y_{12})V_1]M + [(A^T_2y_{31} + B^T_2y_{12})V_1]P
+ [A^T_3y_{31} + B^T_3y_{12})V_1]Q\\[0.5ex]
%
B(V_1) &=& [(A^T_1x_{13} + B^T_1x_{21})V_1]M + [(A^T_2x_{13} + B^T_2x_{21})V_1]P
+ [(A^T_3x_{13} + B^T_3x_{21})V_1]Q\\[0.5ex]
%
C(V_2) &=& [(A^T_1y_{31} + B^T_1y_{12})V_2]M + [(A^T_2y_{31} + B^T_2y_{12})V_2]P
+ [A^T_3y_{31} + B^T_3y_{12})V_2]Q\\[0.5ex]
%
D(V_2) &=& [(A^T_1x_{13} + B^T_1x_{21})V_2]M + [(A^T_2x_{13} + B^T_2x_{21})V_2]P
+ [(A^T_3x_{13} + B^T_3x_{21})V_2]Q
\ea{.}
\]
{\bf(c)}
The nonlinear problem may be solved by a simple iteration method or by {\sc Newton}'s 
method. The latter allows higher {\sc Reynolds} numbers $Re = 1/\nu$. Also, in {\sc 
Newton}'s method the gradient of the nonlinear part must be calculated. From {\bf (b)}
\[
\ba{.}{l}
\grad_{U_1,U_2}
\, \ba{[}{rr} A(U_1) & B(U_1)\\ C(U_2) & D(U_2)\ea{]}
\ba{[}{r}U_1\\ U_2\ea{]}
= \ \ba{[}{rr} A(U_1) & B(U_1)\\ C(U_2) & D(U_2)\ea{]}
\\[3ex]
 + \ \ba{[}{cc} \grad_{U_1}A(U_1)U_1 + \grad_{U_1}B(U_1)U_2 & \text{Null} \\ 
     \text{Null} & \grad_{U_2}C(U_2)U_1 + \grad_{U_2}D(U_2)U_2\ea{]}
\ea{.}     
\]
\par
{\bf (d) Postprozessor}
Let $\u = (u,v)$ be the velocity field then the streamlines $z$ satisfy
%
\[
\int _{\Omega}\nabla \delta z\, \nabla z
= \int_{\Omega}\delta z\, w  + \int_{\Gamma}\delta z \, z_{\n}
\]
\[
\int_T\phi_iw = \int_T\phi_i(v_x - u_y) + \int_{\Gamma}\delta z \, z_{\n}
\]
\[
\ba{.}{l}
\dis \int_T\delta z\, w \,dxdy\\[1ex]
\simeq \dis
JZ\int_S\Psi(\Psi_{\xi}\xi_x + \Psi_{\eta}\eta_x)^T\,d\xi d\eta V
- JZ \int_S\Psi(\Psi_{\xi}\xi_y + \Psi_{\eta}\eta_y)^T\,d\xi d\eta U\\[1ex]
= \dis
JZ \left[\int_S\Psi\Psi_{\xi}^T\,d\xi d\eta\right](V\xi_x - U\xi_y)
+
JZ\left[\int_S\Psi\Psi_{\eta}^T\,d\xi d\eta\right](V\eta_x - U\eta_y)
\ea{.}
\]
Accorcingly, the matrices
\[
\int_S\Psi\Psi_{\xi}^T\,d\xi d\eta\,,\;\;
\int_S\Psi\Psi_{\eta}^T\,d\xi d\eta
\]
are to be calculated.
%
\[
z_x = - v\,,\;\; z_y = u\,,\;\; \grad z\cdot \n = -v\,n_1 + u\,n_2.
\]
%
\[
\xi _x = y_{31}/J, \;\; \xi _y = x_{13}/J, \;\;
\eta _x = y_{12}/J, \;\; \eta _y = x_{21}/J\,.
\]

\end{document}
