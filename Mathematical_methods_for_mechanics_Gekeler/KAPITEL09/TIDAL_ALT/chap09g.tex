\documentclass[12pt,a4paper,leqno,twoside]{book}
\input aaformat
%%%%%%%%%%%%%%%%%%%%%%%%%%%%%%%%%%%%%%%%%
\begin{document}
\newcommand{\Release}{14/04/06}
\addtolength{\abovedisplayshortskip}{-1ex}
\setlength{\fboxsep}{1.5ex}
\parindent0ex
%
{\large\bf Erg"anzungen 5 zu Kap. IX\hfill E.\ Gekeler
}
\par
\vspace{-1mm}
\hfill{\footnotesize\Release\ }
\par
\vspace{-2mm}
\rule{\textwidth}{1pt}
\par\vspace{2ex}
{\bf Fallstudie: Shallow Water Problems}
\par
%
Verwendete Gr"o\ss en im $(x,y,z)$--Koor\-dinaten\-system
%
\bc
\begin{tabular}{|lll|}\hline
$\ h$               && mittlere Wassertiefe\\
$\ w = h + z$  && totale Wassertiefe\\
$\ \wi{u}$               && Str"omung nach Osten\\
$\ \wi{v}$               && Str"omung nach Norden\\
$\ \dis u = \frac{1}{w}\int_{-h}^z \wi{u}\ d\zeta $ && mittlere Geschw. nach
Osten\\
$\ \dis v = \frac{1}{w}\int_{-h}^z \wi{v}\ d\zeta$ && mittlere Geschw. nach
Norden\\
$\ \theta$ & $m/s$ & Windgeschwindigkeit $10 \ m$ \\
&&"uber Wasseroberfl"ache\\
$\ \kappa$ && dim.-loser Koeff. der Oberfl"achenkraft\\
&& durch Windeinfluss\\
$\ \phi $ & $rad$ & Breitengrad\\
$\ \psi$  && Winkel der Windrichtung gegen"uber Osten\\
$\ \omega = 7.292\times 10^{-5}$ & $rad/s$
&  Winkelgeschwindigkeit der Erdrotation\\
$\ f = 2\omega  \sin \phi $ & $rad/s$ & {\sc Coriolis}--Faktor\\
$\ g = 9.81$ & $m/s^2$ & Fallbeschleunigung \\
$\ \mu _e$ & $kg/s$ & {\sc Eddy}-Viskosit"at\\
$\ C$ &$m^{1/2}/s$ & Reibungskoeff. nach {\sc Chezy} gegen"uber\\
&& Meeresgrund \\
$\ n$ && {\sc Manning}-Koeffizient der Rauhigkeit\\
&& des Meeresgrundes\\ \hline
\end{tabular}
\ec
					
In Rechenbeispielen wird etwa
\[
\gamma = \dis n^{-1}h^{1/6}\,, \; n = 0.025\,\;\; \text{oder}\;\;
\gamma = \dis 1.486\,n^{-1}h^{1/6}\,,\; n = 0.0402 \; \; \text{[Peraire]}
\]
gesetzt. F"ur den Koeffizienten $\kappa$ wird in Abh"angigkeit von der 
Windgeschwindigkeit
\[
\kappa = \ \ba{\{}{ll}
1.0 \times 10^{-3} & (\theta \leq 5)\\
1.5 \times 10^{-3} & (5 < \theta \leq 15)\\
2.0 \times  10^{-3} & (15 < \theta \leq  20)
\ea{.}
\]
%
gew"ahlt.
\par
{\bf (a)} Nach \S\, 8.10 gilt f"ur den
den {\sc Green-Lagrange}-Verzer\-rungs\-tensor und
den Spannungstensor 
%
\begin{equation} \label{e1}
\fbox{$\
\ba{.}{rcl}
\eps(\u) &=& \dis \frac{1}{2}\left[\grad \u + (\grad \u)^T\right]\\[1ex]
\sig( \u) &=& 2 \mu_1 \eps (\u) + \mu_2 \spur \eps (\u)\del\\[1ex]
%
&=& \dis 2 \mu_1 \eps (\u ) - \frac{2}{3}\mu_1 (\div \u )\del
 + \left(\frac{2}{3} \mu_1 + \mu_2 \right) \div \u \del\\[1ex]
%
&=& \dis 2 \mu_1\left[ \eps (\u ) - \frac{1}{3} (\div \u )\del\right]
 + \mu_3 \, (\div \u)\del
\ea{.}
$}\; ,
\end{equation}
%
dabei sind
%
\[
\fbox{$\
\ba{.}{rcll}
\mu _1&=& 3(\mu_3 - \mu_2)/2   & \mbox{die Scherz"ahigkeit}\\
\mu_2 &=& (3\mu_3 - 2\mu_1)/3  & \mbox{die Volumenz"ahigkeit}\\
\mu_3&=& (2\mu _1 + 3\mu _2)/3 &\mbox{die Druckz"ahigkeit}\\
\nu &=&\mu_1/\rho & \mbox{die kinematische Z"ahigkeit}
\ea{.}
$} \; .
\]
%
[Becker], S. 114, 148.
In $\sig$ ist die Spur von $[....]$ Null, daher gibt der erste Term die
in\-fini\-tesi\-male Ge\-stalt\-"ande\-rung und der zweite die infinitesimale
Volumen"anderung an.
\par\vspace{0.5ex}
Ferner gilt mit $\grad \phi$ als Zeilenvektor
%
\[
\ba{.}{rcl}
\div (\div (\u)\del) &=& [\grad \div \u]^T \quad \mbox{(Spaltenvektor)}
\\[0.5ex]
\div([\grad \u]^T) &=& [\grad \div \u]^T \quad \mbox{(Spaltenvektor)}\,,
\ea{.}
\]
also
%
\begin{equation}\label{e2}
\ba{.}{lll}
\sig(\u)
&=& \dis \mu_1\, \left[ \grad \u +  [\grad \u]^T\right]
+ \mu_2 \, (\div \u)\del\\[0.5ex]
%
\div \sig(\u)
&=& \dis \mu_1 \, \left[\div \grad \u + [\grad \div \u]^T\right]
+ \mu_2 \, [\grad \div \u]^T\\[0.5ex]
&=& \dis \mu_1 \,\Delta \u + (\mu_1 + \mu_2)[\grad \div \u]^T\,,
\ea{.}\hspace{7ex}
\end{equation}
%
also $\div \sig(\u) = \mu_1\, \Delta \u$ wenn $\div \u = 0$.
Im Weiteren setzen wir voraus, dass die Materialgr"o\ss en $\mu _1$ und $\mu
_2$ {\em konstant} in Raum und Zeit sind.  Ferner soll $\grad p$ ein
Spaltenvektor sein, wie auch aus dem Zusammenhang zu erkennen ist.
Dann folgt
\[
\div(\sig (\u) - p\del)
=  \mu \,\Delta - \grad p\,,\;\; \mu = \mu_1.
\]
Einsetzen in den lokalen Impulserhaltungssatz ergibt die Erhaltungsgleichungen 
f"ur z"ahe Fluide in der nichtkonservativen Form  
%
\begin{equation} \label{e3}
\fbox{$\
 \ba{.}{rclcl}
\dis \: \div \u
&=& 0\\[2ex]
\dis \frac{D \u}{Dt} &-&\dis \frac{\mu}{\rho} \,\Delta \u
+ \frac{1}{\rho}\grad p - \f &=& 0\\[2ex]
\dis \rho \frac{D\epsilon }{Dt} &-& \eps(\u):\sig(\u) + \div \q
- \rho \,r &=& 0
\ea{.}
$}\; .
\end{equation}
\par
oder in der konservativen Form ohne Energieerhaltungssatz
({\sc Einstein}sche Summenkonvention)
\begin{equation} \label{e4}
\div \u = 0\,,\;\; \text{oder} \;\; \frac{\da u_i}{\da x_i} = 0
\end{equation}
%
\begin{equation}\label{e5}
\u_t + (\grad \u)\u - \frac{\mu}{\rho}\,\Delta \u 
+ \frac{1}{\rho}\, \grad p - \f = \ul{0} \in \mb{R}^3
\end{equation}
%
Der Massenerhaltungssatz (Kontinuit"atsgleichung) hat hier die einfache Form
$\div \u = 0$\,.
\par\vspace{0.5ex}

{\bf (b)} Bei Flachwassergleichungen ist die vertikale Geschwindigkeit $u_3$ 
gering und ihre Beschleunigung ebenfalls, so dass die dritte Gleichung in 
(\ref{e5}) durch
%
\begin{equation}\label{e6}
\frac{1}{\rho}\,\frac{\da p}{\da x_3} + g = 0\,,\;\; f_3 = - g
\end{equation}
%
ersetzt werden kann. Integration ergibt mit einer geeigneten Wahl der
Integrationskonstanten und dem atmosph"arischen Luftdruck $p_a$
\[
p = \rho\, g(z - x_3) + p_a
\]
F"ur $x_3 = z$ also auf der Wasseroberfl"ache folgt $p = p_a$ und
wird vielfach vernachl"assigt.
\par
Auf der Wasseroberfl"ache (surface) ist nat"urlich $u^s_3$ die Zeitableitung des
Wasserstandes, daraus folgt
%
\begin{equation}\label{e7}
u^s_3 = \frac{Dz}{D t} = \frac{\da z}{\da t}
+ u^s_1\frac{\da z}{\da x_1} + u^s_2\frac{\da z}{\da x_2}
\end{equation}
und am Meeresgrund (bottom) gilt
%
\begin{equation}\label{e8}
u^b_3 = \frac{D h}{D t} = \frac{\da h}{\da t}
+ u^b_1\frac{\da h}{\da x_1} + u^b_2\frac{\da h}{\da x_2}
\end{equation}
%
Es wird nun angenommen, dass f"ur eine halbwegs z"ahe Fl"ussigkeit
am Boden kein Schlupf vorkommt und 
%
\begin{equation}\label{e9}
u^b_1 = u^b_2 = 0
\end{equation}
%
also auch $u^b_3 = 0$ gesetzt.
\par \vspace{0.5ex}
%%%%%%%%%%%%%%%%%%%%%%%%%%%%%%%%%%%%%%%%%%%%%%%%%%%%%%%%%%%%%%%%%%%%
{\bf (c)} Als n"achstes wird von der Geschwindigkeit $\u = [u_1,u_2]$ zur {\em 
mittleren Geschwindigkeit} "ubergegangen verm"oge einer fiktiven Integration 
"uber $x_3$:
%
\begin{equation}\label{e10}
U := \frac{1}{w}\int_{-h}^z \wi{u}\ d\zeta\,,\;\; 
V = \frac{1}{w}\int_{-h}^z \wi{v}\ d\zeta\,.
\end{equation}
Aus der Kontinuit"atsgleichung (\ref{e4}) folgt dann durch Integrationh%
%
\begin{equation}\label{e11}
\int_{-h}^z\frac{\da u_3}{\da x_3}\,dx_3 +
\int_{-h}^z\frac{\da u_1}{\da x_1}\,dx_3 +
\int_{-h}^z\frac{\da u_2}{\da x_2}\,dx_3 = 0
\end{equation}
%
(wobei eigentlich die Wassertiefe negativ gemessen wird, so dass der gesamte 
Wasserstand $w = z - h$ {\em w"are} mit dem mittlerem Wasserstand $|h|$,
damit der mittlere Wasserstand die $(x_3 = 0)$-Ebene ist). 
\par
Unangefochten von dieser Bemerkung wird nun die mittlere Geschwindigkeit $U_i$ 
definiert zu
%
\begin{equation}\label{e12}
\int_{-h}^z u_i\,dx_3 = U_i(h + z) \equiv U_i w\,,\;\; i = 1:2\,.
\end{equation}
%%%%%%%%%%%%%%%%%%%%%%%%%%%%%%%%%%%%%%%%%
{\bf (d)}
Aus der {\sc Leibniz}-Regel f"ur parameterabh"angige Integrale
%
\begin{equation}\label{e13}
\frac{d}{dx}\int_{a(x)}^{b(x)}f(x,y)\,dy
= f(x,b(x))b'(x) - f(x,a(x))a'(x)
+ \int_{a(x)}^{b(x)}\frac{da}{\da x}f(x,y)\, dy
\end{equation}
%
folgt nun
(Summation "uber i = 1:2)
%
\begin{equation}\label{e14}
\int_{-h}^z\frac{\da u_i}{\da x_i}\,dx_3
= U_iw - u^s_i\frac{\da z}{\da x_i}\,,\;\; 
\end{equation}
und durch direkte Integration mit (\ref{e9}) und $u^b_3 = 0$
%
 \begin{equation}\label{e15}
\int_{-h}^z\frac{\da u_3}{\da x_3}\,dx_3 
= u^s_3 = \frac{\da z}{\da t} + u^s_1\frac{\da z}{\da x_1}
 + u^s_2\frac{\da z}{\da x_2}
\end{equation}

Schlie\ss lich folgt (\ref{e11}) durch Addition von (\ref{e14}) und (\ref{e15}) zur Kontinuit"atsgleichung f"ur die gemittelten Geschwindigkeiten
%
\begin{equation}\label{e16}
\fbox{$ \dis
\frac{\da z}{\da t} + \frac{\da (wU_i)}{\da x_i}
\equiv \frac{\da w}{\da t} + \frac{\da (wU_1)}{\da x_1}
+ \frac{\da (wU_2)}{\da x_2} = 0
$}\;.
\end{equation}
\par
%%%%%%%%%%%%%%%%%%%%%%%%%%%%%%%%%%%%%%%%%%%%%%%%%%%%%%%%%
{\bf (e)} Ebenso werden die Impulserhaltungsgleichungen (momentum equations)
integriert:
%
\begin{equation}\label{e17}
\int_{-h}^z\Big[
\frac{\da u_i}{\da t} + (\grad u_i)\u - \frac{\mu}{\rho}\,\Delta u_i 
+ \frac{1}{\rho}\,\frac{\da p}{\da x_i} - f_i
\Big]\, dx_3 = 0\,,\;\; i = 1:2
\end{equation}
%
Nach \glqq einigen algebraischen Manipulationen\grqq\ [Zienkiewicz] FEM II, S. 606,
erh"alt man
%
\begin{equation}\label{e18}
\ba{.}{l} \dis
\frac{\da (wU_i)}{\da t} 
+ \frac{\da}{\da x_j}
\left[
wU_iU_j + \delta_{ij}\,\frac12\, g\,(w^2 - h^2)
- \frac{1}{\rho}\int_{-h}^z\tau_{ij}\, dx_3\right]\\[1ex]
\dis
-\, \frac{1}{\rho}\,(\tau^s_{3i} - \tau^b_{ij}) - wf_i - g(w - h)\frac{\da h}{\da x_i}
+ \frac{w}{\rho}\frac{\da p_a}{\da x_i} = 0\,,\;\; i = 1:2\,,
\ea{.}
\end{equation}
wobei der Verfasser die Berechnung des konvektiven Terms so nicht nachvollziehen kann.
Im "Ubrigen sind dies {\em keine} Gleichungen in den abh"angigen Variablen
$w,\, wU,\, wV$.
\par
Dabei ist
$
[f_1, f_2] = f[-V,\, u]
$ die {\sc Coriolis}-Kraft und
\[
\tau^b_{3i} = \frac{\rho g |\ul{U}|U_i}{C^2}
\]
mit dem {\sc Chezy}-Koeffizienten $C$ am Meeresgrund, sowie
nach [Zienkiewicz]
\[
\ba{.}{lll}
\tau_{ij} &=& \dis \mu \left(\frac{\da u_i}{\da x_j} + \frac{\da  u_j}{\da x_i}
- \frac23 \, \frac{\da u_i}{\da x_i}\right)\;\;
\text{oder}\;\; 
\tau = \mu \grad \u \;\; \text{f"ur}\;\; \div \u = 0\;\; \text{s.oben}\\[1ex]
\dis \int_{-h}^z\tau_{ij}\,dx_3
&\simeq& \dis \mu \, w
\left(\frac{\da U_i}{\da x_j} + \frac{\da  U_j}{\da x_i}
- \frac23 \, \frac{\da U_i}{\da x_i}\right)
\;\; \text{oder}\;\;
\tau = \mu\,w\, \grad \ul{U}
\,.
\ea{.}
\]
Au\ss erdem ist
\[
\frac12 g(w^2 - h^2)_x - g(w - h)h_x
= (gww_x - ghh_x) - (gwh_x - ghh_x) = gw(w - h)_x = gwz_x
\]

Die folgenden Flachwassergleichungen nach [Pinder \& Gray]
werden allgemein akzeptiert und in der numerischen Behandlung angewendet\,
\fbox{$w = h + z$}:

\begin{equation} \label{e19}
w_t + (wU)_x + (wV)_y = 0\,, \;\; 
\end{equation}
%
\begin{equation}\label{e20}
\fbox{$
\ba{.}{l}
U_t + U\, U_x + V \, U_y - fV + g z_x\\
\dis -  \frac{\mu_e}{\rho w}\big[(wU_x)_x  + (wU_y)_y\big]
 - \frac{\kappa\, \theta^2}{w} \cos \psi
 + \frac{gU(U^2 + V^2)^{1/2}}{w\, C^2} = 0\\[1ex]
V_t + U\, V_x + V \, V_y + fU + g z_y\\
\dis - \frac{\mu_e}{\rho w}\big[(w V_x)_x  + (w V_y)_y\big]
- \frac{\kappa\, \theta^2}{w} \sin \psi
+ \frac{gV(U^2 + V^2)^{1/2}}{w\,C^2} = 0
\ea{.}
$}\;.
\end{equation}
%
Hier ist $U_t + U\, U_x + V \, U_y$ die Materialableitung von $U$.
\par
Wenn (\ref{e20}) mit $w$ multipliziert wird
und (\ref{e19}) mit $U$ bzw. mit $V$
und beide addiert  werden, ergibt sich f"ur die erste Gleichung in (\ref{e20})
%
\[
\ba{.}{l}
U\big(z_t + (wU)_x + (wV)_y\big) +
w\big(U_t + U\, U_x + V \, U_y - fV + g z_x\big)\\[1ex]
\dis  - w\left(  \frac{\mu_e}{\rho w}\big[(wU_x)_x  + (wU_y)_y\big]
+ \frac{\kappa\, \theta^2}{w} \cos \psi
- \frac{gU(U^2 + V^2)^{1/2}}{w\, C^2}\right) = 0
\ea{.}
\]

oder insgesamt
%
\begin{equation}\label{21}
\ba{.}{l}
(wU)_t + (wUU)_x + (wUV)_y \\[1ex]
\dis
+ gwz_x - fwV - \frac{\mu}{\rho}\big[(wU_x)_x + (wU_y)_y\big]
- \kappa \theta^2\cos \psi + \frac{gU|\ul{U}|}{C^2} = 0\\[2ex]
%%%
(wV)_t + (wVU)_x + (wVV)_y \\[1ex]
\dis
+ gwz_x + fwU - \frac{\mu}{\rho}\big[(wV_x)_x + (wV_y)_y\big]
- \kappa \theta^2\sin \psi + \frac{gV|\ul{U}|}{C^2} 
\ea{.}
\end{equation}
Diese Gleichungen stimmen mit (\ref{e18}) "uberein ($p_a = 0$).
\par\vspace{0.5ex}
%%%%%%%%%%%%%%%%%%%%%%%%%%%%%%%%%%%%%%%%%%%%%%%%555
{\bf (f)}
Wir vernachl"assigen nun die Windkraft und die {\sc Coriolis}-Kraft.
Die Koeffizienten des Diffusionsterms und des {\sc Chezy}-Terms sind zwar klein, bei 
einigen Problemen kann aber auf ihren d"ampfenden Einfluss nicht verzichtet werden. 
Zur Integration des {\sc Chezy}-Terms "uber ein Dreieck $T$ wird die
{\sc Gau\ss}-Integration verwendet. Der Diffusionsterm wird linearisiert, z.B. 1. 
Gleichung
%
\[
\frac{\mu_e}{\rho w}\big[(wU_x)_x  + (wU_y)_y\big]
= \frac{\mu_e}{\rho w}\big[wU_{xx}  + wU_{yy} + w_xU_x + w_yU_y\big]
\sim \nu \big[U_{xx}  + U_{yy}\big] = \nu \Delta U\,.
\]
%
Dann ergibt sich aus (\ref{e19}) und (\ref{e20})
%
\begin{equation} \label{e22}
\fbox{$
\ba{.}{l}
w_t = - (wU)_x - (wV)_y\\[1ex]
%
U_t = \dis - U\, U_x - V \, U_y - g z_x
+ \nu \Delta U  - \frac{gU|\ul{U}|}{w\, C^2}\\[2ex]
 %
V_t = \dis - U\, V_x - V \, V_y  - g z_y
+ \nu \Delta V - \frac{gV|\ul{U}|}{w\,C^2} = 0
\ea{.}
$}\;.
\end{equation}
%
\end{document}


