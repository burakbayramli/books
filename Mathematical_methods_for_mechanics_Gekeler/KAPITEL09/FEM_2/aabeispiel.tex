\documentclass[12pt,a4paper]{article}
\input aaformat
\begin{document}
%
\addtolength{\abovedisplayskip}{-1ex}
\addtolength{\belowdisplayskip}{-1ex}
\addtolength{\abovedisplayshortskip}{-1ex}
\addtolength{\belowdisplayshortskip}{-1ex}
%
%\thispagestyle{empty}
%\kopfp
%
\bc
{\bf Kirchhoff-Platte}
\ec
{\large
Beispiel. Batoz et al.: Int. J. Numer. Meth. Eng. 15, S. 1798
\par
\[ \ba{.}{rcl}
1 \; \mbox{inch} &=& 2.54 \; \mbox{cm},\\
1 \; \mbox{psi} &=& 0.070307212 \; \mbox{kp$/$cm$^2$},\\
\cos (\pi/4) = \sin (\pi/4) &=& 1/\sqrt{2},\\
1 \; \mbox{kp} &=& 9.80665 \; \mbox{N},\\
q &=& 1.832 627 787 99\cdot 10^{-2} \; \mbox{kp/cm$^{2}$},\\
E &=& 0.35153606\times 10^7 \; \mbox{kp/cm$^2$},\\
h &=& 0.3175 \; \mbox{cm},\\
\nu &=&  0.3,\\
12 \mbox{inch}\times \cos(\pi/4) &=& 21.5526146906 \mbox{cm}\\
\ea{.}
\]
Testwerte in cm:
\par
\begin{tabular}{cccccc}
1 & 2 & 3 & 4 & 5 & 6\\
0.75438 & 0.51816 & 0.30734 & 0.32766 & 0.14224 & 0.05588
\end{tabular}
\par
}
\end{document}

