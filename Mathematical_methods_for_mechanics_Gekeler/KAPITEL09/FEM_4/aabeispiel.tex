\documentclass[12pt,a4paper,leqno]{article}
\input aaformat
\begin{document}
%
\setlength{\fboxsep}{1.5ex}
\parindent0ex
%
\bc
{\bf Beispiele zur Konvektion}
\ec
\bc
\begin{tabular}{|lll|} \hline
Phys. Gr"o\ss e & SI-Einheit & Definition \\ \hline
Kraft & Newton (N) & $kg\cdot m/s^2$\\
Druck & Pascal (Pa)& $N/m^2 = kg/(m \cdot s^2)$\\
Energie & Joule (J) & $N \cdot m = kg \cdot m^2/s^2$\\
Leistung & Watt (W) & $J/s = kg \cdot m^2/s^3$\\ \hline \hline
Druck & stand. Atmosph"are (atm) & $1\: atm = 101325\: Pa$\\
Temperatur & Grad Celsius &$1 {}^o C = T - 273.15  \: K$\\
Energie    & Kalorie (kal) & $1\: cal\: (15 {}^oC) =  4.1855\: J$\\ \hline
\end{tabular}
\ec
Physikalische Konstanten:
\nopagebreak
\bc
\begin{tabular}{|lll|}\hline
Bezeichnung      & Phys. Gr"o\ss e        & Definition \\ \hline
$g = 9.81$       & Fallbeschl. (Erde)     & $m/s^2$\\
%
$\rho$           & Massendichte           & $kg/m^3$\\
%
$\mu $           & Viskosit"at             & $Pa\cdot s = kg/(m\cdot s)$\\
%
$\nu = \mu/\rho$ & spez. o. kinemat. Viskosit"at    & $m^2/s$\\
%
$\eta$       & mech. Diffusionskoeffizient  & $m^2/s$\\
%
$\kappa$         & W"armeleitf"ahigkeit    &$W/(K\cdot m) = kg \cdot m/(K \cdot s^3)$\\
%
$c$              & spez. W"armekapazit"at        & $J/(K\cdot kg) = m^2/K\cdot s^2)$ \\
%
$\lambda = \kappa/(\rho \cdot c) $          & spez. W"armediffusionskoeffizient
& $m^2/s$ \\
%
$r$       & spezif. W"armequelldichte   & $J/(s \cdot kg)$ \\
%
$h$       & W"armedurchgangskoeffizient & $J/(m^2\cdot s \cdot K)$\\
%
$\beta$      & W"armeleitungskoeffizient & $1/K$\\
\hline
\end{tabular}
\ec
%
Dimensionslose Gr"o\ss en:
($L\,, \: U$ charakteristische L"ange und Geschwindigkeit, $\Delta T$
charakteristische Temperaturdifferenz)
%
\bc
\begin{tabular}{|ll|}\hline
Gr"o\ss e     & Definition \\ \hline
Reynolds-Zahl & $Re = U\cdot L/\nu $\\
Peclet-Zahl   & $Pe = U \cdot L/\lambda $ \\
Grashoff-Zahl  & $Gr = g \cdot \beta \cdot \Delta T \cdot L^3/\nu ^2$\\
Prandl-Zahl   & $Pr = \nu /\lambda $ \\
Schmidt-Zahl  & $Sc = \nu /\eta $\\
Rayleigh-Zahl & $Ra = Gr\cdot Pr$\\
Froude-Zahl   & $Fr = (Re)^2/Gr$\\
\hline
\end{tabular}
\ec
%%%%%%%%%%%%%%%%%%%%%%%%%%%%%%%%%%%%%%55
{\bf Beispiel 7} Instation"arer Fall
$g = 9.81, \; \beta = 0.21\cdot 10^{-3}; \; \Delta T = 40, \; L = 3$,\\
$\nu = \lambda = 1.49 \cdot 10^{-2}$ bei $R_a = 10^4$;\\
$\nu = \lambda = 4.71\cdot 10^{-3}$ bei $R_a = 10^5$;\\
Station"arer Fall: $R_a = 10^3, \; Pr = 1$;
\newpage
\begin{equation}\label{e1107.7}
\fbox{$
\ba{.}{lll}
M_{\N,\N}W_{\N} - K_{\N,\M}Z_{\M}
&=& \dis K_{\N,\R}Z_{\R} - [S]_{\N}[Z]_{\Gamma}\\[2ex]
%
[K + P(Z_{\N})]_{\M,\N}W_{\N} - G_r[C]_{\M,\N}[T]_{\N}
&=& \dis [S]_{\M}[W]_{\Gamma}\\[2ex]
%
[K + P_r P(Z_{\N})]_{\N,\N}T_{\N}
&=& \dis [S]_{\N}[T]_{\Gamma}
\ea{.}
$}\;.
\end{equation}
\begin{equation}\label{e1107.8}
\fbox{$
\ba{.}{lll}
M_{\N,\N} & - K_{\N,\M} & O_{\N,\N}\\
%
K_{\M,\N}         & \epsilon\,M_{\M,\M}   & - G_r[C]_{\M,\N}\\
O_{\N,\N} & O_{\N,\M}   & K_{\N,\N}
\ea{]}
\ba{[}{l}
 W_{\N}\\Z_{\M}\\T_{\N}
\ea{]}
= \ \ba{[}{l}
K_{\N,\R}Z_{\R} - [S]_{\N}[Z]_{\Gamma}\\
- P(Z_{\N})]_{\M,\N}W_{\N} + [S]_{\M}[W]_{\Gamma}\\
-P_r P(Z_{\N})_{\N,\N}T_{\N} + [S]_{\N}[T]_{\Gamma}
\ea{]}
$}\;.
\end{equation}
\begin{equation}\label{e1107.9}
\fbox{$
\ba{.}{l}
\nabla \Phi(W,Z,T) = \\[2ex]
\ba{[}{lll}
M_{\N,\N} & - K_{\N,\M} & O_{\N,\N}\\[0ex]
%
[K + P(Z_{\N})]_{\M,\N}         &
\nabla_Z[P(Z_{\N})_{\M,\N}W_{\N}]_{\M,\M}  &
 - G_r[C]_{\M,\N}\\
O_{\N,\N}  & P_r\nabla_Z[P(Z_{\N})_{\M,\N}W_{\N}]_{\N,\M}& [K + P_rP(Z_{\N})]_{\N,\N}
\ea{]}
\ea{.}
$}\;.
\end{equation}

\end{document}

