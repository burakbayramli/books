% memman.tex    edition 6,  Memoir class user manual
%                Author: Peter Wilson
%                Copyright 2001, 2002, 2003, 2004 Peter R. wilson

\documentclass[10pt,letterpaper]{memoir}
%%%%\usepackage{memfixa}
%%%%\usepackage{memfixc}
%%\usepackage{makeidx}
\usepackage{layouts}[2001/04/29]
%%\usepackage{verbatim}
\usepackage{url}
\usepackage{fixltx2e}
\usepackage{alltt}
\usepackage{graphicx}
\makeindex

%%%%%%\usepackage{amsmath}

\ifpdf
  \pdfoutput=1
  \usepackage[plainpages=false,pdfpagelabels,bookmarksnumbered]{hyperref}
  \usepackage{memhfixc}
\fi

%\showindexmarktrue

\newcommand{\theclass}{memoir}

\providecommand{\tx}{TeX}
\providecommand{\ltx}{LaTeX}


\makeatletter

%%% Print and Index macros
\newcommand{\Ppstyle}[1]{\textsl{#1}}
\newcommand{\pstyle}[1]{\Ppstyle{#1}%
            \index{#1 pages?\Ppstyle{#1} (pagestyle)}%
            \index{pagestyle!#1?\Ppstyle{#1}}}            % pagestyle
\newcommand{\Pcstyle}[1]{\textsl{#1}}
\newcommand{\cstyle}[1]{\Pcstyle{#1}%
            \index{#1 chaps?\Pcstyle{#1} (chapterstyle)}%
            \index{chapterstyle!#1?\Pcstyle{#1}}}          % chapterstyle
\newcommand{\Pclass}[1]{\textsf{#1}}
\newcommand{\Lclass}[1]{\Pclass{#1}%
            \index{#1 class?\Pclass{#1} (class)}%
            \index{class!#1?\Pclass{#1}}}                % class name
\newcommand{\Ppack}[1]{\textsf{#1}}
\newcommand{\Lpack}[1]{\Ppack{#1}%
            \index{#1 pack?\Ppack{#1} (package)}%
            \index{package!#1?\Ppack{#1}}}              % package name
\newcommand{\Popt}[1]{\textsf{#1}}
\newcommand{\Lopt}[1]{\Popt{#1}%
            \index{#1 opt?\Popt{#1} (option)}%
            \index{option!#1?\Popt{#1}}}               % option name
\newcommand{\Ie}[1]{\texttt{#1}\index{#1 env?\texttt{#1} (environment)}%
                               \index{environment!#1?\texttt{#1}}}
\newcommand{\Icn}[1]{\texttt{#1}\index{#1 cou?\texttt{#1} (counter)}%
                                \index{counter!#1?\texttt{#1}}}
\newcommand{\Itt}[1]{\texttt{#1}\index{#1tt?\texttt{#1}}}

% print and index an \if... \piif{if...} or friend (e.g., \else)
\newcommand{\piif}[1]{\cs{#1}\index{#1?\cs{#1}}}

%%%\newcommand{\seealso}[2]{\emph{see also} #1}

\newcommand{\listofx}{`List of\ldots'}

% print a begin environment
\newcommand{\senv}[1]{\texttt{\bs begin\{#1\}}}

% print an end environment
\newcommand{\eenv}[1]{\texttt{\bs end\{#1\}}}

% print a file name
\newcommand{\file}[1]{\texttt{#1}}

% print and index a file name
\newcommand{\ixfile}[1]{\file{#1}%
            \index{#1 file?\file{#1} (file)}%
            \index{file!#1?\file{#1}}}

% print & index CTT
\newcommand{\ctt}{\textsc{ctt}}
\newcommand{\pictt}{\ctt\index{CTT?\ctt}}

% index marking
\newcommand{\idxmark}[1]{#1\markboth{#1}{#1}}

\newcommand{\foredge}{foredge}
\newlength{\pwlayi}\setlength{\pwlayi}{0.45\textwidth} % 
\newlength{\pwlayii}\setlength{\pwlayii}{0.45\pwlayi}

%% definition of \meta is taken from doc.dtx
\begingroup
\obeyspaces%
\catcode`\^^M\active%
\gdef\meta{\begingroup\obeyspaces\catcode`\^^M\active%
\let^^M\do@space\let \do@space%
\def\-{\egroup\discretionary{-}{}{}\hbox\bgroup\itshape}%
\m@ta}%
\endgroup
\def\m@ta#1{\leavevmode\hbox\bgroup$\langle$\itshape#1\/$\rangle$\egroup
    \endgroup}
\def\do@space{\egroup\space
    \hbox\bgroup\itshape\futurelet\next\sp@ce}
\def\sp@ce{\ifx\next\do@space\expandafter\sp@@ce\fi}
\def\sp@@ce#1{\futurelet\next\sp@ce}
%% end of \meta and supports


%% The definition of \MakeShortVerb & \DeleteShortVerb
%% are now in the memoir class.

%% The macros \cmd, \cs, \marg, \oarg, \parg are taken from ltxdoc.dtx
% \cmd{\fred} typesets \fred
\def\cmd#1{\cs{\expandafter\cmd@to@cs\string#1}}
\def\cmd@to@cs#1#2{\char\number`#2\relax}
% \cs{fred} typesets \fred
\DeclareRobustCommand{\cs}[1]{\texttt{\char`\\#1}}
% \marg{arg} typesets {<arg>}
\providecommand{\marg}[1]{%
  {\ttfamily\char`\{}\meta{#1}{\ttfamily\char`\}}}
% \oarg{arg} typesets [<arg>]
\providecommand{\oarg}[1]{%
  {\ttfamily\char`\[}\meta{#1}{\ttfamily\char`\]}}
% \parg{x,y} typesets (<x,y>)
\providecommand{\parg}[1]{%
  {\ttfamily\char`\(}\meta{#1}{\ttfamily\char`\)}}

\def\bs{\texttt{\char`\\}}

%%% macro \cmd from Heiko Oberdiek CTT 2001/05/26 (prints and indexes a command)
\newcommand*{\cmdprint}[1]{\texttt{\string#1}}
\def\cmd#1{\cmdprint{#1}%
  \index{\expandafter\@gobble\string#1?\string\cmdprint{\string#1}}}

% print and index \\!
\newcommand{\pixslashbang}{\cmdprint{\\!}\index{"\"\"!?\string\cmdprint{\\}\texttt{"!}}}

% print and index \!
\newcommand{\pixabang}{\cmdprint{\!}\index{"!?\string\cs{}\texttt{"!}}}


\newcommand{\lnc}[1]{\cmdprint{#1}%
  \index{\expandafter\@gobble\string#1len?\string\cmdprint{\string#1} (length)}%
  \index{length!\expandafter\@gobble\string#1len?\string\cmdprint{\string#1}}}

%%%%%% stuff for new LaTeX code environment

% \zeroseps sets list before/after skips to minimum values
\newcommand{\@zeroseps}{\setlength{\topsep}{\z@}
                        \setlength{\partopsep}{\z@}
                        \setlength{\parskip}{\z@}}
\newlength{\gparindent} \setlength{\gparindent}{\parindent}
% now we can do the new environment. This has no extra before/after spacing.
\newenvironment{lcode}{\@zeroseps
  \renewcommand{\verbatim@startline}{\verbatim@line{\hskip\gparindent}}
  \small\setlength{\baselineskip}{\onelineskip}\verbatim}%
  {\endverbatim
   \vspace{-\baselineskip}%
   \noindent
}

%%%%% LaTeX syntax
\newenvironment{syntax}{\begin{center}
                        \begin{tabular}{|p{0.9\linewidth}|} \hline}%
                       {\hline
                        \end{tabular}
                        \end{center}}

\providecommand{\Note}{\textbf{NOTE:}}



%%%%%%%%%%%%%%%%%%%%%%%%%%%%%%%%%%%%%%%%%%%%%%%%

% Save file as: DROP.STY               Source: FILESERV@SHSU.BITNET  
%%%%%%%%%%%%%%%%%%%%%%%%%%%%%%%%%%%%%%%%%%%%%%%%%%%%%%%%%%%%%%%%%%%%%%%%%%%
%  DROP.DOC <February 17, 1988>
%  Macro for dropping and enlarging the first letter(s) of a paragraph.
%%%%%%%%%%%%%%%%%%%%%%%%%%%%%%%%%%%%%%%%%%%%%%%%%%%%%%%%%%%%%%%%%%%%%%%%%%%
%
%  Macro written by David G. Cantor, and published Fri, 12 Feb 88, in
%  TeXhax, 1988 #16.  
%  Internet:  dgc@math.ucla.edu
%  UUCP:      ...!{ihnp4, randvax, sdcrdcf, ucbvax}!ucla-cs!dgc
%
%  Modified for use with LaTeX by Dominik Wujastyk, February 17, 1988
%  Internet:   dow@wjh12.harvard.edu
%  Bitnet:     dow@harvunxw.bitnet
%%%%%%%%%%%%%%%%%%%%%%%%%%%%%%%%%%%%%%%%%%%%%%%%%%%%%%%%%%%%%%%%%%%%%%%%%%%
%
%  This LaTeX macro is for dropping and enlarging the first letter(s) of a
%  paragraph.  The argument may be one or more letters.
%
%  Here is an example of its usage:
%
%  \documentstyle[drop]{article}
%  \begin{document}
%    \drop{IN} THE beginning God created the heaven and the earth.  Now the
%    earth was unformed and void, and darkness was upon the face of the
%    deep; and the spirit of God hovered over the face of the waters.
%  \end{document}
%
%  Which will produce something along these lines:
%
%  I I\  I THE beginning God  created the heaven and  the earth.
%  I I \ I Now the earth was unformed and void, and darkness was
%  I I  \I upon the face of the deep; and the spirit of God hov-
%  ered over the face of the waters.
%
%  In the first instance the macro will pause during LaTeX processing and 
%  ask you for the font you wish to use for you drop capital.  When you
%  have something that looks good, then comment out box one in DROP.STY,
%  and comment in box two, replacing "cmr10 scaled \magstep5" with the font
%  of your choice.
%
%  In my opinion (DW) there are no fonts available in the standard
%  TeX/LaTeX set that are ideal for this use, unless you go down to 9pt or 
%  8pt for your text face, and this is too small.  If you have Metafont you
%  should consider generating a cmr17 font at a magstep of two (about 25pt)
%  or three (about 30pt), or even more, depending on the point size of your
%  main text.  Why not go the whole hog and design some really fancy 
%  capitals from scratch!
%
%%%%%%%%%%%%%%%%%%%%% BOX ONE %%%%%%%%%%%%%%%%%%%%%%%%%
%\typein[\dropinitialfont]{Font for Dropped initial:}  %
%\font\largefont \dropinitialfont                      %
%%%%%%%%%%%%%%%%%%%%%%%%%%%%%%%%%%%%%%%%%%%%%%%%%%%%%%%
%
%%%%%%%%%%%%%%%%%%%%% BOX TWO %%%%%%%%%%%%%%%%%%%%%%%%%
%\font\largefont= cmr10 scaled \magstep5              %
%\font\largefont= cmbx10 scaled \magstep5              %
%\font\largefont= cmbx17 scaled \magstep3              %
\font\largefont= cmr17 scaled \magstep5              %
%%%%%%%%%%%%%%%%%%%%%%%%%%%%%%%%%%%%%%%%%%%%%%%%%%%%%%%

\def\drop#1#2{{\noindent
    \setbox0\hbox{\largefont #1}\setbox1\hbox{#2}\setbox2\hbox{(}%
    \count0=\ht0\advance\count0 by\dp0\count1\baselineskip
    \advance\count0 by-\ht1\advance\count0by\ht2
    \dimen1=.5ex\advance\count0by\dimen1\divide\count0 by\count1
    \advance\count0 by1\dimen0\wd0
    \advance\dimen0 by.25em\dimen1=\ht0\advance\dimen1 by-\ht1
    \global\hangindent\dimen0\global\hangafter-\count0
    \hskip-\dimen0\setbox0\hbox to\dimen0{\raise-\dimen1\box0\hss}%
    \dp0=0in\ht0=0in\box0}#2}

% end of DROP.STY
%%%%%%%%%%%%%%%%%%%%%%%%%%%%%%%%%%%%%%%%%%%%%%%%

\newcommand{\versal}[1]{{\noindent
    \setbox0\hbox{\largefont #1}%
    \count0=\ht0                   % height of versal
    \count1=\baselineskip          % baselineskip
    \divide\count0 by \count1      % versal height/baselineskip
    \dimen1 = \count0\baselineskip % distance to drop versal
    \advance\count0 by 1\relax     % no of indented lines
    \dimen0=\wd0                   % width of versal
    \global\hangindent\dimen0      % set indentation distance
    \global\hangafter-\count0      % set no of indented lines
    \hskip-\dimen0\setbox0\hbox to\dimen0{\raise-\dimen1\box0\hss}%
    \dp0=0in\ht0=0in\box0}}

\makeatother

\MakeShortVerb{\|}
% \DeleteShortVerb{\|}


%%% need more space for ToC page numbers
\setpnumwidth{2.55em}
\setrmarg{3.55em}

%%% need more space for ToC section numbers
\cftsetindents{section}{1.5em}{3.0em}
\cftsetindents{subsection}{4.5em}{3.9em}
\cftsetindents{subsubsection}{8.4em}{4.8em}
\cftsetindents{paragraph}{10.7em}{5.7em}
\cftsetindents{subparagraph}{12.7em}{6.7em}

%%% need more space for LoF numbers
\cftsetindents{figure}{0em}{3.0em}

%%% and do the same for the LoT
\cftsetindents{table}{0em}{3.0em}

%%% set up the page layout
\settrimmedsize{11in}{210mm}{*}
\setlength{\trimtop}{0pt}
\setlength{\trimedge}{\stockwidth}
\addtolength{\trimedge}{-\paperwidth}
\settypeblocksize{7.75in}{33pc}{*}
\setulmargins{4cm}{*}{*}
\setlrmargins{1.25in}{*}{*}
\setmarginnotes{17pt}{51pt}{\onelineskip}
\setheadfoot{\onelineskip}{2\onelineskip}
\setheaderspaces{*}{2\onelineskip}{*}
\checkandfixthelayout


\bibliographystyle{alpha}

\newcommand{\prtoc}{ToC}             % print ToC
\newcommand{\prlof}{LoF}
\newcommand{\prlot}{LoT}
\newcommand{\ixtoc}{\index{ToC}}     % index ToC
\newcommand{\ixlof}{\index{LoF}}
\newcommand{\ixlot}{\index{LoT}}
\newcommand{\toc}{\prtoc\ixtoc}      % print & index ToC
\newcommand{\lof}{\prlof\index{LoF}}
\newcommand{\lot}{\prlot\index{LoT}}

\newcommand{\noidxnum}[1]{}

%%% subfigures and subtables
\newsubfloat{figure}
\newsubfloat{table}

%% end preamble
%%%%%%%%%%%%%%%%%%%%%%%%%%%%%%%%%%%%%%%%%%%%%%%%%%%%%%%
\begin{document}
%%%%%%%%%%%%%%%%%%%%%%%%%%%%%%%%%%%%%%%%%%%%%%%%%%%%%%%

%% list subfigures
\setcounter{lofdepth}{2}

\index{table of contents|see{ToC}}
\index{list!of figures|see{LoF}}
\index{figure!list of|see{LoF}}
\index{list!of tables|see{LoT}}
\index{table!list of|see{LoT}}
\index{marginal note|see{marginalia}}
\index{footnote!in title|see{thanks}}
\index{counter|noidxnum}
\index{environment|noidxnum}
\index{length|noidxnum}
\index{list|noidxnum}
\index{illustration|seealso{float, figure}}
\index{figure|seealso{float}}
\index{file|noidxnum}
\index{table|seealso{float}}
\index{chapter!style|see{chapterstyle}}
\index{chapter!heading|see{heading}}
\index{page!style|see{pagestyle}}
\index{part!heading|see{heading}}
\index{subfloat|noidxnum}

%%% ToC down to subsections
\settocdepth{subsection}
\frontmatter
\pagestyle{empty}


% half-title page
\vspace*{\fill}
\begin{adjustwidth}{1in}{1in}
\begin{flushleft}
\HUGE\sffamily The
\end{flushleft}
\begin{center}
\HUGE\sffamily  Memoir
\end{center}
\begin{flushright}
\HUGE\sffamily  Class
\end{flushright}
\end{adjustwidth}
\vspace*{\fill}
\cleardoublepage

% title page
\vspace*{\fill}
\begin{center}
\HUGE\textsf{The Memoir Class}\par
\end{center}
\begin{center}
\LARGE\textsf{for}\par
\end{center}
\begin{center}
\HUGE\textsf{Configurable Typesetting}\par
\end{center}

\begin{center}
\Huge\textsf{User Guide}\par
\end{center}
\begin{center}
\LARGE\textsf{Peter Wilson}\par
\end{center}
\vspace*{\fill}
\begin{center}
\textrm{\normalsize T\kern-0.2em H\kern-0.4em P} \\
\textsf{The Herries Press}
\end{center}
\clearpage

% copyright page
\begingroup
\footnotesize
\setlength{\parindent}{0pt}
\setlength{\parskip}{\baselineskip}
%%\ttfamily
\textcopyright{} 2001, 2002, 2003, 2004 Peter R. Wilson \\
All rights reserved

The Herries Press, Normandy Park, WA.

Printed in the World 

The paper used in this publication may meet the minimum requirements
of the American National Standard for Information 
Sciences --- Permanence of Paper for Printed Library Materials, 
ANSI Z39.48--1984.

\begin{center}
10 09 08 07 06 05 04 03 02 01\hspace{2em}15 14 13 12 11 10 9       
\end{center}
\begin{center}
\begin{tabular}{ll}
First edition:                        & 3 June 2001 \\
Second impression, with corrections:    & 2 July 2001 \\
Second edition:                       & 14 July 2001 \\
Second impression, with corrections:    & 3 August 2001 \\
Third impression, with minor additions: & 31 August 2001 \\
Third edition:                        & 17 November 2001 \\
Fourth edition:                       & 16 March 2002 \\
Fifth edition:                        & 10 August 2002 \\
Sixth edition:                        & 31 January 2004 \\
\end{tabular}
\end{center}

\endgroup

\clearpage
\vspace*{\fill}
\begin{quote}
\textbf{memoir,} \textit{n.} a written record set down as material
  for a history or biography: 
  a biographical sketch:
  a record of some study investigated by the writer:
  (in \textit{pl.}) the transactions of a society.
  [Fr. \textit{m\'{e}moire} --- L. \textit{memoria,} memory ---
   \textit{memor}, mindful.] \\[0.5\baselineskip]
  \hspace*{\fill} 
      \textit{Chambers Twentieth Century Dictionary, New Edition}, 1972.
\end{quote}

\vspace{2\baselineskip}

\begin{quote}
\textbf{memoir,} \textit{n.} [Fr. \textit{m\'{e}moire,} masc., a memorandum,
    memoir, fem., memory $<$ L. \textit{memoria,} \textsc{memory}]
  \hspace{1ex} \textbf{1.} a biography or biographical notice, 
      usually written by a relative or personal friend of the subject 
  \hspace{1ex} \textbf{2.} [\textit{pl.}] an autobiography, 
      usually a full or highly personal account
  \hspace{1ex} \textbf{3.} [\textit{pl.}] a report or record of 
      important events based on the writer's personal observation, 
      special knowledge, etc.
  \hspace{1ex} \textbf{4.} a report or record of a scholarly 
      investigation, scientific study, etc.
  \hspace{1ex} \textbf{5.} [\textit{pl.}] the record of the proceedings
      of a learned society \\[0.5\baselineskip]
  \hspace*{\fill} \textit{Webster's New World Dictionary, Second College Edition}.
\end{quote}

%\vspace{2\baselineskip}
%
%\begin{quote}
%\textbf{memoir,} \textit{n.} a fiction designed to flatter the subject 
%  and to impress the reader. \\[0.5\baselineskip]
%\hspace*{\fill} With apologies to Reuben Thomas and Ambrose Bierce
%\end{quote}

\vspace*{\fill}

\cleardoublepage

% ToC, etc
\pagenumbering{roman}
\pagestyle{headings}
%%%%\pagestyle{Ruled}

\tableofcontents
\clearpage
\listoffigures
\clearpage
\listoftables

%\chapter{Foreword}
\chapter{Preface}

    From personal experience and also from lurking on the \url{comp.text.tex}
newsgroup the major problems with using LaTeX are related to document
design. Some years ago most questions on \texttt{ctt} were answered by
someone providing a piece of code that solved a particular problem, and
again and again. More recently these questions are answered along the
lines of `Use the ---------{} package', and again and again.

    I have used many of the more common of these packages but my filing system
is not always well ordered and I tend to mislay the various user manuals,
even for the packages I have written. The \Pclass{memoir} class is an attempt
to integrate some of the more design-related packages with the LaTeX
\Pclass{book} class. I chose the \Pclass{book} class as the \Pclass{report} class
is virtually identical to \Pclass{book}, except that \Pclass{book} does
not have an \Ie{abstract} environment while \Pclass{report} does; however it is 
easy to fake an \Ie{abstract} if it is needed. With a little bit of tweaking,
\Pclass{book} class documents can be made to look just like \Pclass{article}
class documents, and the \Pclass{memoir} class is designed with tweaking very
much in mind.

    The \Pclass{memoir} class effectively incorporates the facilties that
are usually accessed by using external packages. In most cases the class
code is new code reimplementing package functionalities. The exceptions
tend to be where I have cut and pasted code from some of my packages.
I could not have written the \Pclass{memoir} class without the excellent 
work presented by the implementors of LaTeX and its many packages.

    Apart from packages that I happen to have written I have gained many
ideas from the other packages listed in the \bibname. One way or another
their authors have all contributed, albeit unknowingly. 
The participants in the
\url{comp.text.tex} newsgroup have also provided valuable input, partly
by questioning how to do something in LaTeX, and partly by providing
answers. It is a friendly and educational forum.

{\raggedleft{\scshape Peter Wilson} \\ Seattle, WA \\ June 2001\par}

\chapter{Introduction to the first edition}

    This is not a guide to the general use of LaTeX but rather concentrates
on where the \index{class}\Lclass{memoir} class differs from the standard LaTeX
\Lclass{book} and \Lclass{report} classes. There are other sources that deal with LaTeX in 
general, some of which are listed in the \bibname. Lamport~\cite{LAMPORT94}
is of course the original user manual for LaTeX, while the Companion
series, for example~\cite{GOOSSENS94}, go into further details and auxiliary
programs. The Comprehensive TeX Archive Network (CTAN) is a valuable source
of free information and the LaTeX system itself. For general questions see
the FAQ~\cite{FAQ} which also has pointers to information sources. Among
these are \textit{The Not So Short Introduction to LaTeX2e}~\cite{LSHORT},
Keith Reckdahl's \textit{Using imported graphics in LaTeX2e}~\cite{EPSLATEX}
and Piet van Oostrum's \textit{Page layout in LaTeX}~\cite{FANCYHDR}.
The question of how to use different fonts with LaTeX is left strictly alone;
Alan Hoenig's book~\cite{HOENIG98} is the best guide to this that I know of.


    The first part of the manual discusses briefly some aspects of book
design and typography, independently of the means of typesetting. Among
the several books on the subject listed in the \bibname{} I prefer
Bringhurst's \textit{The Elements of Typographic Style}~\cite{BRINGHURST92}.
I have used the LaTeX \Lclass{book} design, which is the default
\Lclass{memoir} class style, in typesetting Part~\ref{part:art}.

    The second part then goes into some detail on how the \Lclass{memoir}
class can be used to implement a particular design.

    With two exceptions, the \Lclass{memoir} class has all the capabilities
of the standard LaTeX \Lclass{book} class. These exceptions are:
\begin{itemize}
\item The old LaTeX v2.09 font commands --- 
      \cmd{\rm} (roman), 
      \cmd{\tt} (\texttt{typewriter}), 
      \cmd{\sf} (\textsf{sans}),
      \cmd{\bf} (\textbf{bold}), 
      \cmd{\sl} (\textsl{slanted}), 
      \cmd{\it} (\textit{italic}),
      and \cmd{\sc} (\textsc{small caps}) ---
      are not supported and will give error messages if used.
 
      The \cmd{\em} (\emph{emphasis}) command is supported but gives 
a warning message if used.

\item There are no commands for making titles. This is because title pages are
      usually designed individually for each book. The 
     \index{package}\Lpack{titling}
      package~\cite{TITLING} can be used if you want the titling commands.
      The package
      provides extended titling facilities when compared to those in the
      standard LaTeX classes.

\end{itemize}
I hope that, apart from these, the class supports everything that the 
\Lclass{book} class provides.

    The major extra functions provided by the class include:
\begin{itemize}
\item Font sizes for the main text of 9, 10, 11, 12 and 14pt.
\item A reasonably intuitive means of setting the page, text and margin sizes.
\item Page trimming marks.
\item If really required, typesetting as though in the olden typewriter days
      (double spaced, raggedright, no hyphenation, typewriter font).
\item Configurable sectional headings, with several predefined styles for
      chapter headings and simple methods for defining new ones.
\item `Anonymous' section breaks (e.g., a blank or decorated line or two).
\item Configurable page headers and footers, with several predefined styles
      and simple methods for defining new ones.
\item Configurable captions, with several predefined captioning styles and
      methods for defining new ones.
\item Ability to define new `List of...' and new floats with their accompanying
      captions and `List of...'.
\item Control over whether the `List of...', bibliography, index, etc., appear
      in the Table of Contents.
\item Support for epigraphs.
\end{itemize}
Also, along the way you will find other more minor but still useful things.

    As Part~\ref{part:practice} progresses I demonstrate some of the changes
that the \Lclass{memoir} class lets you easily make to the normal LaTeX
page and titling style.

\section{Type conventions}

    The following conventions are used in the manual:
\begin{itemize}
\item \Pclass{The names of LaTeX classes\index{class} and 
              packages\index{package} are typeset in this font.}
\item \Popt{Class options\index{option} are typeset in this font.}
\item \Ppstyle{The names of chapterstyles\index{chapterstyle} and 
               pagestyles\index{pagestyle} are typeset in this font.}
\item \texttt{LaTeX code is typeset in this font.}
\end{itemize}

\section{Caveats}

    At the moment both this manual and the class code are in a beta state.
That is, they hopefully serve the purposes they are intended for, but 
it is probable that there are errors, poor explanations and missing
elements. So, be warned that I'm sure that there will be further releases
and these may not be entirely compatible with the current release.

    That said, I will be grateful for any constructive comments that 
anyone\footnote{I have received valuable comments from
Javier Bezos (\url{jbezos@wanadoo.es}), 
Sven Bovin (\url{sven.bovin@chem.kuleuven.ac.be}),
Scott Pakin (\url{pakin@uiuc.edu}),
and
Paul Stanley (\url{pstanley@essexcourt-chambers.co.uk})
.}
may have, especially regarding errors, mispeaking, and desireable 
enhancements. I can be reached either by posting to \url{comp.text.tex}
or more directly at \url{peter.r.wilson@boeing.com}.

    TeX was designed principally for typesetting documents containing a 
lot of mathematics. In such works the mathematics breaks up the flow
of the text on the page, and the vertical space required for displayed
mathematics is arbitrary. Most non-technical books are typeset on a fixed
grid as they do not have arbitrary insertions into the text.

    TeX is designed to handle arbitrary sized inserts in an elegant manner,
and does this by allowing vertical spaces to stretch and shrink a little
so that the actual height of the typeblock is constant. Therefore LaTeX, being
built on top of TeX, does not typeset on a fixed grid, and it would be a 
very major task to try and make it do so; this has been tried but as far as 
I know nobody has been successful.

    The manual includes many unbreakable names of LaTeX commands,
some of which stick out into the margin. The way of getting rid of these
is to rewrite the text so that they don't come at the end of a typeset
line. This is tedious and I haven't done it because I expect the manual
to be revised and that would throw off any hand tweaking done now.

\chapter{Introduction to the second edition}

    Since the first edition of the manual was published the \Lclass{memoir}
class has undergone some changes and extensions. The changes, to remove
some unfortunate errors, are upwards compatible. The extensions, by 
their nature, are not upward compatible.

    The main extensions and changes include:
\begin{itemize}
\item A \index{option}\Lopt{subfigure} option to counteract an unfortunate 
      interplay\footnote{Discovered by Ignasi Furi\'{o} Caldentey 
      (\url{ignasi@ipc4.uib.es}).}
      if the \Lpack{subfigure} package is used with the class.

\item An \Lopt{article} option so that \Lclass{article} class typesetting
      may be simulated.

\item Incorporation of the essential code from the \Lpack{titling}
      package~\cite{TITLING} (to support the \Lopt{article} option).

\item Incorporation of the essential code from the \Lpack{abstract}
      package~\cite{ABSTRACT} (to support the \Lopt{article} option).

\end{itemize}

    The description of how to modify the \prtoc, \prlof{} and \prlot{} headings
in the first edition was completely wrong, as was the corresponding
description of the \cmd{\newlistof} macro. No noticeable
changes have been made to the class code but the descriptions now
reflect reality. I must have been a few bricks short of a full load when
I wrote the original.

    There are other more minor changes and extensions\footnote{With thanks
to, among others, Kevin Lin (\url{kevinlin@runtop.com.tw}) and
Adriano Pascoletti (\url{pascolet@dimi.uniud.it}).} 
which you may find if you recall the first edition.

    There was no mention of typesetting verse in the first edition,
although the class does provide the normal LaTeX \Ie{verse}
environment. A poem
is usually individually typeset as the appearance often has an
effect on the emotional response when reading it. The \Lpack{verse}
package~\cite{VERSE} may be useful when typesetting poetry.

\chapter{Introduction to the third edition}

    Since the second  edition of the manual was published the \Lclass{memoir}
class has been upgraded from beta code to a production release. Extensions
have been made to both the class and this manual.

    The main extensions and changes include:
\begin{itemize}
\item An \Lopt{openleft} option to enable chapters to start
      on verso pages.

\item Incorporation of the essential code from the \Lpack{verse}
      package~\cite{VERSE} for more flexibility when typesetting
      poetry.

\item Replacement of the macro called \cmd{\makepshook} by
      \cmd{\makepsmarks}. Note that this is a non-upward compatible
      change.

\item The first part of the manual has been reorganised and
      extended, principally
      by providing more typesetting examples.

\item As usual, minor glitches have been removed from both the code
      and the manual; each revision, of course, eliminates the gap between
      advertisement and reality.

\end{itemize}


    There are other more minor changes and extensions\footnote{With thanks
to, among others, Ignasi Furi\'{o} Caldentey (\url{ignasi@ipc4.uib.es}),
Daniel Richard G. (\url{skunk@mit.edu}) and
Vladimir Ivanovic (\url{vladimir@acm.org}).} 
which you may find if you recall the second edition.

\chapter{Introduction to the fourth edition}

    Since the third edition of the manual was published the \Lclass{memoir}
class has been upgraded from version~1.0 to version~1.1. Modifications 
have been made to both the class and this manual.

    The main extensions and changes include:
\begin{itemize}
\item The \Lopt{subfigure} option is no longer required.

\item Subfloats have been added to the class. Steve Cochran kindly gave
      permission for me to use some of his \Lpack{subfigure} package
      code for this.

\item Some packages still use the old, deprecated LaTeX version~2.09
      font commands. I have reluctantly introduced an option
      to enable the old, deprecated font commands to be used.

\item The class now works harmoniously with the \Lpack{natbib}
      package~\cite{NATBIB}.

\item As usual, minor glitches have been removed from both the code
      and the manual; each revision hopefully eliminates the gap between
      advertisement and reality.

\end{itemize}


    There are other more minor changes and extensions\footnote{With thanks
to, among others, 
William Adams (\url{WillAdams@aol.com})
Ignasi Furi\'{o} Caldentey (\url{ignasi@ipc4.uib.es}),
Steven Douglas Cochran (\url{sdc+@cs.cmu.edu}),
Henrik Holm (\url{henrik@tele.ntnu.no}),
and Rolf Niepraschk (\url{niepraschk@ptb.de}).
}
which you may find if you have used earlier editions.

\chapter{Introduction to the fifth edition}

    Since the fourth edition of the manual was published the \Lclass{memoir}
class has been upgraded from version~1.1 to version~1.2. Modifications 
have been made to both the class and this manual.

    The main extensions and changes include:
\begin{itemize}
\item The size options have been extended\footnote{At the
      request of Vittorio De Martino whose children use LaTeX
      for their school projects.}
      to include a \Lopt{17pt} option.

\item A few font sizes corresponding to the font size commands (e.g., \verb?\Large?)
      have been changed to give regular steps between sizes.

\item Boxes that can break over pages and/or contain verbatim text.

\item Several ways of dealing with verbatim text, including footnotes
      that can contain verbatim text.

\item Some control over the typesetting of footnotes. Unfortunately
      this necessitated some changes in the methods for styling
      thanks notes.

\item Comment environments.

\item Convenient methods for file input and output.

\item Additional \verb?\provide...? commands.

\item The \Ie{description} environment has been modified to match
      the appearance of the standard environment. The original
      \Lclass{memoir} form is still available as the \Ie{blockdescription}
      environment.

\item A new optional argument has been added to the \cmd{\chapter} and 
      \cmd{\chapter*} commands for setting header texts.

\end{itemize}

     As usual, minor glitches have been removed from both the code
and the manual. Hopefully, new ones have not been introduced.

\chapter{Introduction to the sixth edition}

    Since the fifth edition of the manual was published the \Lclass{memoir}
class has been upgraded from version~1.2 to version~1.6. 
Many new functions have
been added to the class and this manual has been updated to reflect all
the additions.

    The main extensions and changes include:
\begin{itemize}

\item Major extensions for typesetting arrays and tabulars, including
      continuous tabulars and automatic tabulation.

\item Major extensions to footnote styles and the ability to have
      multiple series of footnotes.

\item Major extensions for indexing, including one column and multiple indexes.

\item Major extensions to crop marks. 

\item \verb?\tableofcontents? and friends can be used multiple times.

\item Section titles (as well as numbers) can be referenced.

\item Sheet numbers and access to the numbers of the last sheet and last page.

\item Various methods for formatting numbers.

\item Better cooperation with the \Lpack{chapterbib} and \Lpack{natbib} packages.

\end{itemize}

     There are many more minor additions.
     As usual, glitches have been removed from both the code
and the manual. Hopefully, new ones have not been introduced.


     Many people have contributed to the \Lclass{memoir} class and this manual
in the forms of code, solutions to problems, suggestions for new functions, 
bringing my attention to errors and infelicities in the code 
and manual, and last but not least in simply being encouraging. 
I am very grateful to the following for all they have done, whether they
knew it or not:
Paul Abrahams,
William Adams,
Donald Arseneau,
Jens Berger,
Karl Berry,
Javier Bezos,
Sven Bovin,
Alan Budden,
Ignasi Furi\'{o} Caldenty,
Ezequiel Mart\'{\i}n C\'{a}mara,
David Carlisle,
Steven Douglas Cochran,
Michael Downes,
Victor Eijkhout,
Danie Els,
Robin Fairbairns,
Simon Fear,
Kai von Fintel,
Daniel Richard G,
Romano Giannetti,
Kathryn Hargreaves,
Sven Hartrumpf,
Florence Henry,
Cartsten Heinz,
Peter Heslin,
Morton H\o{}gholm,
Henrik Holm,
Vladimir Ivanovich,
Stefan Kahrs,
J\o{}gen Larsen,
Kevin Lin,
Matthew Lovell,
Daniel Luecking,
Lars Madsen,
Vittorio De Martino,
Frank Mittelbach,
Rolf Niepraschk,
Patrik Nyman,
Heiko Oberdiek, 
Scott Pakin,
Adriano Pascoletti,
Robert,
Chris Rowley,
Bernd Raichle,
Doug Schenck,
Rainer Sch\"{o}pf,
Paul Stanley,
Reuben Thomas,
Bastiaan Niels Veelo,
Emanuele Vicentini,
J\"{u}rgen Vollmer,
and others.

If I have inadvertently left anyone off the list I apologise, 
and please let me know so that I can correct the omisssion.

    Of course, none of this would have been possible without Donald Knuth's
TeX system and the subsequent development of LaTeX by Leslie Lamport.



%%%%%%%%%%%%%%%%%%%%%%%%%%%%%%%%%
\chapter{Terminology}
%%%%%%%%%%%%%%%%%%%%%%%%%%%%%%%%%%

    Like all professions and trades, typographers and printers have their
specialised vocabulary.

    First there is the question of pages, leaves and sheets. 
The trimmed sheets of paper\index{paper} that make up a book are called 
\emph{leaves}\index{leaf},
and I will call the untrimmed sheets the \emph{stock}\index{stock} material. 
A leaf
has two sides, and a \emph{page}\index{page} is one side of a leaf. 
If you think of a book
being opened flat, then you can see two leaves. The front of the righthand
leaf, is called the \emph{recto}\index{recto} page of that leaf, 
and the side of the
lefthand leaf that you see is called the \emph{verso}\index{verso} page 
of that leaf. 
So, a leaf has a recto and a verso page. Recto pages are the odd-numbered 
pages and verso pages are even-numbered.

   Then there is the question of folios. The typographical term for
the number of a page is \emph{folio}\index{folio}.
This is not to be confused with
the same term as used in `Shakespeare's First Folio' where the reference is
to the height and width of the book, nor to its use in the phrase
`\emph{folio} signature'\index{signature} where the term refers to the 
number of times a printed sheet is folded. 
Not every page in a book has a printed
folio, and there may be pages that do not have a folio at all. Pages with
folios, whether printed or not, form the \emph{pagination}\index{pagination} 
of the book. Pages
that are not counted in the pagination have no folios.

   A \emph{font}\index{font} is a set of characters. In the days of 
metal type and hot lead a font meant a complete alphabet and auxiliary
characters in a given size. More recently it is taken to mean a complete
set of characters regardless of size. A font of roman type normally
consists of CAPITAL LETTERS, \textsc{small capitals}, lowercase letters,
numbers, punctuation marks, ligatures (such as `fi' and `ffi'), and a
few special symbols like \&.
   A \emph{font family}\index{font!family} is a set of fonts designed to
work harmoniously together, such as a pair of roman and italic fonts.

   The size of a font\index{font} is expressed in points\index{point} 
(72.27 points equals 1 inch
equals 25.4 millimeters). The size is a rough indication of the height
of the tallest character, but different fonts with the same size may have
very different actual heights.

    The typographers' and printers' term for the vertical space between
the lines of normal text is \emph{leading}\index{leading}, which is also
usually expressed in points and is usually larger than the font size.
A convention for describing the font and leading is give the font size 
and leading separated by a slash; for instance $10/12$ for a
10pt font set with a 12pt leading, or $12/14$ for a 12pt font set with a
14pt leading.

    The normal length of a line of text is often called the 
\emph{measure}\index{measure} and is normally specified in terms of
picas\index{pica} where 1 pica equals 12 points (1pc = 12pt).

    Documents may be described as being typeset with a particular font
with a particular size and a particular leading on a particular measure;
this is normally given in a shorthand form. 
A 10pt font with 11pt leading on a 20pc measure is described as
$10/11 \times 20$, and $14/16 \times 22$ describes a 14pt font
with 16pt leading set on a a 22pc measure.

\section{Units of measurement}

    Typographers and printers use a mixed system of units, some of which
we met above. The fundamental unit is the point; \tref{tab:units} lists 
the most common units employed.

\begin{table}
\centering
\caption{Printers units} \label{tab:units}
\begin{tabular}{ll} \hline
Name (abbreviation) & Value \\ \hline
point (pt)\index{point}\index{pt}          &            \\
pica (pc)\index{pica}\index{pc}           & 1pc = 12pt \\
inch (in)\index{inch}\index{in}           & 1in = 72.27pt \\
centimetre (cm)\index{centimetre}\index{cm}     & 2.54cm = 1in \\
millimetre (mm)\index{millimetre}\index{mm}     & 10mm = 1cm \\ 
big point (bp)\index{big point}\index{bp}      & 72bp = 72.27pt \\
didot point (dd)\index{didot point}\index{dd}    & 1157dd = 1238pt \\
cicero (cc)\index{cicero}\index{cc}         & 1cc = 12dd \\
\hline
\end{tabular}
\end{table}

    Points\index{point} and picas\index{pica} 
are the traditional printers units used in English-speaking countries. 
The didot point\index{didot point} and cicero\index{cicero} are the 
corresponding units used in continental Europe. In Japan `kyus'\index{kyus}
(a quarter of a millimetre) may be used as the unit of measurement.
Inches\index{inch} and centimetres\index{centimetre} are the units that we
are all, or should be, familiar with.

    The point system was invented by Pierre Fournier le jeune in 1737 with
a length of 0.349mm. Later in the same century Fran\c{c}ois-Ambroise Didot
introduced his point system with a length of 0.3759mm. This is the value
still used in Europe. Much later, in 1886, the American Type Founders
Association settled on 0.013837in as the standard size for the point, and
the British followed in 1898. Conveniently for those who are not entirely
metric in their thinking this means that 
six picas are approximately equal to one inch.

    The big point\index{big point} 
is somewhat of an anomaly in that is a recent
invention. It tends to be used
in page markup languages, like PostScript\footnote{PostScript is a 
registered trademark of Adobe Systems Incorporated.\label{fn:ps}},
in order to make to make calculations quicker and easier.

    The above units are all constant in value. There are also some units
whose value depends on the particular font\index{font} being used. 
The \textit{em}\index{em}
is the nominal height of the current font; it is used as a width measure.
An \textit{en}\index{en} is half an em.
The \textit{ex}\index{ex} is
nominally the height of the letter `x' in the current font. You may also
come across the term \textit{quad}\index{quad}, often as in a phrase
like `starts with a quad space'. It is a length defined in terms of
an em, often a quad is 1em.


\cleardoublepage
\pagenumbering{arabic}

% body
\mainmatter

\part{Art and Theory} \label{part:art}

\chapter{The Parts of a Book}

\section{Introduction}

    This chapter describes the various parts of a book, the 
ordering of the parts, and the typical page numbering scheme used
in books. 



\section{Front matter}


    There are three major divisions in a book: 
the front matter\index{front matter} or preliminaries\index{preliminaries}, 
the main matter\index{main matter} or text, 
and the back matter\index{back matter} or references. 
The main differences as
far as appearance goes is that in the front matter the folios\index{folio} are 
expressed as roman numerals and sectional divisions are not numbered. The 
folios\index{folio} are expressed as arabic numerals in the main and back matter. Sectional
divisions are numbered in the main matter but not in the back matter.

    The front matter\index{front matter} consists of such elements as the title
of the book, a table of contents\ixtoc, and similar items. The first few pages
in the front matter are not paginated\index{pagination} and so do not have folios\index{folio}. The remainder
are paginated and the folios\index{folio} are usually expressed as roman numerals. Not all
books have all the elements described below.

    The first page is a recto \emph{half title}\index{half title page} 
page with no folio\index{folio}. 
The page is very simple and displays just the main title of the book --- 
no subtitle, author, or other information. One purported purpose of this
page is to protect the main title page.

    The first verso page, the back of the half-title page, may contain the 
series title, if the book is one in a series, a list of contributors, 
a frontispiece, or may be blank. The series title may instead be put on the 
half-title page or on the copyright page.

   The \emph{title page}\index{title page} is recto and contains the full 
title of the work, the names of the author(s) or editor(s), and often at the
bottom of the page the name of the publisher, together with the publisher's 
logo if it has one.

    The title page(s) may be laid out in a simple manner or can have various
fol-de-rols, depending on the impression the designer wants to give. In
any event the style of this page should give an indication of the style
used in the main body of the work.

    The verso of the title page is the copyright page\index{copyright page}.
This contains the copyright notice, the publishing/printing history, 
the country where printed, ISBN and/or CIP information. The page is usually 
typeset in a smaller font\index{font!change} than the normal text.

    Following the copyright page may come a dedication or an epigraph\index{epigraph}, 
on a recto page, with the following verso page blank.

    This essentially completes the unpaginated pages.

    The headings\index{heading} and textual forms for the paginated 
pages should be the same as those for the main matter, except that 
headings\index{heading} are usually unnumbered.

    The first paginated page,
usually with roman numerals (e.g., this is folio i),
is recto with the Table of Contents (\toc). If the book contains 
figures\index{figure} (illustrations\index{illustration}) 
and/or tables\index{table}, the List of Figures (\lof) and/or List of Tables (\lot) come 
after the \toc, with no blank pages separating them. The \toc{} should contain
an entry for each following major element. If there is a \lot, say, this 
should be listed in the \toc. The main chapters\index{chapter} must be listed, of course, and
so should elements like a preface\index{preface}, bibliography\index{bibliography} or an index\index{index}.

    There may be a foreword\index{foreword} after the listings, with no blank
separator. A foreword is usually written by someone other than the author, 
preferably an eminent person, and is signed by the writer. The writer's
signature is often typeset in small caps after the end of the piece.

   A preface\index{preface} is normally written by the author, in which he
includes reasons why he wrote the work in the first place, and perhaps to 
provide some more personal comments than would be justified in the body. 
A preface starts on the page immediately following a foreword, or the lists.

   If any acknowledgements are required that have not already appeared in the
preface, these may come next in sequence.

   Following may be an introduction if this is not part of the main text. 
The last elements in the front material may be a list of abbreviations, list
of symbols, a chronology of events, a family tree, or other information of
a like sort depending on the particular work.

    Table~\ref{tab:front} summarises the potential elements in the front
matter.

\begin{table}
\centering
\caption{Front matter}\label{tab:front}
\begin{tabular}{llcc} \hline
Element                      & Page  & Paginated & Leaf \\ \hline
Half-title page              & recto & no        & 1 \\
Frontispiece, etc., or blank & verso & no        & 1 \\
Title page                   & recto & no        & 2 \\
Copyright page               & verso & no        & 2 \\
Dedication                   & recto & no        & 3 \\
Blank                        & verso & no        & 3 \\
Table of Contents\ixtoc            & recto & yes       & 3 or 4 \\
List of Figures\ixlof     & recto or verso & yes       & 3 or 4 \\
List of Tables\ixlot      & recto or verso & yes       & etc. \\
Foreword            & recto or verso & yes       & etc. \\
Preface             & recto or verso & yes       & etc. \\
Acknowledgements    & recto or verso & yes       & etc. \\
Introduction        & recto or verso & yes       & etc. \\
Abbreviations, etc  & recto or verso & yes       & etc. \\
\hline
\end{tabular}
\end{table}


    Note that the titles Foreword, Preface and Introduction are somewhat
interchangeable. In some books the title Introduction may be used for what
is described here as the preface, and similar changes may be made among the 
other terms and titles in other books. 

\subsection{Copyright page}

    Most people are familiar with titles, \toc, prefaces, etc., but like
me are probably
less familiar with the contents of the copyright page\index{copyright page|(}. 
In any event this is
usually laid out by the publishing house, but some authors may like to be,
or are forced into being, their own publisher.

    The main point of the copyright page is to display the 
copyright\index{copyright} notice.
The Berne Convention does not require that published works carry a copyright
notice in order to secure copyright protection but most play it on the safe
side and include a copyright\index{copyright} notice.
This usually comes in three parts: the word \textit{Copyright} or more usually
the symbol \textcopyright, 
the year of publication, 
and the name of the copyright owner.
The copyright symbol matches the requirements of the Universal Copyright
Convention to which the USA, the majority of European and many Asian
countries belong.
The phrase `All rights reserved' is often added to ensure protection under the
Buenos Aires Convention, to which most of the Americas belong. A typical
copyright notice may look like: \\
{\footnotesize \textcopyright{} 2035 by Frederick Jones. All rights reserved.}

    Somewhere on the page, but often near the copyright notice, is the name 
and location(s) of the publisher.

    Also on the copyright page is the publishing history, denoting the edition
or editions\footnote{A second edition should be more valuable than a first
edition as there are many fewer of them.} and their dates, 
and often where the book has been printed. One thing that has puzzled me in
the past is the mysterious row of numbers you often see, looking like: \\
\centerline{\footnotesize\texttt{02 01 00 99 98 97\hspace{2em}10 9 8 7 6 5}}
The set on the left, reading from right to left, are the last two digits
of years starting with the original year of publication.
The set on the right, and again reading from right to left, represents the
potential number of new impressions (print runs). The lowest number in each 
group indicates the edition date and the current impression. So, the example
indicates the fifth impression of a book first published in 1997.

    In the USA, the page often includes the Library of Congress 
Cataloging-in-Publication (CIP)\index{CIP} data, 
which has to be obtained from the
Library of Congress. This provides some keywords about the book.

    The copyright page is also the place for the ISBN\index{ISBN} 
(International
Standard Book Number) number. This uniquely identifies the book. For example:
ISBN 0-NNN-NNNNN-2. The initial 0 means that the book was published in an
English-speaking country, the next group of digits identify the publisher,
the third group identifies the particular book by the publisher, and the final
digit, 2 in the example, is a check digit.

    It is left as an exercise for the reader to garner more information about
obtaining CIP and ISBN data.\index{copyright page|)}

\section{Main matter}

    The main matter\index{main matter|(} forms the heart of the book.

    All pages within the main matter are paginated, even though some folios\index{folio} may
not be expressed. The folios\index{folio} are normally presented as arabic numerals, with 
the numbering starting at 1 on the first recto page of the main matter.

    The main matter is at least divided into \emph{chapters}\index{chapter}, 
unless it is something like a 
young child's book which consists of a single short story. The chapters
may be grouped within \emph{parts}\index{part} which would then be the highest level
of division within the book. Typically both parts\index{part!number} and chapters\index{chapter!number} are numbered.
Obviously, part numbering should be continuous throughout the book, but even
with parts the chapter numbering is also continuous throughout the book.

    The title of a part\index{heading!part} is usually on a recto page which just contains the
part title, and number if there is one. Chapters\index{chapter} also start on recto pages
but in this case the text of the chapter\index{chapter} starts on the same page as the chapter
title.

    Chapters\index{chapter} may be divided into sections, either numbered or unnumbered, with
the numbering scheme starting afresh within each chapter. Similarly sections
may be partitioned into subsections but except for more technical works this 
is usually as fine as the subdivisions need go to. Normally there are no 
required page 
breaks before the start of any subdivisions within a chapter\index{chapter}.

    The title page of a part\index{part}
 or chapter\index{chapter} need not have the folio\index{folio} expressed, nor
a possibly textless verso page before the start of a chapter\index{chapter}, but all other 
pages should display their folios\index{folio}.

    There may be a final chapter\index{chapter} in the main matter called Conclusions, 
or similar, which may be a lengthy summary of the work presented, untouched
areas, ideas for future work, and so on.

    If there are any numbered appendices\index{appendix} 
they logically come at the end of
the front matter. Appendices are often `numbered' alphabetically rather
than numerically, so the first might be Appendix A, the second Appendix B,
and so on.

    An epilogue\index{epilogue} or an afterword\index{afterword} is a 
relatively short piece that the author may
include. These are not normally treated as prominently as the preceding
chapters\index{chapter}, and may well be put into the back matter if they are 
unnumbered.\index{main matter|)}

\section{Back matter}

    The back matter\index{back matter|(}
is optional but if present conveys information ancilliary
to that in the main matter.

    An unnumbered appendix\index{appendix} 
would normally come in the back matter.

    Other elements include Notes, a Glossary\index{glossary}
 and/or lists of symbols\index{symbol} or 
abbreviations\index{abbreviation}, which could be in the 
front matter\index{front matter} 
instead. These elements 
are normally unnumbered, as is any list of contributors\index{contributor}, 
Bibliography\index{bibliography} or Index\index{index}.

    In some instances appendices\index{appendix} 
and notes may be given at the end of each
chapter\index{chapter} instead of being lumped at the back.

    The first element in the back matter starts on a recto page but the 
remainder may start on either recto or verso pages.

    In older books it was often the custom to have a colophon\index{colophon}
as the final element in a book. This is an inscription which includes 
information about the production and design of the book and nearly 
always indicates which fonts\index{font} were used.\index{back matter|)}


\section{Signatures and casting off}
\index{signature|(}

    Professionally printed books have many pages printed per sheet of (large)
paper\index{paper}, which is then folded and cut where necessary to produce a 
\emph{signature} of several smaller sheets. An 
unfolded sheet is called a \emph{broadside}\index{broadside}. 
Folding a sheet in half produces a one sheet 
\emph{folio}\index{folio} signature with two leaves and four pages. 
Folding it in half again and cutting along the original fold gives a 
two sheet \emph{quarto}\index{quarto} signature with four leaves
and eight pages. 
Folding in half again, 
results in a four sheet \emph{octavo}\index{octavo} signature with eight
leaves and 16 pages, and so on as listed in \tref{tab:signatures}.

\begin{table}
\centering
\caption{Common signatures} \label{tab:signatures}
\begin{tabular}{lcccrccrccrc} \hline
Name      & Folds & Size             & \multicolumn{3}{c}{Sheets} & 
\multicolumn{3}{c}{Leaves} & \multicolumn{3}{c}{Pages} \\ \hline
Broadside & 0     & $a \times b$     & &  1 & & &  1 & & &   2 & \\
Folio     & 1     & $b/2 \times a$   & &  1 & & &  2 & & &   4 & \\
Quarto, \emph{4to} & 2 & $a/2 \times b/2$ & & 2 & & & 4 & & & 8 & \\
Octavo, \emph{8vo} & 3 & $b/4 \times a/2$ & & 4 & & & 8 & & & 16 & \\
\emph{16mo} & 4   & $a/4 \times b/4$ & &  8 & & & 16 & & &  32 & \\
\emph{32mo} & 5   & $b/8 \times a/4$ & & 16 & & & 32 & & &  64 & \\
\emph{64mo} & 6   & $a/8 \times b/8$ & & 32 & & & 64 & & & 128 & \\ \hline
\end{tabular}%
\index{broadside}\index{folio}\index{quarto}\index{4to}\index{octavo}%
\index{8vo}\index{16mo}\index{32mo}\index{64mo}%
\end{table}

    In \tref{tab:signatures} the Size column is the untrimmed size of a 
leaf\index{leaf} in 
the signature
with respect to the size of the broadside. When made up into a book the
leaves will trimmed to a slightly smaller size, at the discretion of the
designer and publisher; typically a minimum of 1/8 inch or 3 millimetres
would be trimmed from the top, bottom and foredge of a leaf.

    Other folds can produce other signatures. For example a 
\emph{sexto}\index{sexto},
obtained by folding in thirds and then folding in half, is a three sheet
signature with six leaves and 12 pages.

    In making up the book, the pages in each signature are first fastened
together, usually by sewing through the folds. The signatures are then bound
together and the covers, end papers\index{paper!end} and spine are attached to form
the completed whole.


    Publishers like the final typeset book to be of a length that just fits
within an integral number of signatures\index{signature}, 
with few if any blank pages required
to make up the final signature. Casting off\index{casting off} is the
process of determining how many lines a given text will make in a given
size of type, and hence how many pages will be required.

    To cast off you need to know how many characters there will be in
a line, and how many characters there are, or will be, in the text. 
For the purposes of casting off, `characters' includes punctuation as well
as letters and digits. The
first number can be easily obtained, either from copy fitting tables or
by measurement; this is described in more detail in \S\ref{sec:tblock}.
The second is more problematic, especially when the manuscript has yet
to be written. A useful rule of thumb is that words in an English text
average five letters plus one space (i.e., six characters); 
word length in technical texts might be greater than this.

    To determine the number of words it is probably easiest to type a
representative portion of the manuscript, hand count the words and then
divide that result by the proportion of the complete text that you have
typed. For example, if you have typed 1/20\,th of the whole, then divide
by 1/20, which is equivalent to multiplying by 20. To fully estimate
the number of pages required it is also necessary to make allowance for
chapter\index{chapter} titles, illustrations\index{illustration}, and so forth.

    If it turns out, say, that your work will require 3 signatures plus 2
pages then it will be more convenient to make it fit into 3 signatures,
or 4 signatures minus a page or two. This can be done by expanding or cutting
the text and/or by changing the font\index{font!change} 
and/or by changing the number or width
of lines on a page.

    When I was editing a technical journal the authors were given a word 
limit. The primary reason was not that we were interested in the actual
word count but rather so that we could estimate, and possibly limit, 
the number of pages allotted
to each article; we used \emph{octavo}\index{octavo} 
signatures\index{signature} and no blank pages. 
I suspect that it is
the same with most publishers --- it is the page count not the word count
that is important to them.

    In some special cases, extra pages may be `tipped in'\index{tip in} to
the body of the book. This is most likely to occur for 
illustrations\index{illustration} which
require special paper\index{paper} for printing and it would be too costly to use
that paper\index{paper} for the whole work. Another example is for a fold-out of some sort,
a large map, say, or a triple spread illustration\index{illustration}. The tipped in pages
are glued into place in the book and may or may not be paginated. For
tipped in illustrations\index{illustration}, a List of Illustrations may well start with
a phrase like: `Between pages 52 and 53'.

\index{signature|)}

%%%\clearpage
%%%\raggedbottom
\chapter{The page}  \label{chap:lpage}

    Authors usually want their works to be read by others than themselves,
and this implies that their manuscript will be reproduced in some manner.
It is to be hoped that the published version of their work will attract 
readers and there are two aspects to this. The major is the actual content
of the work --- the thoughts of the author couched in an interesting
manner --- if something is boring, then there are too many other interesting
things for the reader to do than to plow on until the bitter end, 
assuming that he
even started to read seriously after an initial scan. The other aspect is
the manner in which the content is displayed. Or, in other words, 
the \emph{typography}
of the book, which is the subject of this chapter.

    The essence of good typography is that it is not noticeable at first,
or even second or later, glances to any without a trained eye. If your
initial reaction when glancing through a book is to exclaim about its layout
then it is most probably badly designed, if it was designed at all. Good
typography is subtle, not strident. 

    With the advent of desktop publishing
many authors are tempted to design their own books. It is seemingly all
too easy to do. Just pick a few of the thousands of fonts\index{font} that are available,
use this one for headings\index{heading}, 
that one for the main text, another one for
captions, decide how big the typeblock\index{typeblock} is to be, and there you are.

    However, just as writing is a skill that has to be learned, typography
is also an art that has to be learned and practised. There are hundreds
of years of experience embodied in the good design of a book. These are
not to be cast aside lightly and many authors who design their own books
do not know what some of the hard-earned lessons are, let alone that what
they are doing may be the very antithesis of these. An expert can break
the rules, but then he is aware that he has good reasons for breaking them.

    The author supplies the message and the typographer supplies the medium.
Contrary to Marshall McLuhan, the medium is \emph{not} the message, 
and the typographer's job is not to
intrude between the message and the audience, but to subtly increase the
reader's enjoyment and involvement. If a book shouts `look at me!' then it
is an advertisement, and a bad one at that, for the designer.


\section{The shape of a book}

    Books come in many shapes and sizes, but over the centuries certain
shapes have been found to be more pleasurable and convenient than others.
Thus books, except for a very very few, are rectangular in shape. The 
exceptions on the whole are books for young children, although I do
have a book edited by Fritz Spiegl and published by Pan Books entitled
\textit{A Small Book of Grave Humour}, which is in the shape of a tombstone
--- this is an anthology of epitaphs. Normally the height of a book, when 
closed, is greater than the width. Apart from any aesthetic reasons, 
a book of this shape is physically more comfortable to hold than one which 
is wider than it is high.

    It might appear that the designer has great freedom in choosing the
size of the work, but for economic reasons this is not normally the case.
Much typographical design is based upon the availabilty of certain 
standard industrial sizes of sheets of paper\index{paper!size}. A page size of $12 \times 8$
inches will be much more expensive than one which fits on a standard
US letter sheet\index{paper!size!letterpaper} of $11 \times 8 \; 1/2$ inches. Similarly, 
one of the standard sizes
for a business envelope is $4 \; 1/8 \times 9 \; 1/2$ inches. 
Brochures for mailing
should be designed so that they can be inserted into the envelope with 
minimal folding. Thus a brochure size of $5 \times 10$ inches will be 
highly inconvenient, no matter how good it looks visually.

    Over the years books have been produced in an almost infinite variety
of proportions,
where by \emph{proportion}\index{proportion} 
I mean the ratio of the height to the width of a
rectangle. However, certain proportions\index{proportion} occur time after time throughout
the centuries and across many different countries and 
civilizations. This is because some proportions\index{proportion} are inherently
more pleasing to the eye than others are. These pleasing proportions\index{proportion} are
also commonly found in nature --- in  physical, biological, and chemical
systems and constructs. 

    Some examples of pleasing proportions\index{proportion} can be
seen in Japanese wood block prints, such as the \textit{Hoso-ye} size
($2 : 1$) which is a double square, the \textit{Oban} ($3 : 2$),
the \textit{Chuban} ($11 : 8$) and the \textit{Koban} size
($\sqrt{2} : 1$). Sometimes these prints were made up into books, but
were often published as stand-alone art work. Similarly Indian paintings,
at least in the 16th to the 18th century,
often come in the range $1.701 : 1$ to $13 : 9$, thus being around
$3 : 2$ in proportion\index{proportion}.

    In medieval Europe page proportions\index{proportion} were generally in the range
$1.25 : 1$ to $1.5 : 1$. Sheets of paper\index{paper} were typically 
produced in the
proportion\index{proportion} $4 : 3$ ($1.33 : 1$) or $3 : 2$ ($1.5 : 1$). 
These proportions\index{proportion}
have the property that they are reproduced with each alternate
folding of the sheet.
For example, if a sheet starts at a size of $60 \times 40$ 
(i.e., $3 : 2$),
then the first fold will make a double sheet of size $30 \times 40$
(i.e., $3 : 4$). The next fold will produce a quadrupled sheet of size
$30 \times 20$, which is again $3 : 2$, and so on. 
It is an interesting fact,
though, that it is impossible to fold a sheet of paper\index{paper}, no matter how large 
and thin, more than six times altogether. The Renaissance typographers
tended to like taller books, and their proportions\index{proportion} would go up 
to $1.87 : 1$
or so. The style nowadays has tended to go back towards the medieval
proportions\index{proportion}.

    The standard ISO page proportions\index{proportion} are $\sqrt{2} : 1$ 
($1.414 : 1$). These
have a similar property to those of medieval times. However, in this case
each fold reproduces the page proportion\index{proportion}. Thus halving an A0 sheet 
(size $1189 \times 841$ mm) produces an A1 size sheet ($594 \times 841$),
which in turn being halved produces the A2 sheet ($420 \times 594$), down
through the A3, A4 ($210 \times 297$ mm), and A5 sheets.

   There is no one perfect proportion\index{proportion} for a page, although some are 
clearly better
than others. For ordinary books both publishers and readers tend to prefer
books whose proportions\index{proportion} range from the light $9 : 5$ ($1.8 : 1$) 
to the heavy
$5 : 4$ ($1.25 : 1$). Some examples are shown in \fref{flpage:prop}.
 Wider pages, those with proportions\index{proportion} less than
$\sqrt{2} : 1$ ($1.414 : 1$), 
are principally useful for documents that need
extra width for tables\index{table}
, marginal notes\index{marginalia}, or where multi-column\index{column!multiple} printing
is preferred. 

\begin{figure}
\centering
\setlength{\unitlength}{1pc}
\begin{picture}(24,38)
\put(0,4){\begin{picture}(24,34)
  \put(0,0){\framebox(24,34){}}
  \thicklines \put(19.78,0){\line(0,1){34}}
  \thinlines
  \put(16,0){\line(0,1){34}}
  \put(17.78,0){\line(0,1){34}}
  \put(18.48,0){\line(0,1){34}}
  \put(19.2,0){\line(0,1){34}}
  \put(20.81,0){\line(0,1){34}}
  \put(21.33,0){\line(0,1){34}}
  \put(22.63,0){\line(0,1){34}}
  \put(0,-0.5){\begin{picture}(24,2)
    \put(16,0){\makebox(0,0){\textsc{a}}}
    \put(17.78,0){\makebox(0,0){\textsc{b}}}
    \put(18.48,0){\makebox(0,0){\textsc{c}}}
    \put(19.2,0){\makebox(0,0){\textsc{d}}}
    \put(19.78,0){\makebox(0,0){{$\varphi$}}}
    \put(20.81,0){\makebox(0,0){\textsc{e}}}
    \put(21.33,0){\makebox(0,0){\textsc{f}}}
    \put(22.63,0){\makebox(0,0){\textsc{g}}}
    \put(24,0){\makebox(0,0){\textsc{h}}}
    \end{picture}}
  \end{picture}}
  \put(0,0){\begin{picture}(24,4)
    \put(0,0){\begin{picture}(8,4)
      \put(0,2){\textsc{a} $2 : 1$}
      \put(0,1){\textsc{b} $9 : 5$}
      \put(0,0){\textsc{c} $1.732 : 1$ ($\sqrt{3}{} : 1$)}
      \end{picture}}
    \put(8,0){\begin{picture}(8,4)
      \put(0,2){\textsc{d} $5 : 3$}
      \put(0,1){{$\varphi$} $1.618 : 1$ ($\varphi{} : 1$)}
      \put(0,0){\textsc{e} $1.538 : 1$}
      \end{picture}}
    \put(16,0){\begin{picture}(8,4)
      \put(0,2){\textsc{f} $3 : 2$}
      \put(0,1){\textsc{g} $1.414 : 1$ ($\sqrt{2}{} : 1$)}
      \put(0,0){\textsc{h} $4 : 3$}
      \end{picture}}
    \end{picture}}
\end{picture}
\setlength{\unitlength}{1pt}
\caption{Some page proportions} \label{flpage:prop}
\end{figure}



    In books where the illustrations\index{illustration} are the primary concern, the shape of
the illustrations\index{illustration} is generally the major influence on the page proportion\index{proportion}.
The page size should be somewhat higher than that of the average 
illustration\index{illustration}.
The extra height is required for the insertion of captions\index{caption} describing the
illustration\index{illustration}. A proportion\index{proportion} of $\pi{} : e$ ($1.156 : 1$), 
which is slightly higher
than a perfect square, is good for square illustrations.\footnote{Both $e$
and $\pi$ are well known mathematical numbers. $e$ ($= 2.718 \ldots$)
is the base of natural logarithms and $\pi$ ($= 3.141 \ldots$) is the
ratio of the circumference of a circle to its diameter.}
The $e : \pi$
($0.864 : 1$) proportion\index{proportion} is useful for landscape photographs  taken with 
a $4 \times 5$
format camera, while those from a 35mm camera (which produces a negative
with a $2 : 3$ proportion\index{proportion}) are better accomodated on 
a $0.83 : 1$ page.

\subsection{The golden section and Fibonacci series}

\index{golden section|(}
    Typographers need a modicum of mathematical ability, but no more
than an average teenager can do --- basically simple arithmetic. You can
skip this section if you wish as it just provides some background 
mathematical material which might be of interest.

    Since ancient Greek times or even before, the golden section, which
is denoted by the Greek letter $\varphi$ (phi), has been considered to be
a particularly harmonious proportion\index{proportion}. It should come as no surprise, then,
that this also has applications in typography.

    The Greeks were interested in geometry (think of Euclid). They discovered
that if you divide a straight line into two unequal parts then a certain
division appeared to have an especially appealing aesthetic quality about it. 
Call the length of the line $l$ and the length of the two parts $a$ and $b$, 
where $a$ is the smaller and $b$ is the larger. The division in question
is when the ratio of the larger to the smaller division ($b/a$) is the same
as the ratio of the whole line to the larger division ($l/b$).
More formally, two elements embody the golden section, symbolised by
$\varphi$, when the ratio of the larger
to the smaller is the same as the ratio of the sum of the two to the larger.
If the two elements are $a$ and $b$, with $a < b$, then
\begin{equation}
\varphi = \frac{b}{a} = \frac{a+b}{b} = (1+\sqrt{5})/2
\end{equation}

    The golden section has been called by a number of different names
during its history. Euclid\index{Euclid}
 called it the `extreme and mean ratio' while
Renaissance writers called it the `divine proportion\index{proportion}'; now it is
called either the `golden section' or the `golden ratio'. The symbol
$\varphi$ is said to come from the name of the Greek artist 
Phidias\index{Phidias}
(C5th \textsc{bc}) who often used the golden section in his sculpture.
A rectangle whose sides are in the same proportion\index{proportion} as the golden section
is often called a `golden rectangle'.
The front of the Parthenon on the Acropolis in Athens is a golden rectangle,
and such rectangles appear often in Greek architecture.
The symbol of the Pythagoran school was the star pentagram, 
%%%%shown in \fref{flpage:spent}, 
where each line is divided in the golden section.


    The approximate decimal value for $\varphi$ is $1.61803$. 
The number has some unusual properties. If you add one to $\varphi$
you get its square, while subtracting one from $\varphi$ gives its 
reciprocal.
\begin{eqnarray}
  \varphi + 1 & = & \varphi^{2} \\
  \varphi - 1 & = & 1/\varphi
\end{eqnarray}
It also has a very simple definition as the continued fraction
\begin{equation}
\varphi = 1 + \frac{1}{\displaystyle 1 + \frac{1}{\displaystyle 1 + \frac{1}{\displaystyle 1 + \frac{1}{1 + \cdots}}}}
\end{equation}


    In 1202 Leonardo Pisano, 
also known as Leonardo Fibonacci, wrote a
book called \textit{Liber Abbaci.}\footnote{Book of the Abacus.} One of the 
topics he was interested in was population growth. The book included
this exercise: \index{Fibonacci series|(}
\begin{quote}
How many pairs of rabbits\index{rabbit} can be produced from a single 
pair in a year?
Assume that each pair produces a new pair of offspring every month,
a rabbit becomes fertile at age one month, and no rabbits die during the
year.
\end{quote}
After a month there will be two pairs. At the end of the next month the
first pair will have produced another pair, so now there are three pairs.
At the end of the following  month the original pair will have produced a
third pair of offspring and their firstborn will also have produced a pair, 
to make five pairs in all. And so on. 
If, like the rabbits, you are not too exhausted
to continue, you can get the following series of 
numbers\footnote{The numbers at the start of the series
depend on whether you consider the initial pair of rabbits to be adults or 
babies.\label{fn:rabbits}}:
\begin{displaymath}
0,1,1,2,3,5,8,13,21,34, 55, 89 \ldots
\end{displaymath}
After the first two terms, each term in the series is the sum of the two
preceding terms. Also, as one progresses along the series, the ratio of
any adjacent pair of terms oscillates around $\varphi$ ($= 1.618 \ldots$),
approaching it ever more closely.
\begin{eqnarray*}
  8/5 & = & 1.6 \\
  13/8 & = & 1.625 \\
  21/13 & = & 1.615 \\
  34/21 & = & 1.619 \\
  55/34 & = & 1.6176 \\
  89/55 & = & 1.6182
\end{eqnarray*}

    For the mathematically inclined there is another, to me, striking
relationship between $\varphi$ and the Fibonacci series. Define the
Fibonacci numbers as $F_{n}$, where
\begin{equation}
\begin{array}{cccc}
F_{0} = 0;\ \ & F_{1}=1;\ \ & F_{n+2}=F_{n+1} + F_{n},  &  n \geq 0.
\end{array}
\end{equation}
Then
\begin{equation}
 F_{n} = \frac{1}{\sqrt{5}}(\varphi^{n} - (- \varphi)^{-n})
\end{equation}

    Both the Fibonacci series and the golden section appear in nature.
The arrangement of seeds in a sunflower, the pattern on the surface of a 
pinecone, and the spacing of leaves around a stalk all exhibit Fibonacci
paterns (for example see~\cite{CONWAY96}). Martin Gardener~\cite{GARDNER66}
reports on studies that claimed that the average ratio of a person's height
to the height of the navel is $1.618+$ --- suspiciously close to $\varphi$.

\index{Fibonacci series|)}
\index{golden section|)}

%%%%%%%%%%%%%%%%%%%%%%%%%%%
%%%%%%\endinput
%%%%%%%%%%%%%%%%%%%%%%%%%%%

\section{The spread} \label{sec:spread}
\index{spread|(}

    The typeblock\index{typeblock} is that part of the page which is normally covered with
type. The same proportions\index{proportion} that are useful for the shape of a page are also
useful for the shape of the typeblock\index{typeblock}. This does not mean, though, that the
proportions\index{proportion} of the page and the typeblock\index{typeblock} should be the same. For instance,
a square typeblock\index{typeblock} on a square page is inherently dull.

    When we first start to learn to read we scan horizontally along each line
of text. As our skills improve we tend to scan vertically rather than
horizontally. A tall column\index{column} of text helps in this process, provided that
the column\index{column} is not too wide.

    A page in a book will typically contain several elements. Principal
among these is the typeblock\index{typeblock}, but there are also items like the folio\index{folio}
(that is, the page number), a running head and/or foot which carries the
chapter\index{chapter} and/or book title, and possibly marginalia\index{marginalia} and footnotes\index{footnote}. These latter
elements, although essential to the content of the book, are minor visual
elements compared to the typeblock\index{typeblock}. But even minor decoration can obscure
or kill an otherwise good design.

  The major concern is the positioning of the typeblock\index{typeblock} on the page. 
The mere 
fact of positioning the typeblock\index{typeblock} also has the result of producing margins\index{margin}
onto the page. Page design is a question of balancing the page proportions\index{proportion}
with the proportions\index{proportion} of the typeblock\index{typeblock} and the proportions\index{proportion} of the margins\index{margin} to 
create an interesting yet harmonious composition. A single page, except
for a title page, is never the subject of a design but rather the design
is in terms of the two pages that are on view when a book is opened --- the
left and right hand pages are considered as a whole. More technically, the
design is in terms of a \emph{double spread}.



\begin{table}
\caption{Some page designs} \label{tlpage:allp}
\centering
\DeleteShortVerb{\|}
%\begin{tabular}{|l|l|l|c|} \hline
\begin{tabular}{|r|r|rrrrr|l|} \hline
\multicolumn{1}{|c|}{$P$} & \multicolumn{1}{c|}{$T$} & \multicolumn{5}{c|}{Margins \& Columns} & 
\multicolumn{1}{c|}{Figure}          \\ 
 & & \multicolumn{1}{c|}{$s$} & \multicolumn{1}{c|}{$t$} & \multicolumn{1}{c|}{$e$} & 
     \multicolumn{1}{c|}{$f$} & \multicolumn{1}{c|}{$g$}     &                 \\ \hline
$\sqrt{3}$ & $2$     & $w/13$   & $8s/5$ & $16s/5$ & $16s/5$ &       & \ref{fb:1} left \\ %Bringhurst
$\sqrt{3}$ & $e/\varphi$ & $w/10$ & $2s$ & $2s$    & $3s$    &       & \ref{fb:1} right \\ % Machie
$12/7$     & $1.701$ & $w/7$    & $8s/5$ & $8s/5$  & $14s/5$ &       & \ref{fb:2} left \\ % Grenfell
$e/\varphi$ & $7/4$  & $w/10$   & $5s/4$ & $5s/3$  & $11s/8$ &       & \ref{fb:2} right \\ % JKJ
$\varphi$  & $1.866$ & $w/9$    & $s$    & $2s$    & $7s/3$  &       & \ref{fb:3} left \\ %Paris
$\varphi$  & $\varphi$ & $w/12$ & $2s$   & $5s/2$  & $4s$    &       & \ref{fb:3} right \\ %Dowding
$8/5$      & $1.634$ & $2w/15$  & $7s/5$ & $9s/5$  & $13s/5$ &       & \ref{fb:4} left \\ %Rogers
$19/12$    & $7/4$   & $2w/15$  & $s$    & $9s/8$  & $11s/8$ &       & \ref{fb:4} right \\ %Anatomy
$19/12$    & $\sqrt{3}$ & $w/7$ & $s$    & $5s/4$  & $1.84s$ &       & \ref{fb:5} left \\ % Cornford
$19/12$    & $8/5$   & $w/12$   & $7s/5$ & $8s/5$  & $2s$    &       & \ref{fb:5} right \\ %Abeced
$\pi/2$    & $9/5$   & $w/9$    & $3s/2$ & $5s/2$  & $3s$    &       & \ref{fb:6} left \\ %Dwiggins
$e/\sqrt{3}$ & $1.71$ & $w/10$  & $11s/8$ & $24s/11$ & $8s/3$ &      & \ref{fb:6} right \\ %Two Men
$1.553$    & $1.658$ & $w/11$   & $\varphi s$ & $\varphi s$ & $\varphi s$ & & \ref{fb:7} left \\ %Express
$1.538$    & $\sqrt{7}$ & $w/10$ & $s$   & $23s/6$ & $3s/2$  &       & \ref{fb:7} right \\ %T&H
$3/2$      & $2$     & $w/5$    & $s/2$  & $s$     & $s$     &       & \ref{fb:8} left \\ %Rome
$3/2$      & $1.701$ & $w/9$    & $s$    & $2s$    & $7s/3$  &       & \ref{fb:8} right \\ %Venice
$3/2$      & $\pi/2$ & $w/13$   & $2s$   & $10s/3$ & $30s/7$ &       & \ref{fb:9} left \\ %Magellan
$3/2$      & $3/2$   & $w/9$    & $3s/2$ & $2s$    & $3s$    &       & \ref{fb:9} right \\ %Gutenberg
$3/2$      & $1.68$  & $w/23$   & $2s$   & $5s$    & $2s$    &       & \ref{fb:10} left \\ % Pers Mss
$3/2$      & $3/2$   & $w/10$   & $2s$   & $5s/2$  & $2.85s$ &       & \ref{fb:10} right \\ % Pers Bk
$1.48$     & $1.376$ & $w/12$   & $7s/4$ & $2s$    & $7s/2$  &       & \ref{fb:11} left \\ %Goudy
$13/9$     & $\sqrt{2}$ & $w/30$ & $2s$  & $9s/2$  & $4s$    & $s/2$ & \ref{fb:11} right \\ %Doomsday
$\sqrt{2}$ & $\varphi$ & $w/9$  & $s$    & $2s$    & $2s$    &       & \ref{fb:12} left \\ %Orig A4
$\sqrt{2}$ & $\varphi$ & $w/8$  & $s$    & $5s/3$  & $5s/3$  &       & \ref{fb:12} right \\ %Mod A4
$7/5$      & $1.641$   & $w/7$  & $s$    & $8s/5$  & $8s/5$  &       & \ref{fb:13} left \\ %Emery Walker
$1.294$    & $\varphi$ & $0.176w$ & $1.03s$ & $1.685s$ & $13s/9$ &   & \ref{fb:13} right \\ %LaTeX
$1.294$    & $13/9$  & $w/12$   & $s$    & $2s$    & $10s/7$ & $s/2$ & \ref{fb:14} left \\ %Wilson
$9/7$      & $19/9$  & $2w/5$   & $5s/8$ & $5s/8$  & $5s/6$  &       & \ref{fb:14} right \\ %Kuniyoshi
$5/4$      & $13/11$ & $w/10$   & $3s/2$ & $2s$    & $8s/3$  &       & \ref{fb:15} left \\ %Fens
$7/6$    & $17/15$ & $w/13$   & $s$    & $s$     & $7s/5$  & $.382$ & \ref{fb:15} right \\ %Durer
%$1.176$    & $1.46$  & $0.107w$ & $5s/6$ & $2.41s$ & $3s/2$  &       & \ref{fb:12} left \\ %Art
$e/\pi$    & $0.951$ & $w/9$    & $s$    & $2s$    & $3s/2$  &       & \ref{fb:16} left \\ %Hammer & Hand
$5/7$      & $2/3$   & $w/9$    & $s/2$  & $2s/3$  & $s$     & $s/3$ & \ref{fb:16} right \\ \hline %Hokusai
\end{tabular}
\MakeShortVerb{\|}
\end{table}

  Table~\ref{tlpage:allp} gives some examples of 
page designs. These are arranged in increasing order of
fatness. In this table, and afterwards, I have just used a single number
to represent the ratio of the page height to the width; that is, for example,
$1.5$ instead of $1.5 : 1$ or $12/7$ instead of $12 : 7$.
The following symbols are used in the table:
\begin{description}
\item[Proportions]:
  \begin{itemize}
  \item[$P$] = page proportion = $h/w$
  \item[$T$] = typeblock proportion = $d/m$
  \end{itemize}
\item[Page size]:
  \begin{itemize}
  \item[$w$] = width of page
  \item[$h$] = height of page
  \end{itemize}
\item[Typeblock]:
  \begin{itemize}
  \item[$m$] = measure (i.e., width) of primary typeblock
  \item[$d$] = depth (excluding folios, running heads, etc.)
%%  \item[$n$] = measure of secondary column
%%  \item[$c$] = column width, when there are two columns
  \end{itemize}
\item[Margins]:
  \begin{itemize}
  \item[$s$] = spine margin (back margin)
  \item[$t$] = top margin (head margin)
  \item[$e$] = fore-edge (front margin)
  \item[$f$] = foot margin (bottom margin)
  \item[$g$] = internal gutter (on a multi-column page)
  \end{itemize}
\end{description}
    The designs are also shown in \figurerefname s~\ref{fb:1} 
to~\ref{fb:16}. Each of these shows a double page spread; the 
page width has been kept constant throughout the series to enable easier
visual comparison --- it is the relative proportions\index{proportion}, not the absolute size, 
that are important. I have only shown the pages and the typeblocks\index{typeblock} to avoid
confusing the diagrams with headers\index{header}, footers\index{footer} 
or folios\index{folio}.


\begin{figure}
\centering
\begin{minipage}[b]{\pwlayi}
\drawaspread{\pwlayii}{1.732}{2}{.0769}{1.6}{3.2}{0} % Bringhurst
\end{minipage}
\hfill
\begin{minipage}[b]{\pwlayi}
\drawaspread{\pwlayii}{1.732}{1.684}{.1}{2}{2}{0} % Machiavelli
\end{minipage}
\caption[Two spreads: Canada, 1992 and England, 1970]%
        {Two spreads: (Left) Canada, 1992. % Bringhurst.
         (Right) England, 1970.} \label{fb:1}
\end{figure}

    Shown in \fref{fb:1} are two modern books. On the left is the layout
for Robert Bringhurst's \textit{The Elements of Typographical Style} published
by Hartley \& Marks in 1992, and designed by Bringhurst. The text face is
Minion set with $12$pt leading on a $21$pc measure. The captions are
set in Syntax. The original
size is $227 \times 132$mm. I highly recommend this book if you are
interested in typography. The layout on the right is The Folio Society's
1970 edition of \textit{The Prince} by Niccol\`{o} Machiavelli. The original
size is $216 \times 125$mm and is set in $12/13 \times 22$ Centaur.



\begin{figure}
\centering
\begin{minipage}[b]{\pwlayi}
\drawaspread{\pwlayii}{1.714}{1.701}{.143}{1.6}{1.6}{0} % Grenfell
\end{minipage}
\hfill
\begin{minipage}[b]{\pwlayi}
\drawaspread{\pwlayii}{1.68}{1.75}{.1}{1.25}{1.667}{0} % JKJ
\end{minipage}
\caption[Two spreads: USA, 1909 and England, 1964.]%
        {Two spreads: (Left) USA, 1909.
         (Right) England, 1964.} \label{fb:2}
\end{figure}

    Figure~\ref{fb:2} (left) illustrates a small book by Wilfred T.~Grenfell
entitled \textit{Adrift on an Ice-Pan} published in 1909 by the Riverside
Press of Boston. The text is set with a leading of $16$pt on a $16$pc measure.
The large leading and small measure combine to give a very open appearance.
The original
size is $184 \times 107$mm. On the right is another book from the
Folio Society --- \textit{Three Men in a Boat} by Jerome K.~Jerome printed
in 1964. The original size is $215 \times 128$mm and is typeset with
Ehrhardt at $11/12 \times 22$.

\begin{figure}
\centering
\begin{minipage}[b]{\pwlayi}
\drawaspread{\pwlayii}{1.618}{1.87}{.111}{1}{2}{0} % Paris
\end{minipage}
\hfill
\begin{minipage}[b]{\pwlayi}
\drawaspread{\pwlayii}{1.618}{1.618}{.0833}{2}{2.5}{0} % Dowding
\end{minipage}
\caption[Two spreads: France, 1559 and Canada, 1995]%
        {Two spreads: (Left) France, 1559.
         (Right) Canada, 1995.} \label{fb:3}
\end{figure}

   Jean de Tourmes, a Parisian publisher, printed \textit{Histoire et Chronique}
by Jean Froissart in 1559. This is a history book with the main text in
roman and sidenotes in italic at roughly 80\% of the size of the main text.
The layout is shown in \fref{fb:3} (left). The gutter (not shown) between the main
text and the sidenote column\index{column} is very small, but the change in fonts and sizes
enables the book to be read with no confusion. Another Hartley \& Marks
typography book --- \textit{Finer Points in the Spacing \& Arrangement
of Type} by Geoffrey Dowding --- is shown at the right of \fref{fb:3}.
This is typeset in Ehrhardt at $10.5/14 \times 23$ on a page size of
$231 \times 143$mm.



\begin{figure}
\centering
\begin{minipage}[b]{\pwlayi}
\drawaspread{\pwlayii}{1.6}{1.634}{.1333}{1.4}{1.8}{0} % Rogers
\end{minipage}
\hfill
\begin{minipage}[b]{\pwlayi}
\drawaspread{\pwlayii}{1.583}{1.75}{.1333}{1}{1.125}{0} % Anatonomy
\end{minipage}
\caption[Two spreads: USA, 1949 and 1990]%
        {Two spreads: (Left) USA, 1949.
         (Right) USA, 1990.} \label{fb:4}
\end{figure}

    Bruce Rogers (1870--1957) described how he came to design his 
Centaur typeface in
\textit{Centaur Types}, a privately published book by his studio October
House in 1949. The layout of this book, which of course was typeset in
Centaur, is shown at the left of \fref{fb:4}. Centaur is an upright
seriffed type based on Nicolas Jenson's type as used in \textit{Eusebius}
published in 1470. \textit{Centaur Types} demonstrates typefaces other than
Centaur, and also includes exact size reproductions of the engraver's 
patterns. It is set at $14/16 \times 22$ on a page size of $240 \times 150$mm.
Figure~\ref{fb:4} (right) is the layout of another book on typefaces.
It is \textit{The Anatomy of a Typeface} by Alexander Lawson published by
David R.~Godine in 1990.
This is set in Galliard with $13$pt leading and a measure of $24$pc on
a page size of $227.5 \times 150$mm.


\begin{figure}
\centering
\begin{minipage}[b]{\pwlayi}
\drawaspread{\pwlayii}{1.583}{1.731}{.143}{1}{1.25}{0} % Cornford
\end{minipage}
\hfill
\begin{minipage}[b]{\pwlayi}
\drawaspread{\pwlayii}{1.583}{1.6}{.0833}{1.4}{1.6}{0} % Abcedarium
\end{minipage}
\caption[Two spreads: England, 1908 and USA, 1993]%
        {Two spreads: (Left) England, 1908.
         (Right) USA, 1993.} \label{fb:5}
\end{figure}

    \textit{Microcosmographica Academia} by F. M. Cornford is shown in
\fref{fb:5}. Despite its title, it is written in English and was published
by Bowes \& Bowes, London, in 1908. It is a dryly humourous look at academic
politics as practised in Cambridge University at the turn of the nineteenth
century (possibly the twentieth as well). It is set with $14$pt leading
on $22$pc. The original page size is $216 \times 136$mm.
The right of this figure illustrates a book with another unusual title ---
\textit{The Alphabet Abcedarium} by Richard A.~Firmage  and published by
David R.~Godine in  1993. It is set in Adobe Garamond on a $27$pc measure
with $14$pt leading. The original page size is $227.5 \times 150$mm. The
book gives a history of each letter of the latin alphabet. One
unusual feature is that there is a deep footer\index{footer} on each page
showing many examples of typefaces of the letter being described.


\begin{figure}
\centering
\begin{minipage}[b]{\pwlayi}
\drawaspread{\pwlayii}{1.571}{1.8}{.111}{1.5}{2.5}{0} % Dwiggins
\end{minipage}
\hfill
\begin{minipage}[b]{\pwlayi}
\drawaspread{\pwlayii}{1.562}{1.709}{.1}{1.375}{2.182}{0} % Two Men
\end{minipage}
\caption[Two spreads: USA, 1931 and England, 1968]%
        {Two spreads: (Left) USA, 1931.
         (Right) England, 1968.} \label{fb:6}
\end{figure}


    W.~A.~Dwiggins was, among many other things, an American book designer.
Figure~\ref{fb:6} (left) shows his layout of H.~G.~Wells' \textit{The Time
Machine} for Random House in 1931. The page size is $231 \times 147$mm.
The right of the figure illustrates the layout of a book called 
\textit{Two Men --- Walter Lewis and Stanley Morrison at Cambridge}
by Brooke Crutchley and published by Cambridge University Press in 1968.
Crutchley was the Cambridge University Printer and each year would produce
a limited edition of a book about Cambridge or typography, and preferably
both together. This is typeset in Monotype Barbon with $17.5$ leading on
a $26$pc measure on a $253 \times 162$mm page.



\begin{figure}
\centering
\begin{minipage}[b]{\pwlayi}
\drawaspread{\pwlayii}{1.553}{1.685}{.0909}{1.618}{1.618}{0} % Express
\end{minipage}
\hfill
\begin{minipage}[b]{\pwlayi}
\drawaspread{\pwlayii}{1.538}{2.647}{.1}{1}{3.833}{0} % Thames & Hudson
\end{minipage}
\caption[Two spreads: USA, 1994 and England, 1988]%
        {Two spreads: (Left) USA, 1994.
         (Right) England 1988.} \label{fb:7}
\end{figure}

    A modern technical book layout is given in \fref{fb:7}. The book
is \textit{Information Modeling the EXPRESS Way} by Douglas Schenck and Peter
Wilson, published by Oxford University Press (New York) in 1994. This is
set in Computer Modern Roman at $10/12 \times 27$ on a page 
$233 \times 150$mm. Ruari McLean's \textit{The Thames and Hudson Manual of
Typography} (1988) is at the right in \fref{fb:7}. This is typeset in
$10/11 \times 20$ Monophoto Garamond on a $240 \times 156$mm page. The wide
\foredge{} is used for small illustrations\index{illustration}. Notes are also set in this
margin\index{margin} rather than at the foot of the page.



\begin{figure}
\centering
\begin{minipage}[b]{\pwlayi}
\drawaspread{\pwlayii}{1.5}{2}{.2}{.5}{1}{0} % Rome
\end{minipage}
\hfill
\begin{minipage}[b]{\pwlayi}
\drawaspread{\pwlayii}{1.5}{1.7}{.111}{1}{2}{0} % Venice
\end{minipage}
\caption[Two spreads: Italy, 1523 and 1499]%
        {Two spreads: (Left) Italy, 1523.
         (Right) Italy 1499.} \label{fb:8}
\end{figure}

 Many page layouts in earlier days were constructed by
drawing with compass and ruler, usually based on regular geometric figures; 
the use of squares, pentagons and hexagons being particularly
prevelant.
    Unusually, the typeblock\index{typeblock} in \fref{fb:8} (left) is centered on the page.
The typeblock\index{typeblock} is based on a square, the depth being twice the measure.
The book, \textit{Canzone} by Giangiorgio Trissino, is a volume of poems
and was published in Rome about 1523 by Ludovico degli Arrighi. Prose works
from the same typographer followed the normal style of having the fore-edge
wider than the spine margin\index{margin}.
    The page proportion\index{proportion} in \fref{fb:8} (right) is also a simple $3 : 2$ 
ratio. The proportions\index{proportion} of the typeblock\index{typeblock}, being $1.7$, are based upon 
a pentagon.
The book is \textit{Hypnerotomachia Poliphili} by Francesco Colonna and was
published by Aldus Manutius in Venice in 1499. The story of this,
including some reproductions from the original, is told by Helen
Barolini~\cite{BAROLINI92}.


\begin{figure}
\centering
\begin{minipage}[b]{\pwlayi}
\drawaspread{\pwlayii}{1.5}{1.571}{.0769}{2}{3.333}{0} % Magellan
\end{minipage}
\hfill
\begin{minipage}[b]{\pwlayi}
\drawaspread{\pwlayii}{1.5}{1.5}{.111}{1.5}{2}{0} % Gutenberg
\end{minipage}
\caption[Two spreads: France/Portugal, 1530 and Gutenberg, C15th]%
        {Two spreads: (Left) France/Portugal, 1530.
         (Right) Gutenberg, C15th.} \label{fb:9}
\end{figure}

    In 1519 the Portugese explorer Ferdinand Magellan set sail from 
Sanl\'{u}car de Barramada, near C\'{a}diz in Spain, 
with five ships and about 270 men.
Three years later one ship and 18 men returned, having made the first
circumnavigation. Among the few survivors was Antonio Pigafetta who recorded
the adventure. 
A very few manuscripts of his report are in existence.
The layout of one of these manuscripts which is in the Beinecke Rare
Book and Manuscript Library at Yale is shown at the left of \fref{fb:9}.
The manuscript, which is written in French, is called 
\textit{Navigation et descouurement de la Inde superieure et isles
de Malueque ou naissent les cloux de Girosle} (Navigation and discovery
of Upper India and the Isles of Molucca where the cloves grow) is written
in a beautiful humanistic miniscule. There are 27 lines to a page, which
is $286 \times 190$mm and made of vellum. The text measure is $29.5$
and the `leading' is $21$pt. The wide fore margin\index{margin} is used for sidenotes
indicating highlights of the story. The manuscript was probably prepared 
soon before 1530; the scribe and where he worked is unknown.

    Many of the books produced by Johannes Gutenberg (1398--1468)
and his early successors 
followed the form shown in
\fref{fb:9} (right). This set of proportions\index{proportion} was also often used in
medieval incunabula\footnote{Early books, especially those printed before
1500.}  and manuscripts. The page and typeblock\index{typeblock} proportions\index{proportion}
are the same ($3 : 2$). The margins\index{margin} are in the proportions\index{proportion}
$2 : 3 : 4 : 6$.
A graphical method for constructing this, and similar designs, is 
shown later in \fref{flpage:lgut}.

\begin{figure}
\centering
\begin{minipage}[b]{\pwlayi}
\drawaspread{\pwlayii}{1.5}{1.68}{.043}{2}{5}{0} % Persian Mss
\end{minipage}
\hfill
\begin{minipage}[b]{\pwlayi}
\drawaspread{\pwlayii}{1.5}{1.5}{.1}{2}{2.5}{0} % Persian book
\end{minipage}
\caption[Two spreads: Persia, 1525 and USA, 1975]%
        {Two spreads: (Left) Persia, 1525.
         (Right) USA, 1975.} \label{fb:10}
\end{figure}

     Two versions of the same publication are shown in \fref{fb:10}.
On the left is a Persian manuscript \textit{Khamsch of Nizami} written
about 1525. The page size is about $324 \times 216$mm. The 
illustrations\index{illustration} and
the typeblock\index{typeblock} are inextricably mixed. On the right is a translation of
some of the manuscript published as \textit{Tales from the Khamsch of Nizami}
by the Metropolitan Museum of Art, New York, in 1975. The modern version
has a page size of $300 \times 200$mm, slightly smaller than the original
but in the same proportions\index{proportion}. The typeblock\index{typeblock} is $32$pc wide and the type is
set with a $15$pt leading.

\begin{figure}
\centering
\begin{minipage}[b]{\pwlayi}
\drawaspread{\pwlayii}{1.48}{1.375}{.0833}{1.75}{2}{0} % Goudy
\end{minipage}
\hfill
\begin{minipage}[b]{\pwlayi}
\drawaspread{\pwlayii}{1.45}{1.414}{.0333}{2}{4.5}{2.175} % Doomsday
\end{minipage}
\caption[Two spreads: USA, 1952 and England, 1087]%
        {Two spreads: (Left) USA, 1952.
         (Right) England, 1087.} \label{fb:11}
\end{figure}

    Frederic Goudy was a prolific American type designer. Shown at the left of
\fref{fb:11} is the layout of his book \textit{The Alphabet and Elements
of Lettering} published by the University of California Press in 1952.
This is typeset in his University of California Old Style, which has
interesting ct and st ligatures. The measure is $36$pc and the leading
is $18$pt. The first half of the book gives a short history of the development
of writing and fonts. The second half consists of 27 plates, one for each 
letter of the alphabet, and the last one for the ampersand character. These 
show the evolution of each letter from Roman times to the mid-twentieth
century.


    Figure~\ref{fb:11} (right) shows the layout of the English 
\textit{Domesday Book} which is a manuscript book written in 1087. 
It records all the
domains won by William the Conqueror in 1066. The book is written
in a Caroline miniscule
in two columns\index{column!double}, with 44 lines per column ragged right. The two columns
have slightly different widths. The first part of the book is more meticulously
written than the later parts, where the scribe appears to be in haste to
finish.


\begin{figure}
\centering
\begin{minipage}[b]{\pwlayi}
\drawaspread{\pwlayii}{1.414}{1.618}{.111}{1}{2}{0} % A4 orig
\end{minipage}
\hfill
\begin{minipage}[b]{\pwlayi}
\drawaspread{\pwlayii}{1.414}{1.618}{.125}{1}{1.667}{0} % A4 mod
\end{minipage}
\caption[Two spreads for ISO page sizes]%
        {Two spreads: (Left) ISO (1).
         (Right) ISO (2).} \label{fb:12}
\end{figure}

    Figure~\ref{fb:12} shows two different layouts for a page corresponding
to the ISO international standard proportion\index{proportion} of $\sqrt{2}$. In each case
the typeblock\index{typeblock} is the same and proportioned in the 
golden section\index{golden section}, 
but the margins\index{margin} are different. The layout on the left provides adequate
room for marginal\index{marginalia} notes in the fore-edge.


\begin{figure}
\centering
\begin{minipage}[b]{\pwlayi}
\drawaspread{\pwlayii}{1.404}{1.641}{.143}{1}{1.6}{0} % Emery Walker
\end{minipage}
\hfill
\begin{minipage}[b]{\pwlayi}
\drawaspread{\pwlayii}{1.294}{1.618}{.176}{1.037}{1.685}{0} % Latex
\end{minipage}
\caption[Two spreads: England, 1973 and LaTeX $10pt$ book style]%
        {Two spreads: (Left) England, 1973.
         (Right) LaTeX $10$pt book style.} \label{fb:13}
\end{figure}

    Another of the Cambridge Printer's Christmas books is at the left
of \fref{fb:13}. In this case it is \textit{Emery Walker --- Some Light
on his Theories of Printing and on his Relations with William Morris
and Cobden-Sanderson} by Colin Franklin and published in 1973. The
page size is $295 \times 210$mm with a measure of $31$pc set with
$15$pt leading. On the right is the default layout provided by the
LaTeX $10$pt book class on US letterpaper\index{paper!size!letterpaper}.

\begin{figure}
\centering
\begin{minipage}[b]{\pwlayi}
\drawaspread{\pwlayii}{1.294}{1.444}{.0833}{1}{2}{0.5} % Adrian Wilson
\end{minipage}
\hfill
\begin{minipage}[b]{\pwlayi}
\drawaspread{\pwlayii}{1.286}{2.11}{.4}{.625}{.625}{0} % Kuniyoshi
\end{minipage}
\caption[Two spreads: USA, 1967 and England, 1982]%
        {Two spreads: (Left) USA, 1967.
         (Right) England, 1982.} \label{fb:14}
\end{figure}

    Adrian Wilson, who died in 1988, was an acclaimed American book designer.
His work on book design, \textit{The Design of Books}, out of print since
1988 but reissued in 1993 by
Chronicle Books, is outlined
at the left of \fref{fb:14}. This is in two columns\index{column!double}, with many illustrations\index{illustration},
on letterpaper\index{paper!size!letterpaper} size pages. It is typeset in Palatino and Linotype Aldus
with $12$pt leading. Each column is $18$pc wide. The other layout in this
figure is B.~W.~Robinson's \textit{Kuniyoshi: The Warrior Prints} published
by Phaidon, Oxford in 1982. The page size is $310 \times 242$mm with a
measure of $28.5$pc. The type is set with $13$pt leading. The wide spine
margin\index{margin} is used for some small reproductions of Japanese woodblock prints, 
some of which extend across the binding itself. The majority of the book
has no text apart from captioning the many reproduced prints which take 
up full pages.

\begin{figure}
\centering
\begin{minipage}[b]{\pwlayi}
\drawaspread{\pwlayii}{1.25}{1.182}{.1}{1.5}{2}{0} % Fens
\end{minipage}
\hfill
\begin{minipage}[b]{\pwlayi}
\drawaspread{\pwlayii}{1.167}{1.133}{.077}{1}{1}{0.382} % Durer
\end{minipage}
\caption[Two spreads: England, 1972 and Switzerland, 1980]%
        {Two spreads: (Left) England, 1972.
         (Right) Switzerland, 1980.} \label{fb:15}
\end{figure}

    \textit{The Waterways of the Fens} by Peter Eden with drawings by
Warwick Hutton is another of the Cambridge Printer's Christmas books.
This is set with $17$pt leading on a measure of $27$pc. The original
page size is $195 \times 150$mm and is illustrated on the left 
of \fref{fb:15}. 
The amount of text on a page varies
and there are many line drawings, some of which take a double spread.
On the right of this figure is another art book, namely \textit{D\"{u}rer}
by Fedja Anzelewsky published by Chartwell Books in 1980. This is set in
two columns\index{column!double} with $14$pt leading on a $23.5$pc measure, 
although there are more illustrations\index{illustration} than text. The page
size is $280 \times 240$mm, considerably larger than its companion in
the figure, yet with much smaller margins\index{margin}.

\begin{figure}
\centering
\begin{minipage}[b]{\pwlayi}
%\drawaspread{\pwlayii}{1.176}{1.46}{.107}{.833}{2.41}{0.25} % Art
\drawaspread{\pwlayii}{.865}{.951}{.111}{1}{2}{0} % Hammer & Hand
\end{minipage}
\hfill
\begin{minipage}[b]{\pwlayi}
\drawaspread{\pwlayii}{.714}{.667}{.111}{0.5}{0.667}{0.333} % Hokusai
\end{minipage}
\caption[Two spreads: England, 1969 and USA 1989]%
        {Two spreads: (Left) England, 1969.
         (Right) USA, 1989.} \label{fb:16}
\end{figure}


    Two more layouts for illustrated books are given in \fref{fb:16}.
In this case the illustrations\index{illustration} are drawings in landscape mode (i.e., they
are wider than they are high); the shape of the drawings has had a major
effect on the page proportions\index{proportion}. In the case on the left the page
proportion\index{proportion} is in the ratio $\pi : e$. The measure is longer than usual
at $37$pc and to compensate for this the leading of $17$pt 
is also larger than customary. It is typeset in Centaur.
The book is
\textit{Hammer and Hand} by Raymond Lister with drawings by Richard Bawden.
It was published in 1969 by Cambridge University Press and is another of
the University Printer's Christmas books.
Shown on the right of \fref{fb:16} is \textit{Hokusai --- One Hundred Poets}
by Peter Morse and published by George Braziller in 1989. The introductory
text is set in two columns\index{column!double} as shown. The body consists of illustrations\index{illustration} of
Japanese wood block prints, originally in the large \textit{oban} size
of about $250 \times 380$ mm.

\subsection{A geometric construction}

    Nowadays it is easy to pick and calculate any kind of page proportion\index{proportion}
that takes your fancy, but how did the early printers do it? They certainly
did not have the use of calculators and I suspect that they had only enough
arithmetic to keep their accounts. Printing was a craft and craftsmen did
not release their trade secrets lightly. I believe that most of the designs
were based on simple geometric figures, which required nothing more than
a ruler and a pair of compasses.


 Jan Tschichold gives a simple construction for the layout of many of Gutenberg's 
books~\cite[pages 44--57]{TSCHICHOLD91}, which is shown in \fref{flpage:lgut}.
The construction actually divides the page up into ninths (the point
\textsc{p} in the diagram, which is at the intersection of the main and half
diagonal construction lines, is one third of the way down and across both the
page and the typeblock\index{typeblock}). This construction can be used no matter what the
page proportions\index{proportion} and will give the same relative result.


\begin{figure}
\centering
\setlength{\unitlength}{1pc}
\begin{picture}(20,15)
\put(0,0){\framebox(20,15){}}
\thicklines
 \put(10,0){\line(0,1){15}} % spine
\put(0,0){\line(4,3){20}} % ll to tr diag
\put(20,0){\line(-4,3){20}} % lr to tl diagonal
\put(0,0){\line(2,3){10}}  % ll to tm line
\put(20,0){\line(-2,3){10}} % lr to tm line
\put(13.333,0){\line(0,1){15}} % vertical line thro' half & full diags
\put(14,10){\makebox(0,0)[tl]{\textsc{p}}}
\put(13.333,15){\line(-4,-3){6.667}} % last line
\thinlines
\put(11.111,3.333){\framebox(6.667,10){}}
\put(2.222,3.333){\framebox(6.667,10){}}
\end{picture}
\setlength{\unitlength}{1pt}
\caption{The construction of the Gutenberg page design}
\label{flpage:lgut}
\end{figure}

\index{spread|)}

\section{The typeblock} \label{sec:tblock}


    The typeblock\index{typeblock} is not just a rectangular block of text. If the typeblock\index{typeblock}
does consist of text, then this will normally be broken up into paragraphs\index{paragraph};
it is not good style to have paragraphs that are longer than a page. Also,
the typeblock\index{typeblock} may include tables\index{table} and illustrations\index{illustration} which provide relief from
straight text. Some pages may have chapter or section headings\index{heading}
 on them which
will also break the run of the text. In general the typeblock\index{typeblock} will
be a mixture of text, white space, and possibly non-text items.

    Consider a typeblock\index{typeblock} that includes no illustrations\index{illustration} or tables\index{table}.
The lines of text must be laid out so that they are easy to read.
Common practice, and more recently psychological testing, has shown that
long lines of text are difficult to read. Thus, there is a physiological
upper limit to the width of the typeblock\index{typeblock}. From a practical viewpoint,
a line should not be too short because then there is difficulty in justifying
the text.

    Experiments have shown that the number of characters in a line of
single column\index{column} text on a page should be
in the range 60 to 70 for ease of reading. The range may be as much
as 45 to 75 characters but 66 characters is often
considered to be the ideal number. Much shorter and the eye is dashing
back and forth between each line. Much longer it is hard to pick up the
start of the next line if the eye has to jump back too far --- the same line
may be read twice or the following line may be inadvertently jumped over.
For double column\index{column!double} text the ideal number of characters is around 45, 
give or take 5 or so.

    Bringhurst~\cite{BRINGHURST92} gives a method for determining the number
of characters in a line for any font\index{font!measuring}: 
measure the length of the lowercase
alphabet and use a copyfitting\index{copyfitting} 
table that shows for a given alphabet 
length and line length, the average number of characters in that line.
 Table~\ref{tab:copyfitting} is an
abridged version of Bringhurt's copyfitting table.
For example, it suggests that a font with a length of 130pt should be
set on a measure of about 26pc for a single column\index{column!double} or in an 18pc wide
column if there are multiple\index{column!multiple} columns.
 

\begin{table}
\DeleteShortVerb{\|}
\centering
\caption{Average characters per line} \label{tab:copyfitting}
\begin{tabular}{r|rrrrrrrr} \hline
Pts. & \multicolumn{8}{c}{Line length in picas} \\
     & 10 & 14 & 18 & 22 & 26  & 30  & 35 & 40 \\ \hline
80   & \textit{40} & \textbf{56} & \textbf{72} & 88 & 104 &     &    &    \\
85   & \textit{38} & \textit{53} & \textbf{68} & 83 & 98 & 113 &    &    \\
90   & \textit{36} & \textit{50} & \textbf{64} & 79 & 86 & 107 &    &    \\
95   & 34 & \textit{48} & \textbf{62} & 75 & 89 & 103 &    &    \\
100  & 33 & \textit{46} & \textbf{59} & \textbf{73} & 86 & 99 & 116 &   \\
105  & 32 & \textit{44} & 57 & \textbf{70} & 82 & 95 & 111 &   \\
110  & 30 & \textit{43} & 55 & \textbf{67} & 79 & 92 & 107 &   \\
115  & 29 & \textit{41} & 53 & \textbf{64} & 76 & 88 & 103 &   \\
120  & 28 & \textit{39} & \textit{50} & \textbf{62} & 73 & 84 & 98 & 112 \\
125  & 27 & 38 & \textit{48} & \textbf{59} & \textbf{70} & 81 & 94 & 108 \\
130  & 26 & 36 & \textit{47} & 57 & \textbf{67} & 78 & 91 & 104 \\
135  & 25 & 35 & \textit{45} & 55 & \textbf{65} & 75 & 88 & 100 \\
140  & 24 & 34 & \textit{44} & 53 & \textbf{63} & 73 & 85 & 97 \\
145  & 23 & 33 & \textit{42} & 51 & \textbf{61} & \textbf{70} & 82 & 94 \\
150  & 23 & 32 & \textit{41} & \textit{51} & \textbf{60} & \textbf{69} & 81 & 92 \\
155  & 22 & 31 & \textit{39} & \textit{49} & 58 & \textbf{67} & 79 & 90 \\
160  & 22 & 30 & 39 & \textit{48} & 56 & \textbf{65} & 76 & 87 \\
165  & 21 & 30 & 38 & \textit{46} & 55 & \textbf{63} & 74 & 84 \\
170  & 21 & 29 & 37 & \textit{45} & 53 & \textbf{62} & 72 & 82 \\
175  & 20 & 28 & 36 & \textit{44} & 52 & \textbf{60} & \textbf{70} & 80 \\
180  & 20 & 27 & 35 & \textit{43} & 51 & 59 & \textbf{68} & 78 \\
185  & 19 & 27 & 34 & \textit{42} & \textit{49} & 57 & \textbf{67} & 76 \\
190  & 19 & 26 & 33 & 41 & \textit{48} & 56 & \textbf{65} & 74 \\
195  & 18 & 25 & 32 & 40 & \textit{47} & 54 & \textbf{63} & 72 \\
200  & 18 & 25 & 32 & 39 & \textit{46} & 53 & \textbf{62} & \textbf{70} \\ 
220  & 16 & 22 & 29 & 35 & \textit{41} & \textit{48} & 56 & \textbf{64} \\
240  & 15 & 20 & 26 & 32 & 38 & \textit{44} & 51 & 58 \\
260  & 14 & 19 & 24 & 30 & 35 & 41 & \textit{48} & 54 \\
280  & 13 & 18 & 23 & 28 & 33 & 38 & \textit{44} & 50 \\
300  & 12 & 17 & 21 & 26 & 31 & 35 & 41 & \textit{47} \\
320  & 11 & 16 & 20 & 25 & 29 & 34 & 39 & \textit{45} \\
340  & 10 & 15 & 19 & 23 & 27 & 32 & 37 & 42 \\
\hline
\end{tabular}
\MakeShortVerb{\|}
\end{table}



   The vertical height of the typeblock\index{typeblock} should be constant from page to
page. The lines of text on facing pages should be aligned
horizontally across the spine, which also means that they will be at
the same place on both sides of a leaf. Alignment across the spine
means that the eye is not distracted by an irregularity at the centre
of a spread, and leaf alignment stops ghosting of text through a thin page,
giving a crisper look to the work. 
So, the spacing between lines should
be constant. This implies that the depth of the typeblock\index{typeblock} should be an
integral multiple of the space required for each line; that is, be specified
as a multiple of the leading. A ten point type, for example, will normally
have two points between lines, to give a leading of 12 points. This can be
written as 10/12. Usefully, one pica is 12 points so with a 12pt leading
vertical distances can be conveniently expressed in picas 
(one pica per line). Another
implication of this is that any space left for illustrations\index{illustration} or tables\index{table}, or
the amount of space taken by chapter and section headings\index{heading} 
should also be
an integer multiple of the leading\index{leading}.

    A ten point type set solid is described as 10/10. The theoretical
face of the type is ten points high, from the top of a \emph{d} to the bottom
of a \emph{p}, and the distance of the baseline of one row of text to the
next row of text is also ten points. Note that if a \emph{p} is vertically
above a \emph{b} then the ascender of the \emph{b} will meet the descender
of the \emph{p}. To avoid this, the vertical separation between baselines 
is increased above the type size. Adding two extra points of vertical space
allows the text to breathe, and gives a leading of 12 points. Few fonts
read well when set solid. Typical settings are 9/11, 10/12, 11/13 and 12/15.
Longer measures require more leading than shorter ones, as do darker and larger
fonts compared with lighter and smaller fonts. More leading is also
useful if the text contains many super-\index{superscript} or sub-scripts, 
or many uppercase 
letters.



\subsection{Page color}

    One of the aims of the typographer is to produce pages that are uniform
in `color'\index{page color}. 
By this they mean that the typeblock\index{typeblock} has a reasonably constant
grayness, not being broken up by too much white space which is a distraction
to the reader. There will be white space around headings\index{heading}, 
which is acceptable
as a heading is meant to attract attention. There may be white space between 
paragraphs\index{paragraph}, but this is usually under the control of the designer. There 
may be vertical rivulets of white space when the interword spaces on
adjacent lines coincide; fixing this usually requires some handwork, either
by the author changing his wording so as to alter the location of
the spaces, or by the typesetter tweaking a little bit. 

    Another form of distraction is if too many lines end with hyphens, or
several adjacent lines start or end with the same text; this not only
will cause a rivulet but will make it harder for the reader to reliably
jump to the next line.

    The main font used controls the depth of the color of a page. To
see what color is produced by a particular font it is necessary to look
at a fairly long, preferably a page, piece of normal text. Fonts from
different families produce different colors, and so may mixed fonts from 
the same family. You can try this yourself by typesetting the same page
in, say, Computer Modern Roman, Italic, and Sans fonts.
The books by Rogers~\cite{ROGERS43}, Lawson~\cite{LAWSON90},
Dowding~\cite{DOWDING98}, and 
Morison~\cite{MORISON99} all show pages set in many different fonts.

    

\subsection{Legibility}

    One of the principle requirements on the typography of a document is that
the document is \emph{legible}. Legibility means that the document is designed
to be easily read under a certain set of circumstances. The criteria for
legibility on a poster that is placed on the side of a bus, for example, are
different from those that apply to a book to be read while sitting in an
easy chair. Essentially, the viewer should be able to read the document
with no physical strain caused by the appearance, but the contents, of course,
may lead to anything ranging from acute mental strain to extreme boredom.

    Type faces and the layout of the typeblock\index{typeblock} must be chosen to optimise
between legibility and `artistic' presentation. The design of the document
should be almost invisible, giving full compliments to the author's
communication. However, if you are a master, like
Hermann Zapf~\cite{ZAPF00}, you can break the rules.

    

\subsubsection{Type faces}

    The first European letter forms that have survived are Greek inscriptions
carved into stone. These were freehand carvings with thin strokes. In time,
the lettering became thicker and serifs started to appear. The Romans
picked up on this later style of letter form. In carving inscriptions, they
first wrote the inscription on the stone using a broad, flat brush. This
naturally led to serifs and differing thicknesses of the letter strokes,
depending on the angle of the stroke with respect to the movement and
orientation of the brush.

    Between the Roman times and Gutenberg there were many changes and
experiments in European letterforms. The scribes used different scripts
for titles, subheads, continuous text, illuminated initial letters, and so
on. In time, two families of letterforms evolved, called \emph{majuscules}
and \emph{miniscules}. The former were larger and more formal, while the latter
were smaller and less formal. We now call these two divisions upper case and
lower case. The upper case derives from Roman times, while the lower case
acquired its fundamental form during the reign of the Holy Roman Emperor
Charlemagne a thousand years later.
A further division also appeared, between black letter (what is
commonly referred to as Gothic or Old English) type and the roman type.

    These types were all upright. Italic letterforms were cut in Italy
in the early sixteenth century, as a more cursive style. Initially these were
lower case only, used in conjunction with upper case roman. By the end of
the century, sloped roman capitals were also in use with italic.

\index{font!seriffed|(}
\index{font!sans|(}
    The late nineteenth century saw the appearance again of sans-serif
typefaces.

    Looking carefully at seriffed and sans-serif fonts it is apparent that
the serifs have three main functions:
\begin{enumerate}
\item They help to keep letters apart.
\item At the same time, they help to keep letters in a word together. This
  helps with legibility as research has shown that we tend to recognize
  words by the shape of the word rather than by individual characters.
\item They help to differentiate between individual but similar letters.
\end{enumerate}

Long experience has shown that a seriffed font is easier to 
read\footnote{This is actually somewhat contentious as some take the view
that with enough practice, sans-serif is just as easy to read.}
than a
sans-serif font, particularly if part of the text is obscured. You can
try an experiment yourself to verify this. Try writing a phrase, once
using a sans-serif font and then with a serifed font. Cover up
the top halves of the two phrases and try to make out what they say. Then
repeat this, except this time cover up the bottom halves of the phrase.
Which is easier to read? Here are some example characters, firstly in san-serif:
\begin{center}
{\Huge\sffamily a c l m n p q o}
\end{center}
and then in roman:
\begin{center}
{\Huge a c l m n p q o}
\end{center}

    Sans-serif fonts often require context to decipher the word. 
For example~\cite{MCLEAN80},
seeing this in isolation
\begin{center}
{\Huge\sffamily lll}
\end{center}
does it stand for `Ill', `one hundred and eleven', `three', or something
completely different like a dingbat or a set of cricket stumps?

    
    There are three generally agreed legibility principles for setting text for
continuous reading.

\begin{enumerate}
\item \emph{Sans-serif type is intrinsically less legible than seriffed type}~\cite{WHEILDON95}.

    We have already seen that this is the case --- there is more variety
among seriffed letters than among sans-serif letters. Further, serifs
perform other functions as well, such as binding letters together within
a word.

    This is not to say that a sans-serif letterform is always more illegible
than a roman one. A poor seriffed form can be much more illegible than
a well used good sans-serif. In general, there is an illegibility factor
associated with sans-serif that must be borne in mind; for general
\emph{continuous} reading, a good seriffed form is more likely to be
easy on the eye than a good sans form.

\item \emph{Well designed upper and lower case roman type is easier to read than
any of its variants.}

    This is a guiding principle with many exceptions. Among the variants
can be considered to be italic and bold types. These have usually been
designed for a special purpose, like emphasing\index{emphasis} 
certain pieces of text, rather
than for general legibility. Some italic types, though, are as legible as their
roman counterparts. In the seventeenth century many books were set entirely
in italic, but we have become accustomed to the roman type.

\item \emph{Words should be set closer together than the space between lines.}

    All text is a mixture of ink and white space. The eye, when reading, 
tends to jump over the white spaces. Given a choice between two spaces, it 
will tend to jump over the smaller of the two. If the word spacing is greater
than the line spacing, then you can find yourself skipping from one line
to the next before finishing the first one.

    Further, if the lines are too long, then when the eye jumps back from
the end of one line to the start of the next, it may have difficulty in 
picking up the correct one.

    Text lines are justified by altering the inter-word spacing, and possibly
by hyphenating the last word on the line if the spacing would be too bad
otherwise. Sans-serif fonts often look best if set ragged right, as this will
keep the inter-word spacing constant. Text set in narrow columns\index{column} also often
looks best when set ragged right.


\end{enumerate}

\subsubsection{Seriffed versus sans-seriffed fonts}

    As noted earlier there seems to be a permanent debate over the use
of seriffed and sans fonts. You will have to make up your own mind as
to what is best for any particular work, but here are some general
comments from some of the literature on the subject.

\begin{itemize}
\def\makelabel#1{\noindent #1}
\item[Bohle~\cite{BOHLE90}] notes: Readers prefer a roman typeface for body
  type because they are most used to seeing that face~\cite{REHE72}.
  Roman type may well also be more readable than sans serif faces because
  the serifs help connect the letters to form the word shape when
  we read~\cite{REHE72}.

\item[Craig~\cite{CRAIG92}] says: You will find that the serifs on a typeface
  facilitate the horizontal flow necessary to comfortable reading.

\item[Degani~\cite{DEGANI92}] in a study of pilots reading checklists
  in emergency cockpit situations decided that sans serif faces were
  better than serif faces.

\item[Schriver~\cite{SCHRIVER97}] notes: Serif and sans serif typefaces 
  are likely to be equally preferred by 
  readers~\cite{HARTLEY83,TINKER63}
  and read equally quickly~\cite{GOULD87,HARTLEY83,ZACHRISSOM69}.
  Serif faces may be easier to read in continuous text than sans
  serif faces~\cite{BURT59,HVISTENDAHL75,ROBINSON71,WHEILDON95}.

\item[Wheildon~\cite{WHEILDON95}] did a series of studies with around
250 readers in Sydney, Australia, asking them to rate serif and sans fonts
in a variety of uses. Among the many results he reported:
\begin{itemize}
  \item More than five times as many readers are likely to show good
  comprehension when a serif body type is used instead of a sans serif
  body type.
  \item The top half of [upper case] letters is more recognizable than
  the bottom half.
  \item There is little difference in legibility between headlines
  [section titles] set in serif and sans serif typefaces, or between
  roman and italic.
  \item Headlines set in capital letters are significantly less legible
  than those set in lower case.
\end{itemize}

\end{itemize}

    The consensus, such as it is, seems to lean towards serifed
typefaces for continuous reading, but for titling the choice is
wide open.

\index{font!sans|)}
\index{font!seriffed|)}

\subsection{Widows and orphans}

    Inconvenient page breaks can also cause a hiatus in the reader's perusal
of a work. These happen when a page break occurs near the start or end of
a paragraph\index{paragraph}. 

    A \emph{widow}\index{widow} is where the last line of a paragraph\index{paragraph} is the first
line on the page. The term is sometimes also used to refer to when the
last word in a paragraph\index{paragraph} is on a line by itself. A widow looks forlorn. 
As Robert Bringhurst said, `A widow has a past but no future'.
Typographically, widows should be avoided. Especially to be avoided 
are widows that are
the only line on a page, for example at the end of a chapter\index{chapter}. Five lines
on the last page of a chapter\index{chapter} is a reasonable minimum.

    An orphan\index{orphan} is not nearly so troubling to typographers as a widow. An
\emph{orphan} is where the first one or two lines of a paragraph\index{paragraph} are at the
bottom of a page. Bringhurt's memory trick for orphans is, `An orphan has
a future but no past'.

\subsection{Paragraphs and versals} \label{sec:versal}

\index{paragraph|(}

    Early books did not have paragraphs as we know them nowadays; the text
was written continuously, except for a break at a major division like the
start of a new book in a bible. Instead
the scribes used a symbol like \P\ (the pilcrow) to mark the beginning of
paragraphs. This symbol is derived from the Greek $\Pi$, for
\textit{par\'{a}graphos}. Mind you, they often did not use any punctuation
at all and were sparing in their use of uppercase letters, 
so you might have seen something like this\footnote{But probably not.
The two `paragraphs' are Latin abecedarian sentences.}

\begin{quote}
usque \P\ te canit adcelebratque polus rex gazifier hymnis 
       \P\ transzephyrique globum scandunt tua facta per axem
\end{quote}

    Nowadays paragraphs are ended by stopping the line of text at 
the end of the
paragraph, and then starting the next paragraph on a new line. The question
then becomes: how do you indicate a new paragraph when the last line of the
previous paragraph fills up the measure? There are two solutions, which 
unfortunately you
sometimes see combined. Either indent the beginning of the first line of
each paragraph, or put additional vertical space between the last and
first lines of paragraphs.

    The traditional technique, which has served well for centuries, is to
indent the first line of a paragraph\index{paragraph!indentation}. The indentation need not be large,
about an em will be enough, but more will be required if the typeblock\index{typeblock}
is wide.

    The other method is used mainly in business letters and is a recent 
invention. The first lines of paragraphs\index{paragraph!indentation} are not indented and typically
 one blank line is left between paragraphs. This may
perhaps be acceptable when using a typewriter, but seems to have no real
justification aesthetically. There is also the problem when a paragraph
both ends with a full line and ends a page. As the next paragraph then starts
at the top of the next page, the blank line separating the two paragraphs
has effectively dissappeared, thus leaving the reader in a possible state
of uncertainty as to whether the paragraph continues across the page break
or not.

    If the paragraph is the first one after a heading\index{heading}, 
then there
is no need to indicate that it is a new paragraph --- it is obvious from its
position. So, the first paragraph after a heading\index{heading} 
should not be indented\index{paragraph!indentation}.
In some novels only chapters are headed yet each chapter is broken into
sections by putting additional vertical blank space between the sections.
Like nonindented paragraphs\index{paragraph!indentation}, this can cause problems where a section division
coincides with a page break. In this case, typographers sometimes use a
decoration to separate sections (for example, a short centered row of a few
asterisks).
   

%%%%%%%%\clearpage
%\subsection{Versals}

\drop{S}{}\textsc{ome typographers}\index{versal|(} 
like to start the first paragraph in a chapter
with a versal. A \emph{versal} is a large initial letter, either raised or
dropped. This comes from the scribal tradition of illuminating the first
letter of a manuscript. The versal may be raised or dropped, as already noted,
or it may be placed in the margin\index{margin}, or otherwise treated in a special manner.

\versal{S}\textsc{ome versals,} especially dropped versals, are very difficult
to typeset correctly. Many attempts of this kind are abject failures, so
be warned. For example, compare the dropped versals at the start of these
first two paragraphs. They are both of the same letter and font, yet the first
one is horrible compared to the one starting this paragraph.


\noindent {\huge I}\textsc{t is easier} to start a paragraph with 
a raised
capital than one that is dropped. A raised versal should only be used
where there is naturally some vertical space above it. As you can see, extra
spacing has had to be inserted before this paragraph to accomodate the versal.
There are still problems with typesetting a raised versal but as these tend
to be subtler than with a dropped versal, readers are less likely
to notice problems.

Typically, small caps are used for a little while following a versal to 
provide a transition between the large versal font and the normal body font.
These should not continue throughout the first line as this tends to divorce
it from the remainder of the paragraph. \index{versal|)}

\noindent \textsc{Another way of starting} a paragraph is to use small
caps for the first few words. The font difference highlights the start
of the paragraph but in a much quieter manner than a versal does. Using
normal sized upper-case instead of the small caps is too much of a 
contrast with the lower-case.

\index{paragraph|)}

\subsection{Footnotes}

\index{footnote|(}
    Footnotes are considered to be part of the typeblock\index{typeblock}. They are
typeset in the space allocated for the typeblock\index{typeblock}, in contrast to 
footers\index{footer}
which are typeset below the typeblock\index{typeblock}.

    Footnotes are normally set in the same type style as the typeblock\index{typeblock}. 
That is, if an upright seriffed font is used for the typeblock\index{typeblock}, it is
also used for the footnote. The
type size is smaller to distinguish the note from the body text and often
the leading in the footnote is also reduced from that in the main text body.
The bottom footnote line should be at the same height as the bottom line
of the typeblock\index{typeblock}. This usually requires some adjustment of the vertical 
space before the footnotes.

    A vertical blank space is often used to set off the footnotes from the
main text, and sometimes a short horizontal line is also used as demarcation.

\index{footnote|)}

\section{Folios}

\index{folio|(}

    The word \emph{folio} is a homonym. It can mean a leaf 
(two back-to-back pages) in a book, the size of a book or a book of
folded sheets (as in Shakespeare's first folio), or the printed page number
in a book. Here I use folio in this last sense.

    Documents should have folios, at a minimum to help the reader know where
he is. Occasionally books have their folios placed near the spine but this
positioning is unhelpful for navigation. The more usual positions are
either centered with respect to the typeblock\index{typeblock} or aligned with the outside
of the typeblock\index{typeblock}, and sometimes even in the outside margin\index{margin}. The folios
can be either at the top or bottom of the page but at least on pages 
with chapter\index{chapter} openings are normally placed at the bottom of the page so that
they do not distract from the title text.

    Every page in a document, except perhaps the title and half title pages
in a book, should be numbered, even if the page does not have a folio. 
In books, the folios for the front material are often in roman numerals.
The main and rear matter folios are arabic numerals, with the sequence
starting from 1 after the front matter. In certain technical documents,
folios may be in the form of chapter number\index{chapter!number} and page number, with the page 
number starting from 1 in each new chapter. Other folio schemes are possible 
but unusual.

    Folios should be placed harmoniously with respect to the typeblock\index{typeblock} and
page margins\index{margin}. The font used for the folios need not be the same as that
for the typeblock\index{typeblock} but must at least be complementary and non-intrusive.

\index{folio|)}

\section{Headers and footers}

\index{header|(}\index{footer|(}
    Headers and footers are repetitive material that is placed at either 
the head or the foot of the page. Typically, folios\index{folio} are headers or footers,
but not always as sometimes they are placed in the margin\index{margin} at or below the
first line in the typeblock.\index{typeblock}

   From now on I will not distinguish between headers and footers and 
just use the word header. Sometimes the header is purely decorative (apart
from a folio\index{folio}) like a horizontal line or some other non-textual marking.
Normally they have a functional use in helping the reader locate himself
in the document.

    The most ubiquitous header is one which gives the title of the document.
If this is the only header, then I consider this to be decorative rather
than functional. As a reader I know what document I am reading and do not
need to be reminded every time I turn a page. More useful are headers that
identify the current part of the document, like a chapter\index{chapter} title or number.
When you put the document down and pick it up later to continue reading, these
help you find your place, or if you need to refer back to a previous chapter
for some reason, then it is a boon to have a chapter heading\index{heading} 
on each 
spread. The minimally functional headers are where the document title
is on one page and the chapter heading is on the facing page. In more technical
documents it may be more useful to have headers of chapter and section titles
on alternate pages. 

    Occasionally both headers and footers are used, in which case one normally
has constant text, like a copyright\index{copyright} notice. 
I have the feeling that using
the latter is only functional for the publishers of the document
when they fear photocopying or some such.

    The header text is usually aligned with the spine side 
of the typeblock\index{typeblock}, but may be centered on top of the typeblock\index{typeblock}. In any event,
it should not interfere with the folio\index{folio}. The type style need not be the same
as the style for the typeblock\index{typeblock}. For example, headers could be set in italic
or small caps, which must blend with the style used for the 
typeblock.\index{typeblock}

\index{footer|)}\index{header|)}

\chapter{Picky points}

\section{Introduction}

    The main elements of good typography are legibility and page color.
This chapter discusses some of the smaller points related to these topics.

\section{Word and line spacing}

    Research has shown that the competent reader recognises words by
their overall shape rather than by stringing together the individual letters
forming the words. A surprisingly narrow gap between words
is sufficient for most to distinguish the word boundaries.

    Most typographers state that the space between words in continuous
text should be about the width of the letter `i'. Any closer and the
words run together and too far apart the page looks speckled with white
spots and the eye finds it difficult to move along the line rather than
jumping to the next word in the next line. 
    Figure~\ref{fig:interword} illustrates different values of interword
spacing.

\setlength{\unitlength}{\fontdimen2\font}
\begin{figure}
\centering
%\fbox{%
\begin{minipage}{\textwidth}
\mbox{}\hrulefill\mbox{}
\begin{quotation}
\fontdimen2\font=2\fontdimen2\font
    The following paragraph is typeset with double the normal interword 
spacing for this font.

    Most typographers state that the space between words in continuous
text should be about the width of the letter `i'. Any closer and the
words run together and too far apart the page looks speckled with white
spots and the eye finds it difficult to move along the line rather than
jumping to the next word in the next line. 
Extra spacing after punctuation is not necessary.
\end{quotation}
\begin{quotation}
\fontdimen2\font=\unitlength
    The following paragraph is typeset with the normal interword spacing 
for this font.

    Most typographers state that the space between words in continuous
text should be about the width of the letter `i'. Any closer and the
words run together and too far apart the page looks speckled with white
spots and the eye finds it difficult to move along the line rather than
jumping to the next word in the next line. 
Extra spacing after punctuation is not necessary.
\end{quotation}
\begin{quotation}
%\settowidth{\unitlength}{i}
\settowidth{\fontdimen2\font}{i}
    The interword spacing in the following paragraph is the width 
of the letter `i'.
 
    Most typographers state that the space between words in continuous
text should be about the width of the letter `i'. Any closer and the
words run together and too far apart the page looks speckled with white
spots and the eye finds it difficult to move along the line rather than
jumping to the next word in the next line. 
Extra spacing after punctuation is not necessary.
\end{quotation}
\mbox{}\hrulefill\mbox{}
\end{minipage}
%} % end fbox
\fontdimen2\font=\unitlength \setlength{\unitlength}{1pt}
\caption{Interword spacings}\label{fig:interword}
\end{figure}

    In keeping with avoiding white spots, many typographers do not
recommend extra spacing after punctuation, although this does depend
partly on a country's typographic history and partly on the individual.
I always found typewritten texts with double spaces after the end
of sentences a particular eyesore. However, with typeset texts any 
extra spacing is usually not as large as that.

    The spacing between lines of text should be greater than the interword
spacing, otherwise there is a tendency for the eye to skip to the
next line instead of the next word. Figure~\ref{fig:interline} illustrates
some text typeset with different line spacings. The normal interword
spacing is used in the samples.

\begin{figure}
\centering
%\fbox{%
\begin{minipage}{\textwidth}
\mbox{}\hrulefill\mbox{}
\normalfont\setlength{\unitlength}{\baselineskip}
\begin{quotation}
\normalfont\setlength{\baselineskip}{1em}
    This paragraph is set solid --- the interline spacing is the same
as the font size. The normal interword spacing is used.
    The spacing between lines of text should be greater than the interword
spacing, otherwise there is a tendency for the eye to skip to the
next line instead of the next word. \par
\end{quotation}
\begin{quotation}
\normalfont\setlength{\baselineskip}{\unitlength}
    This paragraph is set with the normal interline spacing for the font.
The normal interword spacing is used.
    The spacing between lines of text should be greater than the interword
spacing, otherwise there is a tendency for the eye to skip to the
next line instead of the next word. \par
\end{quotation}
\begin{quotation}
\normalfont\setlength{\baselineskip}{1.2\unitlength}
    This paragraph is set with the interline spacing 20\% greater than
is normal for the font.
The normal interword spacing is used.
    The spacing between lines of text should be greater than the interword
spacing, otherwise there is a tendency for the eye to skip to the
next line instead of the next word. \par
\end{quotation}
\mbox{}\hrulefill\mbox{}
\end{minipage}
%} % end fbox
\normalfont\setlength{\baselineskip}{\unitlength}
\setlength{\unitlength}{1pt}
\caption{Interline spacings}\label{fig:interline}
\end{figure}

\section{Abbreviations and acronyms}

\index{abbreviation}
    The English style with abbreviations is to put a full stop (period) after
the abbreviation, unless the abbreviation ends with the same letter as the
full word. Thus, it is Mr for Mister, Dr for Doctor, but Prof.~for Professor.
No extra spacing should be used after the full stop, even if extra
spacing is normally used after punctuation.

    Acroynms\index{acronym} are typeset in uppercase but the 
question is, which uppercase?
The simple way is to use the uppercase of the normal font, like UNICEF, but
if there are too many acronyms scattered around the speckled effect starts
to intrude. If the font family has one, then small caps can be used,
giving \textsc{unicef}. If small caps are not available, or appear
undesireable, then a smaller size of the normal uppercase can be used,
such as {\small UNICEF} or {\footnotesize UNICEF}; some experimentation
may be required to select the appropriate size.

\section{Dashes and ellipses}

\index{dash|(}
    Most fonts provide at least three lengths of dashes. The shortest is
the hyphen (-), then there is the en-dash (--) which is approximately the
width of the letter `n', and the largest is the em-dash (---) which is
approximately twice the length of an en-dash. An expert font may provide
more.

   Unsurprisingly, the hyphen is used for hyphenation, such as in em-dash, or
at the end of a line where a word had to be broken.

    The en-dash is normally used between numerals to indicate a range. For
example a reference may refer to pages 21--27 in some journal or book. There
is no space surrounding the en-dash when used in this manner.

    The em-dash, or the en-dash, is used as 
punctuation --- often when making a side 
remark --- as a phrase separator.
 When en-dashes are used as punctuation it is normal to put spaces around them
but the question of spaces around an em-dash appears to be the subject of
some contention. Roughly half the participants in any discussion advocate
spaces while the other half view them as anathema. If you do use em-dashes
be sure to be consistent in your use, or otherwise, of spaces.

    Ellipses\index{ellipses} are those three, or is it four, 
dots indicating something is
missing or continues somewhat indefinitely. In the middle of a sentence,
or clause or \ldots\ they have a space on either side. At the end of
a sentence the English style is to have no spaces and include the full
stop, making four dots in all, like so\ldots.

   Dashes are also used to indicate missing characters or a word. Missing
characters in the middle of a word are indicated by a 2em-dash (a dash that
is twice as long as an em-dash), as in:
\begin{quote}
\textbf{snafu,} \textit{(U.S. slang)} \textit{n.} chaos. --- \textit{adj.}
  chaotic. [\textit{s}ituation \textit{n}ormal --- \textit{a}ll
  \textit{f}------d \textit{u}p.]
\end{quote}
A 3em-dash is used to indicate a missing word. When I lived in Maryland my
local small town newspaper was the \textit{Frederick Post.} 
The following is from an 
obituary I happened to read; I have hidden the name to protect the 
innocent.
\begin{quote}
  Although he had spent the last 92 years of his life here, 
Mr. --------- was not a Fredericktonian.
\end{quote}

\index{dash|)}

\section{Punctuation}

\subsection{Quotation marks}

\index{quotation marks|(}
    Quotation marks surrounding speech and associated punctuation 
are a fruitful source of confusion.

    The American style is to use double quotes at the start(``) and 
end ('') of spoken words. If the speaker quotes in the speech then single
quote marks (` and ') are used to delineate the internal quotation\index{quotation}.

    The English practice is exactly the opposite. Main speech is delineated
by single quotes and internal quotations\index{quotation} by double quotes. In any event,
if single and double quotes are adjacent they should be separated by a thin
space in order to distinguish one from the other --- a full interword space
is too wide.

    As there are likely to be few internal quotations\index{quotation} it seems to me that
the English practice produces a less spotty appearance than the American.
Figure~\ref{fig:qmarks} shows the same text typeset in both the English
and American styles. The example is from Lewis Carroll's 
\emph{Through the Looking Glass and what Alice Found There} 
and has an unusually large number of internal quotations\index{quotation}. 

\begin{figure}
\centering
\begin{minipage}{\textwidth}
\mbox{}\hrulefill\mbox{}
\begin{quotation}
    `There's glory for you!' 

    `I don't know what you mean by ``glory''\,', Alice said. 

    Humpty Dumpty smiled contemptuously. `Of course you don't --- till I tell
you. I meant ``there's a nice knock-down argument for you''!' 

    `But ``glory'' doesn't mean ``a nice knock-down argument''\,', Alice
objected. 

    `When \emph{I} use a word', Humpty Dumpty said, in a rather scornful
tone, `it means just what I choose it to mean --- neither more nor less'.
\end{quotation}
\mbox{}\hrulefill\mbox{}
\begin{quotation}
     ``There's glory for you!'' 

    ``I don't know what you mean by `glory,'\,'' Alice said. 

    Humpty Dumpty smiled contemptuously. ``Of course you don't --- till I tell
you. I meant `there's a nice knock-down argument for you!'\,'' 

    ``But `glory' doesn't mean `a nice knock-down argument,'\,'' Alice
objected. 

    ``When \emph{I} use a word,'' Humpty Dumpty said, in a rather scornful
tone, ``it means just what I choose it to mean --- neither more nor less.''
\end{quotation}
\mbox{}\hrulefill\mbox{}
\end{minipage}
\caption{Quotation marks: top English, bottom American}\label{fig:qmarks}
\end{figure}

    Where to put punctuation marks with quotes is vexatious. Again the
English and American practice tends to differ. The American tendency is
to put commas and periods inside the closing quote mark and colons and
semicolons after the mark. English editors prefer to put punctuation
after the mark.
In either case, it is difficult
to know exactly what to do. I get the impression that for every example
of the `correct' form there is a counter-example.
Some try and avoid the problem altogether by putting the lower marks, 
like commas or periods, 
directly below the quotation mark but that may cause problems if the 
resulting constructs look like question or exclamation marks. 
In \fref{fig:qmarks} I have tried to use the English and American 
punctuation styles in the respective examples but it is likely that there
are misplacements in both. I think it's basically a question of doing what
you think best conveys the sense, provided there is consistency.
\index{quotation marks|)}

\subsection{Footnote marks}

\index{footnote!mark|(}
    Where to put a footnote marker may be another vexed question in spite
of the general principal being easy to state: The mark goes immediately
after the text element that the note refers to.

    There is no doubt what this means\footnote{Except to some I know.} when
the text element is a word in the middle of other words. Doubt raises
its head when the reference is to a phrase, like this one\footnote{I hope
that this is a phrase.\label{fn:phrase}}, 
which is set off within commas, or when the note refers to a complete 
sentence.\footnote{Is this mark in the correct place?\label{fn:sentence}}

    Like punctuation and quotation marks\index{quotation marks}, 
should a footnote mark come before
or after the punctuation mark at the end of a phrase or a sentence? I have
shown both positions\footnote{Marks \ref{fn:phrase} and \ref{fn:sentence}.}
 in the previous paragraph. The 
general rule that I have deduced is that the mark comes after the 
punctuation, but there are always those who like to prove a rule.

\index{footnote!mark|)}

\index{symbol|(}
   There are other marks that may be associated with a word, like 
(registered) trademarks. These may produce ugly gaps. Sometimes these
cannot be avoided but it may be possible to change the text to minimise
the hiccup. There is an example of this on \pref{fn:ps}. I tried various
schemes in identifying `PostScript' as being a registered trademark of
Adobe Systems Incorporated. Among the discarded trials were:
\begin{quote}
\ldots languages like PostScript\texttrademark, presumably \ldots

\ldots languages like PostScript\textsuperscript{\textregistered}, presumably \ldots

\ldots like the PostScript\textsuperscript{\textregistered}{} language, presumably \ldots

\end{quote}
My final solution was to note the registered trademark information in
a footnote:
\begin{quote}
\ldots languages, like PostScript\footnote{PostScript is a registered 
trademark of Adobe Systems Incorporated.}, presumably \ldots
\end{quote}
In this case I decided that the footnote\index{footnote} was really tied to the word
`PostScript', taking the place of the registered symbol, so I put the
footnote mark\index{footnote!mark} before the comma rather than after it.
\index{symbol|)}

\subsection{Font changes}

\index{font!change|(}

    Sometimes a word or two may be set in a different font from the 
surrounding text, such as when emphasizing\index{emphasis} 
a word by setting it in an
italic font. If the word is followed by a punctuation mark the normal
practice is to set the mark using the new font instead of the normal
font. In some cases the font used for the punctuation may not be
particularly noticeable but sometimes it may be. 

    The frontmatter contains two definitions of the word \textit{memoir,}
which is typeset using a bold font. The definitions thus commence like \\
\hspace*{2em} \textbf{memoir,} \textit{n.} \ldots \\
instead of \\
\hspace*{2em} \textbf{memoir}, \textit{n}. \ldots 

\index{font!change|)}

\section{Narrow measures}

\index{measure!narrow|(}

    Typesetting in a narrow column\index{column} is difficult, especially if you are
trying to make the text flush left and right. As the lines get shorter
it becomes more and more difficult to fit the words in without an excessive
amount of interword spacing or word breaking at the ends of lines. 
In the limit, of course, there
will not be even enough room to put a syllable on a line.

    The best recourse in situations like this is to forget justification
and typeset ragged right. Ragged right looks far better than justified
text with lots of holes in it.
The question then is, to hyphenate or not to hyphenate?\index{hyphenation}

    With no hyphenation there is likely to be increased raggedness at
the line ends when compared with permitting some hyphenation. Hyphenation
can be used to reduce the raggedness but somehow short lines ending with
a hyphen may look a bit odd. This is where you have to exercise your
judgement and design skills.

    Indexes\index{index} are often typeset in double, or even triple or quadruple columns\index{column!multiple},
as each entry is typically short. Also, indexes\index{index} are typically consulted
for a particular entry rather than being read as continuous text. To help
the eye, page numbers are normally typeset immediately after the 
the name of the indexed topic, so indexes\index{index} tend to be naturally ragged right
as a matter of reader convenience.

\index{measure!narrow|)}

    Talking of hyphenation\index{hyphenation},
 each language has its own rules for allowable
hyphenation points. As you might now have come to suspect, English and 
American rules are different even though the language is nominally the same.
Broadly speaking, American English hyphenation points are typically based on
the sound of the word, so the acceptable locations are between syllables.
In British English the hyphenation points tend to be related to the
etymology of the word, so there may be different locations depending on 
whether the word came from the Greek or the Latin. If you are not sure
how a particular word should be hyphenated, look it up in a dictionary
that indicates the potential break points.

\section{Emphasis}

\index{emphasis|(}
    Underlining\index{underline} should \underline{emphatically} \underline{not} be
used to emphasise something in a typeset document. This is a hangover
from the days when manuscripts were typewritten and there was little
that could be done. The other way of emphasising something was to
put extra space between the characters of the w\,o\,r\,d being
e\,m\,p\,h\,a\,s\,i\,s\,e\,d, as has been done twice in this sentence.
As an aside, for me at least, that extra spacing produces the 
illusion that the
characters are slightly larger than normal, which is not the case.

    With the range of fonts and sizes available when typesetting there
are other methods for emphasis, although German typographers have used
letterspacing for emphasis with the fraktur and other similar font types.

    There are basically three aproaches: 
change the {\large size} of the font;
change the \textbf{weight} of the font; or most usually, change the
\emph{shape} of the font. There is a creative tension when trying
to emphasise something --- there is the need to show the reader the 
emphasised
element, but there is also the desire not to interrupt the general flow
of the text. Out of the three basic options, changing the shape seems
to be a reasonable compromise between the need and the desire.
\index{emphasis|)}

\section{Captions and legends}

\index{caption|(}
\index{legend|(}

    I am not entirely sure what is the difference between a
caption and a legend as both terms refer to the title of an illustration\index{illustration}
or table\index{table}. However, legend may also be used to refer to some explanatory 
material within an illustration\index{illustration}, such as the explanation of the symbols
used on a map.

    In any event, captions and legends are usually typeset in a font that
is smaller than the main text font, and which may also be different from the
main font. For example, if the main font is roman and a sans font is used
for chapter titles, then it could be appropriate to use a small size
of the sans font for captions as well.

    The caption for a table\index{table} is normally placed above the table while
captions for illustrations\index{illustration} are placed below.

\index{legend|)}
\index{caption|)}

\section{Tables}

\index{table|(}

    A table is text or numbers arranged in columns\index{column!multiple}, and nearly always
with a `legend'\index{legend} above each column describing the meaning of
the entries in the column. The legends and the column entries are
separated from each other, perhaps by some vertical space but more often
by a horizontal line.

    In general typographers dislike vertical lines in a table, which may
be likely to be used to separate the columns. I'm not sure why this is.
There is an obvious explanation when hand setting the individual characters
as although it would be easy to set horizontal rules it would be very 
difficult to get all the pieces of type with the bits of the vertical rules
aligned properly --- the eye is very sensitive to jags in what is meant to
be a straight line. In the days of digital typography the alignment problem
has gone away, so perhaps the antipathy to vertical lines is a tradition
from earlier days.

    If you want to use vertical lines, just be aware that not everybody
may appreciate your effort.


\index{table|)}



\chapter{Electronic books}
\index{electronic books|(}\index{Ebook|see{electronic books}}
\section{Introduction}

    For want of a better term I am calling electronic books, or Ebooks, 
those documents intended to be read on a computer screen. The vast bulk
of Ebooks are in the form of email but I'm more interested here in 
publications that are akin to hardcopy reports and books that require
more time than a few minutes to read.

    This brief chapter includes some suggestions for the
layout of Ebooks, based on my experience with such works. 
Not considered are internal navigation aids
(e.g., hyperlinks) within and between Ebooks, nor HTML documents where
the visual appearance is meant to set by the viewing software and not 
by the publisher.

\section{Observations}

    Unlike real books which have been available for hundreds of years there
is virtually no experience to act as a guide in suggesting how Ebooks should
appear. 

    The publication medium is obviously very different --- a TV-style 
screen with limited resolution and pretty much fixed in position versus
foldable and markable paper\index{paper} held where the reader finds it best.
These differences lead to the following suggestions.

    A book can be held at whatever distance is comfortable for reading, even
when standing up.
The computer user is normally either sitting in a chair with the monitor
on a desk or table, or is trying to read from a laptop, which may be 
lighter but nobody would want to hold one for any length of time. To try
and alleviate the physical constraints on the Ebook reader the font size
should be larger than normal for a similar printed book. This will provide
a wider viewing range. A larger font will also tend to
increase the sharpness of the print as more pixels will be available for
displaying each character.

    I find it extremely annoying if I have to keep scrolling up and down
to read a page. Each page should fit within the screen, which means that
Ebook pages will be shorter than traditional pages. 
    A suggested size for an Ebook page, in round numbers, is 
about 9 by 6 inches~\cite{ADOBEBOOK} or 23 by 15 centimetres overall.

    The font size should not be less than 12pt. The font may have to be
more robust than you would normally use for printing, as fine hairlines 
or small serifs will not display well unless on a high resolution screen.

    The page design for printed books is based on a double spread. For
Ebooks the design should be based on a single page. The typeblock\index{typeblock} must
be centered on the page otherwise it gets tiring, not to mention
aggravating, if your eyes have to 
flip from side to side when moving from one page to the next. Likewise
any header\index{header} and the top of the typeblock\index{typeblock} must be at a constant height
on the screen. A constant position for the bottom of the text is not
nearly so critical.

    It is more difficult with an Ebook than with a paper\index{paper} book to flip through
it to find a particular place. Navigation aids --- headers\index{header} and footers\index{footer} ---
are therefor more critical. Each page should have both a chapter 
(perhaps also a section) header\index{header} title and a page number. Note that I'm not
considering HTML publications.

    Many viewers for Ebooks let you jump to a particular page. The page
numbers that they use, though, are often based on the sequence number from
the first page, not the displayed folio\index{folio}. In such cases it can be helpful
to arrange for a continuous sequence of page numbers, even if the folios\index{folio}
are printed using different styles. For example, if the front matter uses
roman numerals and the main matter arabic numerals and the last page of the
frontmatter is page xi, then make the first page of the main matter page 12.

    I see no point in Ebooks having any blank pages --- effectively the
concept of recto and verso pages is irrelevant.

    Some printed books have illustrations\index{illustration} that are tipped in, and the tipped
in pages are sometimes excluded from the pagination. In an Ebook the
illustrations\index{illustration} have to be `electronically tipped in' in some fashion, either
by including the electronic source of the illustrations\index{illustration} or by providing
some navigation link to them. Especially in the former case, the
tipped in elements should be included in the pagination.

    Don't forget that a significant percentage of the population is 
color-blind.\index{color|(}\index{color!blind} 
The most common form is a reduced ability to distinguish
between red and green; for example some shades of pink may be perceived
as being a shade of blue, or lemons, oranges and limes may all appear to
be the same color. Along with color-blindness there may be a reduced
capacity to remember colors.

    I have seen Ebooks where color has been liberally used to indicate, say, 
different revisions of the text or different sources for the data in a graph. 
Unless the colors used are really 
distinctive 10\% or more of the potential readership will be lost
or confused. Further,
Ebooks may be printed for reading off-line and if a non-color printer is
used then any colors will appear as shades of grey; these must be such that
they are both readily distinguishable and legible. Yellow on white is almost
as difficult to read as off-white on white or navy blue on black, all of
which I have seen on web sites but rarely after I have tried to print 
the page.

\index{color|)}
\index{electronic books|)}




    







%%%%%%%%%%%%%%%%%%%%%%
%\endinput
%%%%%%%%%%%%%%%%%%%%%%%








%%%%%%%%%%%%%%%%%%%%%%%%%%%%
\part{Practice} \label{part:practice}
%%%%%%%%%%%%%%%%%%%%%%%%%%%%%


\chapter{Starting off} \label{chap:starting}

\pagestyle{headings}
This chapter uses the \pstyle{headings} pagestyle\index{pagestyle}; 
pagestyles are described in \S\ref{chap:signposts}.

    As usual, the \Lclass{memoir} class is called by 
\cmd{\documentclass}\oarg{options}\texttt{\{memoir\}}. The \meta{options}
include being able to select a paper\index{paper!size} size from among a range of sizes, 
selecting a type size, selecting the kind of manuscript, and some related
specifically to the typesetting of mathematics.

\section{Stock paper size options}

    The stock\index{stock} size is the size of a single sheet of the paper\index{paper} you expect to 
put through the printer.
    There are a range of stock\index{stock} paper\index{paper!size} sizes from which to make a selection. 
These include:
\begin{itemize}

\item[\Lopt{a3paper}]\index{paper!size!A3} for a stock size of $420 \times 297$ millimeters
\item[\Lopt{a4paper}]\index{paper!size!A4} for a stock size of $297 \times 210$ millimeters
\item[\Lopt{a5paper}]\index{paper!size!A5} for a stock size of $210 \times 148$ millimeters
\item[\Lopt{a6paper}]\index{paper!size!A6} for a stock size of $148 \times 105$ millimeters
\item[\Lopt{b3paper}]\index{paper!size!B3} for a stock size of $500 \times 353$ millimeters
\item[\Lopt{b4paper}]\index{paper!size!B4} for a stock size of $353 \times 250$ millimeters
\item[\Lopt{b5paper}]\index{paper!size!B5} for a stock size of $250 \times 176$ millimeters
\item[\Lopt{b6paper}]\index{paper!size!B6} for a stock size of $176 \times 125$ millimeters
\item[\Lopt{letterpaper}]\index{paper!size!letterpaper} for a stock size of $11 \times 8.5$ inches
\item[\Lopt{legalpaper}]\index{paper!size!legal} for a stock size of $14 \times 8.5$ inches
\item[\Lopt{executivepaper}]\index{paper!size!executive} for a stock size of $10.5 \times 7.25$ inches
\item[\Lopt{ebook}] for a stock size of $6 \times 9$ inches, principally
                    for `electronic books' intended to be displayed
                    on a computer monitor
\item[\Lopt{landscape}] to interchange the height and width of the stock.

\end{itemize}

    These options, except for \Lopt{landscape}, are mutually exclusive.
The default stock\index{stock} paper\index{paper!size} size is \Lopt{letterpaper}\index{paper!size!letterpaper}.

\section{Type size options}

    The class offers a wider range of type sizes than usual. These are:
\begin{itemize}

\item[\Lopt{9pt}] for 9pt type
\item[\Lopt{10pt}] for 10pt type
\item[\Lopt{11pt}] for 11pt type
\item[\Lopt{12pt}] for 12t type
\item[\Lopt{14pt}] for 14pt type
\item[\Lopt{17pt}] for 17pt type

\end{itemize}

    These options are mutually exclusive.
The default type size is \Lopt{10pt}.

\section{Printing options}

    This group of options includes:

\begin{itemize}

\item[\Lopt{twoside}] for when the document will be published with printing
                        on both sides of the paper\index{paper}.
\item[\Lopt{oneside}] for when the document will be published with only
                        one side of each sheet being printed on.

                        The \Lopt{twoside} and \Lopt{oneside} options
                        are mutually exclusive.

\item[\Lopt{onecolumn}] only one column\index{column!single} of text on a page.
\item[\Lopt{twocolumn}] two equal width columns\index{column!double} of text on a page.

                        The \Lopt{onecolumn} and \Lopt{twocolumn} options
                        are mutually exclusive.

\item[\Lopt{openright}] each chapter\index{chapter} will start on a recto page.
\item[\Lopt{openleft}] each chapter\index{chapter} will start on a verso page.
\item[\Lopt{openany}] a chapter\index{chapter} may start on either a recto or verso page.

                        The \Lopt{openright}, \Lopt{openleft} and 
                        \Lopt{openany} options
                        are mutually exclusive.

\item[\Lopt{final}] for camera-ready copy of your labours.
\item[\Lopt{draft}] this marks overfull lines with black bars and enables
                      some change marking to be shown. There may be other 
                      effects as well, particularly if some packages are used.
\item[\Lopt{ms}] this tries to make the document look as though it was 
                   prepared on a typewriter. Some publishers prefer to receive
                   poor looking submissions.

                   The \Lopt{final}, \Lopt{draft} and \Lopt{ms} options
                   are mutually exclusive.

\item[\Lopt{showtrims}] this option prints marks at the corners of the
                   the sheet so that you can see where the stock\index{stock} must be
                   trimmed to produce the final page size.

\end{itemize}

    The defaults among the printing options are \Lopt{twoside}, 
\Lopt{onecolumn},
\Lopt{openright}, and \Lopt{final}.

\section{Other options}

    The remaining options are:
\begin{itemize}

\item[\Lopt{leqno}] equations will be numbered at the left (the default is
     to number them at the right).

\item[\Lopt{fleqn}] displayed math environments will be indented an amount
                      \cmd{\mathindent} from the left margin\index{margin} (the default is to
                      center the environments).

\item[\Lopt{openbib}] each part of a bibliography\index{bibliography} entry will start on a
                        new line, with second and succeding lines indented
                        by \cmd{\bibindent} (the default is for an entry
                        to run continuously with no indentations).

\item[\Lopt{article}] typesetting simulates the \Lclass{article} class,
  but the \cmd{\chapter} command is not disabled.
  Chapters\index{chapter} do not start a new page and chapter headings\index{heading!chapter} are typeset
  like a section heading\index{heading!sections}. The numbering of 
  figures\index{figure}, etc., is continuous
  and not per chapter. However, a \cmd{\part} command still puts
  its heading\index{heading!part} on a page by itself.

\item[\Lopt{oldfontcommands}] makes the old, deprecated LaTeX version~2.09
  font commands available. Warning messages will be produced whenever
  an old font command is encountered.

\end{itemize}

    None of these options are defaulted.

\section{Remarks}

   Calling the class with no options is equivalent to:
\begin{lcode}
\documentclass[letterpaper,10pt,twoside,onecolumn,openright,final]{memoir}
\end{lcode}
   The source file for this manual starts
\begin{lcode}
\documentclass[letterpaper,10pt]{memoir}
\end{lcode}
which is overkill as both \Lopt{letterpaper} and \Lopt{10pt} are among
the default options.

    Actual typesetting only occurs within the \Ie{document} environment. The
region of the file between the \cmd{\documentclass} command and the start
of the \Ie{document} environment is called the 
\emph{preamble}\index{preamble}. This is where you ask for external packages
and define you own macros if you feel so inclined.

\begin{syntax}
\cmd{\flushbottom} \cmd{\raggedbottom} \\
\end{syntax}
When the \Lopt{twoside} or \Lopt{twocolumn} option is selected then
typesetting is done with \cmd{\flushbottom}, otherwise it is done
with \cmd{\raggedbottom}.

    When \cmd{\raggedbottom} is in effect LaTeX makes little attempt to
keep a constant height for the typeblock\index{typeblock}; pages may run short.

    When \cmd{\flushbottom} is in effect LaTeX ensures that the typeblock\index{typeblock}
on each page is a constant height, except when a page break is deliberately
introduced when the page might run short. In order to maintain a constant
height it may stretch or shrink some vertical spaces 
(e.g., between paragraphs\index{paragraph}, around headings\index{heading} or around floats\index{float} or other inserts).
This may have a deleterious affect on the color\index{page color} 
of some pages. Serendipitously this has happened on \pref{chap:lpage} where
there is additional space between the paragraphs\index{paragraph} (caused by the next sectional
division having to be put at the top of the next page). You may wish to
compare that page with the following one to see the difference in the 
colors. 

    I could have made the page run short by inserting \cmd{\raggedbottom}
at an appropriate place, followed later by a \cmd{\flushbottom}.

    If you get too many strung out pages with \cmd{\flushbottom} you may
want to put \cmd{\raggedbottom} in the preamble\index{preamble}.

    If you use the \Lopt{ebook} option you may well also want to use the
\Lopt{12pt} and \Lopt{oneside} options.



\chapter{Laying out the page} \label{chap:layingpage}

\pagestyle{ruled}

    This chapter is typeset with the \pstyle{ruled} pagestyle.


\section{Introduction}

    The class provides a default page layout, in which the page size is the
same as the stock\index{stock} size and the typeblock\index{typeblock} is roughly in the middle of
the page.
    This chapter describes the commands provided by the class to help you
produce your own page layout if the default is inappropriate.

    The pages of a book carry the text which is intended to educate, entertain
and/or amuse the reader. The page must be designed to serve the purposes of
the author and to ease the reader's task in assimilating the author's ideas.
A good page design is one which the general reader does not notice. If the
reader is constantly noticing the page layout, even unconsciously, it distracts
from the purpose of the book. It is not the job of the designer to 
shout, or even to murmur, `look at my work'.

    There are three main parts to a page: the page itself, the typeblock\index{typeblock}, and 
the margins\index{margin} separating the typeblock\index{typeblock} from the edges of the page. Of slightly
lesser importance are the headers\index{header} and footers\index{footer}, and possibly marginal\index{marginalia} notes.
The art of page design is obtaining a harmonious balance or rhythm between
all these.

    Although the form is different, the facilities described in this chapter 
are similar to those provided by the \Lpack{geometry} 
package~\cite{GEOMETRY}.

\section{Stock material}

   Printing is the act of laying symbols onto a piece of stock\index{stock} material.
Some print on T shirts by a process called silk screening, where the shapes of
the symbols are made in a screen and then fluid is squeezed through the screen
onto the stock material --- in this case the fabric of the T shirt. Whether or
not this is of general interest it is not the sort of printing or stock
material that is usually used in book production. Books, except for the very
particular, are printed on paper\index{paper}. 

    In the desktop publishing world the stock\index{stock}
paper\index{paper!size} is usually one from a range of standard sizes. In the USA it is typically
letterpaper\index{paper!size!letterpaper} (11 by 8.5 inches) and in the rest of the world A4\index{paper!size!A4} paper 
(297 by 210 mm), with one page per piece of stock\index{stock}. In commercial 
printing the stock\index{stock} material is much larger with several pages being 
printed on each stock\index{stock} piece; the stock\index{stock} is then folded, cut and trimmed to form
the final pages for binding. For our purposes we only consider desktop
publishing.


\section{The page}
 
    We only consider one page per piece of stock\index{stock}. 


\begin{figure}
\centering
\drawpage
\caption{LaTeX page layout parameters for a recto page} \label{fig:anoddpage}
\end{figure}


    The parameters used by LaTeX itself to define the page layout are 
illustrated in \fref{fig:anoddpage}. LaTeX does not actually care about the
physical size of a page --- it assumes that, with respect to the top lefthand
corner, the sheet of paper\index{paper} to be printed is infinitely wide and infinitely
long. If you happen to have a typeblock\index{typeblock} that is too wide or too long for
the sheet, LaTeX will merrily position text outside the physical boundaries.

    The LaTeX parameters are often not particularly convenient if, say, the top
of the text must be a certain distance below the top of the page and the 
\foredge{} margin\index{margin} must be twice the spine margin\index{margin}. It is obviously possible
to calculate the necessary values for the parameters, but it is
not a pleasurable task.

    The class provides various means of specifying the page layout, which are
hopefully more convenient to use than the standard ones. Various
adjustable parameters are used that define the stock\index{stock} size, page size, and so 
on. These differ in some respects from the parameters in the standard classes.
Figure~\ref{fig:oddstock} shows the stock\index{stock} for a recto page, with a page layout,
illustrating the main layout parameters. These may be changed individually
by \cmd{\setlength} or by using the commands described below.

\begin{figure}
\centering
\drawmarginparsfalse
\drawstock
\caption{The main \textsf{memoir} class page layout parameters for a recto page} \label{fig:oddstock}
\end{figure}


    In the code for the standard classes it says:
\begin{quotation}
`The variables \cmd{\paperwidth} and \cmd{\paperheight} should reflect
the physical paper\index{paper!size} size after trimming. For desk printer output this is
usually the real paper\index{paper!size} size since there is no post-processing. Classes for
real book production will probably add other paper\index{paper!size} sizes and additionally
the production of crop marks for trimming.'
\end{quotation}

    This class has introduced the additional lengths \cmd{\stockwidth}
and \cmd{\stockheight} to denote the physical paper\index{paper!size} size \emph{before}
trimming.


    The first step in designing the page layout is to decide on the page size 
and then pick an appropriate stock\index{stock} size. Selecting a standard stock\index{stock} size will
be cheaper than having to order specially sized stock\index{stock} material. 

\begin{syntax}
\cmd{\setstocksize}\marg{height}\marg{width} \\
\lnc{\stockheight} \lnc{\stockwidth} \\
\end{syntax}
    The class options provide for some common stock\index{stock} sizes. If you have some
other size that you want to use the 
command \cmd{\setstocksize} can be used to
specify that the stock\index{stock} size is \meta{height} by \meta{width}. For example
the following specifies a stock\index{stock} of 9 by 4 inches:
\begin{lcode}
\setstocksize{9in}{4in}
\end{lcode}

The size of the page must be
no larger than the stock\index{stock} but may be smaller which means that after printing
the stock\index{stock} must be trimmed down to the size of the page.
The page may 
be positioned anywhere within the bounds of the stock\index{stock}.

    Page layout should be conceived in terms of a spread\index{spread}; when 
you open a book in the middle what you see is a spread --- a verso page on the
left and a recto page on the right with the spine between them. Most books when
closed are taller than they are wide; this makes them easier to hold when open
for reading. A squarish page when opened out into a wide spread 
makes for discomfort unless the book is supported on a table.

\begin{syntax}
\cmd{\settrimmedsize}\marg{height}\marg{width}\marg{ratio} \\
\lnc{\paperheight} \lnc{\paperwidth} \\
\end{syntax}
The command \cmd{\settrimmedsize}
can be used to specify 
the height and width of the page (after any trimming). Initially the page
size is made the same as the stock\index{stock} size, as set by the paper\index{paper!size} size option.
The \meta{ratio} argument is 
the amount by which the \meta{height} or the \meta{width} must be multiplied 
by to give the
width or the height. Only two out of the three possible arguments must be 
given values with the other (unvalued) argument given as |*| 
(an asterisk). The lengths \lnc{\paperheight} and \lnc{\paperwidth} are 
calculated according to the given arguments. That is, the command enables
the \lnc{\paperheight} 
and \lnc{\paperwidth} to be specified directly or as one being in a given
ratio to the other.

    If you have used \cmd{\setstocksize} to redefine the stock\index{stock}, then to get
the same page size, do:
\begin{lcode}
\settrimmedsize{\stockheight}{\stockwidth}{*}
\end{lcode}
or for the page to be 90\% of the size of the stock\index{stock}:
\begin{lcode}
\settrimmedsize{0.9\stockheight}{0.9\stockwidth}{*}
\end{lcode}

    The following are three different ways of defining an 8 by 5 inch page.
\begin{lcode}
\settrimmedsize{8in}{5in}{*}
\settrimmedsize{8in}{*}{0.625}  % 5 = 0.625 times 8
\settrimmedsize{*}{5in}{1.6}    % 8 = 1.6 times 5
\end{lcode}

If you look at a well bound hardback book you
can see that the sheets are folded so that they are continuous at the spine, 
where they are sewn together into the binding. The top of the pages should be
smooth so that when the book is upright on a bookshelf dust has a harder 
job seeping between the pages than if the top was all raggedy. Thus, if
the stock\index{stock} is trimmed it will be trimmed at the top. It will also have been
cut at the \foredge s of the pages and at the bottom, otherwise the book
would be unopenable and unreadable.

\begin{syntax}
\cmd{\settrims}\marg{top}\marg{edge} \\
\lnc{\trimtop} \lnc{\trimedge} \\
\end{syntax}
The command \cmd{\settrims} can be used to specify 
the amount intended to 
be trimmed from the top (\meta{top}) and \foredge{} (\meta{edge}) of the stock\index{stock} 
material to produce the top and fore edge of a recto page. 
Note that the combination of \cmd{\settrims} and \cmd{\settrimmedsize}
locate the page with respect to the stock\index{stock}.
By default the top and edge trims are zero, which means that if any trimming
is required it will be at the spine and bottom edges of the stock\index{stock}.

    You can either do any trim calculation for youself or let LaTeX do it for
you. For example, with an 8in by 5in page on 10in by 7in stock\index{stock}
\begin{lcode}
\settrims{2in}{2in}
\end{lcode}
specifies trimming 2in from the top and \foredge{} of the stock\index{stock}, 
giving the desired page size. Taking a design where, say, the page is 90\% of the
stock\index{stock} size it's easy to get LaTeX to do the calculation:
\begin{lcode}
\setlength{\trimtop}{\stockheight}    % \trimtop = \stockheight
\addtolengh{\trimtop}{-\paperheight}  % \trimtop = \stockheight - \paperheight
\setlength{\trimedge}{\stockwidth}
\addtolength{\trimedge}{-\paperwidth}
\end{lcode}
which will set all the trimming to be at the top and \foredge. If you wanted, say,
equal trims at the top and bottom you could go on as
\begin{lcode}
\settrims{0.5\trimtop}{\trimedge}
\end{lcode}


\section{The typeblock} \label{sec:typeblock2}

    Like the page, the typeblock\index{typeblock} is normally rectangular with the height 
greater than the width.


Table~\ref{tab:cmrlengths} gives the lowercase alphabet lengths for a range
of Computer Modern Roman font sizes; this may be used in conjunction
with \tref{tab:copyfitting} on \pref{tab:copyfitting} when deciding
on an appropriate textwidth. 
\begin{syntax}
\lnc{\xlvchars} \lnc{\lxvchars} \\
\end{syntax}
Based on this table, the two lengths 
\lnc{\xlvchars} and 
\lnc{\lxvchars} are approximately the lengths of a line of text with
45 or 65 characters, respectively, for the type size selected for the document.

    If you are using a different font you can use something like the 
following to calculate and print out the length for you.
\begin{lcode}
\newlength{\mylen}                % a length
\newcommand{\alphabet}{abc...xyz} % the lowercase alphabet
\begingroup                       % keep font change local
% font specification e.g., \Large\sffamily
\settowidth{\mylen}{\alphabet}
The length of this alphabet is \the\mylen.  % print in document
\typeout{The length of the Large sans alphabet is \the\mylen} % in log file
\endgroup       % end the grouping
\end{lcode}
The \cmd{\typeout} macro will print the result to the terminal and the
log file.

\begin{table}
\centering
\caption{Length of CMR lowercase alphabets}\label{tab:cmrlengths}
\begin{tabular}{rr}\hline
\multicolumn{1}{c}{Font} & \multicolumn{1}{c}{Length} \\
\multicolumn{1}{c}{size} & \multicolumn{1}{c}{(points)} \\ \hline
5 pt & 87 \\
6 pt & 94 \\
7 pt & 102 \\
8 pt & 108 \\
9 pt & 118 \\
10 pt & 128 \\
11 pt & 139 \\
12 pt & 150 \\
14 pt & 175 \\
17 pt & 207 \\
20 pt & 245 \\
25 pt & 290 \\ 
\hline
\end{tabular}
\end{table}


 Morten H{\o}gholm has done some curve fitting to the data in 
 \tref{tab:copyfitting}.
He found that the expressions
\begin{displaymath}
L_{65} = 2.042\alpha + 33.41 \mbox{pt}
\end{displaymath}
and
\begin{displaymath}
L_{45} = 1.415\alpha + 23.03 \mbox{pt}
\end{displaymath}
fitted aspects of the data, where $\alpha$ is the length of the
alphabet and $L_{i}$ is the desired linewidth when the line should
contain $i$ characters. He suggested the following macros.
\begin{syntax}
\cmd{\setlxvchars}\oarg{fontspec} \\
\cmd{\setxlvchars}\oarg{fontspec} \\
\end{syntax}
The macros \cmd{\setlxvchars} and \cmd{\setxlvchars} set the lengths 
\lnc{\lxvchars} and \lnc{\xlvchars} respectively for the font \meta{fontspec}.
The default for \meta{fontspec} is \cmd{\normalfont}.

    For example, the values of \lnc{\lxvchars} and \lnc{\xlvchars} after
calling \setlxvchars \setxlvchars[\small\sffamily] 
\begin{lcode}
\setlxvchars \setxlvchars[\small\sffamily] 
\end{lcode}
are: \verb?\lxvchars? = \the\lxvchars, and \verb?\xlvchars? = \the\xlvchars.

    Continuing on this theme, Morten also wrote:
\begin{quotation}
\ldots I was defining some environments that had to have \lnc{\parindent}
as their indentation. For some reason I just wrote \texttt{1.5em} instead
of \lnc{\parindent} because I knew that was the value. What I had 
overlooked was that I had loaded the \Lpack{mathpazo} 
package~\cite{MATHPAZO}, thus
altering various \cmd{\fontdimen}s. Conclusion: the environment would
insert \mbox{1.5 em = 18.0 pt}, whereas the \lnc{\parindent} was only
\mbox{17.6207 pt}.

    This, and other related situations can be avoided if one places
\begin{center}
\cmd{\RequirePackage}\marg{font-package}\cmd{\normalfont}
\end{center}
before \cmd{\documentclass}, but I have to this day never seen this suggested.
I would believe that most document classes have settings that depend on
\emph{current} font settings, which they should do for such things as
\lnc{\parindent}. However the decision to let Computer Modern be the 
default font in LaTeX causes these dimensions to be set to erroneous 
values\ldots
\end{quotation}





    The height of the typeblock\index{typeblock} should be equivalent to an integral number
of lines.

\begin{syntax}
\cmd{\settypeblocksize}\marg{height}\marg{width}\marg{ratio} \\
\lnc{\textheight} \lnc{\textwidth} \\
\end{syntax}
The command 
\cmd{\settypeblocksize} is the same as
\cmd{\settrimmedsize} except that it sets the \lnc{\textheight} and 
\lnc{\textwidth} for the typeblock\index{typeblock}. For instance, here are three ways of specifying
a 6in by 3in typeblock\index{typeblock}:
\begin{lcode}
\settypeblocksize{6in}{3in}{*}
\settypeblocksize{6in}{*}{0.5}
\settypeblocksize{*}{3in}{2}
\end{lcode}

    The typeblock\index{typeblock} has to be located on the page. There is a relationship
between the page, typeblock\index{typeblock} and margins\index{margin}. The sum of the spine margin\index{margin},
the \foredge{} margin\index{margin} 
and the width of the typeblock\index{typeblock} must equal the width
of the page. Similarly the sum of the upper margin\index{margin}, the lower margin\index{margin} and the
height of the typeblock\index{typeblock} must equal the height of the page. The process
of locating the typeblock\index{typeblock} with respect to the page can be viewed
either as positioning the typeblock\index{typeblock} with respect to the edges of the page
or as setting the margins\index{margin} between the page and the typeblock\index{typeblock}.

    Remembering that the page layout should be defined in terms of the
appearance as a spread, the spine margin\index{margin} is normally half the \foredge{}
margin\index{margin}, so that the white space is equally distributed around the
sides of the text. 

    There is usually more latitude in choosing the proportions\index{proportion} of the upper and
lower margins\index{margin}, though usually the upper margin\index{margin} is less than the lower margin\index{margin}
so the typeblock\index{typeblock} is not vertically centered.

    Two methods are provided for setting the horizontal dimensions on a page.
One where the width of the typeblock\index{typeblock} is fixed and the margins\index{margin}
are adjustable, and the other where the size of the margins\index{margin} determines the
width of the typeblock\index{typeblock}.

\begin{syntax}
\cmd{\setlrmargins}\marg{spine}\marg{edge}\marg{ratio} \\
\lnc{\spinemargin} \lnc{\foremargin} \\
\end{syntax}
The command \cmd{\setlrmargins}
can be used to specify
the side margins\footnote{Only the spine margin is noted in
\fref{fig:oddstock}; the \foredge{} margin\index{margin} is at the opposite side of the 
typeblock.} with the width of the page and the typeblock being fixed.
 
Either zero or one argument values are required, with any
unvalued argument being denoted by an asterisk. There are several cases to 
consider and these are tabulated in \tref{tab:lrmargins}.

    In the Table, $S$ is the calculated spine margin\index{margin}, $E$ is the calculated
\foredge{} margin\index{margin}, and $P_{w}$ and $B_{w}$ are respectively the page and typeblock\index{typeblock} 
widths. 
The \cmd{\setlrmargins} command maintains the relationship
\begin{displaymath}
S + E = K_{w}  = \textrm{constant} \mbox{\space} (= P_{w} - B_{w}).
\end{displaymath}


\begin{table}
\DeleteShortVerb{\|}
\centering
\caption{Arguments and results for \cs{setlrmargins} } \label{tab:lrmargins}
\begin{tabular}{ccc|l} \hline
\meta{spine} & \meta{edge} & \meta{ratio} & Result \\ \hline
 S   & E & r & ambiguous \\
 S   & E & * & ambiguous \\
 S   & * & r & ambiguous \\
 S   & * & * & $E = K_{w} - S$ \\
{*}  & E & r & ambiguous \\
{*}  & E & * & $S = K_{w} - E$ \\
{*}  & * & r & $E + S = K_{w}$, $E = rS$ \\
{*}  & * & * & $E + S = K_{w}$, $E = S$ \\
\hline
\end{tabular}
\MakeShortVerb{\|}
\end{table}

The cases marked ambiguous in the Table are where the particular combination
of argument values may make it impossible to guarantee the relationship.

    Assuming that we have a 3in wide typeblock\index{typeblock} on a 5in wide page and we want
the spine margin\index{margin} to be 0.8in and the \foredge{} margin\index{margin} to be 1.2in (i.e., the
\foredge{} margin\index{margin} is half as big again as the spine margin\index{margin}) this can be
accomplished in three ways (with the \cmd{\pagewidth} and \lnc{\textwidth} being
previously specified and fixed):
\begin{lcode}
\setlrmargins{0.8in}{*}{*}   % specify spine margin
\setlrmargins{*}{1.2in}{*}   % specify foredge margin
\setlrmargins{*}{*}{1.5}     % specify foredge/spine ratio
\end{lcode}

\begin{syntax}
\cmd{\setlrmarginsandblock}\marg{spine}\marg{edge}\marg{ratio}\\
\end{syntax}
The command \cmd{\setlrmarginsandblock}
can be used to specify the spine and \foredge{} margins\index{margin}, where the page width 
is fixed and the width of the typeblock\index{typeblock} depends on the margins\index{margin}. Results
for this command are given in \tref{tab:lrblock}. The same notation is used,
but in this case \cmd{\setlrmarginsandblock} maintains the
relationship 
\begin{displaymath}
S + B_{w} + E = \textrm{constant} \mbox{\space} (= P_{w}).
\end{displaymath}
The width of the typeblock\index{typeblock} is calculated from $B_{w} = P_{w} - S - E$.

\begin{table}
\DeleteShortVerb{\|}
\centering
\caption{Arguments and results for \cs{setlrmarginsandblock} } \label{tab:lrblock}
\begin{tabular}{ccc|l} \hline
\meta{spine} & \meta{edge} & \meta{ratio} & Result \\ \hline
 S   & E & r & $S$, $E$ \\
 S   & E & * & $S$, $E$ \\
 S   & * & r & $E = rS$ \\
 S   & * & * & $E = S$ \\
{*}  & E & r & $S = rE$ \\
{*}  & E & * & $S = E$ \\
{*}  & * & r & ambiguous \\
{*}  & * & * & ambiguous \\
\hline
\end{tabular}
\MakeShortVerb{\|}
\end{table}

    Assuming that we want a 3in wide typeblock\index{typeblock} on a 5in wide page and we want
the spine margin\index{margin} to be 0.8in and the \foredge{} margin\index{margin} to be 1.2in (i.e., the
\foredge{} margin\index{margin} is half as big again as the spine margin\index{margin}) this can be
accomplished in the following ways (with the \lnc{\textwidth} being calculated
from the previously specified \cmd{\pagewidth} and the specified margins\index{margin}):
\begin{lcode}
\setlrmarginsandblock{0.8in}{1.2in}{*}   % specify both margins
\setlrmarginsandblock{0.8in}{*}{1.5}     % specify spine & foredge/spine ratio
\setlrmarginsandblock{*}{1.2in}{0.667}   % specify foredge & spine/foredge ratio
\end{lcode}
If we wanted the margins to be both 1in instead then any of the following 
will do it:
\begin{lcode}
\setlrmarginsandblock{1in}{1in}{*} % specify both margins
\setlrmarginsandblock{1in}{*}{1}   % specify spine & foredge/spine ratio
\setlrmarginsandblock{1in}{*}{*}   % specify spine (foredge = spine)
\setlrmarginsandblock{*}{1in}{1}   % specify foredge & spine/foredge ratio
\setlrmarginsandblock{*}{1in}{*}   % specify foredge (spine = foredge)
\end{lcode}

    That completes the methods for specifying the horizontal spacings. There 
are similar commmands for setting the vertical spacings which are 
described below.

\begin{syntax}
\cmd{\setulmargins}\marg{upper}\marg{lower}\marg{ratio} \\
\lnc{\uppermargin} \lnc{\lowermargin} \\
\end{syntax}
The command \cmd{\setulmargins} can be used to specify
the upper and lower margins\footnote{Only the upper margin is noted in
\fref{fig:oddstock}; the lower margin\index{margin} is the distance between the bottom
of the typeblock\index{typeblock} and the bottom of the page.} where the heights of the page
and the typeblock\index{typeblock} are fixed. This is similar to \cmd{\setlrmargins}. Using
a slightly different notation this time, with $U$ being the upper margin\index{margin},
$L$ being the lower margin\index{margin}, and $P_{h}$ and $B_{h}$ being the 
\emph{height} of the page
and typeblock\index{typeblock}, respectively, the results are shown in \tref{tab:ulmargins}.
The \cmd{\setlrmargins} command maintains the relationship
\begin{displaymath}
U + L = K_{h}  = \textrm{constant} \mbox{\space} (= P_{h} - B_{h}).
\end{displaymath}


\begin{table}
\DeleteShortVerb{\|}
\centering
\caption{Arguments and results for \cs{setulmargins} } \label{tab:ulmargins}
\begin{tabular}{ccc|l} \hline
\meta{upper} & \meta{lower} & \meta{ratio} & Result \\ \hline
 U   & L & r & ambiguous \\
 U   & L & * & ambiguous \\
 U   & * & r & ambiguous \\
 U   & * & * & $L = K_{h} - U$ \\
{*}  & L & r & ambiguous \\
{*}  & L & * & $U = K_{h} - L$ \\
{*}  & * & r & $L + U = K_{h}$, $L = rU$ \\
{*}  & * & * & $L + U = K_{h}$, $L = U$ \\
\hline
\end{tabular}
\MakeShortVerb{\|}
\end{table}

\begin{syntax}
\cmd{\setulmarginsandblock}\marg{upper}\marg{lower}\marg{ratio} \\
\end{syntax}
The command \cmd{\setulmarginsandblock}
can be used to specify the upper and lower margins\index{margin}, where the page height 
is fixed and the height of the typeblock\index{typeblock} depends on the margins\index{margin}. Results
for this command are given in \tref{tab:ulblock}. The same notation is used,
but in this case \cmd{\setulmarginsandblock} maintains the
relationship 
\begin{displaymath}
U + B_{h} + L = \textrm{constant} \mbox{\space} (P_{h}).
\end{displaymath}
The height of the typeblock\index{typeblock} is calculated from $B_{h} = P_{h} - U - L$.

\begin{table}
\DeleteShortVerb{\|}
\centering
\caption{Arguments and results for \cs{setulmarginsandblock} } \label{tab:ulblock}
\begin{tabular}{ccc|l} \hline
\meta{upper} & \meta{lower} & \meta{ratio} & Result \\ \hline
 U   & L & r & $U$, $L$ \\
 U   & L & * & $U$, $L$ \\
 U   & * & r & $L = rU$ \\
 U   & * & * & $L = U$ \\
{*}  & L & r & $U = rL$ \\
{*}  & L & * & $U = L$ \\
{*}  & * & r & ambiguous \\
{*}  & * & * & ambiguous \\
\hline
\end{tabular}
\MakeShortVerb{\|}
\end{table}

\begin{syntax}
\cmd{\setcolsepandrule}\marg{colsep}\marg{thickness} \\
\lnc{\columnsep} \lnc{\columseprule} \\
\end{syntax}
    For twocolumn\index{column!double} text the width of the gutter between the columns must be
specified. LaTeX also lets you draw a vertical rule in the middle of the 
gutter. The macro \cmd{\setcolsepandrule} 
sets the gutter width, \lnc{\columnsep}, to \meta{colsep} 
and \lnc{\columnseprule}, the thickness of the rule, 
to \meta{thickness}. A \meta{thickness} of 0pt means that the rule will be
invisible. Visible rules usually have a thickness of about 0.4pt.
The total width of the twocolumns\index{column!double} of text and the gutter equals the width of
the typeblock\index{typeblock}.

    This completes the main elements of the page --- the page size, the size
of the typeblock\index{typeblock} and the margins\index{margin} surrounding the typeblock\index{typeblock}. 

\section{Headers, footers and marginal notes}
\index{header|(}\index{footer|(}

    A page may have two additional items, and usually has at least one of 
these. They are the running header and running footer. If the page has a folio\index{folio} then it is located either in the 
header or in the footer. The word `in' is used rather lightly here as the folio\index{folio}
may not be actually \emph{in} the header or footer but is always located 
at some constant relative position. A common position for the folio\index{folio} is towards
the \foredge{} of the page, either in the header of the footer. This makes it 
easy to spot when thumbing through the book. It may be placed at the center of
the footer, but unless you want to really annoy the reader do not place it near
the spine.

    Often a page header contains the current chapter title, with perhaps a 
section title on the opposite header, as aids to the reader in navigating 
around the book. Some books
put the book title into one of the headers, usually the verso one, but I see 
little point in that as presumably the reader knows which particular book he
is reading, and the space would be better used providing more useful signposts.

\begin{syntax}
\cmd{\setheadfoot}\marg{headheight}\marg{footskip} \\
\lnc{\headheight} \lnc{\footskip} \\
\end{syntax}
The command \cmd{\setheadfoot} sets the
\lnc{\headheight} parameter to the value \meta{headheight} and the
\lnc{\footskip} parameter to \meta{footskip}.

\begin{syntax}
\cmd{\setheaderspaces}\marg{headdrop}\marg{headsep}\marg{ratio} \\
\lnc{\headdrop} \lnc{\headsep} \\
\end{syntax}
The command \cmd{\setheaderspaces} is similar to \cmd{\setulmargins}.
Using the notation $U$ for the upper margin\index{margin} (as before), $H_{h}$ for the
\lnc{\headheight}, $H_{s}$ for the \lnc{\headsep} and $D$ for the
\lnc{\headdrop}, where the \lnc{\headdrop} is the distance between the top
of the trimmed page and the top of the header\footnote{The head drop is not
shown in \fref{fig:oddstock}.}, then the macro \cmd{\setheaderspaces}
maintains the relationship
\begin{displaymath}
D + H_{s} = C_{h}  = \textrm{constant} \mbox{\space} (= U - H_{h}).
\end{displaymath}

\begin{table}
\DeleteShortVerb{\|}
\centering
\caption{Arguments and results for \cs{setheaderspaces} } \label{tab:headspaces}
\begin{tabular}{ccc|l} \hline
\meta{headdrop} & \meta{headsep} & \meta{ratio} & Result \\ \hline
 D   & $H_{s}$ & r & ambiguous \\
 D   & $H_{s}$ & * & ambiguous \\
 D   & *       & r & ambiguous \\
 D   & *       & * & $H_{s} = C_{h} - D$ \\
{*}  & $H_{s}$ & r & ambiguous \\
{*}  & $H_{s}$ & * & $D = C_{h} - H_{s}$ \\
{*}  & *       & r & $H_{s} + D = C_{h}$, $H_{s} = rD$ \\
{*}  & *       & * & $H_{s} + D = C_{h}$, $H_{s} = D$ \\
\hline
\end{tabular}
\MakeShortVerb{\|}
\end{table}

The macro \cmd{\setheaderspaces} is for specifying the spacing above and below
the page header.
The possible arguments and results are shown in \tref{tab:headspaces}. Typically,
the \lnc{\headsep} is of more interest than the \lnc{\headdrop}.

\index{footer|)}\index{header|)}

\index{marginalia|(}
   Finally, some works have marginal notes. These really come last in the 
design scheme, providing the margins have been made big enough to accomodate 
them. Figure~\ref{fig:evenstock} shows the marginal note parameters on a verso
page, and also illustrates that some parameters control different 
positions on the stock\index{stock}.


\begin{figure}
\centering
\oddpagelayoutfalse
\drawstock
\caption{The \textsf{memoir} class page layout parameters for a verso page} \label{fig:evenstock}
\end{figure}



\begin{syntax}
\cmd{\setmarginnotes}\marg{separation}\marg{width}\marg{push} \\
\lnc{\marginparsep} \lnc{\marginparwidth} \lnc{\marginparpush} \\
\end{syntax}
The command \cmd{\setmarginnotes}
sets the parameters for marginal notes. The distance \lnc{\marginparsep} 
between the typeblock\index{typeblock} and any note is set to \meta{separation}, the 
maximum width for a note, \lnc{\marginparwidth}, is set to \meta{width}
and the minimum vertical distance between notes, \lnc{\marginparpush},
is set to \meta{push}.

\index{marginalia|)}

\section{Putting it together}

    The page layout parameters just discussed are not always the same
as those that are expected by LaTeX, or by LaTeX packages. The parameters
that LaTeX expects are shown in \fref{fig:anoddpage}. You can either use
the class commands for changing the page layout or change the LaTeX parameters
directly using either \cmd{\setlength} or \cmd{\addtolength} applied to the
parameter(s) to be modified. If you choose the latter route, those class 
parameters which differ from the standard LaTeX parameters will \emph{not} be
modified. 

    The general process of setting up your own page layout is along these
lines:
\begin{itemize}
\item Decide on the required finished page size, and pick a stock\index{stock} size that
      is at least as large as the page.
\item Use \cmd{\setstocksize} to set the values of \lnc{\stockheight}
      and \lnc{\stockwidth}, followed by \cmd{\settrimmedsize} to specify
      the values of \lnc{\paperheight} and \lnc{\paperwidth}.
\item Decide on the location of the page with respect to the stock\index{stock}. If the
      page and stock\index{stock} sizes are the same, then call |\settrims{0pt}{0pt}|.
      If they are not the same size then decide on the appropriate 
      values for \lnc{\trimtop} and \lnc{\trimedge} to position the page on
      the stock\index{stock}, and then set these through \cmd{\settrims}.
\item Decide on the size of the typeblock\index{typeblock} and use \cmd{\settypeblocksize}
      to specify the values of \lnc{\textheight} and \lnc{\textwidth}.
\item Pick the value for the spine margin\index{margin}, and use \cmd{\setlrmargins}
      to set the values for the \lnc{\spinemargin} and \lnc{\foremargin}.

      An alternative sequence is to use \cmd{\setlrmarginsandblock} to
      set the \lnc{\textwidth} for a particular choice of side margins\index{margin}.

\item Pick the value for the upper margin\index{margin} and use \cmd{\setulmargins}
      to set the values for the \lnc{\uppermargin} and \cmd{\lowermargin}.

      An alternative sequence is to use \cmd{\setulmarginsandblock} to
      set the \lnc{\textheight} for a particular choice of upper and
      lower margins\index{margin}.

\item Pick the values for the \lnc{\headheight} and \lnc{\footskip}
      and use \cmd{\setheadfoot} to specify these.

\item Pick your value for \cmd{\headskip} and use \cmd{\setheaderspaces}
      to set this and \lnc{\headdrop}.

\item If you are going to have any marginal\index{marginalia} notes, use \cmd{\setmarginnotes}
      to specify the values for \lnc{\marginparsep}, \lnc{\marginparwidth}
      and \lnc{\marginparpush}.

\end{itemize}

    You can plan and specify your layout in any order that is 
convenient to you. Each of the page layout commands are independent; also if
a value is set at one point, say the \lnc{\textwidth}, and is then
later changed in some way, only the last of the settings is used as the actual
value.


    Comparing \figurerefname s~\ref{fig:oddstock} and~\ref{fig:anoddpage} you can see
the different layout parameters provided by the class and used by standard
LaTeX. For convenience, and because the figures do not show all
the parameters, the two sets of parameters are listed in \tref{tab:stockpage}.

\begin{table}
\centering
\caption{The class and LaTeX page layout parameters}\label{tab:stockpage}
\begin{tabular}{ll} \hline
Class & LaTeX \\ \hline
\lnc{\stockheight}    &  \\
\lnc{\trimtop}        &  \\
\lnc{\trimedge}       &  \\
\lnc{\stockwidth}     &  \\
\lnc{\paperheight}    & \lnc{\paperheight}  \\
\lnc{\paperwidth}     & \lnc{\paperwidth}  \\
\lnc{\textheight}     & \lnc{\textheight}  \\
\lnc{\textwidth}      & \lnc{\textwidth}  \\
\lnc{\columnsep}      & \lnc{\columnsep}  \\
\lnc{\columnseprule}  & \lnc{\columnseprule}  \\
\lnc{\spinemargin}    &  \\
\lnc{\foremargin}     &  \\
                      & \lnc{\oddsidemargin}  \\
                      & \lnc{\evensidemargin}  \\
\lnc{\uppermargin}    &  \\
\lnc{\headdrop}       &  \\
                      & \lnc{\topmargin}  \\
\lnc{\headheight}     & \lnc{\headheight}  \\
\lnc{\headsep}        & \lnc{\headsep}  \\
\lnc{\footskip}       & \lnc{\footskip}  \\
\lnc{\lowermargin}    &  \\
\lnc{\marginparsep}   & \lnc{\marginparsep}  \\
\lnc{\marginparwidth} & \lnc{\marginparwidth}  \\
\lnc{\marginparpush}  & \lnc{\marginparpush}  \\
\hline
\end{tabular}
\end{table}

\begin{syntax}
\cmd{\checkandfixthelayout} \\
\end{syntax}
    The macro \cmd{\checkandfixthelayout} takes all the class layout parameters, 
checks
that they have `sensible' values (e.g., the \lnc{\textwidth} is not negative),
and then calculates the values for all those LaTeX layout parameters that 
differ from the class parameters. If you have used the class macros to change
the layout in any way, you must call \cmd{\checkandfixthelayout} after you have
made all the necessary changes. As an aid, the final layout parameter values
are displayed on the terminal and written out to the log file.

\begin{syntax}
\cmd{\typeoutlayout} \\
\cmd{\typeoutstandardlayout} \\
\end{syntax}
The command \cmd{\typeoutlayout} may be used at any point in the document
to display the current class layout parameter values on the terminal and 
to write
them to the log file. The \cmd{\typeoutstandardlayout} does the same thing,
except that it outputs the values of the standard parameters.

    When pdfLaTeX is used to generate a PDF version of a \Lclass{memoir}
document some special setup must be done.

\begin{syntax}
\cmd{\fixpdflayout} \\
\end{syntax}
The macro \cmd{\fixpdflayout} is automatically called after 
the preamble when pdfLaTeX is used to generate PDF. 
The default definition is effectively:
\begin{lcode}
\newcommand{\fixpdflayout}{\ifpdf\ifnum\pdfoutput>0\relax
  \pdfpageheight=\the\stockheight
  \pdfpagewidth=\the\stockwidth
  \ifdim\pdfvorigin=0pt\pdfvorigin=1in\fi
  \ifdim\pdfhorigin=0pt\pdfhorigin=1in\fi
  \fi\fi}
\end{lcode}
The first settings (\verb?\pdfpage...?) ensure that pdfLaTeX knows the 
size of the physical sheet for printing. The \verb?\...origin? settings
set the pdf origin per the TeX origin, provided that their values are 0pt.
If you have set the origin values yourself, either in a pdfLaTeX 
configuration file or earlier in the preamble, the \cmd{\fixpdflayout} macro 
will not alter them (if you need an origin value to be 0, then set it to
0sp, which is visually indistinguishable from 0pt).

\begin{syntax}
\cmd{\fixdvipslayout} \\
\end{syntax}
The macro \cmd{\fixdvipslayout} is automatically called after the preamble
when PDF output is not being produced. It provides the dvips processor
with information about the stock size which a viewer or printer may use.

    With a landscape document and using the processing route
\texttt{latex -> dvips} the resulting \file{.ps} PostScript file may appear
upside-down when viewed with, say, \texttt{ghostview}. If this happens
try putting the following in your preamble:
\begin{lcode}
\addtodef{\fixdvipslayout}{}{%
  \special{!TeXDict begin /landplus90{true}store end }}
\end{lcode}
If required, additional code can be added to \cmd{\fixdvipslayout}
by repeated applications of \cmd{\addtodef}.
Some other potential specials for PostScript printing may be (at least
for an HP 5SiMx LaserJet duplex printer):
\begin{lcode}
\special{!TeXDict begin <</Duplex true>> setpagedevice end} % duplex
\special{!TeXDict begin <</Tumble true>> setpagedevice end} % short side binding\end{lcode}

\subsection{Example} \label{sec:pexamp}

    Suppose you want a page that will fit on both A4\index{paper!size!A4} and US letterpaper\index{paper!size!letterpaper} stock\index{stock},
wanting to do the least amount of trimming. The width of the typeblock\index{typeblock} should
be such that there are the optimum number of characters per line, and the 
ratio of the height to the width of the typeblock\index{typeblock} should equal the golden
section. The text has to start 50pt below the top of the page.
We will use the default \lnc{\headheight} and \lnc{\footskip}.
The ratio of the \foredge{} margin\index{margin} to the spine margin\index{margin} should equal the
golden section\index{golden section}, 
as should the space above and below the header\index{header}. There is
no interest at all in marginal\index{marginalia} notes, so we can ignore any settings for these.

    We can either do the maths ourselves or get LaTeX to do it for us. Let's
use LaTeX. First we will work out the size of the largest sheet that can be
cut from A4\index{paper!size!A4} and letterpaper\index{paper!size!letterpaper}, whose sizes are $297 \times 210$mm and 
$11 \times 8.5$in; A4 is taller and narrower than letterpaper. 
\begin{lcode}
\settrimmedsize{11in}{210mm}{*}
\end{lcode}
The stocksize is defined by the class option, but we have to work out the
trims to reduce the stock\index{stock} to the page. To make life easier, we will only trim
the \foredge{} and the bottom of the stock\index{stock}, so the \lnc{\trimtop} is zero.
The \lnc{\trimtop} and \lnc{\trimedge} are easily specified by 
\begin{lcode}
\setlength{\trimtop}{0pt} 
\setlength{\trimedge}{\stockwidth}
\addtolength{\trimedge}{-\paperwidth}
\end{lcode}
Specification of the size of the typeblock\index{typeblock} is also easy 
\begin{lcode}
\settypeblocksize{*}{\lxvchars}{1.618}
\end{lcode}
and now the upper and lower margins\index{margin} are specified by
\begin{lcode}
\setulmargins{50pt}{*}{*}
\end{lcode}
The spine and \foredge{} margins\index{margin} are specified just by the value of the golden section\index{golden section}, via 
\begin{lcode}
\setlrmargins{*}{*}{1.618}
\end{lcode}
The only remaining calculation to be done is the \lnc{\headdrop} and 
\lnc{\headsep}. Again this just involves using a ratio
\begin{lcode}
\setheaderspaces{*}{*}{1.618}
\end{lcode}
To finish off we have to make sure that the layout is changed 
\begin{lcode}
\checkandfixthelayout
\end{lcode}

\subsection{The page layout of this manual}

\begin{figure}
\currentstock
\oddpagelayouttrue
\twocolumnlayoutfalse
\drawmarginparstrue
\drawparametersfalse
\drawstock
\caption{The recto page layout for this manual}\label{fig:thispagelay}
\end{figure}



    The page layout for this manual is defined in the preamble\index{preamble} as:
\begin{lcode}
\settrimmedsize{11in}{210mm}{*}
\setlength{\trimtop}{0pt}
\setlength{\trimedge}{\stockwidth}
\addtolength{\trimedge}{-\paperwidth}
\settypeblocksize{7.75in}{33pc}{*}
\setulmargins{4cm}{*}{*}
\setlrmargins{1.25in}{*}{*}
\setmarginnotes{17pt}{51pt}{\onelineskip}
\setheadfoot{\onelineskip}{2\onelineskip}
\setheaderspaces{*}{2\onelineskip}{*}
\checkandfixthelayout
\end{lcode}

An illustration of the layout is shown in \fref{fig:thispagelay} which also
lists the parameter values resulting from the code above, to the nearest point.

I initially used the layout defined in \S\ref{sec:pexamp}, which I thought
looked reasonable, but then I decided to use the one above in order
to save paper\index{paper} when anyone printed out the manual. The wider
typeblock\index{typeblock} also makes it easier for TeX when dealing with lines that include
unhyphenatable text, like the LaTeX code.


    Andreas Mathias, via Rolf Niepraschk,\footnote{Email 
from \url{niepraschk@ptb.de} on 2002/02/05.}
has suggested that the following might be better for typesetting the manual
on A4 paper.
\begin{lcode}
\documentclass[a4paper]{memoir}
...
\settrimmedsize{297mm}{210mm}{*}
\setlength{\trimtop}{0pt}
\setlength{\trimedge}{\stockwidth}
\addtolength{\trimedge}{-\paperwidth}
\settypeblocksize{634pt}{448.13pt}{*}
\setulmargins{4cm}{*}{*}
\setlrmargins{*}{*}{1.5}
\setmarginnotes{17pt}{51pt}{\onelineskip}
\setheadfoot{\onelineskip}{2\onelineskip}
\setheaderspaces{*}{2\onelineskip}{*}
\checkandfixthelayout
\end{lcode}

\chapter{Titles and abstracts}

\section{Introduction}

   The typeset format of the \cmd{\maketitle} command is virtually fixed
within the LaTeX standard classes. This class
provides a set of formatting commands that can be used to modify
the appearance of the title\index{title} information; that is, the contents of
the \cmd{\title}, \cmd{\author} and \cmd{\date} commands. 
It also keeps the values
of these commands so that they may be printed again later in the
document.
   The class also inhibits the normal automatic cancellation of titling
commands after \cmd{\maketitle}. This means that you can have multiple
instances of the same, or perhaps different, titles in one document;
for example on a half title page\index{half title page} and the
full title page\index{title page}.
Hooks are provided so that additional titling elements can be defined
and printed by \cmd{\maketitle}.
  The \cmd{\thanks} command is enhanced to provide various configurations
for both the marker symbol and the layout of the thanks\index{thanks} notes.

Questions about how to have a one-column\index{column!single} abstract\index{abstract} in a 
two-column\index{column!double} paper
seem to pop up fairly regularly on the
\texttt{comp.text.tex} newsgroup. While an answer based on responses
on \texttt{ctt} is provided in the FAQ,
the class provides a more author-friendly means
of accomplishing this. Further, additional controls are provided
for the typesetting of the \Ie{abstract} environment in general.

\section{Titles}

\subsection{Styling the titling}


\index{title!styling|(}

   This part of the class is a reimplementation of the \Lpack{titling}
package~\cite{TITLING}.

The class provides a configurable \cmd{\maketitle} command.
The \cmd{\maketitle} command as defined by the class 
is essentially
\begin{lcode}
\newcommand{\maketitle}{%
   \vspace*{\droptitle}
   \maketitlehooka
   {\pretitle \title \posttitle}
   \maketitlehookb
   {\preauthor \author \postauthor}
   \maketitlehookc
   {\predate \date \postdate}
   \maketitlehookd
   \thispagestyle{title}
}
\end{lcode}
where the \pstyle{title} pagestyle is initially the same as the
\pstyle{plain} pagestyle.
The various macros used within \cmd{\maketitle} are described below.


\begin{syntax}
\cmd{\pretitle}\marg{text} \cmd{\posttitle}\marg{text} \\
\cmd{\preauthor}\marg{text} \cmd{\postauthor}\marg{text} \\
\cmd{\predate}\marg{text} \cmd{\postdate}\marg{text} \\
\end{syntax}
These six commands each have a single argument, \meta{text},
which controls the typesetting of the 
standard elements of the document's \cmd{\maketitle}
command. The \cmd{\title} is effectively processed between the 
\cmd{\pretitle} and \cmd{\posttitle} commands; that is, like:
\begin{lcode}
{\pretitle \title \posttitle}
\end{lcode}
and similarly for the \cmd{\author} and \cmd{\date} commands. The 
commands are initialised to mimic the normal result of \cmd{\maketitle}
typesetting in the \Lclass{report} class.
That is, the default definitions of the commands are:
\begin{lcode}
\pretitle{\begin{center}\LARGE}
\posttitle{\par\end{center}\vskip 0.5em}
\preauthor{\begin{center}
           \large \lineskip 0.5em%
           \begin{tabular}[t]{c}}
\postauthor{\end{tabular}\par\end{center}}
\predate{\begin{center}\large}
\postdate{\par\end{center}}
\end{lcode}

They can be changed to obtain different effects. For example to get
a right justified sans-serif title and a left justifed small caps
date:
\begin{lcode}
\pretitle{\begin{flushright}\LARGE\sffamily}
\posttitle{\par\end{flushright}\vskip 0.5em}
\predate{\begin{flushleft}\large\scshape}
\postdate{\par\end{flushleft}}
\end{lcode}

\begin{syntax}
\lnc{\droptitle} \\
\end{syntax}
 The \cmd{\maketitle} command puts the title at a particular height on the 
page. 
 You can change the vertical position of the title via the length
\lnc{\droptitle}. Giving this a positive value will lower the title and a
negative value will raise it. The default definition is: 
\begin{lcode}
\setlength{\droptitle}{0pt}
\end{lcode}

\begin{syntax}
\cmd{\maketitlehooka} \cmd{\maketitlehookb} \\
\cmd{\maketitlehookc} \cmd{\maketitlehookd} \\
\end{syntax}
 These four hook commands are provided so that additional elements may
be added to \cmd{\maketitle}. These are initially defined to do nothing
but can be renewed. For example, some publications
want a statement about where an article is published immediately
before the actual titling text. The following defines a command
\cmd{\published} that can be used to hold the publishing information
which will then be automatically printed by \cmd{\maketitle}.
\begin{lcode}
\newcommand{\published}[1]{%
   \gdef\puB{#1}}
\newcommand{\puB}{}
\renewcommand{\maketitlehooka}{%
   \par\noindent \puB}
\end{lcode}
You can then say:
\begin{lcode}
\published{Originally published in 
          \textit{The Journal of ...}\thanks{Reprinted with permission}}
...
\maketitle
\end{lcode}
to print both the published and the normal titling information. Note
that nothing extra had to be done in order to use the \cmd{\thanks} command
in the argument to the new \cmd{\published} command.

\index{title page|(}

\begin{syntax}
\senv{titlingpage} text \eenv{titlingpage} \\
\end{syntax}
   When one of the standard classes is used with the \Lopt{titlepage}
option, \cmd{\maketitle} puts the title elements on an unnumbered page
and then starts a new page numbered page 1. 
The standard classes also provide
a \Ie{titlepage} environment which starts a new unnumbered page and at the
end starts a new page numbered 1. You are entirely responsible
for specifying exactly what and where is to go on this title page.
If \cmd{\maketitle} is used  within the \Ie{titlepage} environment it
will start yet another page.

   This class provides neither a \Lopt{titlepage} option nor
a \Ie{titlepage} environment; instead it provides the \Ie{titlingpage}
environment which falls between the \Lopt{titlepage}
option and the \Ie{titlepage} environment. Within the \Ie{titlingpage}
environment you can use the \cmd{\maketitle} command, and any others 
you wish. The \pstyle{titlingpage} pagestyle is used, and 
at the end it starts another ordinary page numbered one. 
The \pstyle{titlingpage} pagestyle is initially defined to be the same
as the \pstyle{empty} pagestyle.

   For example, to put both the title and an abstract\index{abstract} 
on a title page,
with a \pstyle{plain} pagestyle:
\begin{lcode}
\begin{document}
\begin{titlingpage}
\aliaspagestyle{titlingpage}{plain}
\setlength{\droptitle}{30pt} lower the title
\maketitle
\begin{abstract}...\end{abstract}
\end{titlingpage}
\end{lcode}

   By default, titling information is centered with respect to the
width of the typeblock\index{typeblock}.
   Occasionally someone asks on the \texttt{comp.text.tex} newsgroup how
to center the titling information on a title page 
with respect to the width of the physical 
page. If the typeblock\index{typeblock} is centered with respect to the physical page,
then the default centering suffices. If the typeblock\index{typeblock} is not physically
centered, then the titling information either has to be shifted 
horizontally or \cmd{\maketitle} has to be made to think that the typeblock\index{typeblock}
has been shifted horizontally. The simplest solution is to use the
\cmd{\calccentering} and \Ie{adjustwidth*} command and environment. For
example:
\begin{lcode}
\begin{titlingpage}
  \calccentering{\unitlength}
  \begin{adjustwidth*}{\unitlength}{-\unitlength}
    \maketitle
  \end{adjustwidth*}
\end{titlingpage}
\end{lcode}

\index{title page|)}

\begin{syntax}
\cmd{\title}\marg{text} \cmd{\thetitle} \\
\cmd{\author}\marg{text} \cmd{\theauthor} \\
\cmd{\date}\marg{text} \cmd{\thedate} \\
\end{syntax}
   In the usual document classes, the contents (\meta{text}) of
the \cmd{\title}, \cmd{\author} and \cmd{\date}
macros used for \cmd{\maketitle} are unavailable once \cmd{\maketitle} 
has been
issued. The class provides the \cmd{\thetitle},
\cmd{\theauthor} and \cmd{\thedate} commands that can be used for printing
these elements of the title later in the document, 
if desired. 

\begin{syntax}
\cmd{\and} \cmd{\andnext} \\
\end{syntax}
   The macro \cmd{\and} is used within the argument to the
\cmd{\author} command to add some extra space between the author's names.
The class \cmd{\andnext} macro inserts a newline instead of a space.
Within the \cmd{\theauthor} macro both \cmd{\and} and \cmd{\andnext}
are replaced by a comma.

   The class does not follow the the standard classes' habit
of automatically killing the titling
commands after \cmd{\maketitle} has been issued. You can have multiple
\cmd{\title}, \cmd{\author}, \cmd{\date} and \cmd{\maketitle} 
commands in your
document if you wish. For example, some reports are issued with
a title page, followed by an executive summary, and then they
have another, possibly modified, title at the start of
the main body of the report. This can be accomplished like this:
\begin{lcode}
\title{Cover title}
...
\begin{titlingpage}
\maketitle
\end{titlingpage}
...
\title{Body title}
\maketitle
...
\end{lcode}

\begin{syntax}
\cmd{\killtitle} \cmd{\keepthetitle} \\
\cmd{\emptythanks} \\
\end{syntax}
    The \cmd{\killtitle} macro makes all aspects of titling, including
\cmd{\thetitle} etc.,
unavailable from the point that it is issued (using this command will save
some macro space if the \cmd{\thetitle}, etc., commands are not required).
Using this command is the class's manual version
of the automatic killing performed by the standard classes.
The \cmd{\keepthetitle} command performs a similar function, except that
it keeps the \cmd{\thetitle}, \cmd{\theauthor} and \cmd{\thedate} commands,
 while killing everything else.

The \cmd{\emptythanks} command discards any text from prior use of 
\cmd{\thanks}.
This command is useful when \cmd{\maketitle} is used multiple times ---
the \cmd{\thanks} commands in each use just stack up the texts for output
at each use, so each subsequent use of \cmd{\maketitle} will output all 
previous \cmd{\thanks} texts together with any new ones. To avoid this,
put \cmd{\emptythanks} before each \cmd{\maketitle} after the first.

\index{title!styling|)}


\subsection{Styling the thanks} \label{sec:thanks}

\index{thanks}
\index{thanks!styling|(}

    The class provides a configurable \cmd{\thanks} command.

\begin{syntax}
\cmd{\thanksmarkseries}\marg{format} \\
\cmd{\symbolthanksmark} \\
\end{syntax}
 Any |\thanks{}| are marked with symbols in the 
titling and footnotes\index{footnote}.
The command \cmd{\thanksmarkseries} 
can be used to change the marking style. The \meta{format} argument
is the name of one of the formats for printing a counter. The name 
is the same as that of a counter format but without the backslash.
To have the \cmd{\thanks} marks as lowercase letters instead of symbols 
do:
\begin{lcode}
\thanksmarkseries{alph}
\end{lcode}
Just for convenience the \cmd{\symbolthanksmark} command sets the series
to be footnote\index{footnote} symbols.
Using this class the potential names for \meta{format} are:
|arabic|, |roman|, |Roman|, |alph|, |Alph|, and |fnsymbol|. 

\begin{syntax}
\cmd{\continuousmarks} \\
\end{syntax}
The \cmd{\thanks} command uses the \Icn{footnote} counter, 
and normally the counter
is zeroed after the titling so that the footnote marks\index{footnote!mark} start from 1.
If the counter should not be zeroed, then just specify 
\cmd{\continuousmarks}.
This might be required if numerals are used as the thanks markers.

\begin{syntax}
\cmd{\thanksheadextra}\marg{pre}\marg{post} \\
\end{syntax}
The \cmd{\thanksheadextra} command will insert
\meta{pre} and \meta{post} before and after the thanks markers in the
titling block. By default \meta{pre} and \meta{post} are empty.
For example, to put parentheses round the titling markers do:
\begin{lcode}
\thanksheadextra{(}{)}
\end{lcode}


\begin{syntax}
\cmd{\thanksmark}\marg{n} \\
\end{syntax}
It is sometimes desireable to have the same thanks text be applied to,
say, four out of six authors, these being the first 3 and the last one.
The command \cmd{\thanksmark}\marg{n} is similar to 
\cmd{\footnotemark}\oarg{n} in that it prints a thanks mark identical
to that of the \meta{n}'th  \cmd{\thanks} command. No changes are made
to any thanks in the footnotes\index{footnote}. For instance, in the following
the authors Alpha and Omega will have the same mark:
\begin{lcode}
\title{The work\thanks{Draft}}
\author{Alpha\thanks{ABC},
        Beta\thanks{XYZ} and 
        Omega\thanksmark{2}} 
\maketitle
\end{lcode}

%\begin{syntax}
%\cmd{\thanksgap}\marg{length} \\
%\end{syntax}
%The marks in the titling block printed by the 
%\cmd{\thanks} and \cmd{\thanksmark}
%commands take zero space. This is acceptable if they come at the end of
%a line, but not in the middle of a line. Use the 
%\cmd{\thanksgap} command immediately after a mid-line
%\cmd{\thanks} or \cmd{\thanksmark} command to add \meta{length} amount of
%space before the next word. For example, if there are three authors
%from two different institutions:
%\begin{lcode}
%\author{Alpha\thanks{ABC},
%        Omega\thanks{XYZ}\thanksgap{1ex} and 
%        Beta\thanksmark{1}} 
%\end{lcode}

\begin{syntax}
\cmd{\thanksmarkstyle}\marg{defn} \\
\end{syntax}
By default the thanks mark at the foot is typeset as a superscript. In
the class this is specifed via
\begin{lcode}
\thanksmarkstyle{\textsuperscript{#1}}
\end{lcode}
where |#1| will be replaced by the thanks mark symbol. You can change
the mark styling
if you wish. For example, to put parentheses round the
mark and typeset it at normal size on the baseline:
\begin{lcode}
\thanksmarkstyle{(#1)}
\end{lcode}


\begin{syntax}
\lnc{\thanksmarkwidth} \\
\end{syntax}
 The thanks mark in the footnote\index{footnote} is typeset right justified in a box
of width \lnc{\thanksmarkwidth}. The first line of the thanks text starts
after this box. The initialisation is 
\begin{lcode}
\setlength{\thanksmarkwidth}{1.8em}
\end{lcode}
giving the default position.

\begin{syntax}
\lnc{\thanksmarksep} \\
\end{syntax}
The value of the length
 \lnc{\thanksmarksep} controls the indentation the
second and subsequent lines of the thanks text, with respect to
the end of the mark box. As examples: 
\begin{lcode}
\setlength{\thanksmarksep}{0em}
\end{lcode}
 will align the left hand ends of of a multiline thanks text, while: 
\begin{lcode}
\setlength{\thanksmarksep}{-\thanksmarkwidth}
\end{lcode}
will left justify any second and subsequent lines of the thanks text. 
This last
setting is the initialised value, giving the default appearance.

\begin{syntax}
\cmd{\thanksfootmark} \\
\end{syntax}
    A thanks mark in the footnote\index{footnote} region is typeset by \cmd{\thanksfootmark}.
The code for this is roughly:
\begin{lcode}
\newcommand{\thanksfootmark}{%
  \hbox to\thanksmarkwidth{\hfil\normalfont%
     \thanksscript{\thanksfootpre \thefootnote \thanksfootpost}}}
\end{lcode}
where \cmd{\thanksfootpre} and \cmd{\thanksfootpost} are specified via the
\cmd{\thanksfootextra} macro. You should not need to change the definition
of \cmd{\thanksfootmark} 
but you may want to change the default definitions of one or more
of the macros it uses.

\begin{comment}

\begin{syntax}
\cmd{\thanksscript}\marg{text} \\
\end{syntax}
Note that the thanks mark, together with the |\...pre| and |\...post|
commands form the \meta{text} argument to the \cmd{\thanksscript} command. 
This is initially defined as: 
\begin{lcode}
\newcommand{\thanksscript}[1]{\textsuperscript{#1}}
\end{lcode}
so that \cmd{\thanksscript} typesets its argument as a superscript, which
is the default for thanks notes. If you would prefer the mark to be
set at the baseline instead, for example, just do: 
\begin{lcode}
\renewcommand{\thanksscript}[1]{#1}
\end{lcode}
 and if you also wanted very small symbols on the baseline you could do:
\begin{lcode}
\renewcommand{\thanksscript}[1]{\tiny #1}
\end{lcode}

Alternatively 
\begin{lcode}
\renewcommand{\thanksscript}[1]{#1}
\thanksfootextra{\bfseries}{.}
\end{lcode}
will give a bold baseline mark followed by a dot.

   Using combinations of these macros you can easily define
different layouts for the thanks footnotes\index{footnote}. Here are some
sample, but to shorten them I have ignored any
|\renewcommand|s and |\setlength|s, leaving these to be implied
as necessary.
\begin{itemize}
\item Setting |\thanksfootextra{}{\hfill}| left justifies the mark
    in its box:
   \begin{itemize}
   \item |\thanksscript{\llap{#1\space}}|, |\thanksmarkwidth{0em}| and \\
         |\thanksmargin{0em}| puts the baseline mark in the margin\index{margin}
         and the text left justified.
   \item Using |\thanksscript{#1}|, |\thanksmarkwidth{1em}| and \\
         |\thanksmargin{-\thanksmarkwidth}| makes the baseline mark 
         and text left adjusted.
   \item |\thanksscript{#1}|, |\thanksmarkwidth{1em}| and
         |\thanksmargin{0em}| puts the baseline mark left adjusted
         and the text indented and aligned, like this marked
         sentence is typeset.
   \end{itemize}

\item Setting |\thanksfootextra{ }| and |\thanksscript{#1}| 
      right justifies the baseline mark and a space in the mark box:
   \begin{itemize}
   \item The normal style is
         defined by |\thanksmarkwidth{1.8em}| and \\
         |\thanksmargin{-\thanksmarkwidth}| which put the mark 
         indented and the text left adjusted, like a normal indented
         paragraph\index{paragraph!indentation}.
   \item |\thanksmarkwidth{1.8em}| and
         |\thanksmargin{0em}| put the mark indented 
         and the text indented and aligned.
   \end{itemize}

\end{itemize}

%%%%%%%%%%%%%%%%%%%%%
\end{comment}
%%%%%%%%%%%%%%%%%%%%

\begin{syntax}
\cmd{\makethanksmark} \\
\cmd{\makethanksmarkhook} \\
\end{syntax}
The macro \cmd{\makethanksmark} typesets both the thanks marker (via
\cmd{\thanksfootmark}) and the thanks text. You probably will not need
to change its default definition. Just in case, though, 
\cmd{\makethanksmark}
calls the macro \cmd{\makethanksmarkhook} before it does any typesetting.
The default definition of this is: 
\begin{lcode}
\newcommand{\makethanksmarkhook}{}
\end{lcode}
which does nothing.

   You can redefine \cmd{\makethanksmarkhook} to do something useful. For
example, if you wanted a slightly bigger baseline skip you could do:
\begin{lcode}
\renewcommand{\makethanksmarkhook}{\fontsize{8}{11}\selectfont}
\end{lcode}
where the numbers |8| and |11| specify the point size of the font 
and the baseline skip
respectively. In this example 8pt is the normal \cmd{\footnotesize} in
a 10pt document, and 11pt is the baselineskip for \cmd{\footnotesize}
text in an 11pt document (the baseline skip is 9.5pt in a 10pt document); 
adjust these numbers to suit.

\begin{syntax}
\cmd{\thanksrule} \\
\cmd{\usethanksrule} \\
\cmd{\cancelthanksrule} \\
\end{syntax}
By default, there is no rule above \cmd{\thanks}
text that appears in a \Ie{titlingpage} environment.
If you want a rule in that environment, put \cmd{\usethanksrule} 
before the \cmd{\maketitle} command, which will then print a rule according
to the current definition of \cmd{\thanksrule}.
\cmd{\thanksrule} is initialised to be a copy of \cmd{\footnoterule} as it 
is defined at the
end of the preamble\index{preamble}. The definition of \cmd{\thanksrule} can be changed
after |\begin{document}|. If the definition of \cmd{\thanksrule} is modified
and a \cmd{\usethanksrule} command has been issued, then the redefined
rule may also be used for footnotes\index{footnote}. Issuing the command 
\cmd{\cancelthanksrule} will cause the normal \cmd{\footnoterule} definition
to be used from thereon; another \cmd{\usethanksrule}
command can be issued later
if you want to swap back again.

The parameters for the vertical positioning of footnotes\index{footnote} 
and thanks notes, and the default \cmd{\footnoterule} are the same
(see \fref{fig:fn} on \pref{fig:fn}).
You will have to change one or more of these if the vertical spacings of 
footnotes\index{footnote}
and thanks notes are meant to be different.


\index{thanks!styling|)}

\section{Abstracts}

\index{abstract|(}

    Much of this part of the class is a reimplementation of the 
\Lpack{abstract} package~\cite{ABSTRACT}.

   The typeset format of an \Ie{abstract} in a \Lclass{report} or 
\Lclass{article} class\footnote{The \texttt{abstract} environment is not 
available for the \Lclass{book} class.} document depends on the class
options. The formats are:
\begin{itemize}
\item \Lopt{titlepage} class option: The abstract heading\index{heading!abstract} (i.e., value of 
   \cmd{\abstractname}) is typeset centered in a bold font; the text is set in
   the normal font and to the normal width.
\item \Lopt{twocolumn} class option: The abstract heading\index{heading!abstract} is typeset like
   an unnumbered section; the text is set in the normal font and to the
   normal width (of a single column\index{column!single}). 
\item Default (neither of the above class options): The abstract heading\index{heading!abstract}
   is typeset centered in a small bold font; the text is set in a small
   font and indented like the |quotation| environment.
\end{itemize}

   This class provides an \Ie{abstract} environment and
handles to modify the typesetting of an \Ie{abstract}.

\begin{syntax}
\senv{abstract} text \eenv{abstract} \\
\end{syntax}
There is nothing special about using the \Ie{abstract} environment. 
Formatting is controlled by the macros described below.

   The usual advice~\cite{FAQ} about creating a one-column\index{column!double} 
\Ie{abstract} in a 
\Lopt{twocolumn} document is to write code like this:
\begin{lcode}
\documentclass[twocolumn...]{...}
...
\twocolumn[
   \begin{@twocolumnfalse}
     \maketitle               need full-width title
     \begin{abstract}
        abstract text...
     \end{abstract}
   \end{@twocolumnfalse}
]
... hand make footnotes for any \thanks commands
...
\end{lcode}

\begin{syntax}
\senv{onecolabstract} text \eenv{onecolabstract} \\
\cmd{\saythanks} \\
\end{syntax}
The class provides a \Ie{onecolabstract} environment that you can use
for a one column\index{column!single} abstract in a \Lopt{twocolumn} document, and it is used
like this:
\begin{lcode}
\documentclass[twocolumn...]{memoir}
...
\twocolumn[
   \maketitle               need full-width title
   \begin{onecolabstract}
     abstract text...
   \end{onecolabstract}
]
\saythanks  typesets any \thanks commands
...
\end{lcode}
The command \cmd{\saythanks} ensures that any \cmd{\thanks} texts are 
printed out as normal.

    If you want, you can use the \Ie{onecolabstract} environment in place
of the \Ie{abstract} environment --- it doesn't have to be within the 
optional argument of the \cmd{\twocolumn} command. In fact, 
\Ie{onecolabstract} internally uses \Ie{abstract} for the typesetting.

\begin{syntax}
\cmd{\abstractcol} \\
\cmd{\abstractintoc} \\
\cmd{\abstractnum} \\
\cmd{\abstractrunin} \\
\end{syntax}
    The normal format for an abstract is with a centered, bold title
and the text in a small font, inset from the margins\index{margin}.

The \cmd{\abstractcol} declaration specifies that an abstract in a
\Lopt{twocolumn} class option document should be typeset like a
normal, unnumbered chapter.
The \cmd{\abstractintoc} specifies that the abstract title should
be added to the \toc. The declaration \cmd{\abstractnum} specifies that the 
abstract should be typeset like a numbered chapter and 
\cmd{\abstractrunin} specifies that
the title of the abstract should look like a runin heading; these two
declarations are mutually exclusive. Note that the \cmd{\abstractnum}
declaration has no effect if the abstract is in the \cmd{\frontmatter}.

\begin{syntax}
\cmd{\abstractnamefont} \\
\cmd{\abstracttextfont} \\
\end{syntax}
   These two commands can be redefined to change the fonts used for 
typesetting the heading (defined via \cmd{\abstractname}) of the 
\Ie{abstract} 
environment and the font for typesetting the text of the abstract,
respectively. The default definitions are 
\begin{lcode}
\newcommand{\abstractnamefont}{\normalfont\small\bfseries}
\newcommand{\abstracttextfont}{\normalfont\small}
\end{lcode}

\begin{syntax}
\lnc{\absleftindent} \lnc{\absrightindent} \\
\lnc{\absparindent} \lnc{\absparsep} \\
\end{syntax}
   This version of \Ie{abstract} uses a list environment for typesetting
the text. These four lengths can be changed (via \cmd{\setlength}
 or \cmd{\addtolength}) to adjust
the left and right margins\index{margin}, the paragraph indentation\index{paragraph!indentation}, and the vertical skip
between paragraphs in this environment. 
The default values depend on whether or not the \Lopt{twocolumn}
class option is used.

\begin{syntax}
\cmd{\abslabeldelim}\marg{text} \\
\end{syntax}
If the \cmd{\abstractrunin} declaration has been given, 
the heading is typeset as a run-in heading. That is, it is the 
first piece of text on the first line of the abstract text.
The \meta{text} argument of \cmd{\abslabeldelim} is typeset
immediately after the heading. By default it is defined to do nothing, but
if you wanted, for example, the \cmd{\abstractname} to be followed by 
a colon and some extra space you could specify
\begin{lcode}
\abslabeldelim{:\quad}
\end{lcode}

\begin{syntax}
\cmd{\absnamepos} \\
\end{syntax}
   If the \cmd{\abstractrunin} declaration is not used then the heading 
is typeset in its own environment, specified by 
\cmd{\absnamepos}. The default definition is
\begin{lcode}
\newcommand{\absnamepos}{center}
\end{lcode}
It can be defined to be one of \Ie{flushleft}, \Ie{center},
or \Ie{flushright} to give a left, centered or right aligned heading; 
or to any
other appropriate environment which is supported by a used package\index{package}.

\begin{syntax}
\lnc{\abstitleskip} \\
\end{syntax}
   With the \cmd{\abstractrunin} declaration a horizontal space of length 
\lnc{\abstitleskip} is typeset
before the heading. For example, if \lnc{\absparindent} is non-zero, then:
\begin{lcode}
\setlength{\abstitleskip}{-\absparindent}
\end{lcode}
 will typeset the heading flush left.

Without the \cmd{\abstractrunin} declaration, \lnc{\abstitleskip} is 
aditional vertical 
space (either positive
or negative) that is inserted between the abstract name and the text of
the abstract.

\index{abstract|)}



%%%%%%%%%%%%%%%%%%%%%%%%%%%%%
\renewcommand{\printchaptername}{}
\renewcommand{\chapternamenum}{}
\renewcommand{\chapnumfont}{\normalfont\huge\sffamily}
\renewcommand{\chaptitlefont}{\chapnumfont}
\renewcommand{\afterchapternum}{\space}
\setsecheadstyle{\Large\sffamily\raggedright}
\setsubsecheadstyle{\large\sffamily\raggedright}
\setsubsubsecheadstyle{\normalsize\sffamily\raggedright}
%%%%%%%%%%%%%%%%%%%%%%%%%%%%%%%%


\chapter{Document divisions}

    The chapter is typeset using the sectional headings\index{heading} styles specified
at the end of the chapter.


\section{Introduction}

    In this chapter I first discuss the various kinds of divisions
within a book and the commands for typesetting these.

After that I describe the class methods for modifying the appearance
of the chapter and other sectional headings\index{heading}.
    The facilities described here provide roughly the same as you would
get if you used the \Lpack{titlesec}~\cite{TITLESEC} and 
\Lpack{sectsty}~\cite{SECTSTY} packages together; the commands are different,
though.

\section{Book divisions}

    As described earlier there are three main logical divisions to a book;
the front, main and back matter. There are three LaTeX 
commands that correspond to these, namely \cmd{\frontmatter},
\cmd{\mainmatter} and \cmd{\backmatter}.

\begin{syntax}
\cmd{\frontmatter} \cmd{\frontmatter*} \\
\end{syntax}
The \cmd{\frontmatter} declaration sets the folios\index{folio} to be printed in lowercase
roman numerals, starts the page numbering from~i, and prohibits any numbering
of sectional divisions. Caption\index{caption}, equations, etc., will be numbered 
continuously.  The starred version of the command,
\cmd{\frontmatter*}, is similar to the unstarred version except that it
makes no changes to the page numbering or the print style for the folios\index{folio}.

\begin{syntax}
\cmd{\mainmatter} \cmd{\mainmatter*} \\
\end{syntax}
The \cmd{\mainmatter} declaration, which is the default at the start of a document,
sets the folios\index{folio} to be printed in arabic numerals, starts the page numbering 
from~1, and sections and above will be numbered. Float\index{float} captions\index{caption}, equations,
etc.,  will be numbered per chapter\index{chapter}. The starred version of the command,
\cmd{\mainmatter*}, is similar to the unstarred version except that it
makes no changes to the page numbering or the print style for the folios\index{folio}.


\begin{syntax}
\cmd{\backmatter} \\
\end{syntax}
The \cmd{\backmatter} declaration makes no change to the pagination or folios\index{folio}
but does prohibit sectional division numbering, and captions\index{caption}, etc., will
be numbered continuously.


\section{Sectional divisions}

    The \theclass{} class lets you divide a document up into seven levels
of named divisions. They range from part\index{part} through chapter\index{chapter} and down to 
sub-paragraph. A particular sectional division is specified by one of
the commands \cmd{\part}, \cmd{\chapter}, \cmd{\section}, \cmd{\subsection},
which is probably as deep as you want to go. If you really need finer
divisions, they are
 \cmd{\subsubsection}, \cmd{\paragraph} and lastly \cmd{\subparagraph}.
All the sectional commands, except for chapters,
 have the same form, so rather than describing 
each one in turn I will use \cmd{\section} to stand for any one of them.

\begin{syntax}
\cmd{\section}\oarg{toc-title}\oarg{head-title}\marg{title}\\
\cmd{\section*}\marg{title}\\
\end{syntax} 
There are two forms of the command; 
the starred version is simpler, so I'll describe its 
effects first --- it just typesets \meta{title} in the document in the format
for that particular sectional division. Like the starred version, the plain
version also typesets \meta{title} in the document, but it may be numbered.
Diferent forms of the division title are available for the 
Table of Contents (\toc) and a running header\index{header}, as follows:
\begin{itemize}
\item No optional argument: \meta{title} is used for the division title,
      the ToC title and a page header title.
\item One optional argument: \meta{title} is used for the division title;
      \meta{toc-title} is used for the ToC title and a page header\index{header} title.
\item Two optional arguments: \meta{title} is used for the division title;
      \meta{toc-title} is used for the ToC title; \meta{head-title}
      is used for a page header\index{header} title.
\end{itemize}

A \cmd{\section} command restarts the numbering of any \cmd{\subsection}s
from one.
For most of the divisions the \meta{title} is put on the page where the command
was issued. The \cmd{\part} and \cmd{\chapter} commands behave a little 
differently.

The \cmd{\part}\meta{title} command puts the part name (default |Part|),
 number and
\meta{title} on a page by itself. The
numbering of parts\index{part!number} has no effect on the numbering of chapters\index{chapter!number}.

    Later I'll give a list of LaTeX's default names, like |Part|.

\begin{syntax}
\cmd{\chapter}\oarg{toc-title}\oarg{head-title}\marg{title} \\
\cmd{\chapter*}\oarg{head-title}\marg{title} \\
\end{syntax}

The \cmd{\chapter} command starts a new page and puts
the chapter name (default |Chapter|), number and \meta{title}
at the top of the page. It restarts the numbering of any \cmd{\section}s 
from one. If no optional arguments are specified, \meta{title}
is used as the \toc{} entry and for any page headings. If one optional
argument is specified this is \meta{toc-title} and is used for the
\toc{} entry and for page headings. If both optional arguments
are specified the \meta{head-title} is used for page headings.

The \cmd{\chapter*} command starts a new page and puts
\meta{title} at the top of the page. It makes no \toc{} entry, 
changes no numbers and by default changes no page headings.
If the optional \meta{head-title} argument is given, this is used
for page headings. Use of the optional argument has the side-effect
that the \Icn{secnumdepth} counter is set to |maxsecnumdepth| (see below
for an explanation of these).

    When the \Lopt{article} option is in effect, however, things are slightly
different. New chapters do not necessarily start on a new page.
The \cmd{\mainmatter} command
just turns on sectional numbering and starts arabic page numbering; the 
\cmd{\backmatter} command just turns off sectional numbering.
    The \cmd{\tableofcontents} command and friends, as well as any
other commands created via \cmd{\newlistof}, 
\emph{always}\footnote{This is a consequence of the internal
timing of macro calls.}
call |\thispagestyle{chapter}|. If you are using the \Lopt{article} option you
will probably want to ensure that the \pstyle{chapter} pagestyle is the
same as you normally use for the document.

    Unlike the standard classes the \meta{title} is typeset ragged right.
This means that if you need to force a linebreak in the \meta{title} you 
have to use \cmd{\newline} instead of the more usual \cmd{\\}. For instance
\begin{lcode}
\section{A broken\newline title}
\end{lcode}

    In the standard classes a \cmd{\section} or other divisional heading
that is too close to the bottom of a page is moved to the top of the
following page. If this happens and \cmd{\flushbottom} is in effect, the
contents of the short page are stretched to make the last line flush with
the bottom of the typeblock.
\begin{syntax}
\cmd{\raggedbottomsectiontrue} \\
\cmd{\raggedbottomsectionfalse} \\
\lnc{\bottomsectionskip} \\
\end{syntax}
The \cmd{\raggedbottomsectiontrue} declaration will typeset any pages that 
are short because of a moved divisional header as though \cmd{\raggedbottom}
was in effect for the short page; other pages are not affected. The
length \lnc{\bottomsectionskip} controls the amount of stretch on the short
page. Setting it to zero allows the last line to be flush with the bottom
of the typeblock. The default setting of 10mm appears to remove any stretch.

The declaration \cmd{\raggedbottomsectionfalse}, which is the default,
cancels any previous \cmd{\raggedbottomsectiontrue} declaration.


    
\begin{syntax}
\cmd{\plainbreak}\marg{num} \cmd{\plainbreak*}\marg{num}  \\
\cmd{\fancybreak}\marg{text} \cmd{\fancybreak*}\marg{text}   \\
\end{syntax} 
    The \cmd{\plainbreak} is an anonymous division. It puts \meta{num}
blank lines into the typescript and the first line of the following
paragraph is not indented\index{paragraph!indentation}. Another anonymous division is
\cmd{\fancybreak} which puts \meta{text} centered into the typescript and the
initial line of the following paragraph is not indented\index{paragraph!indentation}. For
example:
\begin{lcode}
\fancybreak{{*}\\{* * *}\\{*}}
\end{lcode}
typesets a little diamond made of asterisks.

    The starred versions of the commands indent the first line of the
following paragraph\index{paragraph!indentation}.

\begin{syntax}
\cmd{\plainfancybreak}\marg{space}\marg{num}\marg{text} \\
\cmd{\plainfancybreak*}\marg{space}\marg{num}\marg{text} \\
\end{syntax}
If a plain break comes at the top or bottom of a page then it is very 
difficult for a reader to discern that there is a break at all.
If there is text on the page and enough space left to put some text
after a break the \cmd{\plainfancybreak} command will use a \cmd{\plainbreak}
with \meta{num} lines, 
otherwise (the break would come at the top or bottom of the page) it
will use a \cmd{\fancybreak} with \meta{text}. The \meta{space} argument is a
length specifying the space needed for the \meta{num} blank lines and some
number of text lines for after the plain break. The starred version of
the command uses the starred versions of the \cmd{\plainbreak} and
\cmd{\fancybreak} commands.

    Unfortunately there is an interaction between the requested, plain,
and fancy break spaces.
    Let $P$ be the space (in lines) required for the plain break, 
$F$ the space (in lines)
required for the fancy break, and $S$ the \meta{space} space (in lines). 
From some experiments it appears that the condition for the plain break 
to avoid the top and bottom of the page is that $S - P > 1$. 
Also, the condition for the fancy break to avoid being put in the middle 
of a page (i.e., not at the top or bottom) is that  $S - F < 3$.
For example, if the plain and fancy breaks take the same vertical space
then $S = P +2$ is the only value that matches the conditions. In general, 
if $F = P + n$ then the condition is $1 < S-P < 3+n$, which means that 
for the \cmd{\plainfancybreak} command the
fancy break must always take at least as much space as the plain break.


\fancybreak{\pfbreakdisplay}

    The \cmd{\plainfancybreak} macro inserts a plain break in the middle of
a page or if the break would come at the bottom or top of a page it
inserts a fancy break instead.

\begin{syntax}
\cmd{\pfbreak} \cmd{\pfbreak*} \\
\lnc{\pfbreakskip} \\
\cmd{\pfbreakdisplay}\marg{text} \\
\end{syntax}
The \cmd{\pfbreak} macro is an alternate for \cmd{\plainfancybreak} that may
be more convenient to use. The gap for the plain break is given by the
length \lnc{\pfbreakskip} which is initialised to produce two blank lines.
The fancy break, which takes the same vertical space, is given by the
\meta{text} argument of \cmd{\pfbreakdisplay}. The default definition
typesets three asterisks, as shown a few lines before this.

%%\renewcommand{\pfbreakdisplay}{\huge \ding{167}\quad\ding{167}\quad\ding{167}}
\renewcommand{\pfbreakdisplay}{%
  $\clubsuit$\quad$\diamondsuit$\quad$\clubsuit$}
\fancybreak{\pfbreakdisplay}

    You can change the definition of \cmd{\pfbreakdisplay} for a different
style if you wish. The
fancy break just before this was produced via:
\begin{lcode}
\renewcommand{\pfbreakdisplay}{%
  $\clubsuit$\quad$\diamondsuit$\quad$\clubsuit$}
\fancybreak{\pfbreakdisplay}
\end{lcode}
I used \cmd{\fancybreak} as I'm not sure where the break will come on the
page and the simple \cmd{\pfbreak} macro might just have produced a couple
of blank lines instead of the fancy display.

  The paragraph following \cmd{\pfbreak} is not indented. If you want
it indented use the \cmd{\pfbreak*} starred version.


\begin{syntax}
\cmd{\tableofcontents} \cmd{\tableofcontents*} \\
\cmd{\listoffigures} \cmd{\listoffigures*} \\
\cmd{\listoftables} \cmd{\listoftables*} \\
\end{syntax}
In the standard classes the command \cmd{\tableofcontents} typesets a 
Table of Contents (\toc) at the point in the document where it is issued. 
In the \theclass{} class it also adds \emph{its}
title to the \toc. There is a starred version, \cmd{\tableofcontents*}, 
which does not enter itself into the \toc. So, in this class 
\cmd{\tableofcontents*} is equivalent to the standard classes'
\cmd{\tableofcontents}. The class also provides the \cmd{\listoffigures}
and \cmd{\listoftables} commands which typeset a List of 
Figures (\lof) and
a List of Tables (\lot), also entering their titles into the \toc. The 
starred versions of these make no \toc{} entry.

    You can use \cmd{\tableofcontents}, \cmd{\listoffigures}, etc., more
than once in a \Lclass{memoir} class document.

\begin{syntax}
\cmd{\appendix} \\
\cmd{\appendixname} \\
\end{syntax}
In the standard classes the \cmd{\appendix} command changes the numbering of
chapters\index{chapter!number} to an alphabetic form and changes the names of chapters from 
\cmd{\chaptername} (default |Chapter|)
to the value of \cmd{\appendixname} (default |Appendix|)\index{appendix}. 
Thus, the first and any subsequent \cmd{\chapter}s after 
the \cmd{\appendix} command
will be `Appendix A \ldots', `Appendix B \ldots', and so on. 
This class provides the same command plus some other ways of dealing with
appendices\index{appendix}.

\begin{syntax}
\cmd{\appendixpage}\\
\cmd{\appendixpage*}\\
\cmd{\appendixpagename}\\
\end{syntax}
The \cmd{\appendixpage} command generates a part-like page (but no |Part|
or number) with the title given by the value of \cmd{\appendixpagename} 
(default |Appendices|)\index{appendix}. It also makes an entry in the \toc{} using 
\cmd{\addappheadtotoc} (see below). The starred version generates the 
appendix\index{appendix} page but makes no \toc{} entry.

\begin{syntax}
\cmd{\addappheadtotoc}\\
\cmd{\appendixtocname}\\
\end{syntax}
The command \cmd{\addappheadtotoc} adds an entry to the \toc. The title is
given by the value of \cmd{\appendixtocname} (default |Appendices|)\index{appendix}.

\begin{syntax}
\senv{appendices} text \eenv{appendices}\\
\end{syntax}
The \Ie{appendices} environment acts like the \cmd{\appendix}\index{appendix} command in that it
resets the numbering and naming of chapters\index{chapter}. However, at the end of the
environment, chapters are restored to their original condition and any
chapter numbers continue in sequence as though the \Ie{appendices} environment
had never been there.

\begin{syntax}
\senv{subappendices} text \eenv{subappendices} \\
\cmd{\namesubappendixtrue} \cmd{\namesubappendixfalse} \\
\end{syntax}
The \Ie{subappendices} environment can be used to put appendices\index{appendix} at the end
of a chapter\index{chapter}. Within the environment \cmd{\section} starts a 
new sub-appendix\index{appendix}. You may put \cmd{\addappheadtotoc} at the start
of the environment if you want a heading entry in the \toc.
If you put the declaration \cmd{\namesubappendixtrue}
\emph{before} the \Ie{subappendices} environment, the sub-appendix number
in the body of the document will be preceded by the value of 
\cmd{\appendixname}. The \cmd{\namesubappendixfalse} declaration may be 
used to switch off this behaviour.




\section{Numbering} \label{sec:secnumbers}

    Each type of sectional division has an associated \emph{level} as
shown in \tref{tab:seclevels}.
Divisions are numbered if the value of the \Icn{secnumdepth} counter
is equal to or greater than their level. For example, with
\begin{lcode}
\setcounter{secnumdepth}{2}
\end{lcode}
then subsections up to parts\index{part!number} will be numbered.

\begin{table}
\centering
\caption{Division levels} \label{tab:seclevels}
\begin{tabular}{lr} \hline
Division       & Level \\ \hline
\cmd{\part}           & -1 \\
\cmd{\chapter}        & 0 \\
\cmd{\section}        & 1 \\
\cmd{\subsection}     & 2 \\
\cmd{\subsubsection}  & 3 \\
\cmd{\paragraph}      & 4 \\
\cmd{\subparagraph}   & 5 \\ \hline
\end{tabular}
\end{table}

\begin{syntax}
\cmd{\setsecnumdepth}\marg{secname} \\
\cmd{\maxsecnumdepth}\marg{secname} \\
\end{syntax}
Instead of having to remember the levels if you want to change what
gets numbered you can use the \cmd{\setsecnumdepth} command. It
sets \Icn{secnumdepth} so that divisions \meta{secname} and above
will be numbered. The argument \meta{secname} is the name of a sectional
division without the backslash. For example, to have subsections
and above numbered:
\begin{lcode}
\setsecnumdepth{subsection}
\end{lcode}

    The command \cmd{\maxsecnumdepth} sets the maximum value for
\Icn{secnumdepth} that can be used in the document. You can use
\cmd{\setsecnumdepth} anywhere you want in the document to (temporarily)
change the numbering level.

    By default, the class sets:
\begin{lcode}
\maxsecnumdepth{section}
\setsecnumdepth{section}
\end{lcode}
The commands \cmd{\mainmatter} and \cmd{\mainmatter*} set the
\Icn{secnumdepth} to the value specified by \cmd{\maxsecnumdepth}.

    The number setting commands come from the \Lpack{tocvsec2}
package~\cite{TOCVSEC2}.





\section{Part headings}

\index{heading!part|(}

    Part titles \emph{always} start on a new page with the \pstyle{part}
pagestyle.


    Several aspects of the typesetting of the \cmd{\part} title are 
configurable.

\begin{syntax}
\cmd{\beforepartskip} \cmd{\afterpartskip} \\
\end{syntax}
These two commands effectively control the spacing before and after the part 
title. Their default definitions are:
\begin{lcode}
\newcommand{\beforepartskip}{\null\vfil}
\newcommand{\afterpartskip}{\vfil\newpage}
\end{lcode}
Together, these vertically center any typesetting on the page, and then start
a new page. To move the title upwards on the page, for example, you could do:
\begin{lcode}
\renewcommand{\beforepartskip}{\null\vskip 0pt plus 0.3fil}
\renewcommand{\afterpartskip}{\vskip 0pt plus 0.7fil \newpage}
\end{lcode} 

\begin{syntax}
\lnc{\midpartskip} \\
\end{syntax}
The length \lnc{\midpartskip} is the vertical distance between the part 
number line and the part title. Its default value is 20pt and can be changed
by \cmd{\setlength} or \cmd{\addtolength}. It is probably better to have
the spacing in terms of the \lnc{\baselineskip}, which will change depending
on the font size, instead of a fixed amount.

\begin{syntax}
\cmd{\printpartname} \cmd{\partnamefont} \\
\cmd{\partnamenum} \\
\cmd{\printpartnum} \cmd{\partnumfont} \\
\end{syntax}
The macro \cmd{\printpartname} typesets the part name 
(the value of \cmd{\partname}) using the font 
specified by \cmd{\partnamefont}. The default is the \cmd{\bfseries} font in
the \cmd{\huge} size. Likewise the part number is typeset by \cmd{\printpartnum}
using the font specified by \cmd{\partnumfont}, which has the same default as
\cmd{\partnamefont}. The macro 
\cmd{\partnamenum}, which is defined to be a space, is called between printing
the part name and the number. All these can be changed to obtain different 
effects. For example, to have a large sans font with the part name flush left:
\begin{lcode}
\renewcommand{\partnamefont}{\normalfont\huge\sffamily\raggedright}
\renewcommand{\partnumfont}{\normalfont\huge\sffamily}
\end{lcode}
or to only print the part number in the default font:
\begin{lcode}
\renewcommand{\printpartname}{}
\renewcommand{\partnamenum}{}
\end{lcode}


\begin{syntax}
\cmd{\printparttitle}\marg{title} \cmd{\parttitlefont} \\
\end{syntax}
The title is typeset by \cmd{\printparttitle} using the font specified 
by \cmd{\parttitlefont}. 
By default this is a \cmd{\bfseries} font in the \cmd{\Huge} size. This can
be changed to have, say, the title set raggedleft in a small caps font by
\begin{lcode}
\renewcommmand{\parttitlefont}{\normalfont\Huge\scshape\raggedleft}
\end{lcode}

    The \cmd{\parttitlefont} font is also used by 
\cmd{\appendixpage}, or its starred version, when
typesetting an appendix\index{appendix} page.

\index{heading!part|)}

\section{Chapter headings}

\index{heading!chapter|(}

    The chapter headings are configurable in much the same way as part 
headings, but in addition there are some built in chapter styles that you may
wish to try, or define your own.

    Chapters \emph{always} start on a new page with the \pstyle{chapter}
pagestyle. The particular page, recto or verso, that they start on is
mainly controlled by the class options. If the \Lopt{oneside} option is used
they start on the next new page, but if the \Lopt{twoside} option is
used the starting page may differ, as follows.
\begin{itemize}
\item[\Lopt{openright}] The chapter heading is typeset on the next recto page,
  which may leave a blank verso leaf.
\item[\Lopt{openleft}] The chapter heading is typeset on the next verso page,
  which may leave a blank recto leaf.
\item[\Lopt{openany}] The chapter heading is typeset on the next page and there
  will be no blank leaf.
\end{itemize}

\begin{syntax}
\cmd{\openright} \cmd{\openleft} \cmd{\openany} \\
\end{syntax}
These three declarations have the same effect as the options of the same name.
They can be used anywhere in the document to change the chapter opening page.

\begin{syntax}
\cmd{\clearforchapter} \\
\end{syntax}
The actual macro that sets the opening page for a chapter is 
\cmd{\clearforchapter}. The \cmd{\openright}, \cmd{\openleft} and
\cmd{\openany} define \cmd{\clearforchapter} to be respectively
(see \S\ref{sec:moving}) 
\cmd{\cleartorecto}, \cmd{\cleartoverso} and \cmd{\clearpage}. You can
obviously change \cmd{\clearforchapter} to do more than just start a
new page.

    Some books have the chapter headings on a verso page, with perhaps
an illustration\index{illustration} or epigraph\index{epigraph}, and then the text starts on the opposite
recto page. The effect can be achieved like:
\begin{lcode}
\openleft                % chapter title on verso page
\chapter{The title}      % chapter title
\begin{centering}        % include a centered illustration
\includegraphics{...}
\end{centering}
\clearpage               % go to recto page
Start of the text        % chapter body
\end{lcode}

\begin{syntax}
\cmd{\chapterstyle}\marg{style} \\
\end{syntax}
The macro \cmd{\chapterstyle} is rather like the \cmd{\pagestyle} command in 
that it sets the style of any subsequent chapter headings to be \meta{style}.

    The class provides some predefined chapter styles, including the 
\cstyle{default} style which is the familiar LaTeX \Lclass{book} class chapter
headings style. To use the chapterstyle \cstyle{fred} just issue the commmand
|\chapterstyle{fred}|. Different styles can be used in the same document.
The predefined styles include:
\begin{itemize}
\item[\cstyle{default}] The normal LaTeX style.
\item[\cstyle{section}] The heading is typeset like a section; that is, 
there is just the number and the title on one line. 
    See \Cref{chap:topsandtails} on \pref{chap:topsandtails} for an example.
\item[\cstyle{hangnum}] Like the \pstyle{section} style except that the chapter
number is put in the margin\index{margin}. 
    See \Cref{chap:captions} on \pref{chap:captions} for an example.
\item[\cstyle{companion}] This produces chapter headings like those the the
\textit{LaTeX Companion} series of books.
    See \Cref{chap:signposts} on \pref{chap:signposts} for an example.
\item[\cstyle{article}] The heading is typeset like a \cmd{\section}
heading in the \Lclass{article} class. This is similar to the 
\cstyle{section} style but different fonts and spacings are used.
\item[\cstyle{demo}] Try this one to see what it does. On the other
   hand you can look at \Cref{chap:verse} on \pref{chap:verse} to see
   what it looks like.
\end{itemize}


\begin{syntax}
\lnc{\beforechapskip} \lnc{\afterchapskip} \\
\end{syntax}
These two lengths define the spacing before and after the chapter heading.
Their default values are 50pt and 40pt, respectively.



\begin{syntax}
\lnc{\midchapskip} \\
\end{syntax}
The length \lnc{\midchapskip} is the vertical distance between the chapter 
number line and the chapter title. Its default value is 20pt and can be changed
by \cmd{\setlength} or \cmd{\addtolength}. It is probably better to have
the spacing in terms of the \lnc{\baselineskip}, which will change depending
on the font size, instead of a fixed amount.

\begin{syntax}
\cmd{\printchaptername} \cmd{\chapnamefont} \\
\cmd{\chapternamenum} \\
\cmd{\printchapternum} \cmd{\chapnumfont} \\
\end{syntax}
The macro \cmd{\printchaptername} typesets the chapter name 
(the value of \cmd{\chaptername}, default |Chapter|) using the font 
specified by \cmd{\chapnamefont}. The default is the \cmd{\bfseries} font in
the \cmd{\huge} size. Likewise the chapter number is typeset by 
\cmd{\printchapternum}
using the font specified by \cmd{\chapnumfont}, which has the same default as
\cmd{\chapnamefont}. The macro 
\cmd{\chapternamenum}, which is defined to be a space, is called between printing
the chapter name and the number. 

\begin{syntax}
\cmd{\printchaptertitle}\marg{title} \cmd{\chaptitlefont} \\
\end{syntax}
The title is typeset by \cmd{\printchaptertitle} using the font specified 
by \cmd{\chaptitlefont}. 
By default this is a \cmd{\bfseries} font in the \cmd{\Huge} size. 


\begin{syntax}
\cmd{\insertchapterspace} \\
\end{syntax}
By default a \cmd{\chapter} inserts a small amount of vertical space
into the List of Figures and List of Tables. It calls \cmd{\insertchapterspace}
to do this. The default definition is:
\begin{lcode}
\newcommand{\insertchapterspace}{%
  \addtocontents{lof}{\protect\addvspace{10pt}}%
  \addtocontents{lot}{\protect\addvspace{10pt}}%
}
\end{lcode}
If you would prefer no inserted spaces then 
|\renewcommand{\insertchapterspace}{}| will do the job. 
Different spacing can be inserted by
changing the value of the length arguments to \cmd{\addvspace}.

\index{heading!chapter|)}

\subsection{Defining a chapter style} \label{sec:chapterstyle}

\index{chapterstyle|(}

    The following illustrates the essential parts of the code for typesetting 
a numbered chapter heading.
\begin{lcode}
\chapterheadstart
\printchaptername \chapternamenum \printchapternum
\afterchapternum
\printchaptertitle{The chapter title}
\afterchaptertitle
\end{lcode}

\begin{syntax}
\cmd{\printchapternonum} \\
\end{syntax}

    The essential code for an unnumbered chapter is simpler than 
when there is a number to be typeset.
\begin{lcode}
\chapterheadstart
\pintchapternonum
\printchaptertitle{The chapter title}
\afterchaptertitle
\end{lcode}

The various macros in the above are initially defined as:
\begin{lcode}
\newcommmand{\chapterheadstart}{\vspace*{\beforechapskip}}
\newcommand{\printchaptername}{\chapnamefont \@chapapp}
\newcommand{\chapternamenum}{\space}
\newcommand{\printchapternum}{\chapnumfont \thechapter}
\newcommand{\afterchapternum}{\par\nobreak\vskip \midchapskip}
\newcommand{\printchapternonum}{}
\newcommand{\printchaptertitle}[1]{\chaptitlefont #1}
\newcommand{\afterchaptertitle}{\par\nobreak\vskip \afterchapskip}
\end{lcode}

    A new style is specified by changing the definitions of this last set of 
macros and/or the various font and skip specifications.

\begin{syntax}
\cmd{\makechapterstyle}\marg{style}\marg{text} \\
\end{syntax}
Chapter styles are defined via the \cmd{\makechapterstyle} command, where
\meta{style} is the style being defined and \meta{text} is the LaTeX code 
defining the style.

    As an example, here is the code for defining the \cstyle{section} chapter
style.
\begin{lcode}
\makechapterstyle{section}{%
  \renewcommand{\printchaptername}{}
  \renewcommand{\chapternamenum}{}
  \renewcommand{\chapnumfont}{\chaptitlefont}
  \renewcommand{\printchapternum}{\chapnumfont \thechapter\space}
  \renewcommand{\afterchapternum}{}
}
\end{lcode}
In this style, \cmd{\printchaptername} is vacuous, so the normal `Chapter'
is never typeset. The same font is used for the number and the title, and
the number is typeset with a space after it. The macro \cmd{\afterchapternum}
is vacuous, so the chapter title will be typeset immediately after the number.

    In the standard classes the title of an unnumbered chapter is typeset
at the same position on the page as the word `Chapter' for numbered chapters.
The macro \cmd{\printchapternonum} is called just before an unnumbered 
chapter title text is typeset. By default this does nothing but you can use
\cmd{\renewcommand} to change this. For example, if you wished the title
text for both numbered and unnumbered chapters to be at the same height on
the page then you could redefine \cmd{\printchapternonum} to insert
the amount of vertical space taken by any `Chapter N' line.

    Bastiaan Veelo 
posted the code for a new chapter style to \pictt{} on 2003/07/22 under the
title \textit{[memoir] [contrib] New chapter style}. His code, which
I have slightly modified and changed the name to \cstyle{veelo},
is below. I have also exercised editorial privilege on his comments.

\begin{quote}
 I thought I'd share a new chapter style to be used with the memoir class 
 The style is tailored for documents that are to be trimmed to a smaller 
 width. When the bound document is bent, black tabs will appear on the 
 fore side at the places where new chapters start as a navigational aid.
 We are scaling the chapter number, which most DVI viewers
 will not display accurately. \\[0.5\onelineskip]
Bastiaan.
\end{quote}

\begin{lcode}
\makeatletter
\newlength{\numberheight}
\newlength{\barlength}
\makechapterstyle{veelo}{%
   \setlength{\beforechapskip}{40pt}
   \setlength{\midchapskip}{25pt}
   \setlength{\afterchapskip}{40pt}
   \renewcommand{\chapnamefont}{\normalfont\LARGE\flushright}
   \renewcommand{\chapnumfont}{\normalfont\HUGE}
   \renewcommand{\chaptitlefont}{\normalfont\HUGE\bfseries\flushright}
   \renewcommand{\printchaptername}{%
     \chapnamefont\MakeUppercase{\@chapapp}}
   \renewcommand{\chapternamenum}{}
   \setlength{\numberheight}{18mm}
   \setlength{\barlength}{\paperwidth}
   \addtolength{\barlength}{-\textwidth}
   \addtolength{\barlength}{-\spinemargin}
   \renewcommand{\printchapternum}{%
     \makebox[0pt][l]{%
       \hspace{.8em}%
       \resizebox{!}{\numberheight}{\chapnumfont \thechapter}%
       \hspace{.8em}%
       \rule{\barlength}{\numberheight}
     }
   }
   \makeoddfoot{plain}{}{}{\thepage}
}
\makeatother
\end{lcode}

\makeatletter
\newlength{\numberheight}
\newlength{\barlength}
\makechapterstyle{veelo}{%
   \setlength{\beforechapskip}{40pt}
   \setlength{\midchapskip}{25pt}
   \setlength{\afterchapskip}{40pt}
   \renewcommand{\chapnamefont}{\normalfont\LARGE\flushright}
   \renewcommand{\chapnumfont}{\normalfont\HUGE}
   \renewcommand{\chaptitlefont}{\normalfont\HUGE\bfseries\flushright}
   \renewcommand{\printchaptername}{%
     \chapnamefont\MakeUppercase{\@chapapp}}
   \renewcommand{\chapternamenum}{}
   \setlength{\numberheight}{18mm}
   \setlength{\barlength}{\paperwidth}
   \addtolength{\barlength}{-\textwidth}
   \addtolength{\barlength}{-\spinemargin}
   \renewcommand{\printchapternum}{%
     \makebox[0pt][l]{%
       \hspace{.8em}%
       \resizebox{!}{\numberheight}{\chapnumfont \thechapter}%
       \hspace{.8em}%
       \rule{\barlength}{\numberheight}
     }
   }
   \makeoddfoot{plain}{}{}{\thepage}
}
\makeatother

    If you implement this you will also need to use the \Lpack{graphicx} 
package~\cite{GRAPHICX} because of the \cmd{\resizebox} macro. 
The \cstyle{veelo} style works best for chapters that start 
on recto pages.

    For details of how the other chapter styles are defined, 
look at the
documented class code. This should give you ideas if you want to define
your own style.

    Note that it is not necessary to define a new chapterstyle if you want
to change the chapter headings --- you can just change the individual macros
without putting them into a style.

\index{chapterstyle|)}

\section{Lower level headings}

\index{heading!sections|(}

    The lower level headings --- sections down to subparagraphs --- are also
configurable, but there is nothing corresponding to chapter styles.

    There are essentially three things that may be adjusted for a heading:
(a) the vertical distance between the baseline of the text above the heading to
the baseline of the heading title, (b) the indentation of the heading from the
left hand margin\index{margin}, and (c) the style (font) used for the heading title. 
Additionally, a heading may be runin to the text or as a display before 
the following text;
in the latter case the vertical distance between the heading and the 
following text may also be adjusted. Figure~\ref{fig:displaysechead} shows the
parameters controlling a displayed sectional heading and \fref{fig:runsechead}
shows the parameters for a runin heading.

    In the following I will use |S| to stand for one of |sec|, |subsec|,
|subsubsec|, |para| or |subpara|, which are in turn shorthand for |section|
through to |subparagraph|.

\begin{figure}
\centering
\setlayoutscale{1}
\drawparameterstrue
\drawheading{}
\caption{Displayed sectional headings} \label{fig:displaysechead}
\end{figure}

\begin{figure}
\centering
\setlayoutscale{1}
\drawparameterstrue
\runinheadtrue
\drawheading{}
\caption{Runin sectional headings} \label{fig:runsechead}
\end{figure}

\begin{syntax}
\cmd{\setbeforeSskip}\marg{skip} \\
\end{syntax}
The absolute value of the \meta{skip} length argument is the space to leave
above the heading. If the actual value is negative then the first line 
after the heading will not be indented. The default \meta{skip} depends on the
particular level of heading, but for a \cmd{\section} (i.e., when |S = sec|)
it is 
\begin{lcode}
-3.5ex plus -1ex minus -.2ex
\end{lcode}
where the plus and minus values are the
allowable stretch and shrink; note that all the values are negative so that 
there is no indentation of the following text. If you wanted indentation then
you could do
\begin{lcode}
\setbeforesecskip{3.5ex plus 1ex minus .2ex}
\end{lcode}


\begin{syntax}
\cmd{\setSindent}\marg{length} \\
\end{syntax}
The value of the \meta{length} length argument is the indentation of
the heading (number and title) from the lefthand margin\index{margin}. This is normally
0pt.

\begin{syntax}
\cmd{\setSheadstyle}\marg{text} \\
\end{syntax}
This macro specifies the style (font) for the sectional number and title. 
As before, the default value of the \meta{text} argument depends on the
level of the heading. For a \cmd{\subsection} (i.e., |S=subsec|) it is
|\large\bfseries\raggedright|, to typeset in the \cmd{\bfseries} font
in the \cmd{\large} size; the title will also be set ragged right (i.e.,
there will be no hyphenation in a multiline title).

    Note that the very last element in the \meta{text} argument may be a 
macro that takes one argument (the number and title of the heading). So,
if for some reason you wanted \cmd{\subsubsection} titles to be all uppercase,
centered, and in the normal font, you can do
\begin{lcode}
\setsubsubsecheadtstyle{\normalfont\centering\MakeUppercase}
\end{lcode}

    As another example, although I don't recommend this, you can draw a
horizontal line under section titles via:
\begin{lcode}
\newcommand{\ruledsec}[1]{%
  \Large\bfseries\raggedright #1 \rule{\textwidth}{0.4pt}}
\setsecheadstyle{\ruledsec}
\end{lcode}


\begin{syntax}
\cmd{\setafterSskip}\marg{skip} \\
\end{syntax}
If the value of the \meta{skip} length argument is positive it is the space 
to leave between the display heading and the following text. If it is negative,
then the heading will be runin and the value is the horizontal space
between the end of the heading and the following text.
The default \meta{skip} depends on the
particular level of heading, but for a \cmd{\section} (i.e., when |S = sec|)
it is |2.3ex plus .2ex|, 
and for a \cmd{\subparagraph} (i.e., |S = subpara|), which is a runin heading, it is
|-1em|.

\begin{syntax}
\cmd{\@hangfrom}\marg{stuff} \\
\cmd{\sethangfrom}\marg{code} \\
\end{syntax}
Internally all the titling macros use a macro called \cmd{\@hangfrom} which
by default makes multiline titles look like a hanging paragraph\index{paragraph!hanging}. The
default definition of \cmd{\@hangfrom} (in file \file{ltsect.dtx}) is
effectively:
\begin{lcode}
\newcommand{\@hangfrom}[1]{\setbox\@tempboxa\hbox{{#1}}%
  \hangindent \wd\@tempboxa\noindent\box\@tempboxa}
\end{lcode}
The argument is put into a box and its width is measured, then a hanging
paragraph\index{paragraph!hanging} is started with the argument as the first thing and second and
later lines indented by the argument's width.

The \cmd{\sethangfrom} macro redefines \cmd{\@hangfrom} to be \meta{code}.
For example, to have the titles set as block paragraphs\index{paragraph!block} instead of hanging
paragraphs\index{paragraph!hanging}, simply do:
\begin{lcode}
\sethangfrom{\noindent #1}
\end{lcode}
Note that you have to use |#1| at the position in the replacement
code for \cmd{\@hangfrom} where the argument to \cmd{\@hangfrom}
is to be located.

\begin{syntax}
\cmd{\@seccntformat}\marg{stuff} \\
\cmd{\setsecnumformat}\marg{code} \\
\end{syntax}
Internally all the titling macros use a macro called \cmd{\@seccntformat} 
which defines the formatting of sectional numbers in a title. Its
default definition (in file \file{ltsect.dtx}) is effectively:
\begin{lcode}
\newcommand{\@seccntformat}[1]{\csname the#1\endcsname\quad}
\end{lcode}
which formats the sectional numbers as |\thesec...| with a space afterwards.
The command \cmd{\setsecnumformat} redefines \cmd{\@seccntformat} 
to be \meta{code}.
For example, to put a colon and space after the number
\begin{lcode}
\setsecnumformat{\csname the#1\endcsname:\quad}
\end{lcode}
Note that you have to use |#1| where you want the argument 
(sectional number) of \cmd{\@seccntformat} to go.

\begin{syntax}
\cmd{\hangsecnum} \\
\cmd{\defaultsecnum} \\
\end{syntax}
The macro \cmd{\hangsecnum} is a declaration that makes sectional numbers hang
in the margin\index{margin}. The macro \cmd{\defaultsecnum} is a declaration that reverses the
effect of \cmd{\hangsecnum}, that is, sectional numbers will be typeset in 
their familiar places.


\begin{syntax}
\cmd{\Shook} \\
\cmd{\setShook}\marg{text} \\
\end{syntax}
The macro \cmd{\Shook} is called immediately before the typesetting of the
title; its default definition does nothing. The macro \cmd{\setShook} 
redefines \cmd{\Shook} to be \meta{text}. You can use this hook, for example,
to insert a \cmd{\sethangfrom} or \cmd{\setsecnumformat} command into the
definition of a particular section division command.

\index{heading!sections|)}

\section{Heading styles for this chapter}

    As a reference, \tref{tab:secfonts} lists the default fonts used
for the sectional headings. These fonts are all bold but in different
sizes depending on the division level.

\begin{table}
\centering
\caption{Default fonts for sectional headings}\label{tab:secfonts}
\begin{tabular}{ll} \hline
\cmd{\partnamefont}       & |\huge\bfseries| \\
\cmd{\partnumfont}        & |\huge\bfseries| \\
\cmd{\parttitlefont}      & |\Huge\bfseries| \\
\cmd{\chapnamefont}       & |\normalfont\huge\bfseries| \\
\cmd{\chapnumfont}        & |\normalfont\huge\bfseries| \\
\cmd{\chaptitlefont}      & |\normalfont\Huge\bfseries| \\
\cmd{\secheadstyle}       & |\Large\bfseries\raggedright| \\
\cmd{\subsecheadstyle}    & |\large\bfseries\raggedright| \\
\cmd{\subsubsecheadstyle} & |\normalsize\bfseries\raggedright| \\
\cmd{\paraheadstyle}      & |\normalsize\bfseries| \\
\cmd{\subparaheadstyle}   & |\normalsize\bfseries| \\
\hline
\end{tabular}
\end{table}

\index{heading!chapter|(}
\index{chapterstyle|(}

    The commands described below have been put just before the start
of this chapter.

    For this chapter I have used sans fonts instead of bold fonts. The
commands to do this are shown below for chapters down to subsubsections.
\begin{lcode}
\renewcommand{\chapnumfont}{\normalfont\huge\sffamily}
\renewcommand{\chaptitlefont}{\chapnumfont}
\setsecheadstyle{\Large\sffamily\raggedright}
\setsubsecheadstyle{\large\sffamily\raggedright}
\setsubsubsecheadstyle{\normalsize\sffamily\raggedright}
\end{lcode}

    The chapter heading is typeset like a section heading. I redefine
\cmd{\printchaptername} and \cmd{\chapternamenum} to do nothing, which
eliminates the printing of the chapter name and also anything
that may be between the name and the number. Changing \cmd{\afterchapternum}
to be just \cmd{\space} eliminates the vertical space that is normally
between the number and the title, and instead puts a normal word space 
between the number and the title.
\begin{lcode}
\renewcommand{\printchaptername}{}
\renewcommand{\chapternamenum}{}
\renewcommand{\afterchapternum}{\space}
\end{lcode}

    Before the next chapter starts, I want to revert back to the default
divisional styles, so the following code is used at the end of
this chapter.
\begin{lcode}
\chapterstyle{default}
\setsecheadstyle{\Large\bfseries\raggedright}
\setsubsecheadstyle{\large\bfseries\raggedright}
\setsubsubsecheadstyle{\normalsize\bfseries\raggedright}
\end{lcode}
Calling |\chapterstyle{default}| deals with the chapter heading. The other
commands change the lower level fonts back to their default values.


\index{chapterstyle|)}
\index{heading!chapter|)}

\section{Footnotes and headers}

\index{footnote!in heading|(}

    With the sectioning commands the text of the required argument
\meta{title} is used as the text for the section title in the body
of the document.

    When the optional argument \meta{toc-title} is used in a sectioning
command it is moving and any fragile commands must be \cmd{\protect}ed,
while the \meta{title} argument is fixed. The \meta{toc-title} also
serves double duty:
\begin{enumerate}
\item It is used as the text of the title in the \toc;
\item It is used as the text in page headers\index{header}. 
\end{enumerate}

    If the optional argument is not present, then the \meta{title} is
moving and serves the triple duty of providing the text for the body and \toc{}
titles and for page headers\index{header}.

    Some folk feel an urge to add a footnote\index{footnote} to a sectional title, which
should be resisted. If their flesh is weak, then the optional argument must
be used and the \cmd{\footnote} attached to the required argument only.
If the optional argument is not used then the footnote mark\index{footnote!mark} and text is
likely to be scattered all over the place, on the section page, in the \toc,
on any page that includes \meta{title} in its header\index{header}. This is 
unacceptable to any reader. So, a footnoted\index{footnote} title should look like
this:
\begin{lcode}
\chapter[Title text]{Title text\footnote{Do you really have to do this?}}
\end{lcode}

\index{footnote!in heading|)}


\section{Pagination and folios}

    Every page in a LaTeX document is included in the 
pagination\index{pagination}. That is,
there is a number associated with every page and this is the value of
the \Icn{page} counter. This value can be changed at any time via either
\cmd{\setcounter} or \cmd{\addtocounter}.

\begin{syntax}
\cmd{\pagenumbering}\marg{num-style} \\
\cmd{\pagenumbering*}\marg{num-style} \\
\end{syntax}
The macros \cmd{\pagenumbering} and \cmd{\pagenumbering*} cause 
the folios\index{folio} to be printed using
\meta{num-style} for the page number, where \meta{num-style} can be one of: 
\Itt{Alph}, \Itt{alph}, \Itt{arabic}, \Itt{Roman} or 
\Itt{roman} for uppercase and lowercase letters, arabic numerals, and
uppercase and lowercase Roman numerals, respectively. As there
are only 26~letters, \Itt{Alph} or \Itt{alph} can only be
used for a limited number of pages. Effectively, the macros redefine
\cmd{\thepage} to be |\num-style{page}|. 
Additionally, the \cmd{\pagenumbering}
command resets the \Icn{page} counter to one; the starred version does not 
change the counter.

    It is usual to reset the page number back to one each time the style
is changed, but sometimes it may be desireable to have a continuous sequence
of numbers irrespective of their displayed form, which is where
\cmd{\pagenumbering*} comes in handy.

\begin{syntax}
\cmd{\savepagenumber} \\
\cmd{\restorepagenumber} \\
\end{syntax}
The macro \cmd{\savepagenumber} saves the current page number, and the
macro \cmd{\restorepagenumber} sets the page number to the saved value.
This pair of commands may be used to apparently interrupt the pagination.
For example, perhaps some full page illustrations\index{illustration} will be electronically,
as opposed to physically,
tipped in\index{tip in} to the document and pagination is not required for these.
This could be done along the lines of:
\begin{lcode}
\clearpage          % get onto next page
\savepagenumber     % save the page number
\pagestyle{empty}   % no headers or footers
%% insert the illustrations
\clearpage
\pagestyle{...}
\restorepagenumber
...
\end{lcode}
If you try this sort of thing, you may have to adjust the restored page 
number by one.
\begin{lcode}
\restorepagenumber
% perhaps \addtocounter{page}{1} or \addtocounter{page}{-1}
\end{lcode}
Whether or not this will be necessary depends on the timing of the
|\...pagenumber| commands and TeX's decisions on page breaking.










%%%%%%%%%%%%%%%%%%%%%%%%%%%%%
\chapterstyle{default}
\setsecheadstyle{\Large\bfseries\raggedright}
\setsubsecheadstyle{\large\bfseries\raggedright}
\setsubsubsecheadstyle{\normalsize\bfseries\raggedright}
%%%%%%%%%%%%%%%%%%%%%%%%%%%%%%%%



\chapter{Paragraphs and lists}

%%%%%%%%%%%%%%%%%%%%%
\tightlists
%%%%%%%%%%%%%%%%%%%%%

    This chapter is typeset with the default sectional 
headings\index{heading!default}.

\section{Introduction}

    Within a sectional division the text is typically broken up into
paragraphs. Sometimes there may be text that is set off from the normal
paragraphing, like quotations\index{quotation} or lists.

\section{Paragraphs}

\index{paragraph|(}

    There are basically two parameters that control the appearance of normal
paragraphs.

\begin{syntax}
\lnc{\parindent} \lnc{\parskip} \\
\end{syntax}
 
    The length \lnc{\parindent} is the indentation of the first line of a
paragraph\index{paragraph!indentation} and the length \lnc{\parskip} is the vertical spacing between
paragraphs, as illustrated in \fref{fig:para}. The value of \lnc{\parskip}
is usually 0pt, and \lnc{\parindent} is usually defined in terms of ems so
that the actual indentation depends on the font being used. If \lnc{\parindent}
is set to a negative length, then the first line of the paragraphs will be 
`outdented' into the lefthand margin\index{margin}.

\begin{figure}
\centering
\drawparameterstrue
\drawparagraph
\caption{Paragraphing parameters}\label{fig:para}
\end{figure}

    A block paragraph\index{paragraph!block} is obtained by setting \lnc{\parindent} to 0em; 
\lnc{\parskip} should be set to some positive value so that there is some
space between paragraphs to enable them to be identified. Most typographers
heartily dislike block paragraphs, not only on aesthetical grounds but also
on practical grounds. Consider what happens if the last line of a block 
paragraph is full and also is the last line on the page. The following
block paragraph will start at the top of the next page but there will be
no identifiable space to indicate an inter-paragraph break.

    It is important to know that LaTeX typesets paragraph by paragraph. 
For example, the \lnc{\baselineskip} that is used for a paragraph is the value
that is in effect at the end of the paragraph, and the font size used for a
paragraph is according to the size declaration (e.g., \cmd{\large} or 
\cmd{\normalsize} or \cmd{\small}) in effect at the end of the paragraph.

\subsection{Hanging paragraphs}

\index{paragraph!hanging|(}

    A hanging paragraph is one where the length of the first few lines differs
from the length of the remaining lines. (A normal indented paragraph 
may be considered 
to be a special case of a hanging paragraph where `few = one'). 

\begin{syntax}
\cmd{\hangpara}\marg{indent}\marg{num} \\
\end{syntax}

\hangpara{3em}{-3}%
 Using \cmd{\hangpara} at the start of a paragraph will cause the paragraph
to be hung. If the length \meta{indent} is positive the lefthand end of the 
lines will be indented but if it is negative the righthand ends will be 
indented by the specified amount.
    If the number \meta{num}, say N, is is negative the first N lines of
the paragraph will be indented while if N is positive the N+1 th lines onwards
will be indented. This paragraph was set with |\hangpara{3em}{-3}|. There 
should be no space between the \cmd{\hangpara} command and the start of the
paragraph.

\begin{syntax}
\senv{hangparas}\marg{indent}\marg{num} text \eenv{hangparas} \\
\end{syntax}
    The \Ie{hangparas} environment is like the \cmd{\hangpara} command except
that every paragraph in the environment will be hung.

    The code implementing the hanging paragraphs is the same as for
the \Lpack{hanging} package~\cite{HANGING}. Examples of some uses
can be found in~\cite{TTC199}.


    As noted eleswhere the sectioning commands use the internal 
macro \cmd{\@hangfrom} as part of the formatting of the titles.

\begin{syntax}
\cmd{\hangfrom}\marg{text} \\
\end{syntax}

\hangfrom{Simple hung paragraphs }(like this one) can be specified
using the \cmd{\hangfrom} macro. The macro puts \meta{text} in a box
and then makes a hanging paragraph of the following material. This
paragraph commenced with \\
\verb?\hangfrom{Simple hung paragraphs }(like ...? \\
and you are now reading the result.


\index{paragraph!hanging|)}

\index{paragraph|)}

\section{Flush and ragged}

    Flushleft\index{flushleft} text has the lefthand end of the lines 
aligned vertically at the lefthand margin\index{margin} and
flushright\index{flushright} text has the righthand end of the lines 
aligned vertically at the righthand margin\index{margin}. The
opposites of these are raggedleft\index{raggedleft} text where the 
lefthand ends are not aligned
and raggedright\index{raggedright} where the righthand end of lines are 
not aligned. LaTeX normally typesets flushleft and flushright.

\begin{syntax}
\senv{flushleft} text \eenv{flushleft} \\
\senv{flushright} text \eenv{flushright} \\
\senv{center} text \eenv{center} \\
\end{syntax}
    Text in a \Ie{flushleft} environment is typeset flushleft and raggedright,
while in a \Ie{flushright} environment is typeset raggedleft and flushright.
In a \Ie{center} environment the text is set raggedleft and raggedright, and each
line is centered. A small vertical space is put before and after each of these
environments.

\begin{syntax}
\cmd{\raggedleft} \cmd{\raggedright} \cmd{\centering} \\
\end{syntax} 
    The \cmd{\raggedleft} declaration can be used to have text typeset
raggedleft and flushright, and similary the declaration \cmd{\raggedright}
causes typesetting to be flushleft and raggedright. The declaration 
\cmd{\centering} typesets raggedleft and raggedright with each line centered.
Unlike the environments, no additional space is added. 

\begin{syntax}
\cmd{\raggedyright}\oarg{space} \\
\lnc{\ragrparindent} \\
\end{syntax}
When using \cmd{\raggedright} in narrow columns the right hand edge tends to
be too ragged, and paragraphs are not indented. 
Text set \cmd{\raggedyright} usually fills more of the available
width and paragraphs are indented by \lnc{\ragrparindent}, which is initially
set to \lnc{\parindent}. The optional \meta{space} argument, whose default
is 2em, can be used to adjust the amount of raggedness. As examples:
\begin{lcode}
\raggedyright[0pt]   % typeset flushright 
\raggedyright[1fil]  % same as \raggedright
\raggedyright[0.5em] % less ragged than \raggedright
\end{lcode}

    Remember that LaTeX typesets on a per-paragraph basis, so that putting
the sequence of \cmd{\centering}, \cmd{\raggedleft} declarations in the same
paragraph\index{paragraph} will cause the entire paragraph to be typeset raggedleft and 
flushright --- the \cmd{\centering} declaration is not the one in effect 
at the end of the paragraph.

\section{Quotations}

    LaTeX provides two environments that are typically used for typesetting
quotations\index{quotation}.

\begin{syntax}
\senv{quote} text \eenv{quote} \\
\senv{quotation} text \eenv{quotation} \\
\end{syntax}
     In both of these environments the text is set flushleft and flushright
in a measure that is smaller than the normal textwidth. The only difference
between the two environments is that paragraphs\index{paragraph!indentation} are not indented in the \Ie{quote}
environment but normal paragraphing is used in the \Ie{quotation} environment.

\section{Changing the textwidth}

    The \Ie{quote} and \Ie{quotation} environments both locally change the textwidth,
or more precisely, they temporarily increase the left and right margins\index{margin} by 
equal amounts.
Generally speaking it is not a good idea to change the textwidth but sometimes
it may be called for.

    The commands and environment described below are similar to those
in the \Lpack{chngpage} package~\cite{CHNGPAGE}, but do differ in some 
respects.

\begin{syntax}
\senv{adjustwith}\marg{left}\marg{right} text \eenv{adjustwith} \\
\senv{adjustwith*}\marg{left}\marg{right} text \eenv{adjustwith*} \\
\end{syntax}
The \Ie{adjustwidth} environment temporarily adds the length \meta{left}
to the lefthand margin\index{margin} and \meta{right} to the righthand margin\index{margin}. That is,
a positive length value increases the margin\index{margin} and hence reduces the textwidth,
and a negative value reduces the margin\index{margin} and increases the textwidth. The
\Ie{quotation} environment is roughly equivalent to
\begin{lcode}
\begin{adjustwidth}{2.5em}{2.5em}
\end{lcode}

    The starred version of the environment, \Ie{adjustwidth*},
 is only useful if the left and right
margin\index{margin} adjustments are different. The starred version checks the page number
and if it is odd then adjusts the left (spine) and right (outer) margins\index{margin} 
by \meta{left} and \meta{right} respectively; if the page number is even 
(a verso page) it adjusts the left (outer) and right (spine) margins\index{margin} by
\meta{right} and \meta{left} respectively.

\begin{syntax}
\cmd{\strictpagechecktrue} \cmd{\strictpagecheckfalse} \\
\end{syntax}
Odd/even page checking may be either strict (\cmd{\strictpagechecktrue})
or lazy (\cmd{\strictpagecheckfalse}). Lazy checking works most of the time
but if it fails at any point then the strict checking should be used.

    As an example, if a figure\index{figure} is wider than the textwidth it will stick out
into the righthand margin\index{margin}. It may be desireable to have any wide figure\index{figure}
stick out into the \foredge{} margin\index{margin} where there is usually more room than at
the spine margin\index{margin}. This can be accomplished by
\begin{lcode}
\begin{figure}
\centering
\strictpagechecktrue
\begin{adjustwidth*}{0em}{-3em}
% the illustration
\caption{...}
\end{adjustwidth*}
\end{figure}
\end{lcode}

    A real example in this manual is \tref{tab:fpp} on \pref{tab:fpp},
which is wider than the typeblock\index{typeblock}. In that case I just centered it by
using \Ie{adjustwidth} to decrease each margin\index{margin} equally. In brief, like
\begin{lcode}
\begin{table}
\begin{adjustwidth}{-1cm}{-1cm}
\centering
...
\end{adjustwidth}
\end{table}
\end{lcode}

    Note that the \Ie{adjustwidth} environment applies to complete paragraphs;
you can't change the width of part of a paragraph\index{paragraph} 
except for hanging paragraphs\index{paragraph!hanging} or more esoterically via \cmd{\parshape}. 
Further, if the adjusted paragraph crosses a
page boundary the margin\index{margin} changes are constant; a paragraph that is, say, 
wider at the right on the first page will also be wider at the right as it
continues onto the following page.

    The \Ie{center} environment horizontally centers its contents
with respect to the typeblock\index{typeblock}. 
Sometimes you may wish to horizontally center some text with respect
to the physical page, for example when typesetting a 
colophon\index{colophon} which may look odd centered with respect
to the (unseen) typeblock\index{typeblock}.

    The calculation of the necessary changes to the spine and \foredge{}
margins\index{margin} are simple. Using the same symbols as earlier in 
\S\ref{sec:typeblock2} ($P_{w}$ and
$B_{w}$ are the width of the trimmed page and the typeblock\index{typeblock}, respectively;
$S$ and $E$ are the spine and \foredge{} margins\index{margin}, respectively) then
the amount $M$ to be added to the spine margin\index{margin} and subtracted from the
\foredge{} margin\index{margin} is calculated as:
\begin{displaymath}
 M = 1/2(P_{w} - B_{w}) - S
\end{displaymath}

For example,
assume that the \lnc{\textwidth} is 5 inches and the \lnc{\spinemargin}
is 1 inch. On US letterpaper\index{paper!size!letterpaper} (\lnc{\paperwidth} is 8.5 inches) the
\foredge{} margin\index{margin} is then 2.5 inches, and 0.75 inches\footnote{On A4\index{paper!size!A4}
paper the result would be different.}  must
be added to the spine margin\index{margin} and subtracted from the \foredge{} to
center the typeblock\index{typeblock}. 
The \Ie{adjustwidth} environment can be used to make the (temporary) change.
\begin{lcode}
\begin{adjustwidth*}{0.75in}{-0.75in} ...
\end{lcode}

\begin{syntax}
\cmd{\calccentering}\marg{length} \\
\end{syntax}
 If you don't want to do the above calculations by hand, 
\cmd{\calccentering} will do it for you. 
The \meta{length}
argument must be the name of a pre-existing length command, 
including the backslash. After calling 
|\calccentering|, \meta{length} is the amount to be added to the
spine margin\index{margin} and subtracted from the foredge margin\index{margin} to center the typeblock\index{typeblock}.
An example usage is
\begin{lcode}
\calccentering{\mylength}
\begin{adjustwidth*}{\mylength}{-\mylength}
text horizontally centered on the physical page
\end{adjustwidth*}
\end{lcode}

   You do not necessarily have to define a new length for the purposes
of \cmd{\calccentering}. Any existing length will do, such as
\lnc{\unitlength}, provided it will be otherwise unused between performing
the calculation and changing the margins\index{margin} (and that you can, if necessary
reset it to its original value --- the default value for \lnc{\unitlength}
is 1pt). 

\section{Lists}

\index{list|(}

    Standard LaTeX provides 4 kinds of lists. There is a general \Ie{list}
environment which you can use to define your own particular kind of list,
and the \Ie{description}, \Ie{itemize} and \Ie{enumerate} lists (which are internally
defined in terms of the general \Ie{list} environment).

    This class provides the normal \Ie{description} list but the \Ie{itemize} and
\Ie{enumerate} lists are extended versions of the normal ones.

\begin{syntax}
\senv{description} \cmd{\item}\oarg{label} ... \eenv{description} \\
\cmd{\descriptionlabel}\meta{style} \\
\end{syntax}
In a \Ie{description} list the style of the \meta{label} is given by the
\meta{style} argument of the \cmd{\descriptionlabel} command. The default
definition is
\begin{lcode}
\newcommand*{\descriptionlabel}[1]{\hspace\labelsep
                                   \normalfont\bfseries #1}
\end{lcode}
which gives a bold label. To have, for example, a sans label instead, do
\begin{lcode}
\renewcommand*{\descriptionlabel}[1]{\hspace\labelsep
                                     \normalfont\sffamily #1}
\end{lcode}

    The \Ie{itemize} and \Ie{enumerate} environments below are based on
the \Lpack{enumerate} package~\cite{ENUMERATE}.

\begin{syntax}
\senv{itemize}\oarg{mark} \cmd{\item} ... \eenv{itemize} \\
\end{syntax}
The normal markers for \cmd{\item}s in an \Ie{itemize} list are: 
\begin{enumerate}
\item bullet (\textbullet |\textbullet|), 
\item bold en-dash ({\normalfont\bfseries \textendash} |\bfseries\textendash|),
\item centered asterisk (\textasteriskcentered |\textasteriskcentered|), and
\item centered dot (\textperiodcentered |\textperiodcentered|).
\end{enumerate}
The optional \meta{mark} argument can be used to specify the marker for the
list items in a particular list. If for some reason you wanted to use a 
pilcrow symbol as the item marker for a particular list you could do
\begin{lcode}
\begin{itemize}[\P]
\item ...
...
\end{lcode}



\begin{syntax}
\senv{enumerate}\oarg{style} \cmd{\item} ... \eenv{enumerate} \\
\end{syntax}
The normal markers for, say, the third item in an \Ie{enumerate} list are: 
3., c., iii., and C. The optional \meta{style} argument can be used to
specify the style used to typeset the item counter. An occurrence of
one of the special characters |A|, |a|, |I|, |i|
or |1| in \meta{style} specifies that the counter will be typeset using
uppercase letters (|A|), lowercase letters (|a|), 
uppercase Roman numerals (|I|),
lowercase Roman numerals (|i|), or arabic numerals (|1|). These characters
may be surrounded by any LaTeX commands or characters, but if so the special
characters must be put inside braces (e.g., |{a}|) if they are to be 
considered as ordinary characters instead of as special styling characters.
 For example, to have the
counter typeset as a lowercase Roman numeral followed by a single parenthesis
\begin{lcode}
\begin{enumerate}[i)] 
...
\end{lcode}

\index{list!tight|(}

\begin{syntax}
\cmd{\tightlists} \cmd{\defaultlists} \\
\end{syntax}
The normal LaTeX \Ie{description}, \Ie{itemize} and \Ie{enumerate} lists have an open
look about them when they are typeset as there is significant vertical space
between the items in the lists. After the declaration \cmd{\tightlists} is
issued, the extra vertical spacing between the list items is deleted. The open
list appearance is used after the \cmd{\defaultlists} declaration is issued.
These declarations, if used, must come \emph{before} the relevant list 
environment(s).

    The class initially sets \cmd{\defaultlists}. If you had noticed that
the lists in this chapter have a different appearance than those in earlier
chapters (or even if you hadn't noticed) then that is because 
the declaration \cmd{\tightlists} was put at the start of this chapter.

\begin{syntax}
\cmd{\firmlist} \cmd{\tightlist} \\
\end{syntax}
The command \cmd{\firmlist} or \cmd{\tightlist} can be used immediately
after the start of a list environment to reduce the vertical space within
that list. The \cmd{\tightlist} removes all the spaces while the
\cmd{\firmlist} produces a list that still has some space but not as much
as in an ordinary list.

\index{list!tight|)}


\begin{figure}
\centering
\drawparameterstrue
\drawlist
\caption{The layout parameters for general lists}\label{fig:listlay}
\end{figure}

\index{list!new|(}

\begin{syntax}
\senv{list}\marg{default-label}\marg{code} items \eenv{list} \\
\end{syntax}
LaTeX's list environments are defined in terms of a general \Ie{list}
environment; some other environments, such as the \Ie{quote}, \Ie{quotation}
and \Ie{adjustwidth} are also defined in terms of a \Ie{list}.
Figure~\ref{fig:listlay} shows the parameters controlling the layout
of the \Ie{list} environment.

    The \Ie{list} environment takes two arguments. The \meta{default-label}
argument is the code that should be used when the \cmd{\item} macro is
used without its optional \meta{label} argument. For lists like \Ie{enumerate}
this is specified but often it is left empty, such as for the \Ie{adjustwidth}
environment.

    The \meta{code} argument is typically used for setting the particular
values of the list layout parameters. When defining your own types of lists
it is advisable to set each of the parameters unless you know that the default
values are suitable for your purposes. These parameters can all be modified
with either the \cmd{\setlength} or \cmd{\addtolength} commands.

    As an example, here is the specification for a description-like list
that uses an italic rather than bold font for the items, and is somewhat
tighter than the normal \Ie{description} list.

\begin{lcode}
%%%%% An italic and tighter description environment
\newcommand{\itlabel}[1]{\hspace\labelsep\normalfont\itshape #1}
\newenvironment{itdesc}{%
  \list{}{%
    \setlength{\labelsep}{0.5em}
    \setlength{\itemindent}{0pt}
    \setlength{\leftmargin}{\parindent} 
    \setlength{\labelwidth}{\leftmargin}
    \addtolength{\labelwidth}{-\labelsep}
    \setlength{\listparindent}{\parindent}
    \setlength{\parsep}{\parskip}
    \setlength{\itemsep}{0.5\onelineskip}
    \let\makelabel\itlabel}}{\endlist}
\end{lcode}

    This gets used like any other list:
\begin{lcode}
\begin{itdesc}
\item[label] ....
\end{itdesc}
\end{lcode}

Here is another kind of list called \Ie{aglossary} that could be used for 
a glossary or other similar kind of listing.
\begin{lcode}
% Glossary list
\newenvironment{aglossary}%
               {\begin{list}{}% empty label
                            {\setlength{\topsep}{\baselineskip}
                             \setlength{\partopsep}{0pt}
                             \setlength{\itemsep}{0.5\baselineskip}
                             \setlength{\parsep}{0pt}
                             \setlength{\leftmargin}{2em}
                             \setlength{\rightmargin}{0em}
                             \setlength{\listparindent}{1em}
                             \setlength{\itemindent}{0em}
                             \setlength{\labelwidth}{0em}
                             \setlength{\labelsep}{2em}}}%
               {\end{list}}
\newcommand{\gloss}[1]{\item[#1]\mbox{}\\\nopagebreak}
\end{lcode}
In this case it gets used like this
\begin{lcode}
\begin{aglossary}
\gloss{TERM 1} definition
\gloss{TERM 2} ...
\end{aglossary}
\end{lcode}

\index{list!new|)}

\begin{syntax}
\cmd{\zerotrivseps} \cmd{\restoretrivseps} \cmd{\savetrivseps} \\
\end{syntax}
Several environments, such as \Ie{center}, are defined in terms of a
\Ie{trivlist} (a very simple list form). There is vertical space
before and after such an environment. If you don't want this, then
the declaration \cmd{\zerotrivseps} eliminates such spaces. You can think
of it being defined like:
\begin{lcode}
\newcommand*{\zerotrivseps}{%
  \setlength{\topsep}{0pt}%
  \setlength{\partopsep}{0pt}}
\end{lcode}
To restore the spacing call the \cmd{\restoretrivseps} declaration. 
The command \cmd{\savetrivseps} stores the \lnc{\topsep} and
\lnc{\partopsep} values, and \cmd{\restoretrivseps} sets them to
whatever values were stored. The class itself calls \cmd{\savetrivseps}
to keep the default values. 



\index{list|)}



%%%%%%%%%%%%%%%%%%%%%%%%%%
\chapterstyle{section}
%%%%%%%%%%%%%%%%%%%%%%%%%%
\chapter{Tops and tails} \label{chap:topsandtails}

This chapter uses the \cstyle{section} chapterstyle.

 \section{Introduction}

    The following discussions are focussed on the elements at the start
and end of a document; the Table of Contents at the start and the Index\index{index}
at the end.
    The functionality described is equivalent to the combination
of the \Lpack{tocloft} and \Lpack{tocbibind} 
packages~\cite{TOCLOFT,TOCBIBIND}

 In the standard classes the typographic design of the Table of Contents
 (\toc), the List of Figures (\lof) and List of Tables (\lot) is fixed or,
 more precisely, it is buried within the class definitions.

 \section{LaTeX's \prtoc{} methods}

    This is a general description of how LaTeX does the processing
 for a Table of Contents (\toc). As the processing for List of Figures 
(\lof) and
 List of Tables (\lot) is similar I will just
 discuss the \toc.

    First of all, remember that each sectional division has an
associated level as listed in \tref{tab:seclevels} on 
\pref{tab:seclevels}. LaTeX will not typeset an entry in the \toc{}
unless the value of the \Icn{tocdepth} counter is equal to or greater
than the level of the entry.

\begin{syntax}
\cmd{\maxtocdepth}\marg{secname} \\
\cmd{\settocdepth}\marg{secname} \\
\end{syntax}
The \Lclass{memoir} class \cmd{\maxtocdepth} command sets the maximum 
allowable value
for the \Icn{tocdepth} counter. If used, the command must appear
before the \cmd{\tableofcontents} command. By default, the class
sets |\maxtocdepth{section}|.

    The \Lclass{memoir} class command \cmd{\settocdepth} is somewhat
analagous to the \cmd{\setsecnumdepth} command 
described in \S\ref{sec:secnumbers}.
It sets the value of the \Icn{tocdepth} counter and puts it into
the \toc{} to (temporarily) modify what will appear. 
The command can only be used after the preamble\index{preamble}
but may be used before calling the \cmd{\tableofcontents}.
The \cmd{\settocdepth} and \cmd{\maxtocdepth} macros are from the 
\Lpack{tocvsec2} package~\cite{TOCVSEC2}.

\begin{syntax}
\cmd{\addcontentsline}\marg{file}\marg{kind}\marg{text} \\
\end{syntax}
    LaTeX generates a \file{.toc} file if the document contains a
 \cmd{\tableofcontents} command. The sectioning 
 commands\footnote{For figures and tables it is the 
%% \texttt{\bs caption} command
 \cmd{\caption} command
 that populates the \file{.lof} and \file{.lot} files.}
 put entries into the \file{.toc} file by calling the LaTeX
 \cmd{\addcontentsline} 
 command, where \meta{file} is the file extension (e.g., |toc|),
 \meta{kind} is the kind of entry (e.g., |section| or |subsection|),
 and \meta{text} is the (numbered) title text. In the cases where
 there is a number, the \meta{title} argument is given in the
 form |{\numberline{number} title-text}|.

\begin{syntax}
\cmd{\phantomsection} \\
\end{syntax}
\Note{} The \Lpack{hyperref} package~\cite{HYPERREF} appears to dislike 
authors using 
 \cmd{\addcontentsline}. To get it to work properly with \Lpack{hyperref}
 you normally have to put \cmd{\phantomsection} (a macro defined within
this class and the \Lpack{hyperref} package) immediately 
 before \cmd{\addcontentsline}.

\begin{syntax}
\cmd{\contentsline}\marg{kind}\marg{text}\marg{page} \\
\end{syntax}
     The \cmd{\addcontentsline} command writes an entry to the given file
 in the form: \\
 \cmd{\contentsline}\marg{kind}\marg{text}\marg{page} \\
 where \meta{page} is the page number.
     For each \meta{kind}, LaTeX provides a command: \\
 \cmd{\l@kind}\marg{title}\marg{page} \\
which performs the actual
 typesetting of the \cmd{\contentsline} entry. 


\begin{figure}
\setlayoutscale{0.8}
\drawtoc
\caption{Layout of a \prtoc{} (\prlof, \prlot) entry} \label{fig:ltoc}
\end{figure}

 
\begin{syntax}
\cmd{\@pnumwidth}\marg{length} \\
\cmd{\@tocrmarg}\marg{length} \\
\cmd{\@dotsep}\marg{number} \\
\end{syntax}
 The general layout of a
 typeset entry is illustrated in \fref{fig:ltoc}. There are three
 internal LaTeX commands that are used in the typesetting. The page
 number is typeset flushright in a box of width \cmd{\@pnumwidth}, and the box
 is at the righthand margin\index{margin}. If the page number is too long to fit into
 the box it will stick out into the righthand margin\index{margin}. The title text
 is indented from the righthand margin\index{margin} by an amount given by \cmd{\@tocrmarg}.
 Note that \cmd{\@tocrmarg} should be greater than \cmd{\@pnumwidth}. Some
 entries are typeset with a dotted leader between the end of the title
 title text and the righthand margin\index{margin} indentation. The distance, in
 \emph{math units}\footnote{There are 18mu to 1em.} between the dots
 in the leader is given by the value of \cmd{\@dotsep}. In the standard
 classes the same values are used for the \toc, \lof{} and the \lot.

    The standard values for these internal commands are:
 \begin{itemize}
 \item |\@pnumwidth| = 1.55em
 \item |\@tocrmarg| = 2.55em 
 \item |\@dotsep| = 4.5
 \end{itemize}
 The values can be changed by using \cmd{\renewcommand}, in spite of the
 fact that the first two appear to be lengths.

    Dotted leaders are not available for Part\index{part} and Chapter\index{chapter} \toc{} entries

\begin{syntax}
\cmd{\numberline}\marg{number} \\
\end{syntax}
    Each \cmd{\l@kind} macro is responsible for setting the general 
 \textit{indent} from the lefthand margin\index{margin}, and the \textit{numwidth}.
 The \cmd{\numberline} macro is responsible for typesetting
 the number flushleft in a box of width 
 \textit{numwidth}. If the number is too long for the box then it will
 protrude into the title text. The title text is indented by
 (\textit{indent + numwidth}) from the lefthand margin\index{margin}. That is, the title
 text is typeset in a block of width \\
 (|\linewidth| - \textit{indent} - \textit{numwidth} - |\@tocrmarg|). 

 \begin{table}
 \centering
 \caption[Indents and Numwidths]{Indents and Numwidths (in ems)} \label{tab:indents}
 \begin{tabular}{lcrrrr} \hline
 Entry & Level & \multicolumn{2}{c}{Standard} & \multicolumn{2}{c}{\Lclass{memoir} class} \\
       &       & indent & numwidth & indent & numwidth \\ \hline
 part          & -1 & 0    & --- & 0    & --- \\
 chapter       & 0  & 0    & 1.5 & 0    & 1.5 \\
 section       & 1  & 1.5  & 2.3 & 1.5  & 2.3 \\
 subsection    & 2  & 3.8  & 3.2 & 3.8  & 3.2 \\
 subsubsection & 3  & 7.0  & 4.1 & 7.0  & 4.1 \\
 paragraph     & 4  & 10.0 & 5.0 & 10.0 & 5.0 \\
 subparagraph  & 5  & 12.0 & 6.0 & 12.0 & 6.0 \\
 figure/table  & (1) & 1.5 & 2.3 & 0    & 1.5 \\ 
 subfigure/table & (2) & ---& ---& 1.5  & 2.3 \\ 
\hline
 \end{tabular}
 \end{table}

 Table~\ref{tab:indents} lists the standard values for the \textit{indent}
 and \textit{numwidth}. There is no explicit \textit{numwidth} for a
 part\index{part}; instead a gap of 1em is put between the number and the title text.
 Note that for a sectioning command the values
 depend on whether or not the document class provides the \cmd{\chapter}
 command. Also, which somewhat surprises me, the table\index{table} and figure\index{figure} entries
 are all indented.

\begin{syntax}
\cmd{\@dottedtocline}\marg{level}\marg{indent}\marg{numwidth} \\
\end{syntax}
    Most of the |\l@kind| commands are defined in terms of the
 \cmd{\@dottedtocline} command. This command takes three arguments: 
the \meta{level} argument is the level as shown in \tref{tab:indents},
and \meta{indent} and \meta{numwidth} are the \textit{indent} and 
\textit{numwidth} as illustrated in \fref{fig:ltoc}.
 For example, one definition of the |\l@section| command is: \\
 |\newcommand*{\l@section}{\@dottedtocline{1}{1.5em}{2.3em}}| \\
 If it is necessary to change the default typesetting of the entries,
 then it is usually necessary to change these definitions, but this
class gives you handles to easily alter things without
 having to know the LaTeX internals.

    You can use the \cmd{\addcontentsline} command to add 
\cmd{\contentsline} commands to a file. 

\begin{syntax}
\cmd{\addtocontents}\marg{file}\marg{text} \\
\end{syntax}
    LaTeX also provides the \cmd{\addtocontents}
 command that will insert \meta{text} into \meta{file}. You can use
 this for adding extra text and/or macros into the file, for processing
 when the file is typeset by \cmd{\tableofcontents} (or whatever other
 command is used for \meta{file} processing, such as \cmd{\listoftables}
 for a \file{.lot} file).

 As \cmd{\addcontentsline} and \cmd{\addtocontents} write their arguments to a
 file, any fragile commands used in their arguments must be \cmd{\protect}ed.
 
    You can make certain adjustments to the \toc{} etc., layout by modifying
some of the above macros. Some examples are:
 \begin{itemize}
 \item If your page numbers stick out into the righthand margin\index{margin}
  \begin{lcode}
  \renewcommand{\@pnumwidth}{3em} 
  \renewcommand{\@tocrmarg}{4em}
  \end{lcode}
 but using lengths appropriate to your document.

 \item To have the (sectional) titles in the \toc, etc., typeset ragged 
right with no  hyphenation
 \begin{lcode}
 \renewcommand{\@tocrmarg}{2.55em plus1fil}
 \end{lcode}
 where the value |2.55em| can be changed for whatever margin\index{margin} space you want.

 \item The dots in the leaders can be eliminated by increasing |\@dotsep|
 to a large value:
  \begin{lcode}
  \renewcommand{\@dotsep}{10000}
  \end{lcode}

 \item To have dotted leaders in your \toc{} and \lof{} but not in your \lot:
 \begin{lcode}
 ...
 \tableofcontents
 \makeatletter \renewcommand{\@dotsep}{10000} \makeatother
 \listoftables
 \makeatletter \renewcommand{\@dotsep}{4.5} \makeatother
 \listoffigures
 ...
 \end{lcode}

 \item To add a horizontal line across the whole width of the \toc{} below 
 an entry for a Part\index{part}:
 \begin{lcode}
 \part{Part title}
 \addtocontents{toc}{\protect\mbox{}\protect\hrulefill\par}
 \end{lcode}
 As  said earlier any fragile commands in \cmd{\addtocontents} and 
\cmd{\addcontentsline} 
 their arguments must be protected
 by preceding each fragile command with \cmd{\protect}. 
 The result of the example above
 would be the following two lines in the \file{.toc} file (assuming that it
 is the second Part and is on page 34):
 \begin{lcode}
 \contentsline {part}{II\hspace {1em}Part title}{34}
 \mbox {}\hrulefill \par
 \end{lcode}
 If the \cmd{\protect}s were not used, then the second line would 
instead be:
 \begin{lcode}
 \unhbox \voidb@x \hbox {}\unhbox \voidb@x \leaders \hrule \hfill \kern \z@ \par
 \end{lcode}

\item To change the level of entries printed in the \toc{} (for example
      when normally subsections are listed in the \toc{} but for
      appendices\index{appendix} only the main title is required)
  \begin{lcode}
  \appendix
  \settocdepth{chapter}
  \chapter{First appendix}
  ...
  \end{lcode}

 \end{itemize}
 Remember, if you are modifying any command that includes an |@| sign then this
 must be done in either a \file{.sty} file or if in the document itself
 it must be 
 surrounded by \cmd{\makeatletter} and \cmd{\makeatother}. For example, if you
 want to modify \cmd{\@dotsep} in the preamble\index{preamble} to your document you have
 to do it like this:
 \begin{lcode}
 \makeatletter
 \renewcommand{\@dotsep}{9.0}
 \makeatother
 \end{lcode}
 

 \section{The class \prtoc{} methods} 

  The class provides various means of changing the look of the \toc, etc.,
without having to go through some of the above. 

\begin{syntax}
\cmd{\tableofcontents} \cmd{\tableofcontents*} \\
\cmd{\listoffigures} \cmd{\listoffigures*} \\
\cmd{\listoftables} \cmd{\listoftables*} \\
\end{syntax}
 The \toc, \lof, and \lot{} are printed at the point in the document where
 these commands are called, as per normal LaTeX. However, there are
 two differences between the standard LaTeX behaviour and the behaviour
 with this class. In the standard LaTeX classes
 that have \cmd{\chapter} headings\index{heading}, the \toc, \lof{} and \lot{} each appear on
 a new page. With this class they do not necessarily
 start new pages; if you want them to be on new pages you may have to
 specifically issue an appropriate command beforehand. For example:
 \begin{lcode}
  ...
 \clearpage
 \tableofcontents
 \clearpage
 \listoftables
 ...
 \end{lcode}
Also, the unstarred versions of the commands put their headings\index{heading} into the
\toc, while the starred versions do not.

 \subsection{Changing the titles} \label{sec:titles}

    Commands are provided for controlling the appearance of the
\toc, \lof{} and \lot{} titles. 

\begin{syntax}
\cmd{\contentsname} \cmd{\listfigurename} \cmd{\listtablename} \\
\end{syntax}
Following LaTeX custom, the title texts are the values
 of the \cmd{\contentsname}, \cmd{\listfigurename} and \cmd{\listtablename}
 commands.

 The commands for controlling the typesetting of the \toc, \lof{} and \lot{} titles
all follow a similar pattern So for convenience (certainly mine, 
and hopefully yours)
 in the following
 descriptions I will use |Z| to stand for `toc' or `lof' or `lot'. For
 example, |\Zmark| stands for |\tocmark| or |\lofmark| or |\lotmark|.

    The code for typesetting the \toc{} title looks like:
\begin{lcode}
\tocheadstart
\printtoctitle{\contentsname}
\tocmark
\thispagestyle{chapter}
\aftertoctitle
\end{lcode}
where the macros are described below.

\begin{syntax}
\cmd{\Zheadstart} \\
\end{syntax}
This macro is called before the title is actually printed.
Its default definition is
\begin{lcode}
\newcommand{\Zheadstart}{\chapterheadstart}
\end{lcode}

\begin{syntax}
\cmd{\printZtitle}\marg{title} \\
\end{syntax}
The title is typeset via \cmd{\printZtitle}, which defaults to
using \cmd{\printchaptertitle} for the actual typesetting. 

\begin{syntax}
\cmd{\Zmark} \\
\end{syntax}
 These macros sets the appearance of the running heads on the \toc, \lof, and
 \lot{} pages. The default definition is equivalent to:
\begin{lcode}
\newcommmand{\Zmark}{\@mkboth{\...name}{\...name}}
\end{lcode}
where |\...name| is \cmd{\contentsname} or \cmd{\listfigurename} or
\cmd{\listtablename} as appropriate. You probably don't need to change these, and
in any case they may well be changed by the particular \cmd{\pagestyle} in
use.

\begin{syntax}
\cmd{\afterZtitle} \\
\end{syntax}
 This macro is called after the title is typeset and by
default it is defined to be \cmd{\afterchaptertitle}.

    Essentially, the \toc, \lof{} and \lot{} titles use the same format
as the chapter titles, and will be typeset according to the current
chapterstyle. You can modify their appearance by either using a
different chapterstyle for them than for the actual chapters, or
by changing some of the macros. As examples:
\begin{itemize}
\item Doing
      \begin{lcode}
      \renewcommand{\printZtitle}[1]{\hfill\Large\itshape #1}
      \end{lcode}
      will print the title right justified in a Large italic font.
\item For a Large bold centered title you can do
      \begin{lcode}
      \renewcommand{\printZtitle}[1]{\centering\Large\bfseries #1}
      \end{lcode}
\item Writing
      \begin{lcode}
      \renewcommand{\afterZtitle}{\thispagestyle{empty}\afterchaptertitle}
      \end{lcode}
      will result in the first page of the listing using the \pstyle{empty}
      pagestyle instead of the default \pstyle{chapter} pagestyle.
\item Doing
      \begin{lcode}
      \renewcommand{\afterZtitle}{%
        \par\nobreak \mbox{}\hfill{\normalfont Page}\par\nobreak}
      \end{lcode}
      will put the word `Page' flushright on the line following
      the title.
\end{itemize}




 \subsection{Typesetting the entries} \label{sec:entries}

 Commands are also provided to enable finer control over the typesetting
 of the different kinds of entries. The parameters defining the default
 layout of the entries are illustrated as part of the \Lpack{layouts}
 package~\cite{LAYOUTS} or in~\cite[page 34]{GOOSSENS94}, and are repeated in
 \fref{fig:ltoc}.

\begin{syntax}
 \cmd{\cftdot}\marg{text} \\
\end{syntax}
  In the default \toc{} typsetting only the more minor entries have dotted
 leader lines between the sectioning title and the page number. The
 class provides for general leaders for all entries.
 The `dot' in a leader is given by the value of \cmd{\cftdot}. Its default
 definition is |\newcommand{\cftdot}{.}| which gives the default
 dotted leader. By changing \cmd{\cftdot} you can use symbols other than
 a period in the leader. For example 
 \begin{lcode}
 \renewcommand{\cftdot}{\ensuremath{\ast}}
 \end{lcode}
 will result in a dotted leader using asterisks as the symbol.

\begin{syntax}
 \cmd{\cftdotsep} \\
 \cmd{\cftnodots} \\
\end{syntax}
    Each kind of entry can control the separation between the dots
 in its leader (see below). For consistency though, all dotted leaders
 should use the same spacing. The macro \cmd{\cftdotsep} specifies the
 default spacing. 
 However, if the separation is too large
 then no dots will be actually typeset. The macro \cmd{\cftnodots} is
 a separation value that is `too large'. 

\begin{syntax}
 \cmd{\setpnumwidth}\marg{length} \\
 \cmd{\setrmarg}\marg{length} \\
\end{syntax}
 The page numbers are typeset in a fixed width box. The command
 \cmd{\setpnumwidth} can be used to change the width
 of the box (LaTeX 's internal \cmd{\@pnumwidth}). 
 The title texts will end before reaching the righthand margin\index{margin}.
 \cmd{\setrmarg} can be used to set this distance 
 (LaTeX 's internal \cmd{\@tocrmarg}).
 Note that the length used in \cmd{\setrmarg} should be greater
 than the length set in \cmd{\setpnumwidth}. These values should remain
 constant in any given document.

    This manual requires more space for the page numbers than the default,
so the following was set in the preamble\index{preamble}:
\begin{lcode}
\setpnumwidth{2.55em}
\setrmarg{3.55em}
\end{lcode}


\begin{syntax}
\lnc{\cftparskip} \\
\end{syntax}
 Normally the \lnc{\parskip} in the \toc, etc., is zero. This may be changed
 by changing the length \lnc{\cftparskip}. Note that the current value
 of \lnc{\cftparskip} is used for the \toc, \lof{} and \lot, but you can change
 the value before calling \cmd{\tableofcontents} or \cmd{\listoffigures} or
 \cmd{\listoftables} if one or other of these should have different values
 (which is not a good idea).


    Again for convenience, in the following I will use |X| to stand for any
of the following:
 \begin{itemize}
 \item |part| for |\part| titles
 \item |chapter| for |\chapter| titles
 \item |section| for |\section| titles
 \item |subsection| for |\subsection| titles
 \item |subsubsection| for |\subsubsection| titles
 \item |paragraph| for |\paragraph| titles
 \item |subparagraph| for |\subparagraph| titles
 \item |figure| for figure |\caption| titles
 \item |subfigure| for subfigure |\caption| titles
 \item |table| for table |\caption| titles
 \item |subtable| for subtable |\caption| titles
 \end{itemize}


\begin{syntax}
\cmd{\cftchapterbreak} \\
\end{syntax}
When \cmd{\l@chapter} starts to typeset a \cmd{\chapter} entry in the
\toc{} the first thing it does is to call the macro \cmd{\cftchapterbreak}.
This is defined as:
\begin{lcode}
\newcommand{\cftchapterbreak}{\addpenalty{-\@highpenalty}}
\end{lcode}
which encourages a page break before rather than after the entry. As usual,
you can change \cmd{\cftchapterbreak} to do other things that you feel might
be useful.

\begin{syntax}
\lnc{\cftbeforeXskip} \\
\end{syntax}
 This length controls the vertical space before an entry. It can be changed
 by using \cmd{\setlength}. 

\begin{syntax}
\lnc{\cftXindent} \\
\end{syntax}
 This length controls the indentation of an entry from the left margin\index{margin} 
 (\textit{indent} in \fref{fig:ltoc}). It
 can be changed using \cmd{\setlength}. 

\begin{syntax}
\lnc{\cftXnumwidth} \\
\end{syntax}
 This length controls the space allowed for typesetting title numbers 
 (\textit{numwidth} in \fref{fig:ltoc}). It can
 be changed using \cmd{\setlength}. Second and subsequent lines of a multiline
 title will be indented by this amount.

 The remaining commands are related to the specifics of typesetting
 an entry.
 This is a simplified pseudo-code version for the typesetting of numbered 
 and unnumbered entries.
 \begin{lcode}
 {\cftXfont {\cftXpresnum SNUM\cftXaftersnum\hfil} \cftXaftersnumb TITLE}
         {\cftXleader}{\cftXpagefont PAGE}\cftXafterpnum\par

 {\cftXfont TITLE}{\cftXleader}{\cftXpagefont PAGE}\cftXafterpnum\par
 \end{lcode}
 where |SNUM| is the section number, |TITLE| is the title text and |PAGE| 
 is the page number. In the numbered entry the pseudo-code \\
 |{\cftXpresnum SNUM\cftaftersnum\hfil}| \\
 is typeset within a box of width \lnc{\cftXnumwidth}.

\begin{syntax}
\cmd{\cftXfont} \\
\end{syntax}
 This controls the appearance of the title (and its preceding number, 
 if any). It may be changed using \cmd{\renewcommand}.

\begin{syntax}
\cmd{\cftXpresnum} \cmd{\cftXaftersnum} \cmd{\cftXaftersnumb} \\
\end{syntax}
 The section number is typeset within a box of width \lnc{\cftXnumwidth}.
 Within the box the macro \cmd{\cftXpresnum} is first called, then the
 number is typeset, and the \cmd{\cftXaftersnum}
 macro is called after the number is typeset. The last command
 within the box is \cmd{\hfil} to make the box contents flushleft.
 After the box is
 typeset the \cmd{\cftXaftersnumb} macro is called before typesetting
 the title text. All three of these can be changed by \cmd{\renewcommand}.
 By default they are defined to do nothing.


\begin{syntax}
\cmd{\partnumberline}\marg{num} \\
\cmd{\chapternumberline}\marg{num} \\
\end{syntax}
In the \toc, the macros \cmd{\partnumberline} and \cmd{\chapternumberline}
are responsible respectively for typesetting the \cmd{\part} and \cmd{\chapter}
numbers. Internally they use \cmd{\cftXpresnum}, \cmd{\cftXaftersnum}
and \cmd{\cftaftersnumb} as above. If you do not want, say, 
the \cmd{\chapter} number to appear you
can do:
\begin{lcode}
\renewcommand{\chapternumberline}[1]{}
\end{lcode}

\Note{}  Because the \Lpack{hyperref} package~\cite{HYPERREF} 
does not understand
the \cmd{\partnumberline} and \cmd{\chapternumberline} commands,
if you use the \Lpack{hyperref} package you will also have to use
the \Lpack{memhfixc} package, which comes with memoir.



\begin{syntax}
\cmd{\cftXleader} \\
\cmd{\cftXdotsep} \\
\end{syntax}
 \cmd{\cftXleader} defines the leader between the title and the page number;
 it can be changed by \cmd{\renewcommand}.
 The spacing between any dots in the leader is controlled by \cmd{\cftXdotsep} 
 (\cmd{\@dotsep} in \fref{fig:ltoc}).
 It can be changed by \cmd{\renewcommand} and its value must be either a
 number (e.g., 6.6 or \cmd{\cftdotsep}) or \cmd{\cftnodots} (to disable the dots).
 The spacing
 is in terms of \emph{math units} where there are 18mu to 1em.

\begin{syntax}
\cmd{\cftXpagefont} \\
\end{syntax}
 This defines the font to be used for typesetting the page number. It
 can be changed by \cmd{\renewcommand}.

\begin{syntax}
\cmd{\cftXafterpnum} \\
\end{syntax}
 This macro is called after the page number has been typeset. Its default
 is to do nothing. It can be changed by \cmd{\renewcommand}.

\begin{syntax}
\cmd{\cftsetindents}\marg{entry}\marg{indent}\marg{numwidth} \\
\end{syntax}
 The command 
 \cmd{\cftsetindents} sets the \meta{entry}'s \textit{indent} to the 
length \meta{indent} and its
 \textit{numwidth} to the length \meta{numwidth}. The \meta{entry} argument
 is the name of one of the standard entries (e.g., |subsection|) or the 
name of entry that has been defined within the document.
 For example 
\begin{lcode}
 \cftsetindents{figure}{0em}{1.5em}
\end{lcode}
 will make figure\index{figure} entries left justified.

    This manual requires more space for section numbers in the \toc{} than
the default (which allows for three digits). Consequently the preamble\index{preamble}
contains the following:
\begin{lcode}
\cftsetindents{section}{1.5em}{3.0em}
\cftsetindents{subsection}{4.5em}{3.9em}
\cftsetindents{subsubsection}{8.4em}{4.8em}
\cftsetindents{paragraph}{10.7em}{5.7em}
\cftsetindents{subparagraph}{12.7em}{6.7em}
\end{lcode}
Note that changing the indents at one level implies that any lower level
indents should be changed as well.


 Various effects can be achieved by changing the definitions of \cmd{\cftXfont},
 \cmd{\cftXaftersnum}, \cmd{\cftXaftersnumb}, \cmd{\cftXleader} and 
\cmd{\cftXafterpnum}, 
 either singly or in combination.
 For the sake of some examples, assume that we have the following initial
 definitions
 \begin{lcode}
 \newcommand{\cftXfont}{}
 \newcommand{\cftXaftersnum}{}
 \newcommand{\cftXaftersnumb}{}
 \newcommand{\cftXleader}{\cftdotfill{\cftXdotsep}}
 \newcommand{\cftXdotsep}{\cftdotsep}
 \newcommand{\cftXpagefont}{}
 \newcommand{\cftXafterpnum}{}
 \end{lcode}
 (Note that the same font should be used for the title, leader and page 
 number to provide a coherent appearance).

 \begin{itemize}
 \item To eliminate the dots in the leader:
 \begin{lcode}
 \renewcommand{\cftXdotsep}{\cftnodots}
 \end{lcode}

 \item To put something (e.g., a name) before the title (number):
 \begin{lcode}
 \renewcommand{\cftXpresnum}{SOMETHING }
 \end{lcode}

 \item To add a colon after the section number:
 \begin{lcode}
 \renewcommand{\cftXaftersnum}{:}
 \end{lcode}

 \item To put something before the title number, add a colon after the
    the title number, set everything in bold font,
 and start the title text on the following line:
 \begin{lcode}
 \renewcommand{\cftXfont}{\bfseries}
 \renewcommand{\cftXleader}{\bfseries\cftdotfill{\cftXdotsep}}
 \renewcommand{\cftXpagefont}{\bfseries}
 \renewcommand{\cftXpresnum}{SOMETHING }
 \renewcommand{\cftXaftersnum}{:}
 \renewcommand{\cftXaftersnumb}{\\}
 \end{lcode}

    If you are adding text in the number box in addition to the number,
 then you will probably have to increase the width of the box so that
 multiline titles have a neat vertical alignment; changing box widths
 usually implies that the indents will require modification as 
 well. One possible method of adjusting the box width for the above example
 is:
 \begin{lcode}
 \newlength{\mylen}                  % a "scratch" length
 \settowidth{\mylen}{\bfseries\cftXpresnum\cftXaftersnum}
 \addtolength{\cftXnumwidth}{\mylen} % add the extra space
 \end{lcode} 

 \item To set the section numbers flushright:
 \begin{lcode}
 \setlength{\mylen}{0.5em}    % extra space at end of number
 \renewcommand{\cftXpresnum}{\hfill} % note the double `l'
 \renewcommand{\cftXaftersnum}{\hspace*{\mylen}}
 \addtolength{\cftXnumwidth}{\mylen}
 \end{lcode}
 In the above, the added initial \cmd{\hfill} in the box overrides the
 final \cmd{\hfil} in the box, thus shifting everything to the right hand
 end of the box. The extra space is so that the number is not typeset
 immediately at the left of the title text.

 \item To set the entry ragged left (but this only looks good for single
       line titles):
 \begin{lcode}
 \renewcommand{\cftXfont}{\hfill\bfseries}
 \renewcommand{\cftXleader}{}
 \end{lcode}

 \item To set the page number immediately after the entry text instead of at
       the righthand margin\index{margin}:
 \begin{lcode}
 \renewcommand{\cftXleader}{}
 \renewcommand{\cftXafterpnum}{\cftparfillskip}
 \end{lcode}
 \end{itemize}

\begin{syntax}
\cmd{\cftparfillskip} \\
\end{syntax}
 By default the \cmd{\parfillskip} value is locally set to fill up the last
 line of a paragraph\index{paragraph}. Just changing \cmd{\cftXleader} puts horrible interword
 spaces into the last line of the title. The \cmd{\cftparfillskip} 
 command  is provided just so that the above effect can be achieved.

\begin{syntax}
\cmd{\cftpagenumbersoff}\marg{entry} \\
\cmd{\cftpagenumberson}\marg{entry} \\
\end{syntax}
 The command \cmd{\cftpagenumbersoff} will
 eliminate the page numbers for \meta{entry} in the listing, where
 \meta{entry} is the name of one of the standard
 kinds of entries (e.g., |subsection|, or |figure|) or the name of a
 new entry defined in the document.

    The command \cmd{\cftpagenumberson} reverses
 the effect of a corresponding \cmd{\cftpagenumbersoff} for \meta{entry}.
 
 One question that appeared on the \file{comp.text.tex} newsgroup asked
 how to get the titles of Appendices\index{appendix} list in the \toc{} \emph{without} 
 page numbers. Here is a simple way of doing it.
 \begin{lcode}
 ...
 \appendix
 \addtocontents{toc}{\cftpagenumbersoff{chapter}}
 \chapter{First appendix}
 \end{lcode}
 If there are other chapter type headings\index{heading!chapter} to go into the \toc{} after the 
 appendices\index{appendix} (perhaps a bibliography\index{bibliography} or an index\index{index}), 
 then it will be necessary to do a similar 
 \begin{lcode}
 \addtocontents{toc}{\cftpagenumberson{chapter}}
 \end{lcode}
 after the appendices\index{appendix} to restore the page numbering in the \toc.

  At this point, I leave it up to your ingenuity as to other effects that
 you can achieve. However, if you come up with further examples, let me
 know for possible inclusion in a later version of this document.

 \section{New list of\ldots{} and entries}

\index{list!new list of|(}

 \begin{syntax}
\cmd{\newlistof}\marg{listofcom}\marg{ext}\marg{listofname} \\
\end{syntax}
 The command \cmd{\newlistof} 
 creates a new \listofx, and assorted commands to go along with it.
  The first argument, \meta{listofcom} is used to define a new
 command called |\listofcom| which can then be used like |\listoffigures|
to typeset the \listofx. The \meta{ext} argument is the file extension to
be used for the new listing. The last argument, \meta{listofname} is
the title for the \listofx. Unstarred and starred versions of
|\listofcom| are created. The unstarred version, |\listofcom|, will add
\meta{listofname} to the \toc, while the starred version, |\listofcom*|,
makes no entry in the \toc.

 As an example:
 \begin{lcode}
 \newcommand{\listanswername}{List of Answers}
 \newlistof{listofanswers}{ans}{\listanswername}
 \end{lcode}
 will create a new \cmd{\listofanswers} commmand that can be used
to typeset a listing of answers under the
title \cmd{\listanswername}, where the answer titles are in a \file{.ans}
file. 
   It is up to the author of the document to specify the `answer' code
for the answers in the document. For example:
 \begin{lcode}
 \newcounter{answer}[chapter]
 \renewcommand{\theanswer}{\thechapter.\arabic{answer}}
 \newcommand{\answer}[1]{
   \refstepcounter{answer}
   \par\noindent\textbf{Answer \theanswer. #1}
   \addcontentsline{ans}{answer}{\protect\numberline{\theanswer}#1}\par}
 \end{lcode}
 which, when used like: \\ 
 |\answer{Hard} The \ldots|  \\
 will print as:
\begin{syntax}
 \textbf{Answer 1. Hard} \\
 \hspace*{2em} The \ldots \\
\end{syntax}

    As mentioned above, the \cmd{\newlistof} command creates several 
new commands in addition to |\listofcom|, most of which you should now be 
familiar with. For convenience,
 assume that |\newlistof{...}{Z}{...}| has been issued so that
 |Z| is the new file extension and corresponds to the |Z| in
 \S\ref{sec:titles}. Then in addition to |\listofcom| the following 
new commands will be made available.

 The four commands, |\Zmark|, 
 |\Zheadstart|, 
 |\printZtitle|, and
 |\afterZtitle|, 
are analagous to the commands of the same names
described in \S\ref{sec:titles} (internally the class uses
the \cmd{\newlistof} macro to define the \toc, \lof{} and \lot). 
In particular the default definition of |\Zmark| is equivalent to:
\begin{lcode}
\newcommand{\Zmark}{\@mkboth{listofname}{listofname}}
\end{lcode}
However, this may well be altered by the particular \cmd{\pagestyle} in
use.

\begin{syntax}
|Zdepth| \\
\end{syntax}
 The counter |Zdepth| is analagous to the standard
 \Icn{tocdepth} counter, in that it specifies that entries
 in the new listing should not be typeset if their numbering level 
 is greater
 than |Zdepth|. The default definition is equivalent to
\begin{lcode}
\setcounter{Zdepth}{1}
\end{lcode}

\begin{syntax}
\cmd{\insertchapterspace} \\
\cmd{\addtodef}\marg{macro}\marg{prepend}\marg{append} \\
\end{syntax}
Remember that the \cmd{\chapter} command uses \cmd{\insertchapterspace}
to insert vertical spaces into the \lof{} and \lot. If you want similar
spaces added to your new listing then you have to modify
\cmd{\insertchapterspace}. The easiest way to do this is via
the \cmd{\addtodef} macro, like:
\begin{lcode}
\addtodef{\insertchapterspaces}{}%
  {\addtocontents{ans}{\protect\addvspace{10pt}}}
\end{lcode}
The \cmd{\addtodef} macro is described later in \S\ref{sec:addtodef}.

    The other part of creating a new \listofx, is to specify the 
formatting of the entries, i.e., define an appropriate |\l@kind| macro.

\begin{syntax}
\cmd{\newlistentry}\oarg{within}\marg{cntr}\marg{ext}\marg{level-1} \\
\end{syntax}
 The command \cmd{\newlistentry} creates the commands necessary for
typesetting an entry in a \listofx.
 The first required argument, \meta{cntr} is used to define a new
 counter called |cntr|, unless |cntr| is already defined. 
The optional \meta{within} argument can
 be used so that |cntr| gets reset to one every time the counter called 
 |within| is changed. That is, the first two arguments when |cntr| is not
already defined, are equivalent to
 calling \cmd{\newcounter}\marg{cntr}\oarg{within}. If |cntr| is already
defined, \cmd{\newcounter} is not called. |cntr| is used for the number that
goes along with the title of the entry.

 The second required argument, \meta{ext}, is the file extension for the
 entry listing.
 The last argument, \meta{level-1}, is a number specifying the numbering
 level minus one, 
 of the entry in a listing.


    Calling \cmd{\newlistentry} creates several new commands. Assuming that
 it is called as |\newlistentry[within]{X}{Z}{N}|, where |X| and |Z| are
 similar to the
 previous uses of them (e.g., |Z| is the file extension), 
and |N| is an integer number, then the following
 commands are made available.


  The set of commands |\cftbeforeXskip|, 
 |\cftXfont|, 
 |\cftXpresnum|, 
 |\cftXaftersnum|, 
 |\cftXaftersnumb|, 
 |\cftXleader|, 
 |\cftXdotsep|, 
 |\cftXpagefont|, and
 |\cftXafterpnum|,
  are analagous to the commands of the same names
 described in \S\ref{sec:entries}. Their default values are also
 as described earlier.

 The default values of |\cftXindent| and |\cftXnumwidth| are set according
 to the value of the \meta{level-1} argument (i.e., |N| in this example).
 For |N=0| the settings correspond to those for 
 figures\index{figure} and tables\index{table}, as listed in \tref{tab:indents} for the 
\Lclass{memoir} class.
 For |N=1| the settings correspond
 to subfigures\index{figure!sub-}, and so on.
 For values of |N| less than zero or greater than four, 
 or for non-default values, use the
 \cmd{\cftsetindents} command to set the values.


  |\l@X| is an internal command that typesets an entry in the list, and
 is defined in terms of the above |\cft*X*| commands. It will not typeset
 an entry if |\Zdepth| is |N| or less, where |Z| is the listing's file
 extension.

 The command |\theX| prints the value of the |X| counter. It is initially
 defined so that it prints arabic numerals. If the optional \meta{within}
 argument is used, |\theX| is defined as 
\begin{lcode}
 \renewcommand{\theX}{\thewithin.\arabic{X}}
\end{lcode}
 otherwise as
\begin{lcode}
\renewcommand{\theX}{\arabic{X}}
\end{lcode}


 As an example of the independent use of \cmd{\newlistentry}, the following
 will set up for sub-answers.
 \begin{lcode}
 \newlistentry[answer]{subanswer}{1}
 \renewcommand{\thesubanswer}{\theanswer.\alph{subanswer}}
 \newcommand{\subanswer}[1]{
    \refstepcounter{subanswer}
    \par\textbf{\thesubanswer) #1}
    \addcontentsline{ans}{subanswer{\protect\numberline{\thesubanswer}#1}}
 \setcounter{ansdepth}{2}
 \end{lcode}
 And then:
 \begin{lcode}
 \answer{Harder} The \ldots
   \subanswer{Reformulate the problem} It assists \ldots
 \end{lcode}
 will be typeset as:
\begin{syntax}
\textbf{Answer 2. Harder} \\
\hspace*{2em} The \ldots \\
\hspace*{2em} \textbf{2.a) Reformulate the problem} It assists \ldots \\
\end{syntax}

 By default the answer entries will appear in the List of Answers listing
 (typeset by the |\listofanswers| command).
 In order to get the subanswers to appear, 
 the |\setcounter{ansdepth}{2}| command was used above.

 To turn off page numbering for the subanswers, do \\
 |\cftpagenumbersoff{subanswer}|

    As another example of \cmd{\newlistentry}, suppose that an extra sectioning
 division below |subparagraph| is required, called |subsubpara|.
 The |\subsubpara| command itself can be defined via the LaTeX kernel
 \cmd{\@startsection} command. 
 Also it is necessary to define a |\subsubparamark| macro,
 a new |subsubpara| counter, a |\thesubsubpara| macro and a |\l@subsubpara|
 macro. Using \cmd{\newlistentry} 
 takes care of most of these as shown below (remember
 the caveats about commands with |@| signs in them).
 \begin{lcode}
 \newcommand{\subsubpara}{\@startsection{subpara}
    {6}                            %                    level
    {\parindent}                   %  indent from left margin
    {3.25ex \@plus1ex \@minus .2ex}  %     skip above heading
    {-1em}       runin heading with  % 1em between title & text
    {\normalfont\normalsize\itshape} % italic number and title 
 }
 \newlistentry[subparagraph]{subsubpara}{toc}{5}
 \cftsetindents{subsubpara}{14.0em}{7.0em}
 \newcommand*{\subsubparamark}[1]{}  % gobble heading mark
 \end{lcode}
 

     Each \listofx{} uses a file to store the list entries, and these
 files must remain open for writing throughout the document processing.
 TeX has only a limited number of files that it can keep open, and this
 puts a limit on the number of listings that can be used. For a document
 that includes a \toc{} but no other extra ancilliary files (e.g., no
 index\index{index} or bibliography\index{bibliography} output files) the maximum number of LoX's, including
 a \lof{} and \lot, is no more than about eleven. If you try and create too many
 new listings LaTeX will respond with the error message: 
 \begin{center}
 |No room for a new write| 
 \end{center}
 If you get such a message the only recourse is to redesign your document.

\subsection{Example --- plates}

    As has been mentioned earlier, some illustrations\index{illustration} may be 
tipped in\index{tip in} to a book. Often, these are called `plates' if
they are on glossy paper\index{paper} and the rest of the book is on ordinary paper\index{paper}.
We can define a new kind of Listing for these.

\begin{lcode}
\newcommand{\listplatename}{Plates}
\newlistof{listofplates}{lop}{\listplatename}
\newlistentry{plate}{lop}{0}
\cftpagenumbersoff{plate}
\end{lcode}
This code defines the \cmd{\listofplates} command to start the listing which
will be titled `Plates' from the \cmd{\listplatename} macro. The entry name
is |plate| and the file extension is |lop|. As plate pages typically do
not have printed folios\index{folio}, the \cmd{\cftpagenumbersoff} command has been
used to prohibit page number printing in the listing.

    If pages are tipped in, then they are put between a verso and a recto 
page. The \Lpack{afterpage} package~\cite{AFTERPAGE} lets you specify 
something that should happen after the current page is finished. The next
piece of code uses the package and its \cmd{\afterpage} macro to
define two macros which let you specify something that is to be
done after the next verso or recto page has been completed.
\begin{lcode}
\newcommand{\afternextverso}[1]{%
  \afterpage{\ifodd\c@page #1\else\afterpage{#1}\fi}}
\newcommand{\afternextrecto}[1]{%
  \afterpage{\ifodd\c@page\afterpage{#1}\else #1\fi}}
\end{lcode}


    The \cmd{\pageref}\marg{labelid} command typesets the page number
corresponding to the location in the document where 
\cmd{\label}\marg{labelid} is specified. The following code defines
two macros\footnote{These only work for arabic page numbers.} 
that print the page number before or after that given by
\cmd{\pageref}.
\begin{lcode}
\newcounter{mempref}
\newcommand{\priorpageref}[1]{%
  \setcounter{mempref}{\pageref{#1}}\addtocounter{mempref}{-1}\themempref}
\newcommand{\nextpageref}[1]{%
  \setcounter{mempref}{\pageref{#1}}\addtocounter{mempref}{1}\themempref}
\end{lcode}

    With these preliminaries out of the way, we can use code like the 
following for handling a set of physically tipped in plates.
\begin{lcode}
\afternextverso{\label{tip}
  \addtocontents{lop}{%
    Between pages \priorpageref{tip} and \pageref{tip}
    \par\vspace*{\baselineskip}}
  \addcontentsline{lop}{plate}{First plate}
  \addcontentsline{lop}{plate}{Second plate}
  ...
  \addcontentsline{lop}{plate}{Nth plate}
}
\end{lcode}
This starts off by waiting until the next recto page is started, which
will be the page immediately after the plates, and then
inserts the label |tip|. The \cmd{\addtocontents} macro puts its argument
into the plate list |lop| file, indicating the page numbers before and after
the set of plates. With the plates being physically added to the document
it is not possible to use \cmd{\caption}, instead the \cmd{\addcontentsline} 
macros are used to add the plate titles to the |lop| file.

    With a few modifications the code above can also form the basis 
for listing plates that are
electronically tipped in but do not have printed folios\index{folio} or \cmd{\caption}s.

\index{list!new list of|)}



 \section{Extras}

\index{chapter!precis|(}

   Some old style novels, and even some modern text 
 books,\footnote{For example, Robert Sedgewick, \textit{Algorithms},
 Addison-Wesley, 1983.} include a short synopsis of the contents of 
 the chapter either immediately
 after the chapter heading\index{heading!chapter} or in the \toc, or in both places.

\begin{syntax}
\cmd{\chapterprecis}\marg{text} \\
\end{syntax}
     The command \cmd{\chapterprecis} prints its argument 
 both at the
 point in the document where it is called, and also adds it to the \file{.toc}
 file. For example:
 \begin{lcode}
 ...
 \chapter{}  first chapter
 \chapterprecis{Our hero is introduced; family tree; early days.}
 ...
 \end{lcode}

\begin{syntax}
\cmd{\chapterprecishere}\marg{text} \\
\cmd{\chapterprecistoc}\marg{text} \\
\end{syntax}
 The \cmd{\chapterprecis} command calls these two commands to print the
 \meta{text} in the document (the \cmd{\chapterprecishere} command) 
 and to put it into the \toc{} (the \cmd{\chapterprecistoc} command). 
 These can be used individually if required.

\begin{syntax}
\cmd{\prechapterprecis} \cmd{\postchapterprecis} \\
\end{syntax}
The \cmd{\chapterprecishere} macro is intended for use immediately after 
a \cmd{\chapter}. The \meta{text} argument is typeset in
italics in a \Ie{quote} environment. The macro's definition is:
\begin{lcode}
\newcommand{\chapterprecishere}[1]{%
  \prechapterprecis #1\postchapterprecis}
\end{lcode}
where \cmd{\prechapterprecis} and \cmd{\postchapterprecis} are defined
as:
\begin{lcode}
\newcommand{\prechapterprecis}{%
  \vspace*{-2\baselineskip}%
  \begin{quote}\normalfont\itshape}
\newcommand{\postchapterprecis}{\end{quote}}
\end{lcode}
The \cmd{\prechapterprecis} and \cmd{\postchapterprecis} macros can be 
changed if another style of typesetting is required.

\begin{syntax}
\cmd{\precistoctext}\marg{text} \cmd{\precistocfont} \\
\end{syntax}
The \cmd{\chapterprecistoc} macro puts the macro \cmd{\precistoctext} into 
the \ixfile{toc} file. The default definition is
\begin{lcode}
\DeclareRobustCommand{\precistoctext}[1]{%
  {\leftskip \cftchapterindent\relax
   \advance\leftskip \cftchapternumwidth\relax
   \rightskip \@tocrmarg\relax
   \precistocfont #1\par}}
\end{lcode}
Effectively, in the \toc{} \cmd{\precistoctext} typesets its argument like 
a chapter title using the \cmd{\precistocfont} (default \cmd{\itshape}).

\index{chapter!precis|)}

 Sometimes it may be desirable to make a change to the global parameters
 for an individual entry. For example, a figure\index{figure} might be placed on
 the end paper\index{paper!end} of a book (the inside of the front or back cover), and
 this needs to be placed in a \lof{} with the page number set as, say
 `inside front cover'. If `inside front cover' is typeset as an ordinary
 page number it will stick out into the margin\index{margin}. Therefore, the parameters
 for this particular entry need to be changed.

\begin{syntax}
\cmd{\cftlocalchange}\marg{ext}\marg{pnumwidth}\marg{tocrmarg} \\
\end{syntax}
 The command \cmd{\cftlocalchange} 
 will write an entry into the file with extension \meta{ext} to reset 
the global \cmd{\@pnumwidth} and \cmd{\@tocrmarg} parameter lengths. 
 The command should be called again after any special entry to reset
 the parameters back to their usual values. Any fragile commands used
 in the arguments must be protected.

\begin{syntax}
\cmd{\cftaddtitleline}\marg{ext}\marg{kind}\marg{text}\marg{page} \\
\cmd{\cftaddnumtitleline}\marg{ext}\marg{kind}\marg{num}\marg{title}\marg{page} \\
\end{syntax}
 The command \cmd{\cftaddtitleline} 
 will write a \cmd{\contentsline} entry into \meta{ext} for a \meta{kind}
 entry with title \meta{title} and page number \meta{page}. 
 Any fragile commands used in the arguments must be protected.
That is,
 an entry is made of the form: 
\begin{lcode}
\contentsline{kind}{title}{page}
\end{lcode}
 The command \cmd{\cftaddnumtitleline}
 is similar to \cmd{\cftaddtitleline} except that it also includes 
\meta{num} as the argument to
 \cmd{\numberline}. That is, an entry is made of the form
\begin{lcode}
\contentsline{kind}{\numberline{num} title}{page}
\end{lcode}

 As an example of the use of these commands, 
 noting that the default LaTeX values for 
 \cmd{\@pnumwidth} and \cmd{\@tocrmarg} are 1.55em and 2.55em respectively, 
 one might do the
 following for a figure\index{figure} on the frontispiece\index{frontispiece} page.
 \begin{lcode}
 ...
  this is the frontispiece page with no number
  draw or import the picture (with no \caption)
 \cftlocalchange{lof}{4em}{5em} % make pnumwidth big enough for 
                                % frontispiece and change margin
 \cftaddtitleline{lof}{figure}{The title}{frontispiece}
 \cftlocalchange{lof}{1.55em}{2.55em} % return to normal settings
 \clearpage
 ...
 \end{lcode}
    Recall that a \cmd{\caption} command will put an entry in the \file{.lof}
 file, which is not wanted here. If a caption\index{caption} is required, then you can 
 either craft one youself or, assuming that your general captions\index{caption} are not
 too exotic, use the \cmd{\legend} command (see later).
 If the illustration\index{illustration} is numbered, use 
 \cmd{\cftaddnumtitleline} instead of \cmd{\cftaddtitleline}.


\section{Tails}

    The commands for producing a bibliography\index{bibliography} and an index\index{index} are
the same as for the standard classes, but there are two differences.

\subsection{Bibliography}

\index{bibliography|(}

\begin{syntax}
\senv{thebibliography}\marg{exlabel} \\
\cmd{\bibitem} ... \\
\eenv{thebibliography} \\
\cmd{\bibname} \\
\end{syntax}
The bibliography is typeset within the \Ie{thebibliography} environment.
This takes one required argument, \meta{exlabel}, which is a piece of text
as wide as the widest label in the bibliography. The value of 
\cmd{\bibname} (default `Bibliography') is used
as the title. 

\begin{syntax}
\cmd{\bibintoc} \cmd{\nobibintoc} \\
\end{syntax}
The declaration \cmd{\bibintoc} will cause the |thebibliography|
environment to add the title to the \toc, while the declaration
\cmd{\nobibintoc} ensures that the title is not added to the \toc. 
The default is \cmd{\bibintoc}.


\begin{syntax}
\cmd{\prebibhook} \cmd{\postbibhook} \\
\end{syntax}
The commands \cmd{\prebibhook} and \cmd{postbibhook} are called after 
typesetting the title of the bibliography and after typesetting the list of
entries, respectively. By default, they are defined to do nothing. You may
wish to use one or other of these to provide some general information about
the bibliography. For example:
\begin{lcode}
\renewcommand{\postbibhook}{%
CTAN is the Comprehensive TeX Archive Network and URLS for the 
several CTAN mirrors can be found at \url{http://www.tug.org}.}
\end{lcode}

\begin{syntax}
\cmd{\setbiblabel}\marg{style} \\
\end{syntax}
The style of the labels marking the bibliographic entries can be set
via \cmd{\setbiblabel}. The default definition is
\begin{lcode}
\setbiblabel{[#1]\hfill}
\end{lcode}
where |#1| is the citation mark position, which generates flushleft 
marks enclosed in square brackets. To have marks just
followed by a dot
\begin{lcode}
\setbiblabel{#1.\hfill}
\end{lcode}



    The definition of the \Ie{thebibliography} environment is:
\begin{lcode}
\newenvironment{thebibliography}[1]{%
  \bibsection
  \begin{bibitemlist}{#1}}{\end{bibitemlist}\postbibhook}
\end{lcode}

\begin{syntax}
\cmd{\bibsection} \\
\end{syntax}
The macro \cmd{\bibsection} defines the heading for the \Ie{thebibliography}
environment; that is, everything before the actual list of bibliographic
items starts. Its default definition is:
\begin{lcode}
\newcommand{\bibsection}{%
  \chapter*{\bibname}
  \bibmark
  \ifnobibintoc\else
    \phantomsection
    \addcontentsline{toc}{chapter}{\bibname}
  \fi
  \prebibhook}
\end{lcode}
To change the style of the heading for the bibliography, redefine
\cmd{\bibsection}. For example, to have the bibliography typeset as a
numbered section instead of a chapter, redefine \cmd{\bibsection} as:
\begin{lcode}
\renewcommand{\bibsection}{%
  \section{\bibname}
  \prebibhook}
\end{lcode}

    If you use the \Lpack{natbib} and/or the \Lpack{chapterbib}
packages~\cite{NATBIB,CHAPTERBIB} with the \Lopt{sectionbib} option, 
then \cmd{\bibsection} is
changed appropriately to typeset as a numbered section.

    The \Lpack{jurabib} package~\cite{JURABIB} redefines 
the \Ie{thebibliography}
environment, 
providing its own methods for listing the items. However,
the redefinition also eliminates the opportunity to add the
Bibliography to the Table of Contents and to have some introductory text.
To restore these to the class specification, put the following in your
preamble after loading \Lpack{jurabib}:
\begin{lcode}
\usepackage{jurabib}
\makeatletter
\renewcommand{\bib@heading}{\bibsection}
\makeatother
\end{lcode}
However, thanks to the kindness of Jens Berger, if your version of 
\Lpack{jurabib} is 0.6 or later then 
the fix is not required.


\begin{syntax}
\lnc{\bibitemsep} \\
\end{syntax}
   The \Lpack{natbib} package provides a length, \lnc{\bibsep} which can
be used to alter the vertical spacing between the entries in the bibliography.
Setting \lnc{\bibsep} to 0pt removes any extra space between the entries.
    The equivalent length provided by the class for changing the space between
bibliography entries is \lnc{\bibitemsep}, which by default is set to 
the default value of \lnc{\itemsep}. 

    The bibliography is set as a list, and the spacing between the items is 
(\lnc{\bibitemsep} + \lnc{\parsep}). To eliminate any extra vertical space do
\begin{lcode}
\setlength{\bibitemsep}{-\parsep}
\end{lcode}

\begin{syntax}
\cmd{\biblistextra} \\
\end{syntax}
    A hook, called \cmd{\biblistextra}, is provided that is called at the
end of the bibliography list setup. By default it does nothing but
it can be used, for example, to set all bibliography entries flushleft by
modify the list parameters as shown below.
\begin{lcode}
\renewcommand{\biblistextra}{%
  \setlength{\leftmargin}{0pt}%
  \setlength{\itemindent}{\labelwidth}%
  \setlength{\itemindent}{\labelsep}%
}
\end{lcode}
  


\index{bibliography|)}

\subsection{Index}

\index{index|(}

    The indexing commands have been significantly enhanced with respect
to the standard classes and include the
functionality provided by the \Lpack{makeidx}, \Lpack{showidx} and 
\Lpack{index} packages~\cite{MAKEIDX,INDEX}; these packages should not be used.

    In the standard classes the index is set in two columns.
\begin{syntax}
\cmd{\onecolindextrue} \\
\cmd{\onecolindexfalse} \\
\end{syntax}
The declaration \cmd{\onecolindexfalse}, which is the default, causes
any indexes to be set in two columns. The declaration \cmd{\onecolindextrue}
causes any following indexes to be set in one column. This can be useful
if, for example, you need an index of the first lines of poems.

\begin{syntax}
\cmd{\makeindex}\oarg{file} \\
\cmd{\printindex}\oarg{file} \\
\end{syntax}
The macro \cmd{\makeindex}, which must be put in the preamble if it is used,
opens an \ixfile{idx} file, which by default is called \file{jobname.idx}, 
where \file{jobname} is the name of the main LaTeX source file. 
If the optional \meta{file} argument is given then a file called
\file{file.idx} will be opened instead. 
The macro \cmd{\printindex} reads an \ixfile{ind}
file called \file{jobname.ind}, which should contain an 
\Ie{theindex} environment
and the indexed items. If the optional \meta{file} argument is given then
the \file{file.ind} file will be read. The \textsc{makeindex} program is often
used to convert an \ixfile{idx} file to an \ixfile{ind} file.



\begin{syntax}
\senv{theindex} entries \eenv{theindex} \\
\cmd{\indexname} \\
\end{syntax}
The index\index{index} is typeset in two columns\index{column!double} within the \Ie{theindex} environment.
The index\index{index} title is given by the current value of \cmd{\indexname} (default
`Index').

\begin{syntax}
\cmd{\indexintoc} \cmd{\noindexintoc} \\
\end{syntax}
The declaration \cmd{\indexintoc} will cause the |theindex|
environment to add the title to the \toc, while the declaration
\cmd{\noindexintoc} ensures that the title is not added to the \toc. 
The default is \cmd{\indexintoc}.

\begin{syntax}
\lnc{\indexcolsep} \\
\lnc{\indexrule} \\
\end{syntax}
The length \lnc{\indexcolsep} is the width of the gutter between the two
index columns\index{column!double}. Its default value is 35pt. The length \lnc{\indexrule}, default
value 0pt, is the thickness of a vertical rule separating the two columns.


\begin{syntax}
\cmd{\preindexhook} \\
\end{syntax}
    The macro \cmd{\preindexhook} is called after the title is typeset and
before the twocolumn index\index{index} listing starts. By default it does nothing but
can be changed. For example
\begin{lcode}
\renewcommand{\preindexhook}{Bold page numbers are used to indicate
the main reference for an entry.}
\end{lcode}


\begin{syntax}
\cmd{\index}\oarg{file}\marg{item} \\
\cmd{\specialindex}\marg{file}\marg{counter}\marg{item} \\
\end{syntax}
The macro \cmd{\index} writes its \meta{item} argument to an \ixfile{idx}
file. If the optional \meta{file} argument is given then it will write
to \file{file.idx} otherwise it writes to \file{jobname.idx}. The page for
the \meta{item} is also written to the \file{idx} file. 
The \cmd{\specialindex} macro writes its \meta{item} argument to the 
\file{file.idx} and also writes the page number (in parentheses) and
the value of the \meta{counter}. This means that indexing can be with 
respect to something other than page numbers. However, if the \Lpack{hyperref}
package is used the special index links will be to pages even though they
appear to be with respect to the \meta{counter}; for example, 
if figure numbers are used as the index reference the hyperref link will be
to the page where the figure appears and not the figure itself.

\begin{syntax}
\cmd{\see}\marg{item} \cmd{\seename} \\
\cmd{\seealso}\marg{items} \cmd{\alsoname} \\
\end{syntax}
The macro \cmd{\see} can be used in an \cmd{\index} command to tell the 
reader to `see \meta{item}' instead of printing a page number. Likewise
the \cmd{\seealso} macro directs the reader to `see also \meta{items}'.
For example:
\begin{lcode}
\index{Alf|see{Alfred}}
\index{Frederick|seealso{Fred, Rick}}
\end{lcode}
The actual values for `see' and `see also' are given by the 
\cmd{\seename} and \cmd{\alsoname} macros whose default
definitions are:
\begin{lcode}
\newcommand{\seename}{see}
\newcommand{\alsoname}{see also}
\end{lcode}

\begin{syntax}
\cmd{\reportnoidxfilefalse} \\
\cmd{\reportnoidxfiletrue} \\
\end{syntax}
Following the declaration \cmd{\reportnoidxfilefalse}, which is the default,
LaTeX will silently pass over attempts to use an \ixfile{idx} file which has
not been declared via \cmd{\makeindex}. After the declaration
\cmd{\reportnoidxfiletrue} LaTeX will whinge about any attempts to 
write to an unopened file.

\begin{syntax}
\cmd{\showindexmarktrue} \\
\cmd{\showindexmarkfalse} \\
\end{syntax}
After the declaration \cmd{\showindexmarktrue} (practically) 
all \cmd{\index} and 
\cmd{\specialindex} \meta{item} arguments are listed in the margin of 
the page on which the index command is issued. The default is
\cmd{\showindexmarkfalse}.

\subsubsection{Indexing and the \Lpack{natbib} package}

    The \Lpack{natbib} package~\cite{NATBIB} will make an index 
of citations if
\cmd{\citeindextrue} is put in the preamble after the \Lpack{natbib}
package is called for.

\begin{syntax}
\cmd{\citeindexfile} \\
\end{syntax}
The name of the file for the citation index is stored in the
macro \cmd{\citeindexfile}. This is initially defined as:
\begin{lcode}
\newcommand{\citeindexfile}{\jobname}
\end{lcode}
That is, the citation entries will be written to the default 
\ixfile{idx} file.
This may be not what you want so you can change this, for example to:
\begin{lcode}
\renewcommand{\citeindexfile}{names}
\end{lcode}
If you do change \cmd{\citeindexfile} then you have to put
\begin{lcode}
\makeindex[\citeindex]
\end{lcode}
\emph{before}
\begin{lcode}
\usepackage[...]{natbib}
\end{lcode}

    So, there are effectively two choices, either along the lines of
\begin{lcode}
\renewcommand{\citeindexfile}{authors} % write to authors.idx
\makeindex[\citeindexfile]
\usepackage{natbib}
\citeindextrue
...
\renewcommand{\indexname}{Index of citations}
\printindex[\citeindexfile]
\end{lcode}
or along the lines of
\begin{lcode}
\usepackage{natbib}
\citeindextrue
\makeindex
...
\printindex
\end{lcode}

\subsubsection{Populating the \file{idx} file}

    In the standard classes, indexed items are written directly to an
\ixfile{idx} file. With the class, however, the indexed items are
written to the \ixfile{aux} file and then on the next LaTeX run the
indexed items in the \ixfile{aux} file are written to the designated
\ixfile{idx} file.

   The disadvantage of this two stage process is that after any change to
the indexed items LaTeX has to be run twice to ensure that the change
is propagated to the \ixfile{idx} file. Then, of course, a new \ixfile{ind}
will have to be created and LaTeX run one more time. However, this process
is very similar to what you have to do if you are using BibTeX to create 
a bibliography.

   The advantage of the approach is that indexed items from \cmd{\include}
files that are not processed on a particular run are not lost. The
standard direct write to an \ixfile{idx} file loses any `non-inluded'
indexed items.

\index{index|)}



%%%%%%%%%%%%%%%%%%%%%%%%%%%%%%%%%%%%%%%%%%%%%%%%%
%%%%%%%%%%%%%%%%%%%%%%%%%%%%%%%%
\chapterstyle{hangnum}
%%%%%%%%%%%%%%%%%%%%%%%%%%%%%%%
 \chapter{Captions and floats} \label{chap:captions}

This chapter uses the \cstyle{hangnum} chapterstyle. Section numbers
are also hung into the margin\index{margin} to match via the declaration \cmd{\hangsecnum}.
\hangsecnum

\newcommand{\pname}{ccaption}

 \section{Introduction}

\index{caption|(}

 Some publishers require, and some authors prefer, captioning styles
 other than the one style provided by standard LaTeX. This chapter
describes how you can implement your own captioning style.

 Some publishers require that documents that include multi-part
 tables\index{table} use a \textit{continuation caption} on all but the first
 part of the multi-part table\index{table}. For the times where such a table\index{table}
 is specified by the author as a set of tables\index{table}, the 
 class 
 provides a simple `continuation' caption command to meet this 
 requirement. It also provides
 a facility for an `anonymous' caption which can be used in any
 float\index{float} environment. 
 Captions can be defined that are suitable for use in non-float
 environments, such as placing a picture in a minipage and captioning
 it just as though it had been put into a normal figure\index{figure} environment.
 Further, a mechanism is provided for defining new float\index{float} environments.

    The commands and environments described below are very similar to
those supplied by the \Lpack{ccaption} package~\cite{CCAPTION}.

 \section{Captions} 


 \subsection{Changing the caption style}

\index{caption!style|(}

\begin{syntax}
\cmd{\captiondelim}\marg{delim} \\
\end{syntax}
 The default captioning style is to put a delimeter in the form
 of a colon between the caption
 number and the caption title. The command \cmd{\captiondelim}
 can be used to change the delimeter. For example, to have an en-dash instead
 of the colon, |\captiondelim{-- }| will do the trick. Notice that no space is
 put between the delimeter and the title unless it is specified in the
 \meta{delim} parameter. 
 The class initially specifies |\captiondelim{: }|
 to give the normal delimeter.

\begin{syntax}
\cmd{\captionnamefont}\marg{fontspec} \\
\end{syntax}
 The \meta{fontspec} specified by \cmd{\captionnamefont} is used
 for typesetting the caption name; that is, the first part of the caption
 up to and including the delimeter (e.g., the portion `Table 3:').
 \meta{fontspec} can be any kind of font specification and/or command and/or 
 text. This first part of the caption is treated like: 
|{\captionnamefont Table 3: }|,
 so font declarations, not font text-style commands, are needed for \meta{fontspec}
 For instance, |\captionnamefont{\Large\sffamily}| to specify a large 
sans-serif font.
 The class initially specifies |\captionnamefont{}|
 to give the normal font.
 

\begin{syntax}
\cmd{\captiontitlefont}\marg{fontspec} \\
\end{syntax}
 Similarly, the \meta{fontspec} specified by \cmd{\captiontitlefont}
 is used for typesetting the title text of a caption. For example,
 |\captiontitlefont{\itshape}| for an italic title text.
 The class initially specifies |\captiontitlefont{}|
 to give the normal font.

\begin{syntax}
\cmd{\captionstyle}\oarg{short}\marg{normal} \\
\cmd{\raggedleft} \cmd{\centering} \cmd{\raggedright} \cmd{centerlastline} \\
\end{syntax}
Caption styles are set according to the \cmd{\captionstyle} declaration.
Unless the optional \meta{short} argument is given all captions are typeset
according to \meta{normal}. If the optional \meta{short} argument
is specififed, captions that are less than one line in length are typeset
according to \meta{short}. 
 By default the name and title of a caption are typeset as a block 
 (non-indented)
 paragraph\index{paragraph!block}. 

    Sensible values for the arguments include:
\cmd{\raggedleft}, \cmd{\centering}, \cmd{\raggedright}, and
 \cmd{centerlastline}. The class initially specifies
\begin{lcode}
\captionstyle{}
\end{lcode}
which gives the block paragraph style. The \cmd{\centerlastline} style 
gives a block paragraph\index{paragraph!block} but with the last 
line centered.


\begin{syntax}
\cmd{\hangcaption} \\
\cmd{\indentcaption}\marg{length} \\
\cmd{\normalcaption} \\
\end{syntax}
 The command \cmd{\hangcaption} will cause captions to be typeset 
with the second
 and later lines of a multiline caption title indented by the width
 of the caption name. The command \cmd{\indentcaption}
 will indent title lines after the first by \meta{length}. These
 commands are independent of the |\captionstyle{...}|. Note that a
 caption will not be simultaneously hung and indented. The \cmd{\normalcaption}
 command undoes any previous \cmd{\hangcaption} or \cmd{\indentcaption} command.
 The class initially specifies \cmd{\normalcaption}
 to give the normal non-indented paragraph\index{paragraph!indentation} style.

\begin{syntax}
\cmd{\changecaptionwidth} \\
\cmd{\captionwidth}\marg{length} \\
\cmd{\normalcaptionwidth} \\
\end{syntax}
 Issuing the command \cmd{\changecaptionwidth} will cause the captions to
 be typeset within a total width \meta{length} as specified by
 \cmd{\captionwidth}. Issuing the command \cmd{\normalcaptionwidth}
 will cause captions to be typeset as normal full width captions.
 The class initially specifies \cmd{\normalcaptionwidth} and 
 |\captionwidth{\linewidth}|
 to give the normal width. If a caption is being set within the 
 side captioned environments from the \Lpack{sidecap} package~\cite{SIDECAP}
 then it must be a \cmd{\normalcaptionwidth} caption.

\begin{syntax}
\cmd{\precaption}\marg{pretext} \\
\cmd{\postcaption}\marg{posttext} \\
\end{syntax}
  The commands \cmd{\precaption} and 
 \cmd{\postcaption}
 specify \meta{pretext} and \meta{posttext} that will be processed at the
 start and end of a caption. For example 
\begin{lcode}
\precaption{\rule{\linewidth}{0.4pt}\par}
\postcaption{\rule{\linewidth}{0.4pt}}
\end{lcode}
  will draw a horizontal line above and below the captions.
 The class initially specifies |\precaption{}| and |\postcaption{}|
 to give the normal appearance.

 
    If any of the above commands are used in a float\index{float}, or other, environment
 their effect is limited to the environment. If they are used in the preamble\index{preamble}
 or the main text, their effect persists until replaced by a similar
 command with a different parameter value. The commands do not affect the
 apperance of the title in any \listofx.

\begin{syntax}
\cmd{\\}\oarg{length} \\
\cmd{\\*}\oarg{length} \\
\end{syntax}
 The normal LaTeX command \cmd{\\} can be used within the
 caption text to start a new line. Remember that \cmd{\\} is a fragile 
 command, so if it is
 used within text that will be added to a \listofx{}
 it must be protected.
 As examples: 
\begin{lcode}
\caption{Title with a \protect\\ new line in both the body and List of}
\caption[List of entry with no new line]{Title with a \\ new line}
\caption[List of entry with a \protect\\ new line]{Title text}
\end{lcode}

 Effectively, a caption is typeset as though it were:
 \begin{lcode}
 \precaption
 {\captionnamefont NAME NUMBER \captiondelim}
 {\captionstyle\captiontitlefont THE TITLE}
 \postcaption
 \end{lcode}
 Replacing the above commands by their defaults leads to the simple
 format: \\
 |{NAME NUMBER: }{THE TITLE}|

 As well as using the styling commands to make simple changes to the
 captioning style more noticeable modifications can also be made.
 To change the captioning style so that the name and title are typeset in
 a sans font it is sufficient to do:
 \begin{lcode}
 \captionnamefont{\sffamily}
 \captiontitlefont{\sffamily}
 \end{lcode}

 \begin{table}
 \centering
 \captionnamefont{\sffamily}
 \captiondelim{}
 \captionstyle{\\}
 \captiontitlefont{\scshape}
 \setlength{\belowcaptionskip}{10pt}
 \caption{Redesigned table caption style} \label{tab:style}
 \begin{tabular}{lr} \hline
  three & III \\
  five  & V \\
  eight & VIII \\ \hline
  \end{tabular}
 \end{table}

 A more obvious change in styling is shown in \tref{tab:style},
 which was coded as:
 \begin{lcode}
 \begin{table}
 \centering
 \captionnamefont{\sffamily}
 \captiondelim{}
 \captionstyle{\\}
 \captiontitlefont{\scshape}
 \setlength{\belowcaptionskip}{10pt}
 \caption{Redesigned table caption style} \label{tab:style}
 \begin{tabular}{lr} \hline
  ...
 \end{table}
 \end{lcode}
 This leads to the approximate caption format (processed within |\centering|): \\
 |{\sffamily NAME NUMBER}{\\ \scshape THE TITLE}| \\
 Note that the newline command (\cmd{\\}) cannot be put in the first part
 of the format (i.e., the |{\sffamily NAME NUMBER}|); it has to go into
 the second part, which is why it is specified via |\captionstyle{\\}|
 and not |\captiondelim{\\}|.

    If a mixture of captioning styles will be used you may want to
 define a special caption command for each non-standard style. For
 example for the style of the caption in \tref{tab:style}:
 \begin{lcode}
 \newcommand{\mycaption}[2][\@empty]{
   \captionnamefont{\sffamily\hfill}
   \captiondelim{\hfill}
   \captionstyle{\centerlastline\\}
   \captiontitlefont{\scshape}
   \setlength{\belowcaptionskip}{10pt}
   \ifx #1\@empty \caption{#2}\else \caption[#1]{#2}}
 \end{lcode}
 \textbf{NOTE:}  Any code that involves the |@| sign must be either in
 a package (\file{.sty}) file or enclosed between a \cmd{\makeatletter} \ldots
 \cmd{\makeatother} pairing.

 The code for the \tref{tab:style} example can now be written as:
 \begin{lcode}
 \begin{table}
 \centering
 \mycaption{Redesigned table caption style} \label{tab:style}
 \begin{tabular}{lr} \hline
  ...
 \end{table}
 \end{lcode}
 Note that in the code for |\mycaption| I have added two
 \cmd{\hfill} commands and \cmd{\centerlastline} compared with the original
 specification.
 It turned out that the original definitions
 worked for a single line caption but not for a multiline caption.
 The additional commands makes it work in both cases, forcing the
 name to be centered as well as the last line of a multiline title,
 thus giving a balanced appearence.

 \index{caption!style|)}


 \subsection{Continuation captions and legends}

\index{caption!continuation|(}

\begin{syntax}
\cmd{\contcaption}\marg{text} \\
\end{syntax}
    The \cmd{\contcaption} command can be used to put 
 a `continuation' or `concluded'
 caption into a float\index{float} environment. It neither increments the
 float number nor makes any entry into a float listing, but it
 does repeat the numbering of the previous \cmd{\caption} command.
 

   Table~\ref{tab:m} illustrates the use of the \cmd{\contcaption}
 command. The table\index{table} was produced from the following code.
 \begin{lcode}
   \begin{table}
   \centering
   \caption{A multi-part table} \label{tab:m}
   \begin{tabular}{lc} \hline
    just a single line & 1 \\ \hline
   \end{tabular}
   \end{table}

   \begin{table}
   \centering
   \contcaption{Continued}
   \begin{tabular}{lc} \hline
    just a single line & 2 \\ \hline
   \end{tabular}
   \end{table}

   \begin{table}
   \centering
   \contcaption{Concluded}
   \begin{tabular}{lc} \hline
    just a single line & 3 \\ \hline
   \end{tabular}
   \end{table}
 \end{lcode}

   \begin{table}
   \centering
   \caption{A multi-part table} \label{tab:m}
   \begin{tabular}{lc} \hline
    just a single line & 1 \\ \hline
   \end{tabular}
   \end{table}

   \begin{table}
   \centering
   \contcaption{Continued}
   \begin{tabular}{lc} \hline
    just a single line & 2 \\ \hline
   \end{tabular}
   \end{table}

   \begin{table}
   \centering
   \contcaption{Concluded}
   \begin{tabular}{lc} \hline
    just a single line & 3 \\ \hline
   \end{tabular}
   \end{table}

\index{caption!continuation|)}

\index{legend}
\index{caption!anonymous|(}

\begin{syntax}
\cmd{\legend}\marg{text} \\
\end{syntax}
  The \cmd{\legend} command is intended to be used to put an 
 anonymous 
  caption into a float\index{float} environment, but may be used anywhere.

   \begin{table}
   \centering
   \caption{Another table} \label{tab:legend}
   \begin{tabular}{lc} \hline
    A legendary table & 5 \\
    with two lines    & 6 \\ \hline
   \end{tabular}
   \legend{The legend}
   \end{table}

    For example, the following code was used to produce the two-line
 \tref{tab:legend}. The \cmd{\legend} command can be used within a 
 float\index{float}
 independently of any \cmd{\caption} command.
 \begin{lcode}
   \begin{table}
   \centering
   \caption{Another table} \label{tab:legend}
   \begin{tabular}{lc} \hline
    A legendary table & 5 \\
    with two lines    & 6 \\ \hline
   \end{tabular}
   \legend{The legend}
   \end{table}
 \end{lcode}

     \marginpar{\legend{Title legend}
                This is a marginal note with a legend.}

  Captioned floats\index{float} are usually thought of in terms of the |table|
  and |figure|\index{figure} environments. There can be other kinds of 
float\index{float}.
  As perhaps a more interesting example, the following code produces
  the titled marginal\index{marginalia} note which should be displayed near here.
 \begin{lcode}
     \marginpar{\legend{Title legend}
                This is a marginal note with a legend.}
 \end{lcode}

 If you want the legend text to be included in the \listofx{}
 use the \cmd{\addcontentsline} command in conjunction with the 
 \cmd{\legend}. For example:
 \begin{lcode}
 \addcontentsline{lot}{table}{Titling text} % left justifified
 \addcontentsline{lot}{table}{\protect\numberline{}Titling text} % indented
 \end{lcode}
 The first of these forms will align the first line of the legend text
 under the normal table\index{table} numbers. The second form will align the first
 line of the legend text under the normal table\index{table} titles. In either case,
 second and later lines of a multi-line text will be aligned under
 the normal title lines.

   \begin{table}
   \centering
   \captiontitlefont{\sffamily}
   \legend{Legendary table}
   \addcontentsline{lot}{table}{Legendary table (toc 1)}
   \addcontentsline{lot}{table}{\protect\numberline{}
                                Legendary table (toc 2)}
   \begin{tabular}{lc} \hline
    An anonymous table & 5 \\
    with two lines     & 6 \\ \hline
   \end{tabular}
   \end{table}

 As an example, the \textsf{Legendary table} is produced by the following code:
 \begin{lcode}
   \begin{table}
   \centering
   \captiontitlefont{\sffamily}
   \legend{Legendary table}
   \addcontentsline{lot}{table}{Legendary table (toc 1)}
   \addcontentsline{lot}{table}{\protect\numberline{}Legendary table (toc 2)}
   \begin{tabular}{lc} \hline
    An anonymous table & 5 \\
    with two lines     & 6 \\ \hline
   \end{tabular}
   \end{table}
 \end{lcode}
 Look at the List of Tables to see how the two forms of \cmd{\addcontentsline}
 are typeset.

    As with the \cmd{\caption} command, the spacing before and after
 a legend is controlled by the \lnc{\abovecaptionskip} and 
\lnc{\belowcaptionskip} lengths. 

\begin{syntax}
\cmd{\namedlegend}\oarg{short-title}\marg{long-title} \\
\end{syntax}
 As a convenience, the \cmd{\namedlegend} 
 command is like the \cmd{\caption} command except that it does not number
 the caption and, by default, puts no entry into a \listofx{} file. Like
 the \cmd{\caption} command, it picks up the name to be prepended to the
 title text from the float\index{float} environment in which it is called (e.g.,
 it will use \cmd{\tablename} if called within a |table| environment). The
 following code is the source of the \textit{Named legendary table}.
 \begin{lcode}
 \begin{table}
 \centering
 \captionnamefont{\sffamily}
 \captiontitlefont{\itshape}
 \namedlegend{Named legendary table}
 \begin{tabular}{lr} \hline
 seven & VII \\
 eight & VIII \\ \hline
 \end{tabular}
 \end{table}
 \end{lcode}

 \begin{table}
 \centering
 \captionnamefont{\sffamily}
 \captiontitlefont{\itshape}
 \namedlegend{Named legendary table}
 \begin{tabular}{lr} \hline
 seven & VII \\
 eight & VIII \\ \hline
 \end{tabular}
 \end{table}

\begin{syntax}
\cmd{\flegtype}\marg{name} \\
\cmd{\flegtoctype}\marg{title} \\
\end{syntax}
 The macro \cmd{\flegtype}, where |type| is the name 
 of a float\index{float} environment
 (e.g., |table|) is called by the \cmd{\namedlegend} macro. It is provided
 as a hook that defines the \meta{name} to be used as the name in
 \cmd{\namedlegend}. Two defaults are provided, namely:
 \begin{lcode}
 \newcommand{\flegtable}{\tablename}
 \newcommand{\flegfigure}{\figurename}
 \end{lcode}
 which may be altered via \cmd{\renewcommand} if desired. 
 The macro \cmd{\flegtoctype}, where again |type| is the name 
 of a float\index{float} environment
 (e.g., |table|) is also called by the \cmd{\namedlegend} macro. It is provided
 as a hook that can be used to add \meta{title} to the \listofx.
 By default it is defined to do nothing, and can be changed via
\cmd{\renewcommand}. For instance, it could be changed for tables\index{table} as:
 \begin{lcode}
 \renewcommand{\flegtoctable}[1]{
   \addcontentsline{lot}{table}{#1}}
 \end{lcode}

  The \cmd{\legend} command produces a plain, unnumbered heading. It can also
 be useful sometimes to have named and numbered captions outside
 a floating\index{float} environment, perhaps in a |minipage| if you want the table\index{table}
 or picture\index{illustration} to appear at a precise location in your document.

\index{caption!anonymous|)}

\index{caption!fixed|(}

\begin{syntax}
\cmd{\newfixedcaption}\oarg{capcommand}\marg{command}\marg{float} \\
\cmd{\renewfixedcaption}\oarg{capcommand}\marg{command}\marg{float} \\
\cmd{\providefixedcaption}\oarg{capcommand}\marg{command}\marg{float} \\
\end{syntax}
 The \cmd{\newfixedcaption} 
 command, and its friends, can be used to create a new captioning
 \meta{command} that may be used outside the float\index{float} environment \meta{float}.
 Both the environment \meta{float} and a captioning command, 
 \meta{capcommand}, for that environment must have been defined before
 calling \cmd{\newfixedcaption}. Note that \cmd{\namedlegend} can be used
 as \meta{capcommand}.
 The \cmd{\renewfixedcaption} and \cmd{\providefixedcaption} commands take the same 
 arguments as \cmd{\newfixedcaption}; the three commands are analagous
 to those in the \cmd{\newcommand} family.

 For example, to define a new |\figcaption| command for captioning pictures
 outside the |figure|\index{figure} environment, do\\
 |\newfixedcaption{\figcaption}{figure}| \\
 The optional \meta{capcommand} argument is the name of the float\index{float}
 captioning command that is being aliased. It defaults to \cmd{\caption}.
 As another example, where the optional argument is required, if you
 want to create a new continuation caption command for non-floating
 tables\index{table}, say |\ctabcaption|, then do \\
 |\newfixedcaption[\contcaption]{\ctabcaption}{table}|

 Captioning commands created by \cmd{\newfixedcaption} will be named and
 numbered in the same style as the original \meta{capcommand}, can
 be given a \cmd{\label}, and will appear in the appropriate 
 \listofx. They can also be used within floating\index{float}
 environments, but will not use the environment name as a guide to
 the caption name or entry into the \listofx. For
 example, using |\ctabcaption| in a |figure|\index{figure} environment will still
 produce a \textbf{Table\ldots} named caption.

   Sometimes captions are required on the opposite page to a figure\index{figure}, and
 \cmd{\newfixedcaption} can be useful in this context. For example, if 
figure\index{figure}
 captions should be placed on an otherwise empty page immediately before
 the actual figure\index{figure}, then this can be accomplished by the following hack:
 \begin{lcode}
 \newfixedcaption{\figcaption}{figure}
  ...
 \afterpage{ % fill current page then flush pending floats
   \clearpage
   \begin{midpage}  %  vertically center the caption
   \figcaption{The caption} %  the caption
   \end{midpage}
   \clearpage
   \begin{figure}THE FIGURE, NO CAPTION HERE\end{figure}
   \clearpage
 }  % end of \afterpage
 \end{lcode}
 Note that the \Lpack{afterpage} package~\cite{AFTERPAGE} is required, 
which is part of the
 required tools bundle. The \Lpack{midpage} package supplies the |midpage|
 environment, which can be simply defined as:
 \begin{lcode}
 \newenvironment{midpage}{\vspace*{\fill}}{\vspace*{\fill}}
 \end{lcode}
 The code might need adjusting to meet your particular requirements.
 The \cmd{\cleartoevenpage} command ensures that you get to the next
 even-numbered page (the \cmd{\cleardoublepage} gets you to the next odd-numbered
 page and \cmd{\clearpage} gets you to the next page which may be odd or even).

\index{caption!fixed|)}

 \subsection{Bilingual captions}

\index{caption!bilingual|(}

    Some documents require bilingual (or more) captions. The class 
 provides a set of commands for bilingual captions. Extensions to the
 set, perhaps to support trilingual captioning, are left as an exercise
 for the document author.

\begin{syntax}
\cmd{\bitwonumcaption}\oarg{label}\marg{short1}\marg{long1}\marg{NAME}\marg{short2}\marg{long2} \\
\cmd{\bionenumcaption}\oarg{label}\marg{short1}\marg{long1}\marg{NAME}\marg{short2}\marg{long2} \\
\end{syntax}
  Bilingual captions can be typeset by the \cmd{\bitwonumcaption} 
 command which has six arguments. 
 The first, optional argument \meta{label}, is the name of a label, if
 required.
 \meta{short1} and \meta{long1} are the short (i.e., equivalent
 to the optional argument
 to the \cmd{\caption} command) and long caption texts for
 the main language of the document. The value of the \meta{NAME} argument
 is used as the caption name for the second language caption, while
 \meta{short2} and \meta{long2} are the short and long caption texts
 for the second language. For example, if the main and secondary languages
 are English and German and a figure\index{figure} is being captioned: \\
 |\bitwonumcaption{Short}{Long}{Bild}{Kurz}{Lang}| \\
  If the short title text(s) is not required, then leave the appropriate
 argument(s) either empty or as one or more spaces, like: \\
 |\bitwonumcaption[fig:bi1]{}{Long}{Bild}{  }{Lang}| \\
 Both language texts are entered into the appropriate \listofx,
 and both texts are numbered.

 Figure~\ref{fig:bi1} is an example of using the above code.
 \begin{figure}
 \centering
 EXAMPLE FIGURE WITH BITWONUMCAPTION
 \bitwonumcaption[fig:bi1]{}{Long \cs{bitwonumcaption}}{Bild}{  }{Lang \cs{bitwonumcaption}} 
 \end{figure}

  The \cmd{\bionenumcaption} command takes the same arguments as 
\cmd{\bitwonumcaption}.
 The difference between the two commands is that \cmd{\bionenumcaption} does
 not number the second language text in the \listofx.
 Figure~\ref{fig:bi3} is an example of using \cmd{\bionenumcaption}.
 \begin{figure}
 \centering
 EXAMPLE FIGURE WITH BIONENUMCAPTION
 \bionenumcaption[fig:bi3]{}{Long English \cs{bionenumcaption}}{Bild}{  }{Lang Deutsch \cs{bionenumcaption}} 
 \end{figure}

\begin{syntax}
\cmd{\bicaption}\oarg{label}\marg{short1}\marg{long1}\marg{NAME}\marg{long2} \\
\end{syntax}
  When bilingual captions are typeset via the \cmd{\bicaption} 
 command the second language text is not put into
 the \listofx. The command
 takes 5 arguments. 
 The optional \meta{label} is for a label if required.
 \meta{short1} and \meta{long1} are the short and long caption texts for
 the main language of the document. The value of the \meta{NAME} argument
 is used as the caption name for the second language caption. The last
 argument, \meta{long2}, is the caption text
 for the second language (which is not put into the \listofx). 
 For example, if the main and secondary languages
 are English and German: \\
 |\bicaption{Short}{Long}{Bild}{Langlauf}| \\
 If the short title text is not required, then leave the appropriate
 argument either empty or as one or more spaces.

 Figure~\ref{fig:bi2} is an example of using \cmd{\bicaption} and was 
 produced by the following code:
 \begin{lcode}
 \begin{figure}
 \centering
 EXAMPLE FIGURE WITH A RULED BICAPTION
 \precaption{\rule{\linewidth}{0.4pt}\par}
 \midbicaption{\precaption{}\postcaption{\rule{\linewidth}{0.4pt}}}
 \bicaption[fig:bi2]{Short English \cs{bicaption}}{Longingly}{Bild}{Langlauf}
 \end{figure}
 \end{lcode}

 \begin{figure}
 \centering
 EXAMPLE FIGURE WITH A RULED BICAPTION
 \precaption{\rule{\linewidth}{0.4pt}\par}
 \midbicaption{\precaption{}\postcaption{\rule{\linewidth}{0.4pt}}}
 \bicaption[fig:bi2]{Short English \cs{bicaption}}{Longingly}{Bild}{Langlauf}
 \end{figure}

\begin{syntax}
\cmd{\bicontcaption}\marg{long1}\marg{NAME}\marg{long2} \\
\end{syntax}
  Bilingual continuation captions can be typeset via the \cmd{\bicontcaption} 
 command. In this case, neither language text is put into
 the \listofx. The command
 takes 3 arguments.
 \meta{long1} is the caption text for
 the main language of the document. The value of the \meta{NAME} argument
 is used as the caption name for the second language caption. The last
 argument, \meta{long2}, is the caption text
 for the second language.
 For example, if the main and secondary languages
 are again English and German: \\
 |\bicontcaption{Continued}{Bild}{Fortgefahren}|

\begin{syntax}
 \cmd{\midbicaption}\marg{text} \\
\end{syntax}
 The bilingual captions are implemented by calling \cmd{\caption} twice,
 once for each language. The command \cmd{\midbicaption}, 
 which is similar to the \cmd{\precaption} and \cmd{\postcaption} commands,
 is executed 
 just before calling the second \cmd{\caption}. Among other things,
 this can be used to
 modify the style of the second caption with respect to the first.
 For example, if there is a line above and below normal
 captions, it is probably undesirable to have a double line in the
 middle of a bilingual caption. So, for bilingual captions the
 following may be done within the float\index{float} before the caption:
 \begin{lcode}
 \precaption{\rule{\linewidth}{0.4pt}\par}
 \postcaption{}
 \midbicaption{\precaption{}\postcaption{\rule{\linewidth}{0.4pt}}}
 \end{lcode}
 This sets a line before the first of the two captions, then the
 \cmd{\midbicaption}{...} nulls the pre-caption line and adds a post-caption
 line for the second caption. The class initially specifies
 |\midbicaption{}|.

\index{caption!bilingual|)}

 \subsection{Subcaptions}

\index{caption!subcaption|(}

     The \Lpack{subfigure} package~\cite{SUBFIGURE} enables the captioning of 
sub-figures\index{figure!sub-} within a larger figure\index{figure}, 
and similarly for tables\index{table}. The \Lpack{subfigure} package may 
be used with the class, or you can use the class commands described below.
These commands can only be used inside a float environment for which a
subfloat\footnote{See \S\ref{sec:newsub} on \pref{sec:newsub}.}
has been specified.

\begin{syntax}
\cmd{\subcaption}\oarg{list-entry}\marg{subcaption} \\
\end{syntax}
The \cmd{\subcaption} command is similar to the \cmd{caption} command. It
typesets an identified \meta{subcaption}, where the identification is
an alphabetic character enclosed in parentheses. If the optional
\meta{list-entry} argument is present, \meta{list-entry} is added to
the caption listings for the float. If it is not present, then 
\meta{subcaption} is added to the listing. The \cmd{\subcaption} macro
should only be used in a fixed width box of some sort, for example
a |minipage| environment, as the \meta{subcaption} will be typeset using
a box which is the width of the environment.

    For example:
\begin{lcode}
\begin{figure}
  \centering
  \begin{minipage}{0.3\textwidth}
    \verb?Some verbatim text?
    \subcaption{First text}
  \end{minipage}
  \hfill
  \begin{minipage}{0.3\textwidth}
    \verb?More verbatim text?
    \subcaption{Second text}
  \end{minipage}
  \caption{Verbatim texts}
\end{figure}
\end{lcode}


If a 
figure that\index{figure}
 includes sub-figures\index{figure!sub-} is itself continued then it may be desireable to
 continue the captioning of the sub-figures. For example, if Figure~3
 has three sub-figures, say A, B and C, and Figure~3 is continued then
 the sub-figures\index{figure!sub-} in the continuation should be D, E, etc.
\begin{syntax}
\cmd{\contsubcaption}\oarg{list-entry}\marg{subcaption} \\
\end{syntax}
The \cmd{\contsubcaption} command is the continued version of 
\cmd{\subcaption}.


\begin{syntax}
\cmd{\label}|(|\meta{bookmark}|)|\marg{key} \\
\cmd{\subcaptionref}\marg{key} \\
\end{syntax}
    A \cmd{\label} command may be included in the \meta{subcaption} argument
of \cmd{\subcaption} or \cmd{\contsubcaption}.
Using \cmd{\ref} to refer to the label will typeset the number of the float
(obtained from a labelled \cmd{\caption}) and the subcaption identifier. On
the other hand, using \cmd{\subcaptionref} will typeset just the subcaption
identifier.

    In case the \Lpack{hyperref} package is used, the \cmd{\label} 
command when used inside a \meta{subcaption} argument can take an optional
argument, \emph{enclosed in parentheses \emph{not} square brackets}, 
\meta{bookmark} which will create a bookmark field of the form
`Subfigure 3.2(c)'.

\begin{syntax}
\cmd{\tightsubcaptions} \cmd{\loosesubcations} \\
\end{syntax}
The \cmd{\tightsubcaptions} declaration will put little vertical space 
around the subcaption, while \cmd{\loosesubcaptions} will add more
whitespace. The default is \cmd{\tightsubcaptions}.

\begin{syntax}
\cmd{\subcaptionsize}\marg{font-size} \\
\cmd{\subcaptionlabelfont}\marg{fontspec} \\
\cmd{\subcaptionfont}\marg{fontspec} \\
\end{syntax}
The size of the font used for subcaptions is set by \cmd{\subcaptionsize}.
The default is:
\begin{lcode}
\subcaptionsize{\footnotesize}
\end{lcode}
The fonts used for the identifier and the subcaption are set by
\cmd{\subcaptionlabelfont} and \cmd{\subcaptionfont}, respectively,
where \meta{fontspec} is one or more font style declarations (e.g., 
|\bfseries\slshape| for a bold, slanted font).
The defaults are:
\begin{lcode}
\subcaptionlabelfont{\normalfont}
\subcaptionfont{\normalfont}
\end{lcode}

\begin{syntax}
\cmd{\subcaptionstyle}\marg{style} \\
\end{syntax}
 By default the identifier and title of a subcaption are typeset as a block 
(non-indented) paragraph\index{paragraph!block}. 
\cmd{\subcaptionstyle} can be used to alter this.
Sensible values for \meta{style} are: \cmd{\centering}, \cmd{\raggedleft} or
\cmd{\raggedright} for styles corresponding to these declarations. 
The \cmd{\centerlastline} style gives a block 
paragraph\index{paragraph!block} but with the last line centered.
The class initially specifies |\subcaptionstyle{}|
to give the normal block paragraph style.

\begin{syntax}
\cmd{\hangsubcaption} \\
\cmd{\shortsubcaption} \\
\cmd{\normalsubcaption} \\
\end{syntax}
 The \cmd{\hangsubcaption} declaration causes subcaptions to be typeset 
with the identifier above the title. With the \cmd{\shortsubcaption}
declaration subcaptions that are less than one line in length are 
typeset left justified instead of centered.
The \cmd{\normalsubcaption} command undoes any previous 
\cmd{\hangsubcaption} or \cmd{\shortsubcaption} command.
The class initially specifies \cmd{\normalcaption}.

\begin{syntax}
 \cmd{\subtop}\oarg{list-entry}\oarg{subcaption}\marg{text} \\
 \cmd{\subbottom}\oarg{list-entry}\oarg{subcaption}\marg{text} \\
\end{syntax}
 The command \cmd{\subtop} puts a subcaption number on top of
the \meta{text}. If both optional arguments are present, \meta{list-entry}
will be added to the appropriate listing, and an alphabetic identifier
and \meta{subcaption} will be placed above \meta{text}. If only one
optional argument is present, this is treated as \meta{subcaption}; the
identifier and \meta{subcaption} are placed above \meta{text} and
\meta{subcaption} is added to the listing. In all cases, the identifier
is placed above \meta{text} and added to the listing.

    The \cmd{\subbottom} command is similar, except that the identifier
and any \meta{subcaption} is placed below the \meta{text}.

    The main caption can be at either the top or the bottom of the float.

    For example:
\begin{lcode}
\begin{figure}
\subbottom{...} %  captioned as (a) below
\subtop{...}    %  captioned as (b) above
\caption{...}
\end{figure}
\end{lcode}

    As another example, \fref{subfig:sf} and the next paragraph
was produced by the code below.

\begin{figure}
\centering
\subbottom[Subfigure 1]{\fbox{SUBFIGURE ONE}\label{sf:1}}
\hfill
\subbottom[Subfigure 2]{\fbox{SUBFIGURE TWO}\label{sf:2}}
\caption{Figure with two subfigures} \label{subfig:sf}
\end{figure}

    Figure \ref{subfig:sf} has two subfigures, namely \ref{sf:1}
and \subcaptionref{sf:2}.

\begin{lcode}
    Figure \ref{subfig:sf} has two subfigures, namely \ref{sf:1}
and \subcaptionref{sf:2}.
\begin{figure}
\centering
\subbottom[Subfigure 1]{\fbox{SUBFIGURE ONE}\label{sf:1}}
\hfill
\subbottom[Subfigure 2]{\fbox{SUBFIGURE TWO}\label{sf:2}}
\caption{Figure with two subfigures} \label{subfig:sf}
\end{figure}
\end{lcode}


    The major difference between the \cmd{\subcaption} command and the
\cmd{\subtop} and \cmd{\subbottom} commands, apart from the \meta{text}
argument, is that \cmd{\subcaption}
must be used in a fixed width environment while \cmd{\subtop} uses
the width of \meta{text} for the box in which to typeset the
\meta{subcaption}.

\begin{syntax}
 \cmd{\contsubtop}\oarg{list-entry}\oarg{subcaption}\marg{text} \\
 \cmd{\contsubbottom}\oarg{list-entry}\oarg{subcaption}\marg{text} \\
\end{syntax}
  The command \cmd{\contsubtop} 
 will continue the sub-caption numbering scheme across (continued) floats\index{float},
 putting the \meta{subcaption} at the top of the \meta{text}.
 The \cmd{\contsubbottom} command is similar but puts the \meta{subcaption}
 at the bottom of the \meta{text}. In either case, the main caption can 
 be at the top or bottom of the float\index{float}.

    For example:
\begin{lcode}
 \begin{table}
 \caption{...}
 \subtop{...}  % captioned as (a) above
 \subtop{...}  % captioned as (b) above
 \end{table}
 ...
 \begin{table}
 \contcaption{Concluded}
 \contsubtop{...} % captioned as (c) above
 \contsubtop{...} % captioned as (d) above
 \end{table}
 \end{lcode}

    As with the \cmd{\subcaption} command, a \cmd{\label} command may be
used in either the \meta{subcaption} or the \meta{text} arguments to
\cmd{\subtop} and friends.


\index{caption!subcaption|)}

 \subsection{How LaTeX makes captions} \label{sec:ltx}

\index{caption!LaTeX methods|(}

 This section provides an overview of how LaTeX creates captions and
 gives some examples of how to change the captioning style without
 having to use any package. 
 The section need not be looked at more than once unless you like 
 reading LaTeX code
 or you want to make changes to LaTeX's style of captioning.

 The LaTeX kernel provides tools to help in the definition of captions,
 but it is the particular class that decides on their format.

\begin{syntax}
 \cmd{\caption}\oarg{short}\marg{long} \\
\end{syntax}
 The kernel (in \file{ltfloat.dtx}) defines the caption command via 
\begin{lcode}
\def\caption{\refstepcounter\@captype \@dblarg{\@caption\@captype}}
\end{lcode}

\begin{syntax}
 \cmd{\@captype} \\
\end{syntax}
 \cmd{\@captype} is defined by the code that creates a new float\index{float} environment
 and is set to the environment's name (see the code for |\@xfloat|
 in \file{ltfloat.dtx}). For a |figure| environment,
 there is an equivalent to 
\begin{lcode}
\def\@captype{figure}
\end{lcode}

\begin{syntax}
 \cmd{\@caption}\marg{type}\oarg{short}\marg{long} \\
\end{syntax}
 The kernel also provides the \cmd{\@caption} macro as:
\begin{lcode}
 \long\def\@caption#1[#2]#3{
   \par
   \addcontentsline{\csname ext@#1\endcsname}{#1}    <--
     {\protect\numberline{\csname the#1\endcsname}{\ignorespaces #2}}
   \begingroup
      \@parboxrestore
      \if@minipage
        \@setminipage
      \fi
      \normalsize
      \@makecaption{\csname fnum@#1\endcsname}{\ignorespaces #3}\par  <--
   \endgroup}
 \end{lcode}
 where \meta{type} is the name of the environment in which the caption
 will be used.
 Putting these three commands together results in the user's view of the caption
 command as |\caption[|\meta{short}|]{|\meta{long}|}|.

 It is the responsibilty of the class (or package) which defines floats\index{float}
 to provide definitions for \cmd{\ext@type}, \cmd{\fnum@type} and 
\cmd{\@makecaption}
 which appear in the definition of \cmd{\@caption} (in the lines marked
 |<--| above).

\begin{syntax}
 \cmd{\ext@type} \\
\end{syntax}
 This macro holds the name of the extension for a \listofx{} file.
 For example for the |figure| float\index{float} environment there is the
 definition equivalent to 
\begin{lcode}
 \newcommand{\ext@figure}{lof}
\end{lcode}

\begin{syntax}
 \cmd{\fnum@type} \\
\end{syntax}
 This macro is responsible for typesetting the caption number. For example,
 for the |figure| environment there is the definition equivalent to
\begin{lcode}
 \newcommand{\fnum@figure}{\figurename~\thefigure}
\end{lcode}

\begin{syntax}
 \cmd{\@makecaption}\marg{number}\marg{text} \\
\end{syntax}
 The \cmd{\@makecaption} macro, where \meta{number}
 is a string such as `Table~5.3' and \meta{text} is the caption text,
 performs the typesetting of the caption, and
 is defined in the standard classes (in \file{classes.dtx}) as the
 equivalent of:
\begin{lcode}
 \newcommand{\@makecaption}[2]{
   \vskip\abovecaptionskip         <- 1
   \sbox\@tempboxa{#1: #2}        <- 2
   \ifdim \wd\@tempboxa >\hsize
     #1: #2\par                    <- 3
   \else
     \global \@minipagefalse
     \hb@xt@\hsize{\hfil\box\@tempboxa\hfil}
   \fi
   \vskip\belowcaptionskip}        <- 4
\end{lcode}

\begin{syntax}
 \lnc{\abovecaptionskip}
 \lnc{\belowcaptionskip} \\
\end{syntax}
  Vertical space is added before and after a caption (lines marked 1 and 4
 in the code for \cmd{\@makecaption} above) and the amount of space is given
 by the lengths \lnc{\abovecaptionskip} and \lnc{\belowcaptionskip}. The
 standard classes set these to 10pt and 0pt respectively. If you want
 to change the space before or after a caption, use \cmd{\setlength} to change
 the values. In figures\index{figure}, the caption is usually placed below the
 illustration\index{illustration}. The actual space between the bottom of the illustration\index{illustration}
 and the baseline of the first line of the caption
 is the \lnc{\abovecaptionskip} plus the \lnc{\parskip} plus the \lnc{\baselineskip}.
 If the illustration\index{illustration} is in a |center| environment then additional space
 will be added by the |\end{center}|; it is usually better to use 
 the \cmd{\centering} command rather than the |center| environment.

 The actual typesetting of a caption is effectively performed by the code
 in lines marked 2 and 3 in the code for \cmd{\@makecaption}; note that
 these are where the colon that is typeset after the number is specified. 
 If you want to
 make complex changes to the default captioning style you may have to
 create your own version of \cmd{\@caption} using 
 \cmd{\renewcommand}. On the other hand, many such changes can be achieved
 by changing the definition of the 
 the appropriate \cmd{\fnum@type} command(s). For example, to make the 
 figure\index{figure} name and number bold: 
\begin{lcode}
\renewcommand{\fnum@figure}{\textbf{\figurename~\thefigure}}
\end{lcode}

 REMEMBER: If you are doing anything involving commands that include
 the |@| character, and it's not in a class or package file, you have
 to do it within a \cmd{\makeatletter} and \cmd{\makeatother} pairing. So,
 if you modify the \cmd{\fnum@figure} command anywhere in your document
 it has to be done as:
\begin{lcode}
 \makeatletter
 \renewcommand{\fnum@figure}{......}
 \makeatother
\end{lcode}

 \makeatletter
 \renewcommand{\fnum@figure}{\textsc{\figurename~\thefigure}}
 \makeatother
 \begin{figure}
 \centering
 A THOUSAND WORDS\ldots
 \caption{A picture is worth a thousand words}\label{fig:sc}
 \end{figure}

 As an example, \fref{fig:sc} was created by the following code:
\begin{lcode}
 \makeatletter
 \renewcommand{\fnum@figure}{\textsc{\figurename~\thefigure}}
 \makeatother
 \begin{figure}
 \centering
 A THOUSAND WORDS\ldots
 \caption{A picture is worth a thousand words}\label{fig:sc}
 \end{figure}
\end{lcode}

 As another example, suppose that you needed to typeset the \cmd{\figurename}
 and its number in a bold font, replace the colon that normally appears
 after the number by a long dash, and typeset the actual title text in
 a sans-serif font, as is illustrated by the caption for 
 \fref{fig:sf}. The following code does this.

 \makeatletter
 \renewcommand{\fnum@figure}[1]{\textbf{\figurename~\thefigure} --- \sffamily}
 \makeatother
 \begin{figure}
  \centering
  ANOTHER THOUSAND WORDS\ldots
 \caption{A different kind of figure caption}\label{fig:sf}
 \end{figure}

\begin{lcode}
 \makeatletter
 \renewcommand{\fnum@figure}[1]{\textbf{\figurename~\thefigure} 
                                --- \sffamily}
 \makeatother
 \begin{figure}
  \centering
  ANOTHER THOUSAND WORDS\ldots
 \caption{A different kind of figure caption}\label{fig:sf}
 \end{figure}
\end{lcode}
 Perhaps a little description of how this works is in order.
 Doing a little bit of TeX 's macro processing by hand, the typesetting
 lines in \cmd{\@makecaption} (lines 2 and 3) get instantiated like: \\
 |\fnum@figure{\figurename~\thefigure}: text| \\
 Redefining \cmd{\fnum@figure} to take one argument and then not using the
 value of the argument essentially gobbles up the colon. Using \\
 |\textbf{\figurename~\thefigure}| \\
 in the definition causes \cmd{\figurename} and the number to be typeset in
 a bold font. After this comes the long dash. Finally, putting \cmd{\sffamily}
 at the end of the redefinition causes any following text (i.e., the actual 
 title) to be typeset using the sans-serif font.

 If you do modify \cmd{\@makecaption}, then spaces in the definition may be
 important; also you must use the comment (\%) character in the same
 places as I have done above. Hopefully, though, the class provides the
tools that you need to make most, if not all, of any likely caption styles.

\index{caption!LaTeX methods|)}


 \subsection{Captions with footnotes}

\index{caption!footnote|(}
\index{footnote!in caption|(}

    If you want to have a caption with a footnote, think long and hard
 as to whether this is really essential. It is not normally considered
 to be good typographic practice, and to rub the point in LaTeX does not
 make it necessarily easy to do. However, if you (or your publisher)
 insists, read on.

    If it is present, the optional argument to \cmd{\caption} is put into
 the \lof/lot{} as appropriate. If the argument is not present, then the
 text of the required argument is put into the \lof. In the first case,
 the optional argument is moving, and in the second case the required
 argument is moving. The \cmd{\footnote} command is fragile and must be
 |\protect|ed (i.e., |\protect\footnote{}|) if it is used in a moving 
 argument. If you don't want the footnote to appear in 
the \lof, use a
 footnoteless optional argument and a footnoted required argument.

   You will probably be surprised if you just do, for example:
 \begin{lcode}
 \begin{figure}
 ...
 \caption[For LoF]{For figure\footnote{The footnote}}
 \end{figure}
 \end{lcode}
 because (a) the footnote number may be greater than you thought, and (b)
 the footnote text has vanished. This latter is because LaTeX won't typeset
 footnotes from a float\index{float}. To get an actual footnote within the float you
 have to use a minipage, like:
 \begin{lcode}
 \begin{figure}
 \begin{minipage}{\linewidth}
 ...
 \caption[For LoF]{For figure\footnote{The footnote}}
 \end{minipage}
 \end{figure}
 \end{lcode}
 If you are using the standard classes you may now find that you get two 
footnotes for the price of one. Fortunately this will not occur with this 
class, nor will an unexpected increase of the footnote number.

    When using a minipage as above, the footnote text is typeset at the
 bottom of the minipage (i.e., within the float\index{float}). If you want the footnote
 text typeset at the bottom of the page, then you have to use the
 \cmd{\footnotemark} and \cmd{\footnotetext} commands like:
 \begin{lcode}
 \begin{figure}
 ...
 \caption[For LoF]{For figure\footnotemark}
 \end{figure}
 \footnotetext{The footnote}
 \end{lcode}
 This will typeset the argument of the \cmd{\footnotetext} command at the
 bottom of the page where you called the command. Of course, the figure\index{figure}
 might have floated\index{float} to a later page, and then it's a matter of some
 manual fiddling to get everything on the same page, and possibly
 to get the footnote marks\index{footnote!mark} to match correctly with the footnote text.

 At this point, you are on your own.

\index{footnote!in caption|)}
\index{caption!footnote|)}


\index{caption|)}

\section{Floats}


 \subsection{New float environments}

\index{float!new|(}

\begin{syntax}
 \cmd{\newfloat}\oarg{within}\marg{fenv}\marg{ext}\marg{capname} \\
\end{syntax}
 The \cmd{\newfloat} command 
 creates two new floating environments called \meta{fenv} and \meta{fenv*}. 
If there
is not already a counter defined for \meta{fenv} a new one will be created
to be restarted by \meta{within}, if that is specified.
 A caption within
 the environment will be written out to a file with extension \meta{ext}.
 The caption, if present, will start with \meta{capname}. For example, 
the |figure| float for the class is defined as:
\begin{lcode}
\newfloat[chapter]{figure}{lof}{\figurename}
\renewcommmand{\thefigure}{%
  \ifnum\c@chapter>\z@ \thechapter.\fi \@arabic\c@figure}
\end{lcode}
The last bit of the definition is internal code to make sure that if a
figure\index{figure} is in the document before chapter numbering starts, then the figure\index{figure}
number will not be preceded by a non-existent chapter number.


 The captioning style for floats defined with \cmd{\newfloat} is the same as
 for the figures\index{figure} and tables\index{table}.

    The \cmd{\newfloat} command generates several new commands, some of
which are internal LaTeX commands. For convenience, assume that the command
was called as 
\begin{lcode}
\newfloat{X}{Z}{capname}
\end{lcode}
 so |X| is the name of the float
environment and also the name of the counter for the caption, and |Z| is
the file extension.
The following float environment and related commands are then created.

\begin{syntax}
\senv{X} float material \eenv{X} \\
\senv{X*} float material \eenv{X*} \\
\end{syntax}
 The new float environment is called |X|, and can be used as either
 |\begin{X}| or |\begin{X*}|, with the matching |\end{X}| or |\end{X*}|.
It is given the standard default position specification of |tbp|.

\begin{syntax}
 \Icn{Zdepth} \\
\end{syntax}
 The \Icn{Zdepth} counter is analogous to the standard \Icn{tocdepth} counter
 in that it specifies that entries in a listing should not be
 typeset if their numbering level is greater than \Icn{Zdepth}. The
 default definition is |\setcounter{Zdepth}{1}|. To have a subfloat
 of |Z| appear in the listing do |\setcounter{Zdepth}{2}|.


    As a fuller example, suppose you wanted both 
 figures\index{figure} (which come with the
 class), and diagrams. You could then do something like the following.
 \begin{lcode}
 \newcommand{\diagramname}{Diagram}
 \newcommand{\listdiagramname}{List of Diagrams}
 \newlistof{listofdiagrams}{dgm}{\listdiagramname}
 \newfloat{diagram}{dgm}{\diagramname}
 \newfixedcaption{\fdiagcaption}{diagram}
 \newlistentry{diagram}{dgm}{0}
 \begin{document}
 ...
 \listoffigures
 \listfofdiagrams
 ...
 \begin{diagram}
 \caption{A diagram} \label{diag1}
 ...
 \end{diagram}
 As diagram~\ref{diag1} shows ...
 \begin{minipage}{.9\textwidth}
 \fdiagcaption{Another diagram} \label{diag2}
 ...
 \end{minipage}

 In contrast to diagram~\ref{diag1}, diagram~\ref{diag2} provides ...
 \end{lcode}

    As a word of warning, if you mix both floats and fixed environments with the
 same kind of caption you have to ensure that they get printed in the correct
 order in the final document. If you do not do this, then the \listofx{} 
 captions will come out in the wrong order (the lists are ordered according the
 page number in the typeset document, \emph{not} your source input order).

\index{float!new|)}


\subsection{New subfloats} \label{sec:newsub}

\index{subfloat!new|(}

    The \Lpack{subfigure} package defines the |subfigure| and |subtable|
subfloats. The class does not define any subfloats, if you need them you have
to specify them yourself.

\begin{syntax}
 \cmd{\newsubfloat}\marg{fenv} \\
\end{syntax}
 The \cmd{\newsubfloat} command enables the subcaptioning commands
(\cmd{\subcaption}, \cmd{\subtop}, etc.) to be used within the float 
environment \meta{fenv} previously defined via 
|\newfloat|.

    Calling |\newsubfloat{fenv}| will, among other things,
create a new counter called |subfenv| and the command |\thesubfenv| 
for typesetting the counter. The default definition of |\thesubfenv|
is equivalent to:
\begin{lcode}
\newcommand{\thesubfenv}{(\alph{subfenv})}
\end{lcode}
which typesets a lowercase letter enclosed in parentheses. This is
the identifier for subcaptions, and may be changed via \cmd{\renewcommand}.

    If you are not using the \Lpack{subfigure} package and want, say, 
subfigures then in the preamble call:
\begin{lcode}
\newsubfloat{figure}
\end{lcode}
and if you want the subcaptions to appear in the List of Figures, put:
\begin{lcode}
\setcounter{lofdepth}{2}
\end{lcode}
between the |\begin{document}| and \cmd{\listoffigures} commands.

    Further, if you had some subfigures that were originally planned
for use with the \Lpack{subfigure} package and wanted to use these
but without the package, you could:
\begin{lcode}
\let\subfigure\subbottom
\end{lcode}
which will make |\subfigure| an alias for \cmd{\subbottom}.

\index{subfloat!new|)}


 \subsection{Multiple floats}

\index{float!multiple|(}

     As far as LaTeX is concerned, a float is a box with certain 
 restrictions as to where it can be placed. You can effectively 
 put what you like inside a float box. Normally there is just a single
 picture or tabular in a float but you can put as many of these as will
 fit inside the box.

 \begin{figure}
 \centering
 \hspace*{\fill} 
   {ILLUSTRATION 1} \hfill {ILLUSTRATION 2} 
 \hspace*{\fill}
 \caption{Float with two illustrations} \label{fig:mult1}
 \end{figure}

    Three typical cases of multiple figures\index{figure}/tables\index{table} in a single
 float come to mind:
 \begin{itemize}
 \item Multiple illustrations\index{illustration}/tabulars with a single caption.
 \item Multiple illustrations\index{illustration}/tabulars each individually captioned.
 \item Multiple illustrations\index{illustration}/tabulars with one main caption and
       individual subcaptions.
 \end{itemize}

    The \Lpack{subfigure} package is designed for the last of these cases;
 the others do not require a package.

    Figure~\ref{fig:mult1} is an example of multiple illustrations\index{illustration} 
 in a single float with a single caption.
     This figure was produced by the following code.
 \begin{lcode}
 \begin{figure}
 \centering
 \hspace*{\fill} 
   {ILLUSTRATION 1} \hfill {ILLUSTRATION 2} 
 \hspace*{\fill}
 \caption{Float with two illustrations} \label{fig:mult1}
 \end{figure}
 \end{lcode}
 The |\hspace*{\fill}| and |\hfill| commands were used to space the two
 illustrations\index{illustration} equally. Of course |\includegraphics| or |tabular|
 environments could just as well
 be used instead of the |{ILLUSTRATION N}| text.

    The following code produces \figurerefname s~\ref{fig:mult2} and~\ref{fig:mult3}
 which are examples of two separately captioned illustrations\index{illustration} in one
 float.
 \begin{lcode}
 \begin{figure}
 \centering
 \begin{minipage}{0.4\textwidth}
   \centering
   ILLUSTRATION 3
   \caption{Illustration 3} \label{fig:mult2}
 \end{minipage} 
 \hfill
 \begin{minipage}{0.4\textwidth}
   \centering
   ILLUSTRATION 4
   \caption{Illustration 4} \label{fig:mult3}
 \end{minipage} 
 \end{figure}
 \end{lcode}
 In this case the illustrations\index{illustration} (or graphics or tabulars) are put into
 separate |minipage| environments within the float, and the captions
 are also put within the |minipage|s. Note that any required |\label|
 must also be inside the |minipage|. If you wished, you could add yet
 another caption after the end of the two |minipage|s.

 \begin{figure}
 \centering
 \begin{minipage}{0.4\textwidth}
   \centering
   ILLUSTRATION 3
   \caption{Illustration 3} \label{fig:mult2}
 \end{minipage} 
 \hfill
 \begin{minipage}{0.4\textwidth}
   \centering
   ILLUSTRATION 4
   \caption{Illustration 4} \label{fig:mult3}
 \end{minipage} 
 \end{figure}

  Keith Reckdahl~\cite{EPSLATEX} provides more examples of this
 kind of thing.

\index{float!multiple|)}


 \subsection{Where LaTeX puts floats}

\index{float!placement|(}

 The general format for a float environment is: \\
 |\begin{float}[|\meta{loc}|] ... \end{float}| or for double column\index{column!double} floats: \\
 |\begin{float*}[|\meta{loc}|] ... \end{float*}| \\
 where the optional argument \meta{loc}, consisting of one or more characters,
 specifies a location where the float may be placed. Note that the 
 \Lpack{multicol}\index{column!multiple} package~\cite{MULTICOL} 
only supports the starred floats and it will not 
 let you have a single column\index{column!single} float. The possible \meta{loc} values are one
 or more of the following:
 \begin{itemize}
 \item[\texttt{b}] \textit{bottom}: at the bottom of a page. This does not apply
 to double column\index{column!double} floats as they may only be placed at the top of a page.
 \item[\texttt{h}] \textit{here}: if possible exactly where the float environment
 is defined. It does not apply to double column\index{column!double} floats.
 \item[\texttt{p}] \textit{page}: on a separate page containing only 
 floats (no text).
 \item[\texttt{t}] \textit{top}: at the top of a page. 
 \item[\texttt{!}] make an extra effort to place the float at the earliest place
  specified by the rest of the argument.
 \end{itemize}
 The default for \meta{loc} is |tbp|, so the float may be placed at the top, 
 or bottom, or on a float-only page; the default works well 95\% of the time.
  Floats of the same kind are output in
 definition order, except that a double column\index{column!double} float may be output before
 a later single column\index{column!single} float of the same kind, or 
 \textit{vice-versa}\footnote{This little quirk
 is fixed by the \Lpack{fixltx2e} package, at least for tables and figures.
 The package is part of a normal LaTeX distribution.}. 
 A float is never put on
 an earlier page than its definition but may be put on the same or later page
 of its definition. If a float cannot be placed, all
 suceeding floats will be held up, and LaTeX can store no more than 16 held
 up floats. A float cannot
 be placed if it would cause an overfull page, or it otherwise cannot be fitted
 according the the float parameters.
 A |\clearpage| or |\cleardoublepage| or |\end{document}| flushes
 out all unprocessed floats, irrespective of the \meta{loc} and float
 parameters, putting them on float-only pages. 

\begin{syntax}
 \cmd{\suppressfloats}\oarg{pos} \\
\end{syntax}
    You can use the command \cmd{\suppressfloats} to suppress floats
 at a given \meta{pos} on the current page. |\suppressfloats[t]| prevents
 any floats at the top of the page and |\suppressfloats[b]| prevents any
 floats at the bottom of the page. The simple |\suppressfloats| prevents
 both top and bottom floats.

    The \Lpack{flafter} package, which should have come with your LaTeX
 distribution, provides a means of preventing floats from moving
 backwards from their definition position in the text. This can be useful to
 ensure, for example, that a float early in a |\section{}| is not typeset before
 the section heading\index{heading}.

\begin{figure}
\centering
\drawparameterstrue
\drawfloatpage
\caption{Float and text page parameters}\label{fig:fpp}
\end{figure}

\begin{figure}
\centering
\drawparameterstrue
\setlayoutscale{0.9}
\drawfloat
\caption{Float parameters}\label{fig:flp}
\end{figure}

 \begin{table}
\begin{adjustwidth}{-1cm}{-1cm}
 \centering
 \captionnamefont{\small\sffamily}
 \captiontitlefont{\small\sffamily}
 \setlength{\belowcaptionskip}{10pt}
 \caption{Float placement parameters}\label{tab:fpp}
 \begin{tabular}{lp{0.5\linewidth}r} \hline
 Parameter & Controls & Default \\ \hline
 \multicolumn{3}{c}{Counters --- change with \cs{setcounter} } \\ \hline
 \Icn{topnumber}  & max number of floats at top of a page & 2 \\
 \Icn{bottomnumber} & max number of floats at bottom of a page & 1 \\
 \Icn{totalnumber} & max number of floats on a text page & 3 \\
 \Icn{dbltopnumber} & like \Icn{topnumber} for double column\index{column!double} floats & 2 \\ \hline
 \multicolumn{3}{c}{Commands --- change with \cs{renewcommand} } \\ \hline
 \cmd{\topfraction} & max fraction of page reserved for top floats & 0.7 \\
 \cmd{\bottomfraction} & max fraction of page reserved for bottom floats & 0.3 \\
 \cmd{\textfraction} & min fraction of page that must have text & 0.2 \\
 \cmd{\dbltopfraction} & like \cmd{\topfraction} for double column\index{column!double} floats & 0.7 \\
 \cmd{\floatpagefraction} & min fraction of a float page that must have float(s) & 0.5 \\
 \cmd{\dblfloatpagefraction} & like \cmd{\floatpagefraction} for double column\index{column!double} floats & 0.5 \\ \hline
 \multicolumn{3}{c}{Text page lengths --- change with \cs{setlength} } \\ \hline
 \lnc{\floatsep} & vertical space between floats & 12pt \\
 \lnc{\textfloatsep} & vertical space between a top (bottom) float and suceeding (preceding) text & 20pt  \\
 \lnc{\intextsep} & vertical space above and below an \texttt{h} float & 12pt \\
 \lnc{\dblfloatsep} & like \lnc{\floatsep} for double column\index{column!double} floats & 12pt \\
 \lnc{\dbltextfloatsep} & like  \lnc{\textfloatsep} for double column\index{column!double} floats & 20pt \\ \hline
 \multicolumn{3}{c}{Float page lengths --- change with \cs{setlength} } \\ \hline
 \lnc{\@fptop} & space at the top of the page & |0pt plus 1fil| \\
 \lnc{\@fpsep} & space between floats & |8pt plus 2fil| \\
 \lnc{\@fpbot} & space at the bottom of the page & |0pt plus 1fil| \\
 \lnc{\@dblfptop} & like \lnc{\@fptop} for double column\index{column!double} floats & |0pt plus 1fil| \\
 \lnc{\@dblfpsep} & like \lnc{\@fpsep} for double column\index{column!double} floats & |8pt plus 2fil| \\
 \lnc{\@dblfpbot} & like \lnc{\@fpbot} for double column\index{column!double} floats & |0pt plus 1fil| \\ \hline
 \end{tabular}
\end{adjustwidth}
 \end{table}

 Figures~\ref{fig:fpp} and~\ref{fig:flp} illustrate the many float parameters
and \tref{tab:fpp} lists the float parameters and the typical 
 standard default values. All the lengths are rubber lengths, and the actual 
 defaults depend on both the class and its size option.

    Given the displayed defaults, the height of a top float must be 
 less than 70\% of the textheight and there can be no more than 2 top floats
 on a text page. Similarly, the height of a bottom float must not
 exceed 30\% of the textheight and there can be no more than 1 bottom
 float on a text page. There can be no more than 3 floats (top, bottom and here)
 on the page. At least 20\% of a text page with floats must be text.
 On a float page (one that has no text, only floats) the sum of the heights
 of the floats must be at least 50\% of the textheight. The floats on a float
 page should be vertically centered.

    It can be seen that with the defaults LaTeX might have trouble finding
 a place for a float. Consider what will happen if a float is a bottom float
 whose height is 40\% of the textheight and this is followed by a float whose
 height is 90\% of the textheight. The first is too large to actually go at the
 bottom of a text page but too small to go on a float page by itself. The second
 has to go on a float page but it is too large to share the float page with the
 first float. LaTeX is stuck!

    At this point it is worthwhile to be precise about the effect of a
 one character \meta{loc} argument:
 \begin{itemize}
 \item[\texttt{[b]}] means: `put the float at the bottom of a page with some
      text above it, and nowhere else'. The float must fit into the 
      |\bottomfraction| space otherwise it and subsequent floats will be held up.
 \item[\texttt{[h]}] means: `put the float at this point and nowhere else'.
      The float must fit into the space left on the page otherwise it and 
      subsequent floats will be held up.
 \item[\texttt{[p]}] means: `put the float on a page that has no text but may
      have other floats on it'. There must be at least `|\floatpagefraction|'
      worth of floats to go on a float only page before the float will be
      be output.
 \item[\texttt{[t]}] means: `put the float at the top of a page with some
      text below it, and nowhere else'. The float must fit into the 
      |\topfraction| space otherwise it and subsequent floats will be held up.
 \item[\texttt{[!...]}] means: `ignore the |\...fraction| values for this
      float'.
 \end{itemize}

 You must try and pick a combination from these that will let LaTeX find
 a place to put your floats. However, you can 
 also can change the float parameters to make it easier to find places
 to put floats. Some examples are:
 \begin{itemize}
 \item Decrease \cmd{\textfraction} to get more `float' on a text page, but
 the sum of \cmd{\textfraction} and \cmd{\topfraction} and the sum of \cmd{\textfraction}
 and \cmd{\bottomfraction} should not exceed 1.0, otherwise the placement algorithm
 falls apart. A minimum value for \cmd{\textfraction} is about 0.10 --- a page
 with less than 10\% text looks better with no text at all, just floats.

 \item Both \cmd{\topfraction} and \cmd{\bottomfraction} can be increased, and it does
 not matter if their sum exceeds 1.0. A good typographic style is that floats
 are encouraged to go at the top of a page, and a better balance is achieved
 if the float space on a page is larger
 at the top than the bottom.

 \item Making \cmd{\floatpagefraction} too small might have the effect of a
 float page just having one small float. However, to make sure that a float
 page never has more than one float on it, do: 
\begin{lcode}
\renewcommand{\floatpagefraction}{0.01}
\setlength{\@fpsep}{\textheight}
\end{lcode}

 \item Setting \lnc{\@fptop} to |0pt|, \lnc{\@fpsep} to |8pt| and \lnc{\@fpbot}
 to |0pt plus 1fil| will force floats on a float page to start at the top
 of the page.
 \end{itemize}
 If you are experimenting, a reasonable starting position is:
 \begin{lcode}
 \setcounter{topnumber}{3}
 \setcounter{bottomnumber}{2}
 \setcounter{totalnumber}{4}
 \renewcommand{\topfraction}{0.85}
 \renewcommand{\bottomfraction}{0.5}
 \renewcommand{\textfraction}{0.15}
 \renewcommand{\floatpagefraction}{0.7}
 \end{lcode}
 and similarly for double column\index{column!double} floats if you will have any. Actually, don't
bother to try these settings as they are the default for this class.

    One of LaTeX's little quirks is that on a text page, the `height' of a float
 is its actual height plus \lnc{\textfloatsep} or \lnc{\floatsep}, while on a float
 page the `height' is the actual height. This means that when using the default
 \meta{loc} of |[tbp]| at least one of the text page float fractions 
 (\cmd{\topfraction} and/or \cmd{\bottomfraction}) must be
 larger than the \cmd{\floatpagefraction} by an amount sufficient to take account
 of the maximum text page separation value.
 
\index{float!placement|)}

%%%%%%%%%%%%%%%%%%%%%%%%%%%%%%%%%%%%%%
\DeleteShortVerb{\|}
\MakeShortVerb{\=}
%%%%\input{tabmanbody}  %% rows & columns
%%%%  tabmanbody.tex

 \chapter{Rows and columns}

\section{Introduction}

    The class provides extensions to the standard \Ie{array} and \Ie{tabular}
environments. These are based partly on a merging of the
 \Lpack{array}~\cite{ARRAY}, 
 \Lpack{dcolumn}~\cite{DCOLUMN},
 \Lpack{delarray}~\cite{DELARRAY}, 
 \Lpack{tabularx}~\cite{TABULARX}, and 
 \Lpack{booktabs}~\cite{BOOKTABS} 
 packages. 
    Much of the material in this chapter strongly reflects the
documentation of these packages.

 However, new kinds of tabular environments are also provided.


\section{General}

\begin{syntax}
\cmd{\[} \senv{array}\oarg{pos}\marg{preamble} rows \eenv{array} \cmd{\]} \\
\senv{tabular}\oarg{pos}\marg{preamble} rows \eenv{tabular} \\
\senv{tabular*}\marg{width}\oarg{pos}\marg{preamble} rows \eenv{tabular*} \\
\senv{tabularx}\marg{width}\oarg{pos}\marg{preamble} rows \eenv{tabularx} \\
\end{syntax}
The \Ie{array} and \Ie{tabular} environments are traditional and the others
are extensions. The \Ie{array} is for typesetting
math and has to be within a math environment of some kind. The \Ie{tabular}
series are for typesetting ordinary text.

    The optional \meta{pos} argument can be one of \texttt{t}, \texttt{c},
or \texttt{b} (the default is \texttt{c}), and controls the vertical position
of the array or tabular; either the \texttt{t}op or the \texttt{c}enter,
or the \texttt{b}ottom is aligned with the baseline. 
Each row consists of elements separated by
\&, and finished with \cmd{\\}. There may be as many rows as desired.
The number and style of the columns is specified by the \meta{preamble}.
The width of each column is wide enough to contain its longest entry and
the overall width of the \Ie{array} or \Ie{tabular} is sufficient to contain
all the columns. However, the \Ie{tabular*} and \Ie{tabularx} provide
more control over the width through their \meta{width} argument.

\section{The preamble}

    You use the \meta{preamble} argument to the array and tabular 
environments to specify the number of columns and how you want column
entries to appear. The preamble consists of a sequence of options, which
are listed in \tref{tab:tabpream}.

 \begin{table}
 \begin{center}
 \caption{The array and tabular preamble options.} \label{tab:tabpream}
    \setlength{\extrarowheight}{1pt}
    \begin{tabular}{cp{9cm}} \toprule
 \texttt{l}             &  Left adjusted column. \\
 \texttt{c}             &  Centered adjusted column. \\
 \texttt{r}             &  Right adjusted column. \\
 \texttt{p}\marg{width} &  Equivalent to =\parbox[t]=\marg{width}. \\
 \texttt{m}\marg{width} &  Defines a column of width \meta{width}.
                           Every entry will be centered in proportion to
                           the rest of the line. It is somewhat like
                           =\parbox=\marg{width}. \\
 \texttt{b}\marg{width} &  Coincides with =\parbox[b]=\marg{width}. \\
 \texttt{>}\marg{decl}  &  Can be used before an \texttt{l}, \texttt{r},
                           \texttt{c}, \texttt{p}, \texttt{m} or a
                           \texttt{b} option. It inserts \meta{decl}
                           directly in front of the entry of the column.
                           \\
 \texttt{<}\marg{decl}  &  Can be used after an \texttt{l}, \texttt{r},
                           \texttt{c}, =p{..}=, =m{..}= or a =b{..}=
                           option.  It inserts \meta{decl} right
                           after the entry of the column.  \\
 \texttt{|}             &  Inserts a vertical line. The distance between
                           two columns will be enlarged by the width of
                           the line. \\
 \texttt{@}\marg{decl}  &  Suppresses inter-column space and inserts
                           \meta{decl} instead. \\
 \texttt{!}\marg{decl}  &  Can be used anywhere and corresponds with the
                           \texttt{|} option. The difference is that
                           \meta{decl} is inserted instead of a
                           vertical line, so this option doesn't
                           suppress the normally inserted space between
                           columns in contrast to =@{...}=.\\
 \texttt{D}\marg{ssep}\marg{osep}\marg{places} & Column entries aligned
                           on a `decimal point' \\
       \bottomrule
    \end{tabular}
 \end{center}
 \end{table}

    Examples of the options include:
 \begin{itemize}
    \item A flush left column with bold font can be specified
          with =>{\bfseries}l=.
%% Companion, page 106.
\begin{lcode}
\begin{center}
\begin{tabular}{>{\large}c >{\large\bfseries}l >{\large\itshape}c} \toprule
A   & B  & C \\ 
100 & 10 & 1 \\ \bottomrule
\end{tabular}
\end{center}
\end{lcode}

\begin{center}
\begin{tabular}{>{\large}c >{\large\bfseries}l >{\large\itshape}c} \toprule
A   & B  & C \\ 
100 & 10 & 1 \\ \bottomrule
\end{tabular}
\end{center}

    \item
       In columns which have been generated with \texttt{p}, \texttt{m}
       or \texttt{b}, the default value of =\parindent= is
       \textsf{0pt}.

%Companion, page 106.

\begin{lcode}
\begin{center}
\begin{tabular}{m{1cm}m{1cm}m{1cm}} \toprule
1 1 1 1 1 1 1 1 1 1 1 1 &
2 2 2 2 2 2 2 2         &
3 3 3 3                 \\ \bottomrule
\end{tabular}
\end{center}
\end{lcode}

\begin{center}
\begin{tabular}{m{1cm}m{1cm}m{1cm}} \toprule
1 1 1 1 1 1 1 1 1 1 1 1 &
2 2 2 2 2 2 2 2         &
3 3 3 3                 \\ \bottomrule
\end{tabular}
\end{center}

       The \lnc{\parindent} can be changed with, for example 
       =>{\setlength{\parindent}{1cm}}p=.

%%Companion, page 107.

\begin{lcode}
\begin{center}
\begin{tabular}{>{\setlength{\parindent}{5mm}}p{2cm} p{2cm}} \toprule
1 2 3 4 5 6 7 8 9 0 1 2 3 4 5 6 7 8 9 0 &
1 2 3 4 5 6 7 8 9 0 1 2 3 4 5 6 7 8 9 0 \\ \bottomrule
\end{tabular}
\end{center}
\end{lcode}

\begin{center}
\begin{tabular}{>{\setlength{\parindent}{5mm}}p{2cm} p{2cm}} \toprule
1 2 3 4 5 6 7 8 9 0 1 2 3 4 5 6 7 8 9 0 &
1 2 3 4 5 6 7 8 9 0 1 2 3 4 5 6 7 8 9 0 \\ \bottomrule
\end{tabular}
\end{center}

    \item
       The specification =>{$}c<{$}= generates a column in math
       mode in a \Ie{tabular} environment. When used in an \Ie{array}
       environment the column is in LR mode (because the additional
       $'s cancel the existing $'s).
    \item
       Using =c!{\hspace{1cm}}c= you get space between two
       columns which is enlarged by one centimeter, while
       =c@{\hspace{1cm}}c= gives you exactly one centimeter
       space between two columns.
\item Elsewhere reasons are given why you should not use vertical
      lines (e.g., the \texttt{|} option) in tables. Any examples
      that use vertical lines are for illustrative purposes only where
      it is advantageous to denote column boundaries, for example
      to show different spacing effects.
 \end{itemize}

 \subsection{D column specifiers} \label{sec:dcolumns}

    In financial tables dealing with pounds
and pence or dollars and cents, column entries
should be aligned on the separator between the numbers. The \texttt{D}
column specifier is provided for columns which are to be aligned on 
a `decimal point'. The specifier takes three arguments.
\begin{syntax}
 \texttt{D}\marg{ssep}\marg{osep}\marg{places} \\
\end{syntax}
\begin{itemize}
\item[\meta{ssep}] is the single character which is used as the
 separator in the source \texttt{.tex} file. Thus it will usually 
be `\texttt{.}' or  `\texttt{,}'.

\item[\meta{osep}] is the separator in the output, this may
 be the same as the first argument, but may be any math-mode
 expression, such as \cmd{\cdot}. A \texttt{D} column
 always uses math mode for the digits as well as the separator.

\item[\meta{places}] should be the maximum number of decimal places
 in the column (but see below for more on this). 
If this is negative, any number of decimal places can
 be used in the column, and all entries will be centred on 
 (the leading edge of) the 
 separator. Note that this can cause a column to be too wide; for instance, 
compare  the first two columns in the example below. 
\end{itemize}

    Here are some example specifications which, for convenience, employ
the \cmd{\newcolumntype} macro described later.

\begin{lcode}
\newcolumntype{d}[1]{D{.}{\cdot}{#1}}
\end{lcode}
    This defines \texttt{d} to be a column specifier taking a single argument 
specifying the number of decimal places, 
and the \file{.tex} file should use `\texttt{.}' as the separator, with
\cmd{\cdot} ($\cdot$) being used in the output.

\begin{lcode}
\newcolumntype{.}{D{.}{.}{-1}}
\end{lcode}
The result of this is that `\texttt{.}' specifies a column of entries to 
be centered on the~`$.$'.

\begin{lcode}
\newcolumntype{,}{D{,}{,}{2}}
\end{lcode}
And the result of this is that `\texttt{,}' specifies a column of 
entries with at most two decimal places after a~`$,$'.

 \newcolumntype{d}[1]{D{.}{\cdot}{#1}}
 \newcolumntype{.}{D{.}{.}{-1}}
 \newcolumntype{,}{D{,}{,}{2}}

 The following table is typeset from this code:
\begin{lcode}
\begin{center}
 \begin{tabular}{|d{-1}|d{2}|.|,|}
 1.2   & 1.2   &1.2    &1,2    \\
 1.23  & 1.23  &12.5   &300,2  \\
 1121.2& 1121.2&861.20 &674,29 \\
 184   & 184   &10     &69     \\
 .4    & .4    &       &,4     \\
       &       &.4     &
 \end{tabular}
 \end{center}
\end{lcode}

 \begin{center}
 \begin{tabular}{|d{-1}|d{2}|.|,|}
 1.2   & 1.2   &1.2    &1,2    \\
 1.23  & 1.23  &12.5   &300,2  \\
 1121.2& 1121.2&861.20 &674,29 \\
 184   & 184   &10     &69     \\
 .4    & .4    &       &,4     \\
       &       &.4     &
 \end{tabular}
 \end{center}

 Note that the first column, which had a negative \meta{places}
 argument is wider than the second column, so that the decimal point
 appears in the middle of the column.

 The third
 \meta{places} argument may specify \emph{both} the number of
 digits to the left and to the right of the decimal place. The third
 column in the next table below is set with =D{.}{.}{5.1}= and in the
 second  table,  =D{.}{.}{1.1}=, to specify
 `five places to the left and one to the right' and `one place to the
 left and one to the right' respectively.  (You may use `,' or other
 characters, not necessarily `.' in this argument.) The column of figures
 is then positioned such that a number with the specified numbers of
 digits is centred in the column.
 
 If you have table headings (inserted with \cmd{\multicolumn}={1}{c}{...}=
 to over-ride the =D= column type) then it may be that neither of the
 above `centred' or `right aligned' forms is quite what you want.
\begin{lcode}
 \begin{center}\small
 \begin{tabular}[t]{|D..{-1}|D..{1}|D..{5.1}|}
\multicolumn{1}{|c|}{head}&
\multicolumn{1}{c|}{head}&
\multicolumn{1}{c|}{head}\\[3pt]
 1.2  & 1.2  &1.2 \\
 11212.2& 11212.2&11212.2  \\
 .4    & .4    &.4         
 \end{tabular}
 \hfill
 \begin{tabular}[t]{|D..{-1}|D..{1}|D..{1.1}|}
\multicolumn{1}{|c|}{wide heading}&
\multicolumn{1}{c|}{wide heading}&
\multicolumn{1}{c|}{wide heading}\\[3pt]
 1.2  & 1.2  &1.2 \\
 .4    & .4    &.4  
 \end{tabular}
 \end{center}
\end{lcode}

 \begin{center}\small
 \begin{tabular}[t]{|D..{-1}|D..{1}|D..{5.1}|}
\multicolumn{1}{|c|}{head}&
\multicolumn{1}{c|}{head}&
\multicolumn{1}{c|}{head}\\[3pt]
 1.2  & 1.2  &1.2 \\
 11212.2& 11212.2&11212.2  \\
 .4    & .4    &.4         
 \end{tabular}
 \hfill
 \begin{tabular}[t]{|D..{-1}|D..{1}|D..{1.1}|}
\multicolumn{1}{|c|}{wide heading}&
\multicolumn{1}{c|}{wide heading}&
\multicolumn{1}{c|}{wide heading}\\[3pt]
 1.2  & 1.2  &1.2 \\
 .4    & .4    &.4  
 \end{tabular}
 \end{center}

 In both of these tables the first column is set with =D{.}{.}{-1}=
 to produce a column centered on the `\texttt{.}', and the second column is
 set with =D{.}{.}{1}= to produce a right aligned column.

 The centered (first) column produces columns that are wider than necessary
 to fit in the numbers under a heading as it has to ensure that the
 decimal point is centered. The right aligned (second) column does not have
 this drawback, but under a wide heading a column of small right
 aligned figures is somewhat disconcerting.

 The notation for the \meta{places} argument also enables columns that 
are centred on the mid-point
 of the separator, rather than its leading edge; for example 
 =D{+}{\,\pm\,}{3,3}= will give a symmetric layout of up to three
 digits on either side of a $\pm$ sign.

 \subsection{Defining new column specifiers}

     You can easily type
 \begin{quote}
   =>{=\meta{some declarations}=}{c}<{=\meta{some more
   declarations}=}=
 \end{quote}
when you have a one-off column in a table, but it gets tedious
 if you often use columns of this form. 
The \cmd{\newcolumntype} lets you define a new column option like, say
 \begin{quote}
   =\newcolumntype{x}{>{=\meta{some declarations}=}{c}<{=\meta{some
   more declarations}=}}=\hspace*{-3cm} 
 \end{quote}
and you can then use the \texttt{x} column specifier in the preamble wherever
you want a column of this kind.

\begin{syntax}
\cmd{\newcolumntype}\marg{char}\oarg{nargs}\marg{spec} \\
\end{syntax}
The 
\meta{char} argument is the character that identifies the option and \meta{spec}
is its specification in terms of the regular preamble options.
The \cmd{\newcolumntype} command is similar to \cmd{\newcommand} --- 
\meta{spec} itself can take arguments with the optional \meta{nargs}
argument declaring their number. 

    For example, it is commonly required to have both math-mode and text 
columns in the same alignment. Defining:
\begin{lcode}
\newcolumntype{C}{>{$}c<{$}}
\newcolumntype{L}{>{$}l<{$}}
\newcolumntype{R}{>{$}r<{$}}
\end{lcode}
 Then \texttt{C} can be used to get centred text in an
\Ie{array}, or centred math-mode in a \Ie{tabular}. Similarly
\texttt{L} and \texttt{R} are for left- and right-aligned columns.

 The \meta{spec} in a \cmd{\newcolumntype} command may refer to any of
 the primitive column specifiers (see table \ref{tab:tabpream} on page
 \pageref{tab:tabpream}), or to any new letters defined in other
\cmd{\newcolumntype} commands.

\begin{syntax}
\cmd{\showcols} \\
\end{syntax}
   A list of all the currently active
\cmd{\newcolumntype} definitions is sent to the terminal and log file if
 the \cmd{\showcols} command is given.

\subsection{Notes}

 \begin{itemize}
 \item A preamble of the form ={wx*{0}{abc}yz}= is treated as ={wxyz}=

 \item An incorrect positional argument, such as \texttt{[Q]}, 
 is treated as \texttt{[t]}.

 \item A preamble such as ={cc*{2}}= with an error in
 a $*$-form will generate an error message 
 that is not particularly helpful.

 \item Error messages generated when parsing the column specification
   refer to the preamble argument \emph{after} it has been re-written
   by the \cmd{\newcolumntype} system, not to the preamble entered by the
   user.  

 \item Repeated \texttt{<} or \texttt{>} constructions
 are allowed.  \texttt{>}\marg{decs1}\texttt{>}\marg{decs2}
 is treated the same as \texttt{>}\marg{decs2}\marg{decs1}.

   The treatment of multiple \texttt{<} or \texttt{>}
declarations may seem strange. Using the obvious ordering
of \texttt{>}\marg{decs1}\marg{decs2} has the disadvantage
of not allowing the settings of a  \cmd{\newcolumntype} 
defined using these declarations to be overriden.

 \item The \cmd{\extracolsep} command may be used in \texttt{@}-expressions 
 as in standard \ltx, and also in \texttt{!}-expressions.

   The use of \cmd{\extracolsep} is subject to the following
   two restrictions.  There must be at most one \cmd{\extracolsep}
   command per \texttt{@}, or \texttt{!} expression and the command must be
   directly entered into the \texttt{@} expression, not as part of a macro
   definition. Thus
\begin{lcode}
\newcommand{\ef}{\extracolsep{\fill}} ... @{\ef}
\end{lcode}
 does not work. However you can use
   something like
\begin{lcode}
\newcolumntype{e}{@{\extracolsep{\fill}}
\end{lcode}
instead.

 \item As noted by the \ltx{} book, a 
   \cmd{\multicolumn}, with the exception of the first column,
   consists of the entry and the \emph{following} inter-column
   material. This means that in a tabular with the preamble
   =|l|l|l|l|= input such as =\multicolumn{2}{|c|}= in
   anything other than the first column is incorrect.

   In the standard array/tabular implementation this error is not 
   noticeable as a =|= takes no horizontal space. But in the class the
   vertical lines take up their natural width and you will see two lines if
   two are specified --- another reason to avoid using \verb?|?.

 \end{itemize}

\section{The array environment}

    Mathematical arrays are usually produced using the \Ie{array} environment.

\begin{syntax}
\cmd{\[} \senv{array}\oarg{pos}\marg{preamble} rows \eenv{array} \cmd{\]} \\
\cmd{\[} \senv{array}\oarg{pos}\meta{left}\marg{preamble}\meta{right} rows \eenv{array} \cmd{\]} \\
\end{syntax}
    Math formula are usually centered in the columns, but a column of 
numbers often looks best flush right, or aligned on some distinctive
feature. In the latter case the \texttt{D} column scheme is very handy.
\begin{lcode}
\[ \begin{array}{lcr}
   a +b +c & d - e - f & 123 \\
   g-h     &  j k      & 45 \\
    l      &   m       & 6
  \end{array} \]
\end{lcode}

\[ \begin{array}{lcr}
   a +b +c & d - e - f & 123 \\
   g-h     &  j k      & 45 \\
    l      &   m       & 6
  \end{array} \]

    Arrays are often enclosed in brackets or vertical lines or brackets 
or other symbols to
denote math constructs like matrices. The delimeters are often large and have
to be indicated using \cmd{\left} and \cmd{\right} commands.
\begin{lcode}
\[ \left[ \begin{array}{cc}
          x_{1} & x_{2} \\
          x_{3} & x_{4}
          \end{array} \right] \]
\end{lcode}

\[ \left[ \begin{array}{cc}
          x_{1} & x_{2} \\
          x_{3} & x_{4}
          \end{array} \right] \]


    The \Ie{array} environment is an extension of the standard environment
in that it provides a 
 a system of implicit \cmd{\left} \cmd{\right} pairs. If you want an array
 surrounded by parentheses, you can enter:
\begin{lcode}
 \[  \begin{array}({cc})a&b\\c&d\end{array}   \]
\end{lcode}

 \[  \begin{array}({cc})a&b\\c&d\end{array}   \]

Or, a litle more complex
\begin{lcode}
\[ \begin{array}({c})
     \begin{array}|{cc}|
       x_{1} & x_{2} \\
       x_{3} & x_{4}
     \end{array} \\
         y \\
         z
   \end{array} \]
\end{lcode}

\[ \begin{array}({c})
     \begin{array}|{cc}|
       x_{1} & x_{2} \\
       x_{3} & x_{4}
     \end{array} \\
         y \\
         z
   \end{array} \]

 Similarly an environment equivalent to plain TeX's \cmd{\cases} could
 be defined by:\\
\begin{lcode}
 \[  f(x)=\begin{array}\{{lL}.
           0        &if $x=0$\\
           \sin(x)/x&otherwise
           \end{array}  \]
\end{lcode}

 \DeleteShortVerb{\=}
 \newcolumntype{L}{>{$}l<{$}}
 \[  f(x)=\begin{array}\{{lL}.
           0        &if $x=0$\\
           \sin(x)/x&otherwise
           \end{array}  \]
 \MakeShortVerb{\=}

 Here \texttt{L} denotes a column of left aligned L-R text, as described
earlier.
 Note that as the delimiters must always be used in pairs, the  `=.='
 must be used to denote a  `null delimiter'.

 This feature is especially useful if the =[t]= or =[b]=
 arguments are also used. In these cases the result is not equivalent
 to surrounding the environment by \cmd{\left}\ldots\cmd{\right}, as
 can be seen from the following example:
\begin{lcode}
 \begin{array}[t]({c}) 1\\2\\3 \end{array}
 \begin{array}[c]({c}) 1\\2\\3 \end{array}
 \begin{array}[b]({c}) 1\\2\\3 \end{array}
 \quad\mbox{not}\quad
 \left(\begin{array}[t]{c} 1\\2\\3 \end{array}\right)
 \left(\begin{array}[c]{c} 1\\2\\3 \end{array}\right)
 \left(\begin{array}[b]{c} 1\\2\\3 \end{array}\right)
\end{lcode}


 \[
 \begin{array}[t]({c}) 1\\2\\3 \end{array}
 \begin{array}[c]({c}) 1\\2\\3 \end{array}
 \begin{array}[b]({c}) 1\\2\\3 \end{array}
 \quad\mbox{not}\quad
 \left(\begin{array}[t]{c} 1\\2\\3 \end{array}\right)
 \left(\begin{array}[c]{c} 1\\2\\3 \end{array}\right)
 \left(\begin{array}[b]{c} 1\\2\\3 \end{array}\right)
 \]

\section{Tables}

    A table is one way of presenting a large amount of information
in a limited space. Even a simple table can presents facts that could
require several wordy paragraphs --- it is not only a picture that is worth
a thousand words.

    A table should have at least two columns, otherwise it is really a list,
and often has more. The leftmost column is often called the \emph{stub}
and it typically contains a vertical listing of the information categories
 in the other columns. The columns have \emph{heads} (or \emph{headings}) at
the top indicating the nature of the entries in the column, although
it is not always necessary to provide a head for the stub if the
heading is obvious from the table's caption. Column heads
may include subheadings, often to specify the unit of measurement for numeric
data. 

   A less simple table may need two or more levels of headings, in which
case \emph{decked heads} are used. A decked head consists of a \emph{spanner
head} and the two or more column heads it applies to. A horizontal 
\emph{spanner rule} is set between the the spanner and column heads to 
show the columns belonging to the spanner.

  Double decking, and certainly triple decking or more, should be avoided
as it can make it difficult following them down the table. It may be possible
to use a \emph{cut-in head} instead of double decking. A cut-in head is
one that cuts across the columns of the table and applies to all the 
matter below it.

   No mention has been made of any vertical rules in a table, and this is
deliberate. There should be no vertical rules in a table. Rules, 
if used at all, should be horizontal only, and these should be single, 
not double or triple. It's not that ink is expensive, or that practically
no typesetting is done by hand any more, it's that the visual clutter should
be eliminated. 

    For example, in \tref{tab:2views}, \tref{tab:before} is from the \ltx{} book and 
\tref{tab:after} is how Simon Fear~\cite{BOOKTABS} suggests it
should be cleaned up.
\begin{lcode}
\begin{table}
\centering
\caption{Two views of one table} \label{tab:2views}
\subtop[Before]{\label{tab:before}%
 \begin{tabular}{||l|lr||} \hline
 gnats     & gram      & $13.65 \\ \cline{2-3}
           & each      & .01 \\ \hline
 gnu       & stuffed   & 92.50 \\ \cline{1-1} \cline{3-3}
 emu       &           & 33.33 \\ \hline
 armadillo & frozen    & 8.99 \\ \hline
 \end{tabular}}
\hfill
\subtop[After]{\label{tab:after}%
 \begin{tabular}{@{}llr@{}} \toprule
 \multicolumn{2}{c}{Item} \\ \cmidrule(r){1-2}
 Animal & Description & Price ($)\\ \midrule
 Gnat  & per gram  & 13.65 \\
       & each      & 0.01 \\
 Gnu   & stuffed   & 92.50 \\
 Emu   & stuffed   & 33.33 \\
 Armadillo & frozen & 8.99 \\ \bottomrule
 \end{tabular}
}
\end{table}
\end{lcode}

\begin{table}
\centering
\caption{Two views of one table} \label{tab:2views}
\subtop[Before]{\label{tab:before}%
 \begin{tabular}{||l|lr||} \hline
 gnats     & gram      & $13.65 \\ \cline{2-3}
           & each      & .01 \\ \hline
 gnu       & stuffed   & 92.50 \\ \cline{1-1} \cline{3-3}
 emu       &           & 33.33 \\ \hline
 armadillo & frozen    & 8.99 \\ \hline
 \end{tabular}}
\hfill
\subtop[After]{\label{tab:after}%
 \begin{tabular}{@{}llr@{}} \toprule
 \multicolumn{2}{c}{Item} \\ \cmidrule(r){1-2}
 Animal & Description & Price ($)\\ \midrule
 Gnat  & per gram  & 13.65 \\
       & each      & 0.01 \\
 Gnu   & stuffed   & 92.50 \\
 Emu   & stuffed   & 33.33 \\
 Armadillo & frozen & 8.99 \\ \bottomrule
 \end{tabular}
}
\end{table}
 
\subsection{Fear's rules} \label{sec:fear}

    Simon Fear disapproves of the default LaTeX table rules and
 wrote the \Lpack{booktabs} package~\cite{BOOKTABS} to provide
 better horizontal rules. Like many typographers, he abhors vertical rules.

     In a simple case a table begins with a \cmd{\toprule}, has
 a single row of column headings, then a dividing rule called
 here a \cmd{\midrule}; after the columns of data the table is 
finished off with
 a \cmd{\bottomrule}. Most book publishers set the \cmd{\toprule} and 
 \cmd{\bottomrule} heavier (i.e., thicker, or darker)
 than the intermediate \cmd{\midrule}. 

\begin{syntax}
\cmd{\toprule}\oarg{wd} 
\cmd{\bottomrule}\oarg{wd} 
\lnc{\heavyrulewidth} \\
\cmd{\midrule}\oarg{wd} 
\lnc{\lightrulewidth} \\
\end{syntax}

 The rule commands here all take a default width (thickness)
 which may be reset within the document.
For the top and bottom rules this
 is \lnc{\heavyrulewidth} and for midrules it is \lnc{\lightrulewidth}
 (fully described below). In very rare cases where you need to have
a special width, you may use the optional argument \meta{wd} length
argument.

    The rule commands go immediately after the closing
\cmd{\\} of the preceding row (except of course \cmd{\toprule}, which
 comes right after the =\tabular{}= command); in other words,
 exactly where plain \ltx{} allows \cmd{\hline} or \cmd{\cline}.

\begin{syntax}
\cmd{\cmidrule}\oarg{wd}=(=\meta{trim}=)=\marg{a--b} 
\lnc{\cmidrulewidth} \\
\end{syntax}
Similar to the normal \cmd{\cline} macro, \cmd{\cmidrule} draws a 
rule across the columns (numbered) \meta{a--b}. 
Generally, this rule
 should not come to the full width of the end columns, and this
 is especially the case when we need to begin a \cmd{\cmidrule}
 straight after the end of another one.
The optional `trimming' commands, which are =(r)=, =(l)= and =(rl)=
 or =(lr)=, indicate whether the right and/or left ends of the
 rule should be trimmed. Note the exceptional use of parentheses
 instead of braces or brackets for this optional argument. For example,
the code
\begin{lcode}
\cmidrule(r){1-2}
\end{lcode}
is used for \tref{tab:after}.

The default width is \lnc{\cmidrulewidth} but the optional
\meta{wd} argument can be used to override this. 

 An example of the commands in use is given by the code used to
 produce the `after' example in \tref{tab:after} above.

 Occasionally extra space is required between certain rows
 of a table; for example, before the last row when it is a
 total of numbers above.
This is simply a matter of inserting \cmd{\addlinespace}
 after the \cmd{\\} alignment marker.

\begin{syntax}
\cmd{\addlinespace}\oarg{wd}
\lnc{\defaultaddspace} \\
\end{syntax}
Think of \cmd{\addlinespace} as being a white rule of width \meta{wd}.
 The default space is \cmd{\defaultaddspace} which gives rather
 less than a whole line space.

\begin{syntax}
\cmd{\specialrule}\marg{wd}\marg{abovespace}\marg{belowspace} \\
\end{syntax}
    You should not use double rules in a table; use rules with different 
thicknesses instead. The \cmd{specialrule} can be used for drawing
rules with special thickness and spacing. 
However, the rules already described should be sufficient without
having to resort to \cmd{\specialrule}.

\begin{syntax}
\cmd{\morecmidrules} \\
\end{syntax}
Nevertheless, if you insist on having 
double \cmd{\cmidrule}s you will need the extra command
 \cmd{\morecmidrules} to do so properly, because two
 \cmd{\cmidrule}s  in a row calls for two rules on the same `rule row'. 
Thus in
\begin{lcode}
\cmidrule{1-2}\cmidrule{1-2}
\end{lcode}
 the second command writes a rule that just overwrites the first
 one. Use
\begin{lcode}
\cmidrule{1-2}\morecmidrules\cmidrule{1-2}
\end{lcode}
 which gives you a double rule between columns one and two,
 separated by \cmd{\cmidrulesep}.
Finish off a whole row of rules before giving the
 \cmd{\morecmidrules} command. Note that \cmd{\morecmidrules} has no
 effect whatsoever if it does not immediately follow a
 \cmd{\cmidrule} (i.e., it is not a general space-generating command).

 The new rule commands are not guaranteed to work with \cmd{\hline}
 or \cmd{\cline}.
 More significantly the rules generated by the new commands are pretty
well guaranteed not to connect with verticals generated by ={|}=
 characters in the preamble. This is a feature --- you
 should not use vertical rules in tables.

\subsection{Tabular environments}

\begin{syntax}
\senv{tabular}\oarg{pos}\marg{preamble} rows \eenv{tabular} \\
\senv{tabular*}\marg{width}\oarg{pos}\marg{preamble} rows \eenv{tabular*} \\
\senv{tabularx}\marg{width}\oarg{pos}\marg{preamble} rows \eenv{tabularx} \\
\end{syntax}

    A table created using the \Ie{tabular} environment comes out as
wide as it has to be to accomodate the entries. On the other hand,
both the \Ie{tabular*} and \Ie{tabularx} environments let you specify
the overall width of the table via the additional \meta{width} atrgument.

    The \Ie{tabular*} environment makes any necessary adjustment by altering
the intercolumn spaces while the \Ie{tabularx} environment alters
the column widths. Those columns that can be adjusted are noted by
using the letter \texttt{X} as the column specifier in the \meta{preamble}.
Once the correct column widths have been calculated the \texttt{X}
columns are converted to \texttt{p} columns.

    The following code is used for a regular \Ie{tabular}.
\begin{lcode}
\begin{center}
\begin{tabular}{|c|p{5.5pc}|c|p{5.5pc}|}  \hline
\multicolumn{2}{|c|}{Multicolumn entry!} & THREE & FOUR \\ \hline
one &
\raggedright\arraybackslash The width of this column is fixed (5.5pc). &
three &
\raggedright\arraybackslash Column four will act in the same way as
  column two, with the same width.\\
\hline
\end{tabular}
\end{center}
\end{lcode}

\begin{center}
\begin{tabular}{|c|p{5.5pc}|c|p{5.5pc}|}   \hline
\multicolumn{2}{|c|}{Multicolumn entry!} & THREE & FOUR \\  \hline
one &
\raggedright\arraybackslash The width of this column is fixed (5.5pc). &
three &
\raggedright\arraybackslash Column four will act in the same way as
  column two, with the same width.\\
 \hline
\end{tabular}
\end{center}

    The following examples use virtually the same contents, the major
differences are the specifications of the environment.

 The following \Ie{tabularx} is set with
\begin{lcode}
\begin{tabularx}{250pt}{|c|X|c|X|}
\end{lcode}
and the result is:

 \begin{center}
 \begin{tabularx}{250pt}{|c|X|c|X|}
 \hline
 \multicolumn{2}{|c|}{Multicolumn entry!}&
 THREE&
 FOUR\\
 \hline
 one&
 \raggedright\arraybackslash The width of this column depends on the
 width of the table.\footnote{You can use footnotes in a \texttt{tabularx}!}
&
 three&
 \raggedright\arraybackslash Column four will act in the same way as
 column two, with the same width.\\
 \hline
 \end{tabularx}
 \end{center}

The following \Ie{tabular*} is set with
\begin{lcode}
 \begin{tabular*}{250pt}{|c|p{5.5pc}|c|p{5.5pc}|}
\end{lcode}
Notice that there is no \texttt{X} column; the total width required
for the tabular is less than the specfied width and hence the horizontal
lines spill over the right hand end of the apparent tabular width.
 \begin{center}
 \begin{tabular*}{250pt}{|c|p{5.5pc}|c|p{5.5pc}|}
 \hline
 \multicolumn{2}{|c|}{Multicolumn entry!}&
 THREE&
 FOUR\\
 \hline
 one&
 \raggedright\arraybackslash The width of this column is fixed (5.5pc). &
 three &
 \raggedright\arraybackslash Column four will act in the same way as
 column two, with the same width.\\
 \hline
 \end{tabular*}
 \end{center}

The next \Ie{tabular*} table is set with 
\begin{lcode}
 \begin{tabular*}{300pt}{|@{\extracolsep{\fill}}c|p{5.5pc}|c|p{5.5pc}|}
\end{lcode}

 \begin{center}
 \begin{tabular*}{300pt}{|@{\extracolsep{\fill}}c|p{5.5pc}|c|p{5.5pc}|}
 \hline
 \multicolumn{2}{|c|}{Multicolumn entry!}&
 THREE&
 FOUR\\
 \hline
 one&
 \raggedright\arraybackslash The width of this column's text is fixed (5.5pc). &
 three &
 \raggedright\arraybackslash Column four will act in the same way as
 column two, with the same width.\\
 \hline
 \end{tabular*}
 \end{center}


The following \Ie{tabularx} is set with
\begin{lcode}
\begin{tabularx}{300pt}{|c|X|c|X|}
\end{lcode}
the result is:

 \begin{center}
 \begin{tabularx}{300pt}{|c|X|c|X|}
 \hline
 \multicolumn{2}{|c|}{Multicolumn entry!}&
 THREE&
 FOUR\\
 \hline
 one&
 \raggedright\arraybackslash The width of this column depends on the
 width of the table.&
 three&
 \raggedright\arraybackslash Column four will act in the same way as
 column two, with the same width.\\
 \hline
 \end{tabularx}
 \end{center}

     The main differences between the \Ie{tabularx} and \Ie{tabular*}
environments are:
 \begin{itemize}
 \item \Ie{tabularx} modifies the widths of the \emph{columns},
 whereas \Ie{tabular*} modifies the widths of the inter-column
 \emph{spaces}.
 \item \Ie{tabular} and \Ie{tabular*} environments may be
 nested with no restriction, however if one \Ie{tabularx}
 environment occurs inside another, then the inner one \emph{must} be
 enclosed by ={ }=.
 \item The body of the \Ie{tabularx} environment is in fact the
 argument to a command, and so certain constructions which are not
 allowed in command arguments (like \cmd{\verb}) may not be used.\footnote
 {Actually, \texttt{verb} and \texttt{verb*} may be used, but they may 
 treat spaces incorrectly, and the
 argument can not contain an unmatched {\ttfamily\char`\{} or
 {\ttfamily\char`\}}, or a  {\ttfamily\char`\%} character.}
 \item \Ie{tabular*} uses a primitive capability of \tx{} to
 modify the inter column space of an alignment. \Ie{tabularx}
 has to set the table several times as it searches for the best column
 widths, and is therefore much slower. Also the fact that the body is
 expanded several times may break certain \tx{} constructs.
 \end{itemize}

\begin{syntax}
\cmd{\tracingtabularx} \\
\end{syntax}
Following the \cmd{\tracingtabularx} declaration all
later \Ie{tabularx} environments will print information
 about column widths as they repeatedly re-set the tables to find the
 correct widths.

 By default the \texttt{X} specification is turned into
 =p{=\meta{some value}=}=. Such narrow columns often
 require a special format, which can be achieved by using the =>= syntax.
For example, =>{\small}X=. Another format which is useful in narrow 
columns is  ragged right, however \ltx's \cmd{\raggedright} macro redefines
\cmd{\\} in a way which conflicts with its use in  \Ie{tabular} or 
\Ie{array}  environments.

\begin{syntax}
\cmd{\arraybackslash} \\
\end{syntax}
 For this reason the command \cmd{\arraybackslash} is provided;
this may be used after a \cmd{\raggedright}, \cmd{\raggedleft}  or
\cmd{\centering} declaration. Thus a \Ie{tabularx} preamble may include
\begin{lcode}
>{\raggedright\arraybackslash}X
\end{lcode}
 These preamble specifications may of course be saved using the
 command, \cmd{\newcolumntype}. After specifying, say,
\begin{lcode}
\newcolumntype{Y}{>{\small\raggedright\arraybackslash}X}
\end{lcode}
then \texttt{Y} could be used in the \Ie{tabularx} preamble
 argument.
 
\begin{syntax}
\cmd{\tabularxcolumn} \\
\end{syntax}
 The \texttt{X} columns are set using the \texttt{p} column, which
 corresponds  to \cmd{\parbox}=[t]=. You may want them set using, say, the
\texttt{m} column, which corresponds to \cmd{\parbox}=[c]=. It is not
 possible to change the column type using the \texttt{>} syntax, so another
 system is provided.  \cmd{\tabularxcolumn} should be defined to be a macro
 with one argument, which expands to the \Ie{tabular} preamble
 specification that you want to correspond to \texttt{X}. The
 argument will be replaced by the calculated width of a column.

 The default definition is
\begin{lcode}
\newcommand{\tabularxcolumn}[1]{p{#1}}
\end{lcode}
This may be changed, for instance
\begin{lcode}
\renewcommand{\tabularxcolumn}[1]{>{\small}m{#1}}
\end{lcode}
so that \texttt{X} columns will be typeset as \texttt{m} columns using
the \cmd{\small} font.

 Normally all \texttt{X} columns in a single table are set to the
 same width, however it is possible to make \Ie{tabularx} set
 them to different widths.
 A preamble argument of 
\begin{lcode}
{>{\hsize=.5\hsize}X>{\hsize=1.5\hsize}X}
\end{lcode}
 specifies two columns, the second will be three times as wide as the
 first. However if you want to play games like this you should follow
 the following two rules.
 \begin{enumerate}
 \item Make sure that the sum of the widths of all the \texttt{X}
 columns is unchanged. (In the above example, the new widths still add
 up to twice the default width, the same as two standard \texttt{X}
 columns.)
 \item Do not use \cmd{\multicolumn} entries which cross any \texttt{X}
 column.
 \end{enumerate}
 As with most rules, these may be broken if you know what you are
 doing.

 It may be that the widths of the `normal' columns of the table
 already total more  than the requested total
 width. \Ie{tabularx} refuses to set the 
 \texttt{X} columns to a negative width, so in this case you get a 
 warning ``X Columns too narrow (table too wide)''.

 The \texttt{X} columns will in this case be set to a width of 1em
 and so the table itself will be wider than the requested total width
 given in the argument to the environment.
% This behaviour of the package can be customised slightly
% as noted in the documentation of the code section.

    The standard \cmd{\verb} macro does not work inside a \Ie{tabularx},
just as it does not work in the argument to any macro.

\begin{syntax}
\cmd{\TX@verb} \\
\end{syntax}
\cmd{\TX@verb} is the `poor man's \cmd{\verb}' 
and is based on page 382 of the \tx Book. As
 explained there, doing verbatim this way means that spaces are not
 treated correctly, and so a \cmd{\verb*} version may well be useless. 
 The mechanism is quite general, and any macro which wants to allow a
 form of \cmd{\verb} to be used within its argument may
\begin{lcode}
\let\verb=\TX@verb
\end{lcode}
It must ensure that the real definition is restored afterwards.

    This version of \cmd{\verb} is subject to the 
following restictions:
 \begin{enumerate}
 \item Spaces in the argument are not read verbatim, but may be skipped
       according to \tx's usual rules.
 \item Spaces will be added to the output after control words, even if
       they were not present in the input.
 \item Unless the argument is a single space, any trailing space,
       whether in the original argument, or added as in (2),
       will be omitted.
 \item The argument must not end with =\=, so =\verb|\|= is not
      allowed, however, because of (3), =\verb|\ |= produces
      =\=.
 \item The argument must be balanced with respect to ={= and =}=. So
      =\verb|{|= is not allowed.
 \item A comment character like \verb?%? will not appear verbatim. It will
       act as usual, commenting out the rest of the input line!
 \item The combinations =?`= and =!`= will appear as
       {\ttfamily?`} and {\ttfamily!`} if the \texttt{cmtt} font is
       being used.
 \end{enumerate}

\section{Free tabulars}

    All the tabular environments described so far put the table
into a box, which \ltx{} treats like a large, even though complex,
 character, and characters are not broken across pages. If you
have a long table that runs off the bottom of the page you can turn
to, say, the \Lpack{longtable}~\cite{LONGTABLE} or \Lpack{xtab}~\cite{XTAB}
packages which will automatically break tables across page boundaries.
These have various bells and whistles, such as automatically putting
on a caption at the top of each page, repeating the column heads, and 
so forth. 

\subsection{Continuous tabulars}

\begin{syntax}
\senv{ctabular}\oarg{pos}\marg{preamble} rows \eenv{ctabular} \\
\senv{ctabular*}\oarg{pos}\marg{width}\marg{preamble} rows \eenv{ctabular*} \\
\end{syntax}
The \Ie{ctabular} environment is similar to \Ie{tabular}, but with a few
differences, the main one 
being that the table will merrily continue across page breaks.
The \meta{preamble} argument is the same as for the previous \Ie{array}
and \Ie{tabular} environments, but the optional \meta{pos} argument
controls the horizontal position of the table, not the vertical. The
possible argument values are 
\texttt{l} (left justified), 
\texttt{c} (centered), or
\texttt{r} (right justified); the default is \texttt{c}.

\begin{lcode}
\begin{ctabular}{lcr}  \toprule
LEFT & CENTER & RIGHT \\  \midrule
l & c & r \\
l & c & r \\
l & c & r \\
l & c & r \\  \bottomrule
\end{ctabular}
\end{lcode}
  
\begin{ctabular}[c]{lcr}  \toprule
LEFT & CENTER & RIGHT \\  \midrule
l & c & r \\
l & c & r \\
l & c & r \\
l & c & r \\  \bottomrule
\end{ctabular}

    The \Ie{ctabular} environment will probably not be used within
a \Ie{table} environment (which defeats the possibility of the table
crossing page boundaries). To caption a \Ie{ctabular} you can define a 
fixed caption. For example:
\begin{lcode}
\newfixedcaption{\freetabcaption}{table}
\end{lcode}
And then \cmd{\freetabcaption} can be used like the normal \cmd{\caption}
within a \Ie{table} float.
\newfixedcaption{\freetabcaption}{table}

\subsection{Automatic tabulars}

    A tabular format may be used just to list things, for example the
names of the members of a particular organisation, or the names of
\ltx{} environments. 

    Especially when drafting a document, or when the number of entries
is likely to change, it is convenient to be able to tabulate a list
of items without having to explicitly mark the end of each row.

\begin{syntax}
\cmd{\autorows}\oarg{width}\marg{pos}\marg{num}\marg{style}\marg{entries} \\
\end{syntax}
The \cmd{\autorows} macro lists the \meta{entries} in rows; that is,
the entries are typeset left to right and top to bottom. 
The table below was set by \cmd{\autorows} using:
\begin{lcode}
\freetabcaption{Example automatic row ordered table}
\autorows{c}{5}{c}{one, two, three, four, five,
                   six, seven, eight, nine, ten,
                   eleven, twelve, thirteen, fourteen }
\end{lcode}

\freetabcaption{Example automatic row ordered table}
\autorows{c}{5}{c}{one, two, three, four, five,
                   six, seven, eight, nine, ten,
                   eleven, twelve, thirteen, fourteen }

    The \meta{pos} (\texttt{l}, \texttt{c}, or \texttt{r})
argument controls the horizontal position of the
tabular and \meta{num} is the number of columns. The \meta{style}
(\texttt{l}, \texttt{c}, or \texttt{r}) argument specifies the location
of the entries in the columns; each column is treated identically.
The \meta{entries} argument is a comma-separated list of the names to be
tabulated; there must be no comma after the last of the names before the
closing brace.

 Each column is normaly the same width which is large enough to accomodate
the widest entry in the list.
A positive \meta{width} (e.g., \verb?[0.8\textwidth]?), defines the overall
width of the table, and the column width is calculated by dividing \meta{width}
by the number of columns. Any negative value for the \meta{width} width lets 
each column be wide enough for the widest entry in that column; the column width 
is no longer a constant. 

   The next examples illustrate the effect of the \meta{width} argument
(the default value is 0pt).

\begin{lcode}
\freetabcaption{Changing the width of a row ordered table}
\autorows[-1pt]{c}{5}{c}{one, two, three, four, five,
                   six, seven, eight, nine, ten,
                   eleven, twelve, thirteen, fourteen }
\autorows[0pt]{c}{5}{c}{one, two, three, four, five, ... }
\autorows[0.9\textwidth]{c}{5}{c}{one, two, three, four, five, ... }
\end{lcode}

\freetabcaption{Changing the width of a row ordered table}
\autorows[-1pt]{c}{5}{c}{one, two, three, four, five,
                   six, seven, eight, nine, ten,
                   eleven, twelve, thirteen, fourteen }

 \autorows[0pt]{c}{5}{c}{one, two, three, four, five,
                   six, seven, eight, nine, ten,
                   eleven, twelve, thirteen, fourteen }

\autorows[0.9\textwidth]{c}{5}{c}{one, two, three, four, five,
                   six, seven, eight, nine, ten,
                   eleven, twelve, thirteen, fourteen }


\begin{syntax}
\cmd{\autocols}\oarg{width}\marg{pos}\marg{num}\marg{style}\marg{entries} \\
\end{syntax}
The \cmd{\autocols} macro lists its \meta{entries} in columns, proceeding
top to bottom and left to right. The arguments, except for \meta{width},
are the same as for \cmd{\autorows}. The column width is always constant throughout
the table and is normally sufficient for the
widest entry. A positive \meta{width} has the same effect as for \cmd{\autorows}
but a negative value is ignored.

    If you need to include a comma within one of the entries in the list
for either \cmd{\autorows} or \cmd{\autocols} you have to use a macro. 
For instance:
\begin{lcode}
\newcommand*{\comma}{,}
\end{lcode}
\newcommand*{\comma}{,}

The following examples illustrate these points.

\begin{lcode}
\freetabcaption{Changing the width of a column ordered table}
\autocols{c}{5}{c}{one\comma{} two, three, four, five,
                   six, seven, eight, nine, ten,
                   eleven, twelve, thirteen, fourteen }
\autocols[0.9\textwidth]{c}{5}{c}{one\comma{} two, three, four, five,
                   six, seven, eight, nine, ten,
                   eleven, twelve, thirteen, fourteen }
\end{lcode}

\freetabcaption{Changing the width of a column ordered table}
\autocols{c}{5}{c}{one\comma{} two, three, four, five,
                   six, seven, eight, nine, ten,
                   eleven, twelve, thirteen, fourteen }

\autocols[0.9\textwidth]{c}{5}{c}{one\comma{} two, three, four, five,
                   six, seven, eight, nine, ten,
                   eleven, twelve, thirteen, fourteen }


\section{Spacing}

    Sometimes tabular rows appear vertically challenged.
\begin{syntax}
\lnc{\extrarowheight} \\
\end{syntax}
The length called \lnc{\extrarowheight} can be set to modify
the normal height of \Ie{tabular} (and \Ie{array}) rows. 
If it is a positive
 length, the value is added to the normal height of
 every row of the table, while
 the depth will remain the same. This is important for tables
 with horizontal lines because those lines normally touch the
 capital letters.
 For example, =\setlength{\extrarowheight}{1pt}= was used in the
definition of \tref{tab:tabpream}.

 \subsection{Special variations of \texttt{\textbackslash hline}}

 The family of \Ie{tabular} environments allows
 vertical positioning with respect to the baseline of
 the text in which the environment appears.  By default the
 environment appears centered, but this can be changed to
 align with the first or last line in the environment by
 supplying a \texttt{t} or \texttt{b} value to the
 optional position argument. However, this does not work
 when the first or last element in the environment is a
 \cmd{\hline} command---in that case the environment is
 aligned at the horizontal rule.

 \pagebreak[3]

 Here is an example:
 \begin{center}
 \begin{minipage}[t]{.4\linewidth}
 Tables
 \begin{tabular}[t]{l}
   with no\\ hline \\ commands \\ used
 \end{tabular} versus \\ tables
 \begin{tabular}[t]{|l|}
  \hline
   with some \\ hline \\ commands \\
  \hline
 \end{tabular} used.
 \end{minipage}
 \begin{minipage}[t]{.5\linewidth}
 \begin{verbatim}
 Tables
 \begin{tabular}[t]{l}
  with no\\ hline \\ commands \\ used
 \end{tabular} versus tables
 \begin{tabular}[t]{|l|}
  \hline
   with some \\ hline \\ commands \\
  \hline
 \end{tabular} used.
 \end{verbatim}
 \end{minipage}
\end{center}

\begin{syntax}
\cmd{\firsthline} \cmd{\lasthline} \\
\lnc{\extratabsurround} \\
\end{syntax}
 Using \cmd{\firsthline} and \cmd{\lasthline} will 
 cure the problem, and the tables will align properly as long
 as their first or last line does not contain extremely large
 objects.
 \begin{center}
 \begin{minipage}[t]{.4\linewidth}
 Tables
 \begin{tabular}[t]{l}
   with no\\ line \\ commands \\ used
 \end{tabular} versus \\ tables
 \begin{tabular}[t]{|l|}
  \firsthline
   with some \\ line   \\ commands \\
  \lasthline
 \end{tabular} used.
 \end{minipage}
 \begin{minipage}[t]{.5\linewidth}
 \begin{verbatim}
 Tables
 \begin{tabular}[t]{l}
   with no\\ line \\ commands \\ used
 \end{tabular} versus tables
 \begin{tabular}[t]{|l|}
  \firsthline
   with some \\ line   \\ commands \\
  \lasthline
 \end{tabular} used.
 \end{verbatim}
 \end{minipage}
 \end{center}

 The implementation of these two commands contains an extra
 dimension, which is called \cmd{\extratabsurround}, to add some
 additional space at the top and the bottom of such an environment.
 This is useful if such tables are nested.
 

 \subsection{Handling of rules}

 There are two possible approaches to the handling of horizontal and
 vertical rules in tables:
 \begin{enumerate}
   \item rules can be placed into the available space without
   enlarging the table, or
   \item rules can be placed between columns or rows thereby enlarging
   the table.
 \end{enumerate}
 This class implements the second possibility while the
 default implementation in the \ltx{} kernel implements the first
 concept.

   With standard \ltx{} adding rules to a table will not affect the
   width or height of the table (unless double rules are used), e.g.,
   changing a preamble from \verb=lll= to \verb=l|l|l= does not
   affect the document other than adding rules to the table. In
   contrast, with this class a table that just fitted the
   \verb=\textwidth= might now produce an overfull box. (But you shouldn't
   have vertical rules in the first place.)


%%%%%%%%%%%%%%%%%%%%%%%%
% end of tabman.tex
%%%%%%%%%%%%%%%%%%%%%%%

\DeleteShortVerb{\=}
\MakeShortVerb{\|}
%%%%%%%%%%%%%%%%%%%%%%%%%%%%%%%%%%%%%%%%


\chapter{Margin and foot notes} \label{chap:mnotes}

   The standard classes provide the \cmd{\marginpar} command for putting
things into the margin. The class supports two extra kinds of side notes.
It also provides extended footnoting capabilities.


\section{Footnotes}

    A footnote can be considered to be a special kind of float\index{float} 
that is put at the bottom of a page.

\begin{syntax}
\cmd{\footnote}\oarg{num}\marg{text} \\
\cmd{\footnotemark}\oarg{num} \cmd{\footnotetext}\oarg{num}\marg{text} \\
\end{syntax}
In the main text, the \cmd{\footnote} command puts a marker at the
point where it is called, and puts the \meta{text}, preceded by the same
mark, at the bottom of the page. If the optional \meta{num} is used
then its value is used for the mark, otherwise the \Icn{footnote}
 counter is stepped and provides the mark's value. 

    You can use \cmd{\footnotemark} to put a marker in the main text; the value
is determined just like that for \cmd{\footnote}. Footnote text can be put 
at the bottom of the page via \cmd{\footnotetext}; if the optional \meta{num}
is given it is used as the mark's value, otherwise the value of the
\Icn{footnote} counter is used.
   It may be helpful, but completely untrue, to think of \cmd{\footnote} being
defined like:
\begin{lcode}
\newcommand{\footnote}[1]{\footnotemark\footnotetext{#1}}
\end{lcode}


\begin{syntax}
\cmd{\footref}\marg{label} \\
\end{syntax}
    On occasions it may be desireable to make more than one reference
to the text of a footnote\index{footnote!reference}. This can be done by putting a \cmd{\label}
in the footnote and then using \cmd{\footref} to refer to the label; this
prints the footnote label. For example:
\begin{lcode}
...\footnote{... adults or babies.\label{fn:rabbits}}
...
... The footnote\footref{fn:rabbits} on \pref{fn:rabbits} ...
\end{lcode}
In this manual, the last line above prints:
\begin{syntax}
... The footnote\footref{fn:rabbits} on \pref{fn:rabbits} ... \\
\end{syntax}

    The parameters provided by the standard classes for controlling 
footnotes are illustrated in \fref{fig:fn}.

\begin{figure}
\centering
\drawparameterstrue
\setlayoutscale{0.4}
\drawfootnote
\caption{Footnote layout parameters for the standard classes}\label{fig:fn}
\end{figure}

\begin{syntax}
\lnc{\footnotesep} \\
\verb!\skip\footins! \\
\end{syntax}
The length \lnc{\footnotesep} controls the vertical spacing between footnotes\index{footnote}
(and thanks notes),
and is initialised by the class to give no extra spacing for a
\cmd{\footnotesize} font. You can change the spacing by 
\begin{lcode}
\addtolength{\footnotesep}{...}
\end{lcode}
The length |\skip\footins| defines the vertical spacing between the
bottom of the main text and the top of the first line of the first
footnote\index{footnote}. Likewise this can be changed by 
\begin{lcode}
\addtolength{\skip\footins}{...}
\end{lcode}
The total vertical distance between the bottom of the main text and
the baseline of the first line of the first footnote\index{footnote} is
|\footnotesep + \skip\footins|.

\begin{syntax}
\cmd{\footnoterule} \\
\end{syntax}
 The \cmd{\footnoterule} macro is defined in the LaTeX kernel and
redefined in the standard classes. An `\texttt{@}less' definition, which is slightly
less efficient than that in the classes, is:
\begin{lcode}
\renewcommand{\footnoterule}{%
   \kern -3pt                   % call this kerna
   \hrule height 0.4pt width 0.4\columnwidth
   \kern 2.6pt                  % call this kernb
}
\end{lcode}
This produces a horizontal rule from the left margin\index{margin} to 40\% of the
\cmd{\columnwidth} with thickness 0.4pt. The rule must not take up any
vertical space, which means that |kerna + kernb| must equal the thickness
of the rule. The rule is located a distance |\skip\footins + kerna|
below the bottom of the main text. So, to move the rule upwards, decrease
the value of |kerna| and increase that of |kernb|, and the reverse to
move the rule downwards.

    This class provides some additional parameters but for a wider
variety of footnote styles you may like to use the \Lpack{footmisc} 
package~\cite{FOOTMISC}.
    The additional class parameters have been chosen so as not to clash with
the \Lpack{footmisc} package. Similar concepts apply to the
\cmd{\thanks} command described in \Sref{sec:thanks} 
starting on \pref{sec:thanks}.

    The \cmd{\footnote} macro essentially does three things:
\begin{itemize}
\item Typesets a marker at the point where \cmd{\footnote} is called;
\item Typesets a marker at the bottom of the page on which \cmd{\footnote}
      is called;
\item Following the marker at the bottom of the page, typesets the text 
      of the footnote.
\end{itemize}

\begin{syntax}
\cmd{\@makefnmark} \\
\end{syntax}
The \cmd{\footnote} macro calls the kernel command \cmd{\@makefnmark} to
typeset the footnote marker at the point where \cmd{\footnote} is called
(the value of the marker is kept in the macro \cmd{\@thefnmark}). The default
definition typesets the mark as a superscript and is essentially
\begin{lcode}
\def\@makefnmark{\hbox{\textsuperscript{\@thefnmark}}}
\end{lcode}
To get, say, the marker to be typeset on the baseline in the normal font
and enclosed in brackets ---
\begin{lcode}
\renewcommand*{\@makefnmark}{ [\@thefnmark]}
\end{lcode}

\begin{syntax}
\cmd{\footfootmark} \\
\lnc{\footmarkwidth} \cmd{\footmarkstyle}\marg{arg} \\
\end{syntax}
The class macro for typesetting the marker at the foot of the page is
\cmd{\footfootmark}. The mark is set in a box of width
\lnc{\footmarkwidth}. If this is negative, the mark is outdented
into the margin, if zero the mark is flush left, and when positive
the mark is indented. The appearance of the mark is contolled by
\cmd{\footmarkstyle}. The default specification is
\begin{lcode}
\footmarkstyle{\textsuperscript{#1}}
\end{lcode}
where the |#1| indicates the position of \cmd{\@thefnmark} in the style.
The default results in the mark being set as a superscript.
For example, to have the marker set on the baseline 
and followed by a right parenthesis, do
\begin{lcode}
\footmarkstyle{#1) }
\end{lcode}

\begin{syntax}
\lnc{\footmarksep} \lnc{\footparindent} \\
\end{syntax}
The mark is typeset in a box of width \lnc{\footmarkwidth} and this is
then followed by the text of the footnote. Second and later lines of the
text are offset by the length \lnc{\footmarksep} from the end of the box.
The first line of a paragraph within a footnote is indented by
\lnc{\footparindent}. 

\begin{syntax}
\cmd{\foottextfont} \\
\end{syntax}
The text in the footnote is typeset using the \cmd{\foottextfont} font.
The default is \cmd{\footnotesize}.

    Altogether, the class specifies
\begin{lcode}
\footmarkstyle{\textsuperscript{#1}}
\setlength{\footmarkwidth}{1.8em}
\setlength{\footmarksep}{-1.8em}
\setlength{\footparindent}{1em}
\newcommand{\foottextfont}{\footnotesize}
\end{lcode}
to replicate the standard footnote layout. 

    You might like to try the
following combinations of \lnc{\footmarkwidth} and \lnc{\footmarksep}
to see if you prefer the layout produced by one of them to the standard 
layout (which is produced by the first pairing in the list below):
\begin{lcode}
\setlength{\footmarkwidth}{1.8em}  \setlength{\footmarksep}{-1.8em}
\setlength{\footmarkwidth}{1.8em}  \setlength{\footmarksep}{0em}
\setlength{\footmarkwidth}{0em}    \setlength{\footmarksep}{0em}
\setlength{\footmarkwidth}{-1.8em} \setlength{\footmarksep}{1.8em}
\setlength{\footmarkwidth}{0em} \setlength{\footmarksep}{1.8em} \footmarkstyle{#1)\hfill}
\end{lcode}
% width = 1.8, sep = 0 -> indented block para & hung number
% width = sep = 0 -> block paragraph at margin
% width = -1.8, sep = -width -> mark in margin, block para at margin


\begin{syntax}
\cmd{\makefootmarkhook} \\
\end{syntax}
Just before a footnote is typeset the macro \cmd{\makefootmarkhook} is
called, which by default does nothing, but you can renew it to do something.
For example:
\begin{lcode}
\renewcommand{\makefootmarkhook}{\raggedright}
\end{lcode}
will cause footnotes to be typeset raggedright.

    Any footnotes after this point will be set according to:
\begin{lcode}
\setlength{\footmarkwidth}{-1.0em}
\setlength{\footmarksep}{-\footmarkwidth}
\footmarkstyle{#1}
\end{lcode}
\setlength{\footmarkwidth}{-1.0em}
\setlength{\footmarksep}{-\footmarkwidth}
\footmarkstyle{#1}



\begin{syntax}
\cmd{\multfootsep} \\
\end{syntax}
In the standard classes if two or more footnotes are applied 
sequentially\footnote{One footnote}\footnote{Immediately followed by another}
then the markers in the text are just run together. The class, like the
\Lpack{footmisc}~\cite{FOOTMISC} and \Lpack{ledmac}~\cite{LEDMAC} packages, 
inserts a separator
between the marks. In the class the macro \cmd{\multfootsep} is used as
the separator. Its default definition is:
\begin{lcode}
\newcommand*{\multfootsep}{\textsuperscript{\normalfont,}}
\end{lcode}

\begin{syntax}
\cmd{\feetabovefloat} \\
\cmd{\feetbelowfloat} \\
\end{syntax}
In the standard classes, footnotes on a page that has a float at the
bottom are typeset before the float. I think that this looks
peculiar. Following the \cmd{\feetbelowfloat} declaration footnotes will be
typeset at the bottom of the page below any bottom floats; they will also
be typeset at the bottom of \cmd{\raggedbottom} pages as opposed to being
put just after the bottom line of text. The standard positioning is
used following the \cmd{\feetabovefloat} declaration, which is the default.

\begin{syntax}
\cmd{\plainfootnotes} \\
\cmd{\twocolumnfootnotes} \\
\cmd{\threecolumnfootnotes} \\
\cmd{\paragraphfootnotes} \\
\end{syntax}
Normally, each footnote starts a new paragraph. The class provides three
other styles, making four in all. Following the \cmd{\twocolumnfootnotes}
declaration footnotes will be typeset in two columns, and similarly
they are typeset in three columns after the \cmd{\threecolumnfootnotes}
declaration. Footnotes are run together as a single paragraph after the
\cmd{\paragraphfootnotes} declaration. The default style is used after
the \cmd{\plainfootnotes} declaration. 

   The style can be changed at any 
time but there may be odd effects if the change is made in the middle of
a page when there are footnotes before and after the declaration. You may
find it interesting to try changing styles in an article type document 
that uses \cmd{\maketitle} and \cmd{\thanks}, and some footnotes on the 
page with the title:
\begin{lcode}
\title{...\thanks{...}}
\author{...\thanks{...}...}
...
\begin{document}
\paragraphfootnotes
\maketitle
\plainfootnotes
...
\end{lcode}

\begin{syntax}
\cmd{\newfootnoteseries}\marg{series} \\
\cmd{\plainfootstyle}\marg{series} \\
\cmd{\twocolumnfootstyle}\marg{series} \\
\cmd{\threecolumnfootstyle}\marg{series} \\
\cmd{\paragraphfootstyle}\marg{series} \\
\end{syntax}

    The class provides for additional series of footnotes; perhaps the
normal footnotes are required, flagged with arabic numerals, and another 
kind of footnote flagged with roman numerals. A new footnote series is
created by the \cmd{\newfootseries} macro, where \meta{series} is an
alphabetic identifier for the series. This is most conveniently a 
single (upper case) letter, for example \texttt{P}. 

    Calling, say, \verb?\newfootnoteseries{Q}? creates a set of macros
equivalent to those for the normal \cmd{\footnote} but with the \meta{series}
appended. These include \cs{footnoteQ}, \cs{footnotemarkQ},
\cs{footnotetextQ} and so on. These are used just like the normal
\cmd{\footnote} and companions.

    By default, a series is set to typeset using the normal style
of a paragraph per note. The series' style can be changed by using one
of the \cs{...footstyle} commands.

    For example, to have a `P' (for paragraph) series using roman numerals 
as markers which, in the main text are superscript with a closing parenthesis
and at the foot are on the baseline followed by an endash, and the text is
set in italics at the normal footnote size:
\begin{lcode}
\newfootnoteseries{P}
\paragraphfootstyle{P}
\renewcommand{\thefootnoteP}{\roman{footnoteP}}
\footmarkstyleP{#1--}
\renewcommand{\@makefnmarkS}{\hbox{\textsuperscript{\@thefnmarkP)}}}
\renewcommand{\foottextfont}{\itshape\footnotesize}
\end{lcode}
This can then be used like:
\begin{lcode}
.... this sentence\footnoteP{A `p' footnote\label{fnp}} 
includes footnote~\footrefP{fnp}.
\end{lcode}
In a minipage a different counter (\texttt{mpfootn...}) is used for a footnote 
than in the main text (\texttt{footn...}).
By default its value is a letter. To change it to, say, a numeral for the
`P' series, do:
\begin{lcode}
\renewcommand{\thempfootnoteP}{\arabic{mpfootnoteP}}
\end{lcode}

   The \cmd{\newfootnoteseries} macro does not create series versions
of the footnote-related length commands, such as \lnc{\footmarkwidth}
and \lnc{\footmarksep}, nor does it create versions of \cmd{\footnoterule}.

   At the foot of the page footnotes are grouped according to their series;
all ordinary footnotes are typeset, then all the first series footnotes 
(if any), then the second series, and so on. The ordering corresponds to
the order of \cmd{\newfootnoteseries} commands.

\begin{syntax}
\cmd{\footfudgefiddle} \\
\end{syntax}
For paragraphed footnotes \tx{} has to estimate the amount of space they
will take. Unfortunately this \emph{is} an estimate and if it is an
underestimate then the footnotes will flow too low on the page, for example
below the page number. Increasing \cmd{\footfudgefiddle} from its default
vaue of 64, causes \tx{} to allot more space. For instance
\begin{lcode}
\renewcommand{\footfufgefiddle}{70}
\end{lcode}
will at least go some way to curing the problem. You will probably have
to do some experimentation to get a good value.

\begin{syntax}
\cmd{\fnsymbol}\marg{counter} \\
\cmd{\@fnsymbol}\marg{num} \\
\end{syntax}
The \cmd{\fnsymbol} macro typesets the representation of the counter \meta{counter}
like a footnote symbol. Internally it uses the kernel \cmd{\@fnsymbol} macro
which converts a positive integer \meta{num} to a symbol. If you are not fond of
the standard ordering of the footnote symbols, this is the macro to change. Its
original definition is:
\begin{lcode}
\def\@fnsymbol#1{\ensuremath{\ifcase#1\or *\or \dagger\or \ddagger\or
  \mathsection\or \mathparagraph\or \|\or **\or \dagger\dagger
  \or \ddagger\ddagger \else\@ctrerr\fi}}
\end{lcode}
This, as shown by \verb?\@fnsymbol{1},...\@fnsymbol{9}? produces the series: \makeatletter
\@fnsymbol{1},
\@fnsymbol{2},
\@fnsymbol{3},
\@fnsymbol{4},
\@fnsymbol{5},
\@fnsymbol{6},
\@fnsymbol{7},
\@fnsymbol{8}, and
\@fnsymbol{9}.
%\@fnsymbol{10}  % out of bounds
\makeatother 

\makeatletter
\let\m@mold@fnsymbol\@fnsymbol
\renewcommand{\@fnsymbol}[1]{\ensuremath{%
  \ifcase#1\or *\or \dagger\or \ddagger\or
  \mathsection\or \|\or \mathparagraph\or **\or \dagger\dagger
  \or \ddagger\ddagger \else\@ctrerr\fi}}
\makeatother
    Robert Bringhurst quotes 
the following as the traditional ordering (at least up
to \makeatletter\@fnsymbol{6}\makeatother): \makeatletter
\@fnsymbol{1},
\@fnsymbol{2},
\@fnsymbol{3},
\@fnsymbol{4},
\@fnsymbol{5},
\@fnsymbol{6},
\@fnsymbol{7},
\@fnsymbol{8}, and
\@fnsymbol{9}.
\makeatother 
You can obtain this sequence by redefining \cmd{\@fnsymbol} as:
\begin{lcode}
\renewcommand{\@fnsymbol}[1]{\ensuremath{%
  \ifcase#1\or *\or \dagger\or \ddagger\or
  \mathsection\or \|\or \mathparagraph\or **\or \dagger\dagger
  \or \ddagger\ddagger \else\@ctrerr\fi}}
\end{lcode}
not forgetting judicious use of \cmd{\makeatletter} and \cmd{\makeatother}
if you do this in the preamble\index{preamble}. Other authorities or publishers
may prefer other sequences and symbols.
\makeatletter\let\@fnsymbol\m@mold@fnsymbol\makeatother







\section{Verbatim footnotes}
    
    The macro \cmd{\verbfootnote} can typeset a footnote that includes
some verbatim text.

\begin{syntax}
\cmd{\verbfootnote}\oarg{num}\marg{text} \\
\end{syntax}

    The next two paragraph are typeset by this code:
\begin{lcode}
    Below, footnote\footref{fn1} is a \cmd{\footnote} while 
footnote~\ref{fn2} is a \cmd{\verbfootnote}.

    The \cmd{\verbfootnote} command should 
appear\footnote{There may be some problems if color is 
                used.\label{fn1}}
to give identical results as the normal \cmd{\footnote}, 
but it can include some verbatim 
text\verbfootnote{The \verb?\footnote? macro, like all others 
                  except for \verb?\verbfootnote?, can not 
                  contain verbatim text in its argument.\label{fn2}}
in the \meta{text} argument.
\end{lcode}

    Below, footnote\footref{fn1} is a \cmd{\footnote} while 
footnote~\ref{fn2} is a \cmd{\verbfootnote}.

    The \cmd{\verbfootnote} command should 
appear\footnote{There may be some problems if color is 
                used.\label{fn1}}
to give identical results as the normal \cmd{\footnote}, 
but it can include some verbatim 
text\verbfootnote{The \verb?\footnote? macro, like all others 
                  except for \verb?\verbfootnote?, can not 
                  contain verbatim text in its argument.\label{fn2}}
in the \meta{text} argument.



\section{Sidebars}

    It appears that the \cmd{\sidebar} command, which was originally noted
as being experimental, has been used successfully for a variety of tasks.

\begin{syntax}
\cmd{\sidebar}\marg{text} \\
\cmd{\sidebarfont} \\
\end{syntax}
The \cmd{\sidebar} command is like the \cmd{\marginpar} command in that
it typesets its \meta{text} argument, using the font specified
by \cmd{\sidebarfont}, in the margin. Unlike |\marginpar|,
the \meta{text} will start near the top of the page and may continue onto
later pages if it is too long to go on a single page.

   The default definition of \cmd{\sidebarfont} is 
\begin{lcode}
\newcommand{\sidebarfont}{\normalfont}
\end{lcode}


\begin{syntax}
\cmd{\sidebarform} \\
\end{syntax}

    Sidebars are normally narrow so text is set ragged right. More
accurately, the text is set according to \cmd{\sidebarform} which is defined
as:
\begin{lcode}
\newcommand{\sidebarform}{\rightskip=\z@ \@plus 2em}
\end{lcode}
which is ragged right but with less raggedness than \cmd{\raggedright}
would allow. As usual, you can change \cmd{\sidebarform}, for example:
\begin{lcode}
\renewcommand{\sidebarform}{}            % justified text
\end{lcode}

    You are likely to run into problems if the \cmd{\sidebar} command comes
near a pagebreak, or if the sidebar text gets typeset alongside main
text that has non-uniform line spacing (e.g., around a \cmd{\section}).
Also, the contents of sidebars may not be typeset if they are too near
to the end of the document.

\begin{syntax}
\lnc{\sidebarwidth} \\
\lnc{\sidebarhsep} \\
\lnc{\sidebarvsep} \\
\end{syntax}
The \meta{text} of a \cmd{\sidebar} is typeset in a column of width
\lnc{\sidebarwidth} and there will be a horizontal gap of length 
\lnc{\sidebarhsep} between the typeblock and the \meta{text}. The
length \lnc{\sidebarvsep} is the vertical gap between sidebars that fall
on the same page.

\begin{syntax}
\cmd{\setsidebarheight}\marg{length} \\
\end{syntax}
The macro \cmd{\setsidebarheight} can be used to specify the height
of a sidebar. The \meta{length} argument should be equivalent to
an integral number of lines. For example:
\begin{lcode}
\setsidebarheight{15\onelineskip}
\end{lcode}


Currently the alignment of the text in a sidebar with the main text
is not particularly good and you may want to do some 
experimentation (possibly through a combination of
\lnc{sidebarvsep} and \cmd{\setsidebarheight}) to make it 
better.\footnote{If you do improve the alignment, 
please let me know how you did it.}

\section{Side notes}

    The vertical position of side notes specified via \cmd{\marginpar}
is flexible so that adjacent notes are prevented from overlapping.

\begin{syntax}
\cmd{\sidepar}\oarg{left}\marg{right} \\
\lnc{\sideparvshift} \\
\end{syntax}

    The \cmd{\sidepar} macro is similar to \cmd{\marginpar} except that
it produces side notes that do not float --- they may overlap. The same
spacing is used for both \cmd{\marginpar} and \cmd{\sidepar}, namely
the lengths \lnc{\marginparsep} and \lnc{\marginparwidth}. The length
\lnc{\sideparvshift} can be used to make vertical adjustments to the
position of \cmd{\sidepar} notes. By default this is set to a value
of -2.08ex which seems to be the magical length that aligns the top of
the note with the text line.

    By default the \meta{right} argument is put in the right margin. When
the \Lopt{twoside} option is used the \meta{right} argument is put into
the left margin on the verso (even numbered) pages; however, for these pages
the optional \meta{left} argument is used instead if it is present. For
two column text the relevent argument is put into the `outer' margin with 
respect to the column.

\begin{syntax}
\cmd{\sideparswitchtrue} \cmd{\sideparswitchfalse} \\
\cmd{\reversesidepartrue} \cmd{\reversesideparfalse} \\
\cmd{\parnopar} \\
\end{syntax}

    The default placement can be changed by judicious use of the
\cs{reversidepar*} and \cs{sideparswitch*} declarations. For two sided
documents the default is \cmd{\sideparswitchtrue}, which specifies that
notes should be put into the outer margins; in one sided documents
the default is \cmd{\sideparswitchfalse} which specifies that notes
should always be put into the right hand margin. Following the
declaration \cmd{\reversesidepartrue} notes are put into opposite margin
than that described above 
(the default declaration is \cmd{\reversesideparfalse}).

    When LaTeX is deciding where to place the side notes it checks whether
it is on an odd or even page and sometimes TeX doesn't realise that it has just
moved onto the next page. Effectively TeX typesets paragraph by paragraph 
(including any side notes) and at the end of each paragraph sees if there
should have been a page break in the middle of the paragraph. If there was
it outputs the first part of the paragraph, inserts the page break, and retains
the second part of the paragraph, without retypesetting it, for eventual
output at the top of the new page. This means that side notes for any given
paragraph are in the same margin, either left or right. A side note at the
end of a paragraph may then end up in the wrong margin. The macro 
\cmd{\parnopar} forces a new paragraph but without appearing to (the first
line in the following paragraph follows immediately after the last element
in the prior paragraph with no line break). You can use \cmd{\parnopar}
to make TeX to do its page break calculation when you want it to, by splitting
what appears to be one paragraph into two paragraphs.

    Bastiaan Veelo has kindly provided example code for another
form of a side note, as follows.
\begin{lcode}
%% A new command that allows you to note down ideas or annotations in
%% the margin of the draft. If you are printing on a stock that is wider
%% than the final page width, we will go to some length to utilise the
%% paper that would otherwise be trimmed away, assuming you will not be
%% trimming the draft. These notes will not be printed when we are not
%% in draft mode.
\makeatletter
   \ifdraftdoc
     \newlength{\draftnotewidth}
     \newlength{\draftnotesignwidth}
     \newcommand{\draftnote}[1]{\@bsphack%
       {%% do not interfere with settings for other marginal notes
         \strictpagechecktrue%
         \checkoddpage%
         \setlength{\draftnotewidth}{\foremargin}%
         \addtolength{\draftnotewidth}{\trimedge}%
         \addtolength{\draftnotewidth}{-3\marginparsep}%
         \ifoddpage
           \setlength{\marginparwidth}{\draftnotewidth}%
           \marginpar{\flushleft\textbf{\textit{\HUGE !\ }}\small #1}%
         \else
           \settowidth{\draftnotesignwidth}{\textbf{\textit{\HUGE\ !}}}%
           \addtolength{\draftnotewidth}{-\draftnotesignwidth}%
           \marginpar{\raggedleft\makebox[0pt][r]{%% hack around
               \parbox[t]{\draftnotewidth}{%%%%%%%%% funny behaviour
                 \raggedleft\small\hspace{0pt}#1%
               }}\textbf{\textit{\HUGE\ !}}%
           }%
         \fi
       }\@esphack}
   \else
     \newcommand{\draftnote}[1]{\@bsphack\@esphack}
   \fi
\makeatother
\end{lcode}

    Bastiaan also noted that it provided an example of using the
\lnc{\foremargin} length.
    If you want to try it out, either put the code in your preamble,
or put it into a package (i.e., \file{.sty} file) without the 
\cs{makeat...} commands. 



%%%%%%%%%%%%%%%%%%%%%%%%%%%%%%%%%%
\clearpage
\pagestyle{companion}
\chapterstyle{companion}
\defaultsecnum
%%%%%%%%%%%%%%%%%%%%%%%%%%%%%%%%%%
\chapter{Signposts and decoration} \label{chap:signposts}


\newcommand{\tepi}[2]{\epigraph{#1}{#2}}

 \tepi{My soul, seek not the life of immortals; but enjoy to the full
       the resources that are within your reach}
      {\textit{Pythian Odes \\ Pindar}}

    This chapter is typeset using the \cstyle{companion} chapterstyle 
and the \pstyle{companion} pagestyle.

\section{Introduction}

    By now we have covered most aspects of typesetting. As far as
the class is concerned this chapter describes the remaining major element, 
namely headers\index{header} and footers\index{footer}, and the slightly more fun task of typesetting
epigraphs\index{epigraph}.


\section{Page styles}

    The class provides a selection of pagestyles that you can use and if 
they don't suit, then there are means to define your own.

    These facilities were inspired by the \Lpack{fancyhdr} 
package~\cite{FANCYHDR}, although the command set is different.

    The standard classes provide for a footer\index{footer} and header\index{header} for odd and even 
pages. Thus there are four elements to be specified for a pagestyle.
This class partitions the headers\index{header} and footers\index{footer} into left, center and right
portions, so there are a total of 12 elements that have to be specified
for a pagestyle. You may find, though, that one of the built in pagestyles
meets your needs so you don't have to worry about all these specifications.

\begin{syntax}
\cmd{\pagestyle}\marg{style} \\
\cmd{\thispagestyle}\marg{style} \\
\end{syntax}
   \cmd{\pagestyle} sets the current pagestyle to \meta{style}. On a particular
page \cmd{\thispagestyle} can be used to override the cuurent pagestyle for
the one page.

    Some of the class' commands automatically call \cmd{\thispagestyle}:
\begin{itemize}
\item \cmd{\part} calls |\thispagestyle{part}|; 
\item \cmd{\chapter} calls |\thispagestyle{chapter}|; 
\item if \cmd{\cleardoublepage} will result in an empty verso page it calls
    |\thispagestyle{cleared}| for the empty page.
\end{itemize}

    The page styles provided by the class are:
\begin{itemize}
\item[\pstyle{empty}] The headers\index{header} and footers\index{footer} 
  are empty.
\item[\pstyle{plain}] The header\index{header} is empty and the 
  folio\index{folio} (page number) is centered at the bottom of the page.
\item[\pstyle{headings}] The footer\index{footer} is empty and the 
  header\index{header} contains the the folio\index{folio} at the outer 
  side of the page, and either the chapter or section title. 
  Chapter~\ref{chap:starting} uses this style.
\item[\pstyle{myheadings}] This is like the \pstyle{headings} style except you
     have to specify the titles, if any, to go into the header\index{header}.
\item[\pstyle{ruled}] The footer\index{footer} contains the 
  folio\index{folio} at the outside. The header\index{header} contains 
  either the chapter or section titles, and a line is drawn underneath the 
  header\index{header}. 
  This style is used in \Cref{chap:layingpage}.
\item[\pstyle{Ruled}] This is like the \pstyle{ruled} style except that 
  the headers\index{header} and footers\index{footer} extend into the 
  \foredge{} margin\index{margin}.
  Chapter~\ref{chap:verse} uses this style.
\item[\pstyle{companion}] This is like the pagestyle in the 
     \textit{Companion} series (e.g., see~\cite{GOOSSENS94}).
    For an example, see \Cref{chap:signposts}.
\item[\pstyle{part}] This is the same as the \pstyle{plain} pagestyle.
\item[\pstyle{chapter}]  This is the same as the \pstyle{plain} pagestyle.
\item[\pstyle{cleared}]  This is the same as the \pstyle{empty} pagestyle.
\item[\pstyle{title}]   This is the same as the \pstyle{plain} pagestyle.
\item[\pstyle{titlingpage}] This is the same as the \pstyle{empty} pagestyle.
\end{itemize}

    For the \pstyle{headings} pagestyle above, you have to define your own
titles to go into the header\index{header}. Each sectioning command, say |\sec|, 
calls a macro called |\secmark|. The pagestyles usually define this command
so that it picks up the title, and perhaps the number, of the |\sec|. 

\begin{syntax}
\cmd{\markboth}\marg{left}\marg{right} \\
\cmd{\markright}\marg{right} \\
\end{syntax}
    \cmd{\markboth} sets the values of two \emph{markers}\index{markers}
to \meta{left} and \meta{right} respectively, at the point in the text 
where it is called. Similarly, \cmd{\markright} sets the value of a
marker to \meta{right}.

\begin{syntax}
\cmd{\leftmark} \cmd{\rightmark} \\
\end{syntax}
The macro \cmd{\leftmark} contains the value of the \meta{left} argument
of the \emph{last} \cmd{\markboth} on the page. The macro \cmd{\rightmark}
contains the value of the \meta{right} argument of the \emph{first}
\cmd{\markboth} or \cmd{\markright} on the page, or if there is not one it
contains the value of the most recent \meta{right} argument.

    So, a pagestyle can define the |\secmark| commands in terms of 
\cmd{\markboth} or \cmd{\markright}, and then use \cmd{\leftmark} and/or
\cmd{\rightmark} in the headers\index{header} or footers\index{footer}. I'll show examples of how this
works later, and this is often how the \pstyle{myheadings} style gets 
implemented.


\section{Making headers and footers}

    As mentioned, the class provides for left, center, and right slots in
even and odd headers\index{header} and footers\index{footer}. This section describes how you can make
your own pagestyle using these 12 slots.

\begin{syntax}
\cmd{\makepagestyle}\marg{style} \\
\cmd{\aliaspagestyle}\marg{alias}\marg{original} \\
\cmd{\copypagestyle}\marg{new}\marg{original} \\
\end{syntax}
The command \cmd{\makepagestyle} specifies a pagestyle \meta{style} which
is initially equivalent to the \pstyle{empty} pagestyle. 
The \cmd{\aliaspagestyle} macro defines the \meta{alias} pagestyle to be 
the same as the \meta{original} pagestyle. As an example of the latter, 
the class includes the code
\begin{lcode}
\aliaspagestyle{part}{plain}
\aliaspagestyle{chapter}{plain}
\aliaspagestyle{cleared}{empty}
\end{lcode}
The macro \cmd{\copypagestyle} creates a \meta{new} pagestyle with the 
definitions being copies of those for the \meta{original} pagestyle.


\begin{syntax}
\cmd{\makeevenhead}\marg{style}\marg{left}\marg{center}\marg{right} \\
\cmd{\makeoddhead}\marg{style}\marg{left}\marg{center}\marg{right} \\
\cmd{\makeevenfoot}\marg{style}\marg{left}\marg{center}\marg{right} \\
\cmd{\makeoddfoot}\marg{style}\marg{left}\marg{center}\marg{right} \\
\end{syntax}
The macro \cmd{\makeevenhead} defines the \meta{left}, \meta{center}, and
\meta{right} portions of the \meta{style} pagestyle header\index{header} for even numbered
(verso) pages. Similarly \cmd{\makeoddhead}, \cmd{\makeevenfoot}, and
\cmd{\makeoddfoot} define the \meta{left}, \meta{center} and \meta{right}
portions of the \meta{style} header\index{header} for odd numbered (recto) pages,
and the footers\index{footer} for verso and recto pages. These commmands for \meta{style}
should be used after the corresponding \cmd{\makepagestyle} for \meta{style}.

\begin{syntax}
\cmd{\makerunningwidth}\marg{style}\marg{length} \\
\lnc{\headwidth} \\
\end{syntax}
The macro \cmd{\makerunningwidth} sets the width of the \meta{style}
pagestyle headers\index{header} and footers\index{footer} to be \meta{length}. The \cmd{\makepagestyle}
sets the width to be the textwidth, so the macro need only be used if some
other width is desired. The length \lnc{\headwidth} is provided as a
(scratch) length that may be used for headers\index{header} or footers\index{footer}, or any other
purpose.

\begin{syntax}
\cmd{\makeheadrule}\marg{style}\marg{width}\marg{thickness} \\
\cmd{\makefootrule}\marg{style}\marg{width}\marg{thickness}\marg{skip} \\
\end{syntax}
A header\index{header} may have a rule drawn between it and the top of the typeblock\index{typeblock},
and similarly a rule may be drawn between the bottom of the typeblock\index{typeblock}
and the footer\index{footer}. The \cmd{\makeheadrule} macro specifies the the \meta{width}
and \meta{thickness} of the rule below the \meta{style} pagestyle header\index{header}, and
the \cmd{\makefootrule} does the same for the rule above the footer\index{footer}; the
additional \meta{skip} argument is a distance that specifies the vertical
positioning of the foot rule (see \cmd{\footruleskip}).
The \cmd{\makepagestyle} macro initialises the \meta{width} to the 
\lnc{\textwidth} and the \meta{thickness} to 0pt, so by default no rules
are visible.

\begin{syntax}
\lnc{\normalrulethickness} \\
\end{syntax}
\lnc{\normalrulethickness} is the normal thickness of a visible rule, by 
default 0.4pt. It can be changed using \cmd{\setlength}, although I suggest 
that you do not unless perhaps when using the \Lopt{14pt} or \Lopt{17pt}
class options. 

\begin{syntax}
\cmd{\footruleheight} \\
\cmd{\footruleskip} \\
\end{syntax}
The macro \cmd{\footruleheight} is the height of a normal
rule above a footer\index{footer} (default zero). \cmd{\footruleskip} is a distance 
sufficient to ensure that a foot rule will be placed between the bottom
of the typeblock\index{typeblock} and the footer\index{footer}. Despite appearing to be lengths, if
you really need to change the values use \cmd{\renewcommand}, 
not \cmd{\setlength}.

\begin{syntax}
\cmd{\makeheadposition}\marg{style}\\
    \marg{eheadpos}\marg{oheadpos}\marg{efootpos}\marg{ofootpos} \\
\end{syntax}
The \cmd{\makeheadposition} macro specifies the horizontal positioning
of the even and odd headers\index{header} and footers\index{footer}, respectively, for the \meta{style}
pagestyle. Each of the \meta{...pos} arguments may be |flushleft|, |center|,
or |flushright|, with the obvious meanings. An empty, or unrecognised, argument
is equivalent to |center|. This macro is really only of use if the 
header/footer\index{header}\index{footer} width is not the same as the \lnc{\textwidth}.

\begin{syntax}
\cmd{\makepsmarks}\marg{style}\marg{code} \\
\end{syntax}
The last thing that the \cmd{\pagestyle}\marg{style} does is call the
\meta{code} argument of the \cmd{\makepsmarks} for \meta{style}.
This is normally used for specifying non-default code 
(i.e., code not specifiable via any of the previous macros) for the particular
pagestyle. The code normally defines the marks, if any, that will be used in
the headers\index{header} and footers\index{footer}.


\subsection{Example pagestyles}

    This first example demonstrates most of the page styling commands.
In the \textit{LaTeX Companion} series of 
books~\cite{GOOSSENS94,GOOSSENS97,GOOSSENS99} the header\index{header} is wider 
than the typeblock\index{typeblock}, sticking out into the foredge margin\index{margin}, and has a rule 
underneath it. The page number is in bold and at the outer end of the header\index{header}.
Chapter titles are in verso headers\index{header} and section titles in recto headers\index{header}, both
in bold font and at the spine margin\index{margin}. The footers\index{footer} are empty.

    The first thing to do in implementing this style is to calculate 
the width of the headers\index{header}, which extend to cover any marginal\index{marginalia} notes.
\begin{lcode}
\setlength{\headwidth}{\textwidth}
  \addtolength{\headwidth}{\marginparsep}
  \addtolength{\headwidth}{\marginparwidth}
\end{lcode}
Now we can set up an empty \pstyle{companion} pagestyle and start to change
it by specifying the new header\index{header} and footer\index{footer} width:
\begin{lcode}
\makepagestyle{companion}
\makerunningwidth{companion}{\headwidth}
\end{lcode}
and specify the width and thickness for the header\index{header} rule, otherwise it will 
be invisible.
\begin{lcode}
\makeheadrule{companion}{\headwidth}{\normalrulethickness}
\end{lcode}

    In order to get the header\index{header} to stick out into the foredge margin\index{margin}, 
verso headers\index{header} have to be flushright (raggedleft) and recto headers\index{header}
to be flushleft (raggedright). As the footers\index{footer} are empty, their position
is immaterial.
\begin{lcode}
\makeheadposition{companion}{flushright}{flushleft}{}{}
\end{lcode}

    The current chapter and section titles are obtained from the 
\cmd{\leftmark} and \cmd{\rightmark} macros which are defined via the
\cmd{\chaptermark} and \cmd{\sectionmark} macros. Remember that
\cmd{\leftmark} is the last \meta{left} marker and \cmd{\rightmark}
is the first \meta{right} marker\index{markers} on the page.

Chapter numbers are not
put into the header\index{header} but the section number, if there is one, is put into
the header\index{header}. We have to make sure that
the correct definitions are used for these, and this is where the 
\cmd{\makepsmarks} macro comes into play.
\begin{lcode}
\makepsmarks{companion}{%
  \let\@mkboth\markboth
  \def\chaptermark##1{\markboth{##1}{##1}}% % left & right marks
  \def\sectionmark##1{\markright{%          % right mark
    \ifnum \c@secnumdepth>\z@
      \thesection. \%                       % section number
    \fi
    ##1}}
}
\end{lcode}

    The preliminaries have all been completed, and it just remains to specify
what goes into each header\index{header} and footer\index{footer} slot (but the footers\index{footer} are empty).
\begin{lcode}
\makeevenhead{companion}%
  {\normalfont\bfseries\thepage}{}{\normalfont\bfseries\leftmark}
\makeoddhead{companion}%
  {\normalfont\bfseries\rightmark}{}{\normalfont\bfseries\thepage}
\end{lcode}

    Now issuing the command |\pagestyle{companion}| will produce pages 
typeset with \pstyle{companion} pagestyle headers\index{header}.

    This definition is actually part of the class. If you want to alter it, or
use it as the basis for a style of your own, remember the caveats about
macros with |@| in their names.

\subsection{Special headers}

    If you look at the Index\index{index} you will see that the header\index{header} shows the first and
last entries on the page.
A main entry in the index\index{index} looks like:
\begin{lcode}
\item \idxmark{entry}, page number(s)
\end{lcode}
and in the preamble\index{preamble} to this manual \cmd{\idxmark} is defined as
\begin{lcode}
\newcommand{\idxmark}[1]{#1\markboth{#1}{#1}}
\end{lcode}
This typesets the entry and also uses the entry as markers so that
the first entry on a page is held in \cmd{\rightmark} and the last
is in \cmd{\leftmark}.

    As index\index{index} entries are usually very short, the Index\index{index} is set in 
two columns\index{column!double}. Unfortunately LaTeX's marking mechanism can be a little
fragile on twocolumn\index{column!double} pages, but the standard 
\Lpack{fixltx2e} package~\cite{FIXLTX2E} corrects this.

    The index\index{index} itself is called by\index{indexing}
\begin{lcode}
\clearpage
\pagestyle{index}
\renewcommand{\preindexhook}{%
The first page number is usually, but not always, the primary reference to
the indexed topic.\vskip\onelineskip}
\printindex
\end{lcode}

\makepagestyle{index}
  \makeheadrule{index}{\textwidth}{\normalrulethickness}
  \makeevenhead{index}{\rightmark}{}{\leftmark}
  \makeoddhead{index}{\rightmark}{}{\leftmark}
  \makeevenfoot{index}{\thepage}{}{}
  \makeoddfoot{index}{}{}{\thepage}

    The \pstyle{index} pagestyle, which is the crux of
this example, is defined here as:
\begin{lcode}
\makepagestyle{index}
  \makeheadrule{index}{\textwidth}{\normalrulethickness}
  \makeevenhead{index}{\rightmark}{}{\leftmark}
  \makeoddhead{index}{\rightmark}{}{\leftmark}
  \makeevenfoot{index}{\thepage}{}{}
  \makeoddfoot{index}{}{}{\thepage}
\end{lcode}
This, as you can hopefully see, puts the first and last index\index{index} entries
on the page into the header\index{header} at the left and right, with the folios\index{folio}
in the footers\index{footer} at the \foredge{} margin\index{margin}.

    I used Pehong Chen's 
\textit{MakeIndex}\index{MakeIndex?\textit{MakeIndex}} program~\cite{CHEN88} 
for converting the raw index\index{index}
data in the \file{.idx} file into the form expected by LaTeX in 
the \file{.ind} file. \textit{MakeIndex} can be configured via an
\file{.ist} file, and the \file{memman.ist} file that I created for
this manual includes the mysterious lines
\begin{lcode}
% output main entry <entry> as: \item \idxmark{<entry>},
item_0  "\n\\item \\idxmark{"
delim_0 "},"
\end{lcode}
Look up either Chen's paper~\cite{CHEN88} or Chapter~12 in~\cite{GOOSSENS94}
for an explanation of this cryptic code.

\index{float!page|(}

\begin{syntax}
\piif{ifonlyfloats}\marg{yes}\marg{no} \\
\end{syntax}
    There are occasions when it is desireable to have different headers\index{header} on
pages that only contain figures\index{figure} or tables\index{table}. If the command \piif{ifonlyfloats}
is issued on a page that contains no text and only floats then the \meta{yes}
argument is processed, otherwise on a normal page the \meta{no} argument
is processed. The command is most useful when defining a pagestyle that 
should be different on a float-only page\index{page!of floats}.

    For example, assume that the \pstyle{companion} pagestyle is to be
generally used, but on float-only pages all that is required is a pagestyle
similar to \pstyle{plain}. Borrowing some code from the \pstyle{companion}
specification this can be accomplished like:
\begin{lcode}
\makepagestyle{floatcomp}
% \headwidth has already been defined for the companion style
\makeheadrule{floatcomp}{\headwidth}%
  {\ifonlyfloats{0pt}{\normalrulethickness}}
\makeheadposition{floatcomp}{flushright}{flushleft}{}{}
\makepsmarks{floatcomp}{\companionpshook}
\makeevenhead{floatcomp}{\ifonlyfloats{}{\normalfont\bfseries\thepage}}%
           {}{\ifonlyfloats{}{\normalfont\bfseries\leftmark}}
\makeoddhead{floatcomp}{\ifonlyfloats{}{\normalfont\bfseries\rightmark}}%
           {}{\ifonlyfloats{}{\normalfont\bfseries\thepage}}
\makeevenfoot{floatcomp}{}{\ifonlyfloats{\thepage}{}}{}
\makeoddfoot{floatcomp}{}{\ifonlyfloats{\thepage}{}}{}
\end{lcode}
The code above for the \pstyle{floatcomp} style should be compared with 
that for the earlier \pstyle{companion} style.

    The headrule is invisible on float pages by giving it zero thickness, 
otherwise it has the \cmd{\normalrulethickness}. The head position is 
identical for both pagestyles. However, the headers\index{header} are empty for
\pstyle{floatcomp} and the footers\index{footer} have centered page numbers 
on float pages; on ordinary pages the footers\index{footer} are empty while the headers\index{header}
are the same as the \pstyle{companion} headers\index{header}.

    The code includes one `trick'. The macro \cmd{\makepsmarks}|{X}{code}|
is equivalent to
\begin{lcode}
\newcommand{\Xpshook}{code}
\end{lcode}
I have used this knowledge in the line:
\begin{lcode}
\makepsmarks{floatcomp}{\companionpshook}
\end{lcode}
which avoids having to retype the code from |\makepsmarks{companion}{...}|,
and ensures that the code is actually the same for the two pagestyles.


\begin{syntax}
\cmd{\mergepagefloatstyle}\marg{style}\marg{textstyle}\marg{floatstyle} \\
\end{syntax}
    If you have two pre-existing pagestyles, one that will be used for
text pages and the other that can be used for float pages, then the
\cmd{\mergepagefloatstyle} command provides a simpler means of combining
them than the above example code for \pstyle{floatcomp}. The argument
\meta{style} is the name of the pagestyle being defined. The
argument \meta{textstyle}
is the name of the pagestyle for text pages and \meta{floatstyle} is the
name of the pagestyle for float-only pages. Both of these must have been 
defined before calling \cmd{\mergepagefloatstyle}. So, instead of the long
winded, and possibly tricky, code I could have simply said:
\begin{lcode}
\mergepagefloatstyle{floatcomp}{companion}{plain}
\end{lcode}

\index{float!page|)}


 \section{Epigraphs} 
\index{epigraph|(}

 \tepi{The whole is more than the sum of the parts.}
      {\textit{Metaphysica \\ Aristotle}}

 Some authors like to add an interesting quotation\index{quotation} at either the start
 or end of a chapter. The class provides commands
 to assist in the typesetting of a single epigraph. Other authors like to 
 add many such quotations\index{quotation} and the class provides environments to
 cater for these as well.
 Epigraphs can be typeset at either the left, the center or the right of 
 the typeblock\index{typeblock}. A few example eipgraphs are exhibited here, and
 others can be found in an article by
 Christina Thiele~\cite{TTC199} where she reviewed the \Lpack{epigraph}
package~\cite{EPIGRAPH} which is included in the class.

 \subsection{The \texttt{epigraph} command}

  The original inspiration for \cmd{\epigraph} was Doug Schenck's
 for the epigraphs in our book~\cite{EBOOK}. That was hard wired for
 the purpose at hand. The version here provides much more flexibility.


\begin{syntax}
\cmd{\epigraph}\marg{text}\marg{source} \\
\end{syntax}
  The command \cmd{\epigraph}
 typesets
 an epigraph using \meta{text} as the main text of the epigraph and
 \meta{source} being the original author (or book, article, etc.)
 of the quoted text. By default the epigraph is placed at the right hand 
 side of the typeblock\index{typeblock}, and the \meta{source} is typeset at the bottom
 right of the \meta{text}.


 \subsection{The \texttt{epigraphs} environment}

\begin{syntax}
\senv{epigraphs}  \\
  \cmd{\qitem}\marg{text}\marg{source} \\
  ... \\
\eenv{epigraphs} \\
\end{syntax}
 The \Ie{epigraphs} environment typesets a list of epigraphs, and by default
 places them at the right hand side of the typeblock\index{typeblock}.
  Each epigraph in an \Ie{epigraphs} environment is specified by a 
 \cmd{\qitem} (analagous to the \cmd{\item}
 command in ordinary list environments).
 By default, the \meta{source} is typeset at the bottom right of the
 \meta{text}. 

 

 \subsection{General}

 \tepi{Example is the school of mankind, and they will learn at no other.}
      {\textit{Letters on a Regicide Peace}\\ \textsc{Edmund Burke}}

   The commands described in this section apply to both the \cmd{\epigraph}
 command and the |epigraphs| environment. But first of all, note that an
 epigraph immediately after a heading\index{heading} will cause the first paragraph\index{paragraph!indentation}
 of the following text to be indented. If you want the initial paragraph
 to have no indentation, then start it with the \cmd{\noindent} command.

\begin{syntax}
 \lnc{\epigraphwidth} \\
\cmd{\epigraphtextposition}\marg{flush} \\
\end{syntax}
  The epigraphs are typeset in a minipage of width \lnc{\epigraphwidth}. 
The default
 value for this can be changed using the \cmd{\setlength} command. Typically,
 epigraphs are typset in a measure much less than the width of the typeblock\index{typeblock}.
 In order to avoid bad line breaks, the \meta{text} is normally typeset 
 raggedright. 

    The \meta{flush} argument to the \cmd{\epigraphtextposition} 
declaration
controls the \meta{text} typesetting style. By default this is 
\texttt{flushleft} (which produces raggedright text). The sensible values
are \texttt{center} for centered text, \texttt{flushright} for raggedleft
text, and \texttt{flushleftright} for normal justified text.

    If by any chance you want the \meta{text} to be typeset in some
other layout style, the easiest
way to do this is by defining a new environment which sets the paragraphing
parameters to your desired values. For example, as the \meta{text} is
typeset in a minipage, there is no paragraph indentation. If you
want the paragraphs to be indented and justified then define
a new environment like:
\begin{lcode}
\newenvironment{myparastyle}{\setlength{\parindent}{1em}}{}
\end{lcode}
 and use it as: 
\begin{lcode}
\epigraphtextposition{myparastyle}
\end{lcode}

\begin{syntax}
 \cmd{\epigraphposition}\marg{flush} \\
\end{syntax}
  As noted, the default position of epigraphs is at the right hand side
 of the typeblock\index{typeblock}. The positioning is 
controlled by 
the \meta{flush} argument to the \cmd{\epigraphposition} declaration.
The default value is \texttt{flushright}. This can be changed
 to \texttt{flushleft} for positioning at the left hand side or to
 \texttt{center} for positioning at the center of the typeblock\index{typeblock}.

\begin{syntax}
 \cmd{\epigraphsourceposition}\marg{flush} \\
\end{syntax}
 The \meta{flush} argument to the \cmd{\epigraphsourceposition}
declaration controls the position 
of the \meta{source}.
 The default value is \texttt{flushright}. It can be changed to
 \texttt{flushleft}, \texttt{center} or \texttt{flushleftright}.

 For example, to have epigraphs centered with the \meta{source} at the left,
 add the following to your document.
 \begin{lcode}
 \epigraphposition{center}
 \epigraphsourceposition{flushleft}
 \end{lcode}

\begin{syntax}
\cmd{\epigraphfontsize}\marg{fontsize} \\
\end{syntax}
 Epigraphs are often typeset in a smaller font than the main text. The
\meta{fontsize} argument to the \cmd{\epigraphfontsize}
declaration sets the font size to be used.
 If you don't like the default value (\cmd{\small}), you can easily change
it to, say \cmd{\footnotesize} by:
\begin{lcode}
\epigraphfontsize{\footnotesize}
\end{lcode}
		
\begin{syntax}
 \lnc{\epigraphrule} \\
\end{syntax}
 By default, a rule is drawn between the \meta{text} and \meta{source},
 with the rule thickness being given by the value of \lnc{\epigraphrule}.
 The value can be changed by using \cmd{\setlength}.
 A value of \texttt{0pt} will eliminate the rule. Personally, I dislike
 the rule in the list environments.

\begin{syntax}
 \lnc{\beforeepigraphskip} \\
 \lnc{\afterepigraphskip} \\
\end{syntax}
 The two |...skip| commands specify the amount of vertical space inserted
 before and after typeset epigraphs. Again, these can be changed by
 \cmd{\setlength}. It is desireable that the sum of their values should be an 
 integer multiple of the \lnc{\baselineskip}.

 Note that you can use normal LaTeX commands in the \meta{text} and
 \meta{source} arguments. You may wish to use different fonts for the
 \meta{text} (say roman) and the \meta{source} (say italic).

 The epigraph at the start of this section was specified as:
 \begin{lcode}
 \epigraph{Example is the school of mankind,
           and they will learn at no other.}
  {\textit{Letters on a Regicide Peace}\\ \textsc{Edmund Burke}}
 \end{lcode}

 \subsection{Epigraphs before chapter headings}

    The \cmd{\epigraph} command and the |epigraphs| environment typeset
 an epigraph at the point in the text where they are placed. The
 first thing that a \cmd{\chapter} command does is to start off a new page,
 so another mechanism is provided for placing an epigraph just before
 a chapter heading\index{heading!chapter}.
    
\begin{syntax}
 \cmd{\epigraphhead}\oarg{distance}\marg{text} \\
\end{syntax}
  The \cmd{\epigraphhead} macro  stores \meta{text} 
 for printing at \meta{distance} below the header\index{header} on a page.
 \meta{text} can be ordinary text or, more likely, can be either an
 \cmd{\epigraph} command or an |epigraphs| environment. By default, the 
 epigraph will be typeset at the righthand margin\index{margin}.
 If the command is immediately preceded by a \cmd{\chapter} or \cmd{\chapter*} 
 command, the epigraph is typeset on the chapter title page.

    The default value for the optional \meta{distance} argument is set so
 that an \cmd{\epigraph} consisting of a single line of quotation\index{quotation} and a single
 line denoting the source is aligned with the bottom of the `Chapter X'
 line produced by the \cmd{\chapter} command. In other cases you will
 have to experiment with the \meta{distance} value. The value for
 \meta{distance} can be either a integer or a real number. The units
 are in terms of the current value for \lnc{\unitlength}. A typical value
 for \meta{distance} for a single line quotation\index{quotation} and source for 
 a \cmd{\chapter*} might be about 70 (points). A positive value
 of \meta{distance} places the epigraph below the page heading and a negative
 value will raise it above the page heading.

    Here's some example code:
 \begin{lcode}
 \chapter*{Celestial navigation}
 \epigraphhead[70]{\epigraph{Star crossed lovers.}{\textit{The Bard}}}
 \end{lcode}


 The \meta{text} argument is put into a minipage of width \lnc{\epigraphwidth}.
 If you use something other than \cmd{\epigraph} or |epigraphs| for the
 \meta{text} argument, you may have to so some positioning of the text
 yourself so that it is properly located in the minipage. For example
 \begin{lcode}
 \chapter{Short}
 \renewcommand{\epigraphflush}{center}
 \epigraphhead{\centerline{Short quote}}
 \end{lcode}

 The \cmd{\epigraphhead} command changes the page style for the page on
 which it is specified, so there should be no text between the
 \cmd{\chapter} and the \cmd{\epigraphhead} commands. The page style
is identical to the \pstyle{plain} page style except for the inclusion of
the epigraph.
    If you want a more fancy style for epigraphed chapters you will have
to do some work yourself.

\begin{syntax}
\cmd{\epigraphforheader}\oarg{distance}\marg{text} \\
\cmd{\epigraphpicture} \\
\end{syntax}
The \cmd{\epigraphforheader} macro takes the same arguments as
\cmd{\epigraphhead} but puts \meta{text} into a zero-sized picture at
the coordinate position |(0,-<distance>)|; the macro \cmd{\epigraphpicture}
holds the resulting picture. This can then be used as part of a 
chapter pagestyle, as in
\begin{lcode}
\makepagestyle{mychapterpagestyle}
...
\makeoddhead{mychapterpagestyle}{}{}{\epigraphpicture}
\end{lcode}


\begin{syntax}
 \cmd{\dropchapter}\marg{length} \\
 \cmd{\undodrop} \\
\end{syntax}
 If a long epigraph is placed before a chapter title it is possible that the
 bottom of the epigraph may interfere with the chapter title. The command
 \cmd{\dropchapter} will lower any subsequent chapter titles by 
 \meta{length}; a negative \meta{length} will raise the titles.
 The command \cmd{\undodrop} restores subsequent chapter titles to their default
 positions. For example:
 \begin{lcode}
 \dropchapter{2in}
 \chapter{Title}
 \epigraphhead{long epigraph}
 \undodrop
 \end{lcode}

\index{epigraph|)}

\begin{syntax}
 \cmd{\cleartoevenpage}\oarg{text} \\
\end{syntax}
 On occasions it may be desirable to put something (e.g., an epigraph, a map,
 a picture) on the page facing the start
 of a chapter, where the something belongs to the chapter that is about to 
 start rather than the chapter that has just ended. In order to do this 
 in a document that is going to be printed
 doublesided, the chapter must start on an odd numbered page and the 
 pre-chapter material put on the immediately preceding even numbered page.
 The \cmd{\cleartoevenpage} command is like \cmd{\cleardoublepage} except
 that the page following the command will be an even numbered page, and the
 command takes an optional argument 
 which is applied to the skipped page (if any).

    Here is an example:
\begin{lcode}
 ... end previous chapter.
 \cleartoevenpage
 \begin{center}
 \begin{picture}... \end{picture}
 \end{center}
 \chapter{Next chapter}
\end{lcode}
 If the style is such that chapter headings\index{heading!chapter} are put at the top of the pages,
 then it would be advisable to include |\thispagestyle{empty}| (or |plain|)
 immediately after |\cleartoevenpage| to avoid a heading related to the
 previous chapter from appearing on the page. 

 If the something is like a figure\index{figure} with a numbered caption and the numbering
 depends on the chapter numbering, then the numbers have to be hand set (unless
 you define a special chapter command for the purpose). For example:
\begin{lcode}
 ... end previous chapter.
 \cleartoevenpage[\thispagestyle{empty}] % skipped page, if any, to be empty
 \thispagestyle{plain}
 \addtocounter{chapter}{1} % increment the chapter number
 \setcounter{figure}{0}    % initialise figure counter
 \begin{figure}
 ...
 \caption{Pre chapter figure}
 \end{figure}

 \addtocounter{chapter}{-1} % decrement the chapter number
 \chapter{Next chapter}     % increments chapter & initialises figure numbers
 \addtocounter{figure}{1}   % to account for pre-chapter figure
\end{lcode}
 

 \subsection{Epigraphs on part pages}

\index{epigraph|(}

     The \Lpack{epigraph} package as it stands cannot put an epigraph on the
 same page as a |\part| or |\part*| title page in 
 a \Lclass{book} or \Lclass{report} class. This is because the |\part| command
 internally does some page flipping before and after the title page.
 However, it is easy enough to put epigraphs on part\index{part} pages.

 \begin{itemize}
 \item Create a file called, say, \file{epipart.sty} which looks like this:
 \begin{verbatim}
 % epipart.sty
 \let\@epipart\@endpart
 \renewcommand{\@endpart}{\thispagestyle{epigraph}\@epipart}
 \endinput
 \end{verbatim}

 \item Start your document like:
 \begin{verbatim}
 \documentclass[...]{...}
 \usepackage{epigraph}
 \usepackage{epipart}
 \end{verbatim}

 \item Immediately \emph{before} each |\part| command put an 
 |\epigraphhead| command. For example:
 \begin{verbatim}
 \epigraphhead[300]{Epigraph text}
 \part{Part title}
 \end{verbatim}
 The value of the optional argument may need changing to vertically adjust
 the position of the epigraph. If there is any |\part| that does not have an
 epigraph then an empty |\epigraphhead| command (i.e., |\epigraphhead{}|)
 must be placed immediately before the |\part| command.

 \end{itemize}

     A similar scheme may be used for epigraphs on other kinds of pages. 
 The essential
 trick is to make sure that the \pstyle{epigraph} pagestyle is used for
 the page.
    For an epigraphed bibliography\index{bibliography} 
or index\index{index}, the macros \cmd{\prebibhook}
or \cmd{\preindexhook} can be appropriately modified to do this.

\index{epigraph|)}


%%%%%%%%%%%%%%%%%%%%%%%%%%%%%%%%%%%%%%%
\clearpage
\pagestyle{Ruled}
\chapterstyle{demo}
%%%%%%%%%%%%%%%%%%%%%%%%%%%%%%%%%%%%%%%%

\chapter{Typesetting verse} \label{chap:verse}

    This chapter uses the \cstyle{demo} chapterstyle and the \pstyle{Ruled}
pagestyle. 

\section{Introduction}

    The typesetting of a poem should be really be dependent on the
particular poem. Individual problems do not usually admit of a
general solution, so this manual and code should be used more
as a guide towards some solutions rather than providing a ready
made solution for any particular piece of verse.

    The doggerel used as illustrative material has been taken 
from~\cite{RUMOUR}.

    Note that for the examples in this section I have made no attempt
to do other than use the minimal (La)TeX capabilities; in particular
I have made no attempt to do any special page breaking so some stanzas
may cross onto the next page --- most undesireable for publication.

    The standard LaTeX classes provide the \Ie{verse} environment which 
is defined as a particular
kind of list. Within the environment you use \cmd{\\} to end a line, 
and a blank line will end a stanza. For example, here is a single
stanza poem:
\begin{lcode}
\newcommand{\garden}{
I used to love my garden \\
But now my love is dead \\
For I found a bachelor's button \\
In black-eyed Susan's bed.
}
\end{lcode}
\newcommand{\garden}{
I used to love my garden \\
But now my love is dead \\
For I found a bachelor's button \\
In black-eyed Susan's bed.
}
When this is typeset as a normal LaTeX paragraph\index{paragraph} (with no paragraph
indentation) it looks like: \\[\baselineskip]
\garden{}

\vspace*{\onelineskip}

   Typesetting it within the \Ie{verse} environment produces:% \\[\baselineskip]

\begin{verse}  
\garden
\end{verse}

%\ablankline

The stanza could also be typeset within the \Ie{alltt} environment, defined
in the standard \Lpack{alltt} package~\cite{ALLTT}, 
using a normal font and no \cmd{\\} line endings.
\begin{lcode}
\begin{alltt}\normalfont
I used to love my garden 
But now my love is dead 
For I found a bachelor's button 
In black-eyed Susan's bed.
\end{alltt}
\end{lcode}
which produces:

\begin{alltt}\normalfont
I used to love my garden 
But now my love is dead 
For I found a bachelor's button 
In black-eyed Susan's bed.
\end{alltt}

The \Ie{alltt} environment is like the \Ie{verbatim} environment except that
you can use LaTeX macros inside it. 
   In the \Ie{verse} environment long lines will be wrapped and indented
but in the \Ie{alltt} environment there is no indentation. 

Some stanzas have certain lines indented, often alternate ones. To
typeset stanzas like this you have to add your own spacing. For
instance:
\begin{lcode}
\begin{verse}
There was an old party of Lyme \\
Who married three wives at one time. \\
\hspace{2em} When asked: `Why the third?' \\
\hspace{2em} He replied: `One's absurd, \\
And bigamy, sir, is a crime.'
\end{verse}
\end{lcode}
is typeset as: 

\begin{verse}
There was an old party of Lyme \\
Who married three wives at one time. \\
\hspace{2em} When asked: `Why the third?' \\
\hspace{2em} He replied: `One's absurd, \\
And bigamy, sir, is a crime.'
\end{verse}

%\ablankline

Using the \Ie{alltt} environment you can put in the spacing via ordinary
spaces. That is, this:
\begin{lcode}
\begin{alltt}\normalfont
There was an old party of Lyme
Who married three wives at one time.
      When asked: `Why the third?' 
      He replied: `One's absurd, 
And bigamy, sir, is a crime.'
\end{alltt}
\end{lcode}
is typeset as

\begin{alltt}
\normalfont
There was an old party of Lyme
Who married three wives at one time.
      When asked: `Why the third?' 
      He replied: `One's absurd, 
And bigamy, sir, is a crime.'
\end{alltt}

More exotically you could use the TeX \cmd{\parshape} 
command\footnote{See the \textit{TeXbook} for how to use this.}:
\begin{lcode}
\parshape = 5 0pt \linewidth 0pt \linewidth 
              2em \linewidth 2em \linewidth 0pt \linewidth
\noindent There was an old party of Lyme \\
Who married three wives at one time. \\
When asked: `Why the third?' \\
He replied: `One's absurd, \\
And bigamy, sir, is a crime.' \par
\end{lcode}
which will be typeset as:

\vspace*{\onelineskip}

\parshape = 5 0pt \linewidth 0pt \linewidth 
              2em \linewidth 2em \linewidth 0pt \linewidth
\noindent There was an old party of Lyme \\
Who married three wives at one time. \\
When asked: `Why the third?' \\
He replied: `One's absurd, \\
And bigamy, sir, is a crime.' \par


\vspace*{\onelineskip}

   This is about as much assistance as standard (La)TeX provides, except
to note that in the \Ie{verse} environment the \cmd{\\*} version of \cmd{\\}
will prevent a following page break. You can also make judicious use
of the \cmd{\needspace} macro to keep things together.

   Some books of poetry, and especially anthologies, have two or more
indexes\index{index}, one, say for the poem titles and another for the first lines.
The \Lpack{index}~\cite{INDEX} and \Lpack{multind}~\cite{MULTIND}
packages provide support for multiple indexes\index{index} in one document.

%\clearpage
\section{Classy verse} 

    The code provided by the \Lclass{memoir} class is meant to help
with some aspects of typesetting poetry but does not, and cannot,
provide a comprehensive solution to all the requirements that
will arise.

    The main aspects of typesetting poetry that differ from typesetting
plain text are:
\begin{itemize}
\item Poems are usually visually centered on the page.
\item Some lines are indented, and often there is a pattern to the
      indentation.
\item When a line is too wide for the page it is broken and the
      remaining portion indented with respect to the original start
      of the line.
\end{itemize}
These are the ones that the class attempts to deal with.

\begin{syntax}
\senv{verse}\oarg{length} ... \eenv{verse} \\
\lnc{\versewidth} \\
\lnc{\stanzaskip} \\
\end{syntax}
The \Ie{verse} environment provided by the class is an extension
of the usual LaTeX environment. The environment takes one optional
parameter, which is a length; for example |\begin{verse}[4em]|.
You may have noticed that the earlier verse examples are all
near the left margin\index{margin}, whereas verses usually look better if they
are typeset about the center of the page. The length parameter,
if given, should be about the length of an average line, and then
the entire contents will be typeset with the mid point of the length
centered horizontally on the page.

The length \lnc{\versewidth} is provided as a convenience. It may be used,
for example, to calculate the length of a line of text for use
as the optional argument to the \Ie{verse} environment: 
\begin{lcode}
\settowidth{\versewidth}{This is the average line,}
\begin{verse}[\versewidth]
\end{lcode}

    The vertical space between stanzas is the length \lnc{\stanzaskip}.
It can be changed by the usual methods.

\begin{syntax}
\cmd{\\} \cmd{\\*} \cmd{\\>} \verb?\\!? \\
\end{syntax}
Each line in the \Ie{verse} environment, except possibly for the last line
in a stanza, 
must be ended with either \cmd{\\} or \cmd{\\*}. The starred
version prevents a page break before the following line.
    The \cmd{\\>} command causes a break in the line (see the 
description of the \cmd{\verselinebreak} macro). 

The \pixslashbang{}  macro
may be used at the end of a line to signal the end of the stanza. This
would normally be followed by a blank line before the next stanza.

    Each of the |\\...| macros take an optional length argument. In the
case of the \cmd{\\}, \cmd{\\*} and \pixslashbang{} macros it introduces
the specified amount of vertical space. For \cmd{\\>} it specifies
a horizontal space.

\begin{syntax}
\cmd{\vin} \\
\lnc{\vgap} \\
\lnc{\vindent} \\
\end{syntax}
The command \cmd{\vin} is shorthand for |\hspace*{\vgap}| for use
at the start of an indented line of verse. The length \lnc{\vgap}
(initially 1.5em) can be changed by \cmd{\setlength} or \cmd{\addtolength}.

    Verse lines are sometimes indented according to the space taken by
the text on the previous line.
\begin{syntax}
\cmd{\vinphantom}\marg{text} \\
\end{syntax}
The macro \cmd{\vinphantom} can be used at the start of a line of verse to
give an indentation as though the line started with \meta{text}. For example:
\begin{verse}
\ldots \\
Come away with me.
\end{verse}
\begin{verse}
\vinphantom{Come away with me.} Impossible! \\
\ldots
\end{verse}
The above fragment from a poem was typeset by:
\begin{lcode}
\begin{verse}
\ldots \\
Come away with me.
\end{verse}
\begin{verse}
\vinphantom{Come away with me.} Impossible! \\
\ldots
\end{verse}
\end{lcode}

    \cmd{\vinphantom} may also be used in the middle of any line to 
leave some blank space. For example, compare the two lines below, 
which are produced by this code:
\begin{lcode}
\noindent Come away with me and be my love --- Impossible. \\
          Come away with me \vinphantom{and be my love} --- Impossible.
\end{lcode}

\noindent Come away with me and be my love --- Impossible. \\
Come away with me \vinphantom{and be my love} --- Impossible.


When a verse line is too long to fit within the typeblock\index{typeblock} it is
wrapped onto the next line with a space, given by the value of the
length \lnc{\vindent}. The initial value of \lnc{\vindent} is twice 
\lnc{\vgap}.

\begin{syntax}
\cmd{\verselinebreak}\oarg{length} \\
\end{syntax}
Using the command \cmd{\verselinebreak} will cause 
later text in the
line of the verse to be typeset indented on the following line.
If the optional length argument is given then its value is added
to the normal indentation.
The broken
line will count as a single line as far as the \Ie{altverse}, 
\Ie{patverse}, and \Ie{patverse*} environments are concerned.
The \cmd{\\>} macro is shorthand for \cmd{\verselinebreak}.

\begin{syntax}
\senv{altverse} ... \eenv{altverse} \\
\end{syntax}
Within the \Ie{verse} environment stanzas are separated by a blank line
in the input. Individual stanzas within \Ie{verse} may, however, 
be enclosed in the \Ie{altverse} environment. This has the effect of
indenting the 2nd, 4th, etc., lines of the stanza by the length \lnc{\vgap}.

\begin{syntax}
\senv{patverse} ... \eenv{patverse} \\
\senv{patverse*} ... \eenv{patverse*} \\
\cmd{\indentpattern}\marg{digits} \\
\end{syntax}
As an alternative to the \Ie{altverse} environment, 
individual stanzas within the \Ie{verse} environment may be enclosed
in the \Ie{patverse} environment. Within this environment the indentation
of each line is specified by an indentation pattern, which consists
of an array of digits, $d_{1}$ to $d_{n}$, and the $n^{th}$ line is
indented by $d_{n}$ times |\vgap|. The first line is
not indented, irrespective of the value of $d_{1}$ and if the number of
lines is greater than the pattern length, the trailing lines are not indented.

    The \Ie{patverse*} environment is similar to \Ie{patverse} except
that the indentation pattern will keep repeating until the end of the
environment.

    The indentation pattern for a \Ie{patverse} environment is specified
via the \cmd{\indentpattern} command, where \meta{digits} is a string
of digits (e.g., |3213245281|). 

\begin{syntax}
\cmd{\linenumberfrequency}\marg{nth} \\
\cmd{\thepoemline} \\
\cmd{\linenumberfont}\marg{font-decl} \\
\lnc{\vrightskip} \\
\end{syntax}
    The lines in a poem may be numbered. The \cmd{\linenumberfrequency} 
declaration
specifies that every \meta{nth} line is to be numbered. If \meta{nth} is
less than 1 then line numbering is switched off, if \meta{nth} is 1 then 
every line will be numbered, and if \meta{nth} is, say 5, every fifth
line will be numbered. The default is |\linenumberfrequency{0}|.

    The counter for the lines is \Icn{poemline}, so the typeset form of the 
line number is specified via \cmd{\thepoemline}
which defaults to arabic numbers. The number is positioned
in the right hand margin at a distance \lnc{\vrightskip} 
(default 1em) from the edge
of the typeblock. The font used for the line numbers is specified
by \cmd{\linenumberfont}. The default definition is: \\
|\linenumberfont{\small\rmfamily}| \\
to produce small numbers in the roman font.


\begin{syntax}
\cmd{\flagverse}\marg{text} \\
\lnc{\vleftskip} \\
\end{syntax}
The \cmd{\flagverse} macro can be used at the start of a verse line, and
it typesets its \meta{text} argument at a distance \lnc{\vleftskip} 
(default 3em) to the
left of the verse line. This could be used, for example, to number stanzas.

\begin{syntax}
\cmd{\poemtitle}\oarg{short}\marg{long} \\
\cmd{\poemtitle*}\marg{long} \\
\cmd{\poemtitlefont}\marg{font} \\
\end{syntax}
\cmd{\poemtitle} typesets the title
of a poem and makes an entry into the \toc. The starred version,
\cmd{\poemtitle*}, makes no \toc{} entry.
The \cmd{\poemtitlefont} macro specifies the font and positioning of 
the poem title. Its initial definition is: \\
|\newcommand{\poemtitlefont}{\normalfont\bfseries\large\centering}| \\
to give a large bold centered title. This can of course be renewed
if you want something else.

\begin{syntax}
\lnc{\beforepoemtitleskip} \\
\lnc{\afterpoemtitleskip} \\
\end{syntax}
These two lengths are the vertical space before and after the 
\cmd{\poemtitle}
title text. They are initially defined to give the same spacing
as for a \cmd{\section} title. They can be changed by \cmd{\setlength} or
\cmd{\addtolength} for different spacings.

\begin{syntax}
\cmd{\poemtitlemark}\marg{title} \\
\end{syntax}
The \cmd{\poemtitle} macro, but not \cmd{\poemtitle*},
calls the \cmd{\poemtitlemark}\marg{title} macro,
which is defined to do nothing. This would probably be changed by a
pagestyle definition (like \cmd{\sectionmark} or \cmd{\chaptermark} may be
modified).

\begin{syntax}
\cmd{\poemtoc}\marg{sec} \\
\end{syntax}
The kind of entry made in the \toc{} by the \cmd{\poemtitle} command is
defined by \cmd{\poemtoc}. The initial definition is: \\
|\newcommand{\poemtoc}{section}| \\
for a section-like \toc{} entry. This can be changed to, say, |chapter|
or |subsection| or \ldots.


%\clearpage
\subsection{Examples}

   Here are some sample verses using the class facilities.

First our old Limerick friend, but titled and centered:
\begin{lcode}
\renewcommand{\poemtoc}{subsection}
\settocdepth{subsection}
\poemtitle{A Limerick}
\settowidth{\versewidth}{There was an old party of Lyme}
\begin{verse}[\versewidth]
There was an old party of Lyme \\
Who married three wives at one time. \\
\vin When asked: `Why the third?' \\
\vin He replied: `One's absurd, \\
And bigamy, sir, is a crime.'
\end{verse}
\end{lcode}
which gets typeset as below. The \cmd{\poemtoc} is redefined
to |subsection| so that the \cmd{\poemtitle} titles
are entered into the \toc{} as unnumbered subsections.
However, the \toc{} normally only lists sections and above, so
I also used \cmd{\settocdepth} to change this to subsections and 
above\footnote{It is changed back at the end of this chapter.}.

\renewcommand{\poemtoc}{subsection}
\settocdepth{subsection}
\poemtitle{A Limerick}
\settowidth{\versewidth}{There was an old party of Lyme}
\begin{verse}[\versewidth]
There was an old party of Lyme \\
Who married three wives at one time. \\
\vin When asked: `Why the third?' \\
\vin He replied: `One's absurd, \\
And bigamy, sir, is a crime.'
\end{verse}

%\ablankline

    Next is the Garden verse within the |altverse| environment. It
is titled and centered. 
\begin{lcode}
\settowidth{\versewidth}{But now my love is dead}
\poemtitle{Loves Lost}
\begin{verse}[\versewidth]
\begin{altverse}
\garden
\end{altverse}
\end{verse}
\end{lcode}
Note how the alternates lines are automatically indented in the 
typeset result below.

\settowidth{\versewidth}{But now my love is dead}
\poemtitle{Loves Lost}
\begin{verse}[\versewidth]
\begin{altverse}
\garden
\end{altverse}
\end{verse}

% \ablankline

It is left up to you how you might want to add information about
the author of a poem. Here is one example of a macro for this:
\begin{lcode}
\newcommand{\attrib}[1]{%
   \nopagebreak{\raggedleft\footnotesize #1\par}}
\end{lcode}
\newcommand{\attrib}[1]{%
   \nopagebreak{\raggedleft\footnotesize #1\par}}

   This can be used as in the next bit of doggerel.
\begin{lcode}
\poemtitle{Fleas}
\settowidth{\versewidth}{What a funny thing is a flea}
\begin{verse}[\versewidth]
What a funny thing is a flea. \\
You can't tell a he from a she. \\
But he can. And she can. \\
Whoopee!
\end{verse}
\attrib{Anonymous}
\end{lcode}

\poemtitle{Fleas}
\settowidth{\versewidth}{What a funny thing is a flea}
\begin{verse}[\versewidth]
What a funny thing is a flea. \\
You can't tell a he from a she. \\
But he can. And she can. \\
Whoopee!
\end{verse}
\attrib{Anonymous}

%\ablankline

The next example demonstrates the automatic line wrapping for overlong
lines.
\begin{lcode}
\poemtitle{In the Beginning}
\settowidth{\versewidth}{And objects at rest tended to remain at rest}
\begin{verse}[\versewidth]
Then God created Newton, \\
And objects at rest tended to remain at rest, \\
And objects in motion tended to remain in motion, \\
And energy was conserved
   and momentum was conserved
   and matter was conserved \\
And God saw that it was conservative.
\end{verse}
\attrib{Possibly from \textit{Analog}, circa 1950}
\end{lcode}

%%\enlargethispage{\baselineskip}
\poemtitle{In the Beginning}
\settowidth{\versewidth}{And objects at rest tended to remain at rest}
\begin{verse}[\versewidth]
Then God created Newton, \\
And objects at rest tended to remain at rest, \\
And objects in motion tended to remain in motion, \\
And energy was conserved
   and momentum was conserved
   and matter was conserved \\
And God saw that it was conservative.
\end{verse}
\attrib{Possibly from \textit{Analog}, circa 1950}

%\ablankline

The following verse demonstrates the use of a forced linebreak. It also
has a slightly different title style.
\begin{lcode}
\renewcommand{\poemtitlefont}{\normalfont\large\itshape\centering}
\poemtitle{Mathematics}
\settowidth{\versewidth}{Than Tycho Brahe, or Erra Pater:}
\begin{verse}[\versewidth]
In mathematics he was greater \\
Than Tycho Brahe, or Erra Pater: \\
For he, by geometric scale, \\
Could take the size of pots of ale;\\ \settowidth{\versewidth}{Resolve by}
Resolve, by sines \\>[\versewidth] and tangents straight, \\
If bread or butter wanted weight; \\
And wisely tell what hour o' the day \\
The clock does strike, by Algebra.
\end{verse}
\attrib{Samuel Butler (1612--1680)}
\end{lcode}

\renewcommand{\poemtitlefont}{\normalfont\large\itshape\centering}
\poemtitle{Mathematics}
\settowidth{\versewidth}{Than Tycho Brahe, or Erra Pater:}
\begin{verse}[\versewidth]
In mathematics he was greater \\
Than Tycho Brahe, or Erra Pater: \\
For he, by geometric scale, \\
Could take the size of pots of ale;\\ \settowidth{\versewidth}{Resolve by}
Resolve, by sines \\>[\versewidth] and tangents straight, \\
If bread or butter wanted weight; \\
And wisely tell what hour o' the day \\
The clock does strike, by Algebra.
\end{verse}
\attrib{Samuel Butler (1612--1680)}

%\ablankline
%\clearpage

Another limerick, but this time taking advantage of the |patverse|
environment. If you are typesetting a series of limericks you only
need specify one \cmd{\indentpattern} for all of them.
\begin{lcode}
\settowidth{\versewidth}{There was a young lady of Ryde}
\indentpattern{00110}
\poemtitle{The Young Lady of Ryde}
\begin{verse}[\versewidth]
\begin{patverse}
There was a young lady of Ryde \\
Who ate some apples and died. \\
The apples fermented \\
Inside the lamented \\
And made cider inside her inside. 
\end{patverse}
\end{verse}
\end{lcode}

\settowidth{\versewidth}{There was a young lady of Ryde}
\indentpattern{00110}
\poemtitle{The Young Lady of Ryde}
\begin{verse}[\versewidth]
\begin{patverse}
There was a young lady of Ryde \\
Who ate some apples and died. \\
The apples fermented \\
Inside the lamented \\
And made cider inside her inside. 
\end{patverse}
\end{verse}


The next example is a song that I have known since childhood. 
The `forty-niner' in line~\ref{vs:1} refers to the 1849 gold rush.

\clearpage

\begin{lcode}
\settowidth{\versewidth}{Oh my darling, Clementine}
\poemtitle{Clementine}
\begin{verse}[\versewidth]
\linenumberfrequency{3}
\flagverse{1.} In a cavern, in a canyon, \\
Excavating for a mine, \\
Lived a miner, forty-niner, \label{vs:1} \\
And his daughter, Clementine. \\!
\flagverse{\textsc{chorus}} Oh my darling, Oh my darling, \\
Oh my darling, Clementine. \\
Thou art lost and gone for ever. \\
Dreadful sorry, Clementine. \\!
\linenumberfrequency{0}
\end{verse}
\end{lcode}

\settowidth{\versewidth}{Oh my darling, Clementine}
\poemtitle{Clementine}
\begin{verse}[\versewidth]
\linenumberfrequency{3}
\flagverse{1.} In a cavern, in a canyon, \\
Excavating for a mine, \\
Lived a miner, forty-niner, \label{vs:1} \\
And his daughter, Clementine. \\!
\flagverse{\textsc{chorus}} Oh my darling, Oh my darling, \\
Oh my darling, Clementine. \\
Thou art lost and gone for ever. \\
Dreadful sorry, Clementine. \\!
\linenumberfrequency{0}
\end{verse}


 The last example is a much more ambitious use of |\indentpattern|. In
this case it is defined as: \\
|\indentpattern{0135554322112346898779775545653222345544456688778899}| \\
and the result is shown on the next page.


\clearpage
\poemtitle{Mouse's Tale}
\settowidth{\versewidth}{a mouse that morning}
\indentpattern{0135554322112346898779775545653222345544456688778899}
\begin{verse}[\versewidth]
\setlength{\vgap}{1em}
\begin{patverse}
\large Fury said to \\
  a mouse, That \\
  he met \\
  in the \\
  house, \\
\normalsize `Let us \\
  both go \\
  to law: \\
  \emph{I} will \\
  prosecute \\
  \textit{you.} --- \\
  Come, I'll \\
\small take no \\
  denial; \\
  We must \\
  have a \\
  trial: \\
  For \\
\footnotesize really \\
  this \\
  morning \\
  I've \\
  nothing \\
  to do.' \\
  Said the \\
  mouse to \\
\scriptsize the cur, \\
  Such a \\
  trial, \\
  dear sir, \\
  With no \\
  jury or \\
  judge, \\
  would be \\
  wasting \\
  our breath.' \\
\tiny  `I'll be \\
  judge, \\
  I'll be \\
  jury.' \\
  Said \\
  cunning \\
  old Fury; \\
  `I'll try \\
  the whole \\
  cause \\
  and \\
  condemn \\
  you \\
  to \\
  death.'  \par
\end{patverse}
\end{verse}
\attrib{Lewis Carrol, \textit{Alice's Adventures in Wonderland}, 1865}

%%%%%%%%%%%%%%%%%%%%%%%%%
\settocdepth{section}
%%%%%%%%%%%%%%%%%%%%%%%%

\chapter{Verbatims, boxes, and files}

\section{Introduction}

    This chapter describes some new facilities for handling verbatim
text; new boxes to go around things, including verbatims; and a more 
friendly interface for writing to and reading from files and in some
cases being able to read files verbatim, while putting them
into a box with line numbers.

\section{Verbatims}

    I have included in the class the code for the \Lpack{shortvrb}
package~\cite{SHORTVRB} and an extended version of the \Lpack{verbatim}
package~\cite{VERBATIM}. I have also borrowed from the \Lpack{moreverb}
package~\cite{MOREVERB}.

    When processing text verbatim LaTeX ignores any special meaning
that a character may have. As a special case of this, LaTeX ignores 
\emph{any} character in a comment.
\begin{syntax}
\cmd{\newcomment}\marg{comenv} \\
\end{syntax}
The macro \cmd{\newcomment} creates a new comment environment called
\meta{comenv}. All text within the \meta{comenv} environment will be ignored.
\begin{syntax}
\senv{comment} anything \eenv{comment} \\
\end{syntax}
The class provides one comment environment, namely the \Ie{comment} 
environment which is specified by
\begin{lcode}
\newcomment{comment}
\end{lcode}
This may be useful for `commenting out' large chunks of 
a source document.

\begin{syntax}
\cmd{\commentsoff}\marg{comenv} \\
\cmd{\commentson}\marg{comenv} \\
\end{syntax}
The declaration \cmd{\commentsoff} switches off the commenting within
the \meta{comenv} comment environment, and the declaration
\cmd{\commentson} switches on commenting within the \meta{comenv}
comment environment. In either case, \meta{comenv} must have been previously
defined as a comment environment via \cmd{\newcomment}.

    Suppose, for example, that you are preparing a draft document for 
review by some others and you want to include some notes for the reviewers.
Also, you want to include some private comments in the source for yourself.
You could use the \Ie{comment} environment for your private comments and
create another environment for the notes to the reviewers. These notes
should not appear in the final document. Your source might then look like:
\begin{lcode}
\newcomment{review}
\ifdraftdoc\else
  \commentsoff{review}
\fi
...
\begin{comment}
Remember to finagle the wingle!
\end{comment}
...
\begin{review}
\textit{REVIEWERS: Please pay particular attention to this section.}
\end{review}
...
\end{lcode}

    LaTeX provides the \cmd{\verb} command for
typesetting short pieces of text verbatim; `short' means less than one
line. If you have to type a lot of verbatim bits, like macro names,
it becomes very tedious to keep on doing \verb!\verb?\amacro?!.
The \cmd{\verb*} macro is similar to \cmd{\verb} except that it
typesets the symbol \verb*? ? in place of a space.

\begin{syntax}
\cmd{\MakeShortVerb}\marg{verbchar} \\
\cmd{\DeleteShortVerb}\marg{verbchar} \\
\end{syntax}

    With the \cmd{\MakeShortVerb} declaration you can use a single
character instead of |\verb|\meta{char}. The \meta{verbchar}
argument to \cmd{\MakeShortVerb} is a backslash followed by a single 
character, for example |\MakeShortVerb{\?}|. 
You can then use |?...?| instead of |\verb?...?|. To turn off
the use of |?| in this special manner, call |\DeleteShortVerb{\?}|.

\begin{syntax}
\senv{verbatim} anything \eenv{verbatim} \\
\senv{verbatim*} anything \eenv{verbatim*} \\
\end{syntax}
The \Ie{verbatim} and \Ie{verbatim*} environments are used for
typesetting any length of verbatim text.

\begin{syntax}
\cmd{\setverbatimfont}\marg{font-decl} \\
\end{syntax}
The font that is used for \emph{all} verbatim text is specified
by \cmd{\setverbatimfont}. The default is: \\
|\setverbatimfont{\normalfont\ttfamily}| \\
If you wanted verbatims to be set in a smaller size, then this will
do the trick:
\begin{lcode}
\setverbatimfont{\normalfont\small\ttfamily}
\end{lcode}

\begin{syntax}
\cmd{\tabson}\oarg{num} \\
\cmd{\tabsoff} \\
\end{syntax}
The standard \Ie{verbatim} environment ignores any TABs, or rather
treats a sequence of TABs as a single space. In the class,
if you use the declaration \cmd{\tabson} then TABs will not be ignored
in subsequent verbatims. The declaration \cmd{\tabsoff} turns off
tabbing inside verbatims. The default is |\tabsoff|. By default the class
uses four spaces for each tab. The optional \meta{num} argument to
\cmd{\tabson} will use \meta{num} spaces for each TAB.

\begin{syntax}
\cmd{\wrappingon} \\
\cmd{\wrappingoff} \\
\lnc{\verbatimindent} \\
\cmd{\verbatimbreakchar} \\
\end{syntax}
Very occasionally it may be useful to wrap long verbatim lines round onto
the following line. After the declaration \cmd{\wrappingon}, verbatim lines
that would extend into the margin will be wrapped (\cmd{\wrappingoff}
returns the behaviour to normal). The second and later parts of a 
wrapped line are indented by the length \lnc{\verbatimindent} which
can be altered in the usual manner. 

    It may be desireable to indicate that a line is continued on the 
next line. That is the role of \cmd{\verbatimbreakchar}. By default
this is defined as: \\
|\newcommand*{\verbatimbreakchar}{\char`\%}| \\
which typesets \% at the end of each line that is wrapped. To have
|/| at the end of each line instead, do:
\begin{lcode}
\renewcommand*{\verbatimbreakchar}{\char`\/}
\end{lcode}

    Note that trying both tabbing and wrapping together does not 
always work well. It is best to do only one out of the two, and make
that tabbing.

    There is some more on verbatims in later sections.

\section{Framed boxes}

    You can use framed boxes, looking like the syntax boxes here,
in the standard LaTeX classes but they cannot break over a page.

    The \Ie{framed}, \Ie{shaded} and \Ie{leftbar} environments, which
can break over a page, are from the \Lpack{framed} package~\cite{FRAMED}.

\begin{syntax}
\senv{framed} text \eenv{framed} \\
\senv{shaded} text \eenv{shaded} \\
\senv{leftbar} text \eenv{leftbar} \\
\end{syntax}
The \Ie{framed} environment puts the text into an \cmd{\fbox}-like 
framed box the same width as the text width. The
\Ie{shaded} environment puts the text into a coloured box, and the
\Ie{leftbar} environment draws a vertical line at the left of the text.
In all cases the text and boxes can include page breaks.

\begin{syntax}
\lnc{\FrameRule} \lnc{\FrameSep} \cmd{\FrameHeightAdjust} \\
\Itt{shadecolor} \\
\end{syntax}
The thickness of the rules is the length \lnc{\FrameRule} and the separation
between the text and the box is given by the length \lnc{\FrameSep}.
The height of the frame above the baseline at the top of a page is specified
by the macro \cmd{\FrameHeightAdjust}. The default definitions being:
\begin{lcode}
\setlength{\FrameRule}{\fboxrule}
\setlength{\FrameSep}{3\fboxsep}
\newcommand{\FrameHeightAdjust}{0.6em}
\end{lcode}
The background color in the \Ie{shaded} environment is specified by
\Itt{shadecolor} which you have to specify using the \Lpack{color}
package~\cite{COLOR}. For example:
\begin{lcode}
\usepackage{color}
\definecolor{shadecolor}{gray}{0.75}
\end{lcode}

\begin{syntax}
\cmd{\frameasnormaltrue} \cmd{\frameasnormalfalse} \\
\end{syntax}
Following the declaration \cmd{\frameasnormaltrue} paragraphing within
the environments will be as specified for the main text. After the declaration
\cmd{\frameasnormalfalse} paragraphing will be as though the environments
were like a \Ie{minipage}. The initial declaration is 
\cmd{\frameasnormaltrue}.

    There is one known problem with framing: when framing is used on 
a page where the page header is in a smaller type than the body, the 
header may be moved slightly up or down. This can be avoided by putting
the font size change in a group, but it seems that a larger font does not need
to be grouped. For example:
\begin{lcode}
\makeoddhead{myheadings}{{\tiny Tiny header}}{}{\LARGE Large header}
\end{lcode}
The environments are a little
delicate. You cannot use floats, footnotes or marginpars inside them,
and they do not work in two column mode except for the standard
LaTeX supplied one.

    You can use the \Lpack{framed} package with the \Lclass{memoir} class, in
which case you will get whatever functionality is provided by the package as
it will override the class' code.

\begin{syntax}
\senv{fboxverbatim} anything \eenv{fboxverbatim} \\
\end{syntax}
The \Ie{fboxverbatim} environment is, except for its name,
identical to the \Ie{boxedverbatim} environment from the \Lpack{moreverb}
package~\cite{MOREVERB}. It puts a close fitting rectangular box
around its contents, which are typeset verbatim.


\begin{syntax}
\senv{boxedverbatim} anything \eenv{boxedverbatim} \\
\end{syntax}
The \Ie{boxedverbatim} environment has the flavour of both the
\Ie{framed} and \Ie{fboxverbatim} environments, but it adds its
own bells and whistles. In its simplest usage the contents of the
environment are typeset verbatim in a rectangular box (or boxes as
it allows pagebreaks) like a \Ie{framed} box.

\begin{syntax}
\cmd{\bvbox} \cmd{\bvsides} \cmd{\bvtopandtail} \cmd{\nobvbox} \\
\end{syntax}
Four different kinds of framing styles are provided for the
\Ie{boxedverbatim} environment. After a \cmd{\bvbox}
declaration the default framing style is used, which draws rectangular
boxes. After the \cmd{\bvsides} declarations, rules are drawn on each
side but there are no top or bottom  rules; the converse is
the \cmd{\bvtopandtail} declaration when only an initial and final 
horizontal rule is drawn and no side rules. No rules at all are
drawn after \cmd{\nobvbox}.

\begin{syntax}
\cmd{\bvtopofpage}\marg{text} \\
\end{syntax}
With the default framing style (\cmd{\bvbox}) a heading can be put
at the top of continuation boxes if there is too much to fit onto
a page. The heading is the \meta{text} of \cmd{\bvtopofpage}.
The default specification is |\bvtopofpage{}| but if you wanted, say, to
indicate that the verbatim was a continuation from the previous
page you could do
\begin{lcode}
\bvtopofpage{\begin{center}\normalfont (Continued)\end{center}}
\end{lcode}

\begin{syntax}
\cmd{\linenumberfrequency}\marg{nth} \\
\cmd{\linenumberfont}\marg{font-decl} \\
\cmd{\resetbvlinenumber} \\
\end{syntax}
Exactly
as in the \Ie{verse} environment line numbers are printed
if the \meta{nth} argument to |\linenumberfrequency| is greater
than zero, and the numbers are set in the font defined via 
\cmd{\linenumberfont}.
At any point outside the environment the line numbers may be reset
to the initial value by \cmd{\resetbvlinenumber}. 

\begin{syntax}
\cmd{\bvnumbersinside} \\
\cmd{\bvnumbersoutside} \\
\end{syntax}
If the lines are numbered, the numbers can be put inside the box
(\cmd{\bvnumbersinside}) or outside (\cmd{\bvnumbersoutside}).
In either case, numbers are set at the left of the lines.
The default is for numbers inside the box.

    Verbatim tabbing, but not wrapping, applies to the \Ie{boxedverbatim}
environment.

\section{Files}

    Latex gives you the \cmd{\input} command to read a file but is really
not much obvious help if you want to write stuff out to a file.

    TeX has 16 `streams' that can be open simultaneously for writing
out, and there are also 16 input streams. These streams can be attached
to files so that data can be read out and read in from external files.

\begin{syntax}
\cmd{\newoutputstream}\marg{stream} \\
\cmd{\newinputstream}\marg{stream} \\
\end{syntax}
    The command \cmd{\newoutputstream} creates a new stream
called \meta{stream} for writing out text and commands. 
The \meta{stream} argument must be just alphabetic characters with 
no spaces; for example |myout|.
Similarly, the command \cmd{\newinputstream} creates a new stream for
reading from a file. The \meta{stream} names must be unique --- you cannot
use the same name for both an input and an output stream.

   If you try and create too many streams, TeX will tell you via an error 
message.

\begin{syntax}
\cmd{\IfStreamOpen}\marg{stream}\marg{true-code}\marg{false-code} \\
\end{syntax}
You can check if a stream \meta{stream} is currently open 
by \cmd{\IfStreamOpen}.
If the stream is open then the \meta{true-code} argument will be processed,
otherwise the stream is closed and the \meta{false-code} argument 
is processed.


\begin{syntax}
\cmd{\openoutputfile}\marg{filename}\marg{stream} \\
\cmd{\closeoutputstream}\marg{stream} \\
\end{syntax}
     The macro \cmd{\openoutputfile} opens the
file called \meta{filename} and the output stream \meta{stream}. It
then attaches the file to the stream for writing. Any pre-existing 
contents of \meta{filname} are deleted.

 The macro \cmd{\closeoutputstream} closes the output stream
\meta{stream} and closes whatever file is currently attached to 
\meta{stream}. It then detaches the file from
 the stream.

\begin{syntax}
\senv{writeverbatim}\marg{stream} anything \eenv{writeverbatim} \\
\end{syntax}
 The \Ie{writeverbatim} environment takes one argument, 
the name of an output stream, which must be open. 
The contents of the environment are written
out verbatim to the file currently attached to the stream.

\begin{syntax}
\cmd{\addtostream}\marg{stream}\marg{text} \\
\end{syntax}
 The command \cmd{\addtostream} writes \meta{text}
to the file currently attached to the output stream \meta{stream}, which
must be open. Any commands within \meta{text} will be processed before
being written. To prevent this, put \cmd{\protect} before any command that
you do not want expanding.

\begin{syntax}
\cmd{\openinputfile}\marg{filename}\marg{stream} \\
\cmd{\closeinputstream}\marg{stream} \\
\end{syntax}

 The macro \cmd{\openinputfile} opens the
file called \meta{filename} and the input stream \meta{stream}. It
then attaches the file to the stream for reading.
It is an error if \meta{filename} can not be found.

 The macro \cmd{\closeinputstream}closes the input stream
\meta{stream} and closes whatever file
is currently attached to \meta{stream}. It then detaches the file from
the stream.

\begin{syntax}
\cmd{\readstream}\marg{stream} \\
\cmd{\readaline}\marg{stream} \\
\end{syntax}
 The macro \cmd{\readstream} reads the contents of the file that
is currently associated with the input stream \meta{stream}. This provides
the same functionality as \cmd{\input}\marg{filename} does.

 The macro \cmd{\readaline} reads what TeX considers to be
one line from the file that
is currently associated with the input stream \meta{stream}. Multiple lines
can be read by calling \cmd{\readaline} multiple times. A warning is issued
if there are no more lines to be read (i.e., the end of the file has
been reached).

\begin{syntax}
\cmd{\verbatiminput}\marg{filename} \\
\cmd{\readverbatim}\marg{stream} \\
\cmd{\boxedverbatiminput}\marg{filename} \\
\cmd{\readboxedverbatim}\marg{stream} \\
\end{syntax}
The macro \cmd{\verbatiminput} is similar to \cmd{\input} except
that it inputs the contents of the file \meta{filename} as verbatim text.
Similarly, \cmd{\readverbatim} reads the contents of the file that
is currently associated with the input stream \meta{stream} as verbatim
text. These two commands are equivalent to
\begin{lcode}
\begin{verbatim}
  \input{...}
  \readstream{...}
\end{verbatim}
\end{lcode}
except, of course, that you cannot do that.

    To round things out, \cmd{\boxedverbatiminput} inputs the contents
of the \meta{filename} file as boxed verbatim text, and
\cmd{\readboxedverbatim} reads the contents
of the file attached to \meta{stream} as a boxed verbatim. If you could
do it, these commands are equivalent to
\begin{lcode}
\begin{boxedverbatim}
  \input{...}
  \readstream{...}
\end{boxedverbatim}
\end{lcode}

   Tabbing, wrapping and numbering are just as applicable to verbatim
input texts as they are to the corresponding verbatim environments.











%%%%%%%%%%%%%%%%%%%%%%%%%%%%%%%%%%%%%%%
\clearpage
\pagestyle{Ruled}
\chapterstyle{demo}
%\chapterstyle{veelo}
%%%%%%%%%%%%%%%%%%%%%%%%%%%%%%%%%%%%%%%%

\chapter{Miscellaneous} \label{chap:misc}

\chapterprecis{In which we talk of many things, but not ships
               or shoes or sealing wax, nor cabbages and kings.}

    This chapter uses the \cstyle{demo} chapterstyle and the \pstyle{Ruled}
pagestyle. It started with the \cmd{\chapterprecis} command.

\section{Introduction}

    The class provides some minor additional facilities, which are described
here.

\section{Draft documents}

  The \Lopt{draft} option marks any overfull lines with black rectangles,
otherwise the appearance is the same as for a \Lopt{final} document.

\begin{syntax}
\piif{ifdraftdoc} \\
\end{syntax}
The \piif{ifdraftdoc} macro is provided so that you can add extra
things that you might want to happen when processing a draft; for example
you might want to have each page header\index{header} (or footer\index{footer}) include the word `DRAFT'.
The code to do this can be put into a construct like the following:
\begin{lcode}
\ifdraftdoc
  % special things for a draft document
\else
  % perhaps special things for a non-draft document
\fi
\end{lcode}


\section{Change marks}

    When preparing a manuscript it normally goes through several iterations.
The macros described in this section can be used to identify changes made to 
a document throughout its lifecycle.

   The commands below implement a simplified version of the change
marking in the \Lclass{iso} class~\cite{ISOCLASS}.

\begin{syntax}
\cmd{\changemarkstrue} \cmd{\changemarksfalse} \\
\end{syntax}
The change marking macros only work properly when the \Lopt{draft} class
option is used. Additionaly, the marks are only printed when the 
\cmd{\changemarkstrue} declaration is in effect. Using \cmd{\changemarksfalse}
switches off any marking.

\begin{syntax}
\cmd{\added}\marg{change-id} \\
\cmd{\deleted}\marg{change-id} \\
\cmd{\changed}\marg{change-id} \\
\end{syntax}
Each of these macros puts a mark and \meta{change-id} into the margin\index{margin} near
where the command is given. The \cmd{\added} macro indicates that something
has been added to the manuscript and uses $\oplus$ as its symbol. The
\cmd{\deleted} macro is for indicating that something has been deleted and uses
the $\neq$ symbol. The macro \cmd{\changed} uses the $\Leftrightarrow$ symbol
to indicate that something has been changed.

    These marking commands should be attached to some word or punctuation
mark in the text otherwise extraneous spaces may creep into the typeset
document.

\section{Trim marks}

    When the \Lclass{memoir} class \Lopt{showtrims} option is used, trim
marks can be placed on each page, usually to indicate where the stock should
be trimmed to obtain the planned page size.

    Trim marks can be placed at each corner of the (trimmed) page, and also
at the middle of each side of the page.

\begin{syntax}
\cmd{\trimXmarks} \\
\cmd{\trimLmarks} \\
\cmd{\trimFrame} \\
\cmd{\trimNone} \\
\end{syntax}
Some predefined trimming styles are provided. After the declaration
\cmd{\trimXmarks} marks in the shape of a cross are placed at the four
corners of the page. The declaration \cmd{\trimLmarks} calls for corner marks
in the shape of an `L', in various orientations depending on the particular
corner. After \cmd{\trimFrame} a frame will be drawn around each page, 
coinciding with the page boundaries. The declaration \cmd{\trimNone}
disables all kinds of trim marking.

\begin{syntax}
\cmd{\trimmarks} \\
\cmd{\tmarktl} \cmd{\tmarktr} \cmd{\tmarkbr} \cmd{\tmarkbl} \\
\cmd{\tmarktm} \cmd{\tmarkmr} \cmd{\tmarkbm} \cmd{\tmarkml} \\
\end{syntax}
The \cmd{\trimmarks} macro is responsible for displaying up to 8 marks. The
marks are defined as zero-sized pictures which are placed strategically
around the borders of the page. 

    The command \cmd{\trimmarks} places the pictures \cmd{\tmarktl}, 
\cmd{\tmarktr},
\cmd{\tmarkbl}, and \cmd{\tmarkbr} at the top left, top right,
bottom right and bottom left corners of the page. The pictures
\cmd{\tmarktm}, \cmd{\tmarkmr}, \cmd{\tmarkbm}, and \cmd{\tmarkml} are placed
at the top middle, middle right, bottom middle and middle left of the
edges of the page. All these |\tmark..| macros should expand to zero-sized
pictures.

\begin{syntax}
\cmd{\trimmark} \\
\end{syntax}
The declaration \cmd{\trimXmarks} uses \cmd{\trimmark} for the corner 
crosses. This is defined as
\begin{lcode}
\newcommand{\trimmark}{%
  \begin{picture}(0,0)
    \setlength{\unitlength}{1cm}\thicklines
    \put(-2,0){\line(1,0){4}}
    \put(0,-2){\line(0,1){4}}
  \end{picture}}
\end{lcode}
which produces a zero-sized picture of a 4cm cross.

    As an example, to draw short lines marking the half-height of the page, 
try this:
\begin{lcode}
\providecommand{\tmarkml}{}
\renewcommand{\tmarkml}{%
  \begin{picture}(0,0){%
    \unitlength 1mm
    \thinlines
    \put(-2,0){\line(-1,0){10}}
  \end{picture}}}
\providecommand{\tmarkmr}{}
\renewcommand{\tmarkmr}{%
  \begin{picture}(0,0){%
    \unitlength 1mm
    \thinlines
    \put(2,0){\line(1,0){10}}
  \end{picture}}}
\end{lcode}
Thin horizontal lines of length 10mm will be drawn at the middle left and
middle right of the page, starting 2mm outside the page boundary.

\section{Sheet numbering}

    One purpose of trim marks is to show a printer where the stock
should be trimmed. In this application it can be useful to also note the
sheet number on each page, where the sheet number is 1 for the first page 
and increases by 1 for each page thereafter. The sheet number is independent
of the page number.

\begin{syntax}
\cmd{\thesheetsequence} \\
\end{syntax}
The macro \cmd{\thesheetsequence} typesets the current sheet sequence number
and is analogous to the \cmd{\thepage} macro.

\begin{syntax}
\Icn{lastsheet} \\
\Icn{lastpage} \\
\end{syntax}
The counter \Icn{lastsheet} holds the number of sheets processed during
the \emph{previous} run of LaTeX. Similarly, the counter \Icn{lastpage}
holds the number of the last page processed during the previous run.
Note that the last page number is not necessarily the same as the last
sheet number. For example: \\
\textit{In this document this is 
        sheet \thesheetsequence\ of \thelastsheet\ sheets, 
        and page \thepage\ of \thelastpage.}

The previous sentence was the result of processing the following
code 
\begin{lcode}
\textit{In this document this is 
        sheet \thesheetsequence\ of \thelastsheet\ sheets, 
        and page \thepage\ of \thelastpage.}
\end{lcode}

    You may wish to use the sheet and/or page numbers as part of some
trim marks. The following will note the sheet numbers above the page.
\begin{lcode}
\newcommand{\trimseqpage}{%
  \begin{picture}(0,0)
    \unitlength 1mm
    \put(0,2){\makebox(0,0)[b]{Sheet: \thesheetsequence\ of \thelastsheet}}
  \end{picture}}
\let\tmarktm\trimseqpage
\end{lcode}


\section{Page breaks before lists}

   A sentence or two may be used to introduce a list (e.g., |itemize|)
and it can be annoying if there is a page break between the introduction
and the first item.
\begin{syntax}
\cmd{\noprelistbreak} \\
\end{syntax}
Putting \cmd{\noprelistbreak} immediately before the |\begin{itemize}|
should prevent a page break. Ideally, there sould be no blank lines
between the end of the introduction and the start of the list. 

\section{Changing counters}

    This is effectively a bundling of the \Lpack{chngcntr} 
package~\cite{CHNGCNTR}.

\begin{syntax}
\cmd{\newcounter}\marg{ctr}\oarg{within} \\
\cmd{\thectr} \\
\end{syntax}
    In LaTeX a new counter called, say |ctr|, is created by the 
\cmd{\newcounter} command as |\newcounter{ctr}|. If the optional \meta{within}
argument is given, the counter \meta{ctr} is reset to zero each time the 
counter called |within| is changed; the |within| counter must exist before
it is used as the optional argument. The command |\thectr| typesets the value
of the counter |ctr|. This is automatically defined for you by the 
\cmd{\newcounter} command to typeset arabic numerals.

\begin{syntax}
\cmd{\counterwithin}\marg{ctr}\marg{within} \\
\cmd{\counterwithin*}\marg{ctr}\marg{within} \\
\end{syntax}
The \cmd{\counterwithin} macro makes a \meta{ctr} that has been initially
defined by |\newcounter{ctr}| act as though it had been defined by
|\newcounter{ctr}[within]|. It also redefines the |\thectr| command
to be |\thewithin.\arabic{ctr}|. The starred version of the command
does nothing to the original definition of |\thectr|.

\begin{syntax}
\cmd{\counterwithout}\marg{ctr}\marg{within} \\
\cmd{\counterwithout*}\marg{ctr}\marg{within} \\
\end{syntax}
The \cmd{\counterwithout} macro makes a \meta{ctr} that has been initially
defined by |\newcounter{ctr}[within]| act as though it had been defined by
|\newcounter{ctr}|. It also redefines the |\thectr| command
to be |\arabic{ctr}|. The starred version of the command
does nothing to the original definition of |\thectr|.

    Any number of |\counterwithin{ctr}{...}| and |\counterwithout{ctr}{...}|
commands can be issued for a given counter |ctr| if you wish to toggle
between the two styles. The current value of |ctr| is unaffected by these
commands. If you want to change the value use \cmd{\setcounter}, and to change
the typesetting style use \cmd{\renewcommand} on |\thectr|.

\section{New new and provide commands}

\begin{syntax}
\cmd{\newloglike}\marg{cmd}\marg{string} \\
\cmd{\newloglike*}\marg{cmd}\marg{string} \\
\end{syntax}
The class provides means of creating new math log-like functions. For
example you might want to do
\begin{lcode}
\newloglike{\Ei}{Ei}
\end{lcode}
if you are using the exponential integral function in your work.
The starred version of the command creates a function that takes limits
(like the \cmd{\max} function).

    The LaTeX kernel defines the \cmd{\providecommand} macro that acts
like \cmd{\newcommand} if the designated macro has not been defined
previously, otherwise it does nothing. The class adds to that limited
repetoire.

\begin{syntax}
\cmd{\provideenvironment}\marg{name}\oarg{numargs}\oarg{optarg}\marg{begindef}\marg{enddef} \\
\cmd{\providelength}\marg{cmd} \\
\cmd{\providecounter}\marg{ctr}\oarg{within} \\
\cmd{\provideloglike}\marg{cmd}\marg{string} \\
\cmd{\provideloglike*}\marg{cmd}\marg{string} \\
\end{syntax}

    The macros \cmd{\provideenvironment}, \cmd{\providelength}
and \cmd{\providecounter} take the same arguments as their |\new...|
counterparts. If the environment, length or counter has not been defined
then it is defined accordingly, otherwise the macros do nothing except
produce a warning message for information purposes.

   The \cmd{\provideloglike} commands are for math log-like functions,
but they do not produce any warning messages.

\section{Changing macros} \label{sec:addtodef}

     Commands are provided for extending simple macro definitions. 
Get the \Lpack{patchcmd} package~\cite{PATCHCMD} if you need
to make other additions to definitions.

\begin{syntax}
\cmd{\addtodef}\marg{macro}\marg{prepend}\marg{append} \\
\cmd{\addtoiargdef}\marg{macro}\marg{prepend}\marg{append} \\
\end{syntax}
The macro \cmd{\addtodef} inserts \meta{prepend} at the start of the
current definition of \meta{macro} and puts \meta{append} at the end,
where \meta{macro} is the name of a macro (including the backslash) which 
takes no arguments. The \cmd{\addtoiargdef} macro is similar except that
\meta{macro} is the name of a macro that takes one argument.

 For example the following are two equivalent
definitions of |\mymacro|:
\begin{lcode}
\newcommand{\mymacro}[1]{# is a violinist in spite of being tone deaf}
\end{lcode}
and
\begin{lcode}
\newcommand{\mymacro}[1]{#1 is a violinist}
\addtoiargdef{\mymacro}{}{ in spite of being tone deaf}
\end{lcode}

    The \cmd{\addtoiargdef} (and \cmd{\addtodef}) commands
can be applied several times to the same \meta{macro}. Revising the
previous example we could have
\begin{lcode}
\newcommand{\mymacro}[1]{#1 is a violinist}
\addtoiargdef{\mymacro}{Although somewhat elderly, }%
                       { in spite of being tone deaf}
\addtoiargdef{\mymacro}{}{ and a bagpiper}
\end{lcode}
which is equivalent to
\begin{lcode}
\newcommand{\mymacro}[1]{%
  Although somewhat elderly, #1 is a violinist
  in spite of being tone deaf and a bagpiper}
\end{lcode}

The \meta{prepend} and \meta{append} arguments may include LaTeX code, 
as shown in this extract from the class code:
\begin{lcode}
\addtoiargdef{\date}{}{%
  \begingroup
    \renewcommand{\thanks}[1]{}
    \renewcommand{\thanksmark}[1]{}
    \renewcommand{\thanksgap}[1]{}
    \protected@xdef\thedate{#1}
  \endgroup}
\end{lcode}
Note that in the case of \cmd{\addtoiargdef} an argument can also refer
to the original argument of the \meta{macro}.

\begin{syntax}
\cmd{\addtodef*}\marg{macro}\marg{prepend}\marg{append} \\
\cmd{\addtoiargdef*}\marg{macro}\marg{prepend}\marg{append} \\
\end{syntax}
These starred versions are for use when the original \meta{macro}
was defined via \cmd{\newcommand*}. Using the starred versions is
like using \cmd{\renewcommand*} and the unstarred versions are like
having used \cmd{\renewcommand}. It is the version (starred or unstarred)
of a sequence of |\addto...| commands that counts when determining whether
the equivalent |\renew...| is treated as starred or unstarred.

    The |\addto...| macros cannot be used to delete any code from 
\meta{macro} nor to add anything except at the start and end. Also,
in general, they cannot be used to change the definition of a macro that
takes an optional argument, or that is starred.

\section{String arguments}

    In the code for the class I have sometimes used macro arguments
that consist of a `string', like the |*| arguments in the page layout
macros (e.g., \cmd{\settypeblocksize}), or the |flushleft|, |center| and
|flushright| strings for the \cmd{\makeheadposition} macro.

\begin{syntax}
\cmd{\nametest}\marg{str1}\marg{str2} \\
\piif{ifsamename} \\
\end{syntax}
The macro \cmd{\nametest} takes two strings 
as the arguments \meta{str1} and \meta{str2}. It sets \piif{ifsamename}
TRUE if \meta{str1} is the same as \meta{str2}, otherwise it sets
\piif{ifsamename} FALSE. For the purposes of \cmd{\nametest}, a string is a
sequence of characters which may include spaces and may include
the |\| backslash character; strings are equal if and only if their
character sequences are identical.


    Typically, I have used it within macros for checking on argument
values. For example:
\begin{lcode}
\newcommand{\amacro}[1]{%
  \nametest{#1}{green}
  \ifsamename
%    code for green
  \fi
  \nametest{#1}{red}
  \ifsamename
%    code for red
  \fi
  ...
}
\end{lcode}

\section{Odd/even page checking}

    It is difficult to check robustly if the current page is odd or even but 
the class does provide a robust method based on writing out a label and then
checking the page reference for the label. This requires at least two LaTeX
runs to stabilise. This has been extracted from the original 
\Lpack{chngpage} package~\cite{CHNGPAGE}.

\begin{syntax}
\cmd{\checkoddpage} \\
\piif{ifoddpage} \\
\cmd{\strictpagechecktrue} \cmd{\strictpagecheckfalse} \\
\end{syntax}
The macro \cmd{\checkoddpage} sets 
\piif{ifoddpage} 
to TRUE if the page number 
is odd, otherwise it sets it to FALSE (the page number is even). The robust
checking methos involves writing and reading labels, which is what is done
after the command \cmd{\strictpagechecktrue} is issued. The simple method is just
to check the current page number which, because of TeX's asynchronous page
breaking algorithm, may not correspond to the actual page number where the
\cmd{\checkoddpage} commmand was issued. The simple, but faster, page checking
method is used after the \cmd{\strictpagecheckfalse} command is issued.

\begin{syntax}
\cmd{\cplabel} \\
\end{syntax}
When strict page checking is used the labels consist of a number preceded
by the value of \cmd{\cplabel}, whose default definition is |^_| (e.g.,
a label may consist of the characters |^_21|). If this
might clash with any of your labels, change \cmd{\cplabel} with \cmd{\renewcommand}, but
the definition of \cmd{\cplabel} must be constant for any given document.

\section{Moving to another page} \label{sec:moving}

   Standard LaTeX provides the \cmd{\newpage}, \cmd{\clearpage}
and \cmd{\cleardoublepage} commands for discontinuing the current 
page and starting a new one. The following is a bundling of the
\Lpack{nextpage} package~\cite{NEXTPAGE}.

\begin{syntax}
\cmd{\needspace}\marg{length} \\
\end{syntax}
This macro decides if there is \meta{length} space at the bottom of the 
current page. If there is it does nothing, otherwise it starts a new page.
This is useful if \meta{length} amount of material is to be kept together
on one page. The \cmd{\needspace} macro 
depends on penalties for deciding what to do which means that the reserved
space is an approximation. However, except for the odd occasion, the
macro gives adequate results. 

\begin{syntax}
\cmd{\Needspace}\marg{length} \\
\cmd{\Needspace*}\marg{length} \\
\end{syntax}
    Like \cmd{\needspace}, the \cmd{\Needspace} macro checks if there is
\meta{length} space at the bottom of the current page and if there is not
it starts a new page. The command should only be used between paragraphs;
indeed, the first thing it does is to call \cs{par}. The \cmd{\Needspace}
command checks for the actual space left on the page and is more exacting
than \cmd{\needspace}.

    If either \cmd{\needspace} or \cmd{\Needspace} produce a short page it
will be ragged bottom even if \cmd{\flushbottom} is in effect. With
the starred \cmd{\Needspace*} version, short pages will be flush bottom
if \cmd{\flushbottom} is in effect and will be ragged bottom
when \cmd{\raggedbottom} is in effect.

    Generally speaking, use \cmd{\needspace} in preference to \cmd{\Needspace}
unless it gives a bad break or the pages must be flush bottom.


\begin{syntax}
\cmd{\movetoevenpage}\oarg{text} \\
\cmd{\cleartoevenpage}\oarg{text} \\
\end{syntax}
The \cmd{\movetoevenpage} stops the current page and starts typesetting on the
next even numbered page. The |\clear...| version flushes out all 
floats\index{float} before
going to the next even page. The optional \meta{text} is put on the skipped
page (if there is one).

\begin{syntax}
\cmd{\movetooddpage}\oarg{text} \\
\cmd{\cleartooddpage}\oarg{text} \\
\end{syntax}
These macros are similar to the |\...evenpage| ones except that they jump
to the next odd numbered page.

    A likely example for the optional \meta{text} argument is
\begin{lcode}
\cleartooddpage[\vspace*{\vfill}THIS PAGE LEFT BLANK\vspace*{\vfill}]
\end{lcode}
which will put `THIS PAGE LEFT BLANK' in the centre of any
potential skipped (empty) even numbered page.

\begin{syntax}
\cmd{\cleartorecto} \cmd{\cleartoverso} \\
\end{syntax}
These are slightly simpler forms\footnote{Perhaps more robust.} of
\cmd{\cleartooddpage} and \cmd{\cleartoevenpage}. For example, if you wanted
the Table of Contents to start on a verso page, like in 
\textit{The TeXbook} \cite{KNUTH84a}, then do this:
\begin{lcode}
\cleartoverso
\tableofcontents
\end{lcode}

\section{Number formatting}

    Several methods are provided for formatting numbers. Two classes
of number representations are catered for. A `numeric number' is 
typeset using arabic digits and a `named number' is typeset using
words.

    The argument to the number formatting macros is a `number', 
essentially something that resolves to a series of arabic digits. Typical
arguments might be:
\begin{itemize}
\item Some digits, e.g., \verb?\ordinal{123} ->? 
      \ordinal{123} 
\item A macro expanding to digits, e.g., \verb?\def\temp{3}\ordinal{\temp} ->? 
      \begingroup\def\temp{3}\ordinal{\temp}\endgroup % \\

%      Or even, for example, \verb?\ordinal{\pageref{chap:numf}} ->? 
%      \ordinal{\pageref{chap:numf}}
\item The value of a counter, e.g., \verb?\ordinal{\value{page}} ->? 
      \ordinal{\value{page}} 
\item The arabic representation of a counter, e.g., \verb?\ordinal{\thepage} ->? 
      \ordinal{\thepage} 

However, if the representation of a counter is not completely in arabic 
digits, such as \verb?\thesection? which here prints as \thesection, it will 
produce odd errors or peculiar results if it is used as the argument.
For instance: \\
\verb?\ordinal{\thesection} ->? \ordinal{\thesection}

\end{itemize}

\subsection{Numeric numbers}

\begin{syntax}
\cmd{\cardinal}\marg{number} \\
\cmd{\fcardinal}\marg{number} \\
\cmd{\fnumbersep} \\
\end{syntax}
The macro \cmd{\fcardinal} prints its \meta{number} argument formatted using
\cmd{\fnumbersep} between each triple of digits. The default definition
of \cmd{\fnumbersep} is:
\begin{lcode}
\newcommand{\fnumbersep}{,}
\end{lcode}

    Here are some examples: \\
\verb?\fcardinal{12} ->? \fcardinal{12} \\
\verb?\fcardinal{1234} ->? \fcardinal{1234} \\
\verb?\fcardinal{1234567} ->? \fcardinal{1234567} \\
\verb?\renewcommand{\fnumbersep}{ }\fcardinal{12345678} ->?
  \renewcommand{\fnumbersep}{ }\fcardinal{12345678}
\renewcommand{\fnumbersep}{,}

    The \cmd{\cardinal} macro is like \cmd{\fcardinal} except that there
is no separation between any of the digits.

\begin{syntax}
\cmd{\ordinal}\marg{number} \\
\cmd{\fordinal}\marg{number} \\
\cmd{\ordscript}\marg{chars} \\
\end{syntax}
The \cmd{\fordinal} macro typesets its \meta{number} argument as a formatted
ordinal, using \cmd{\fnumbersep} as the separator. The macro \cmd{\ordinal}
is similar except that there is no separation between any of the digits.

    Some examples are: \\
\verb?\fordinal{3} ->? \fordinal{3} \\
\verb?\fordinal{12341} ->? \fordinal{12341} \\
\verb?\renewcommand{\ordscript}[1]{\textsuperscript{#1}}\fordinal{2} ->?
  \renewcommand{\ordscript}[1]{\textsuperscript{#1}}\fordinal{2} \\
\verb?\ordinal{1234567} ->? \ordinal{1234567} \\
\verb?This is the \ordinal{\value{chapter}} chapter. ->? 
  This is the \ordinal{\value{chapter}} chapter.


    The characters denoting the ordinal (ordination?) are typeset as 
the argument of \cmd{\ordscript}, whose default definition is:
\begin{lcode}
\newcommand{\ordscript}[1]{#1}
\end{lcode}
As the above examples show, this can be changed to give a different
appearance.

\begin{syntax}
\cmd{\nthstring} \cmd{\iststring} \cmd{\iindstring} \cmd{\iiirdstring} \\
\end{syntax}
The ordinal characters are the values of the macros \cmd{\nthstring}
(default: th) for most ordinals, \cmd{\iststring} (default: st) for
ordinals ending in 1 like \ordinal{21}, \cmd{\iindstring} (default: nd)
for ordinals like \ordinal{22}, and \cmd{\iiirdstring} (default: rd)
for ordinals like \ordinal{23}.


\subsection{Named numbers}



\begin{syntax}
\cmd{\numtoname}\marg{number} \\
\cmd{\numtoName}\marg{number} \\
\cmd{\NumToName}\marg{number} \\
\end{syntax}
The macro \cmd{\numtoname} typesets its \meta{number} argument using 
lowercase words. The other two macros are similar, except that 
\cmd{\numtoName} uses uppercase for the initial letter of the first word and
\cmd{\NumToName} uses uppercase for the initial letters of all the words.

    As examples: \\
\verb?\numtoname{12345} ->? \numtoname{12345} \\
\verb?\numtoName{12345} ->? \numtoName{12345} \\
\verb?\NumToName{12345} ->? \NumToName{12345} \\
\verb?The minimum number in TeX is \numtoname{-2147483647}? \\
\verb?    (i.e., \fcardinal{-2147483647}) ->? \\
  The minimum number in TeX is \numtoname{-2147483647} 
  (i.e., \fcardinal{-2147483647})

\begin{syntax}
\cmd{\ordinaltoname}\marg{number} \\
\cmd{\ordinaltoName}\marg{number} \\
\cmd{\OrdinalToName}\marg{number} \\
\end{syntax}
These three macros are similar to \cmd{\numtoname}, etc., except that they
typeset the argument as a wordy ordinal.

    For example: \\
\verb?This is the \ordinaltoname{\value{chapter}} chapter. ->? 
  This is the \ordinaltoname{\value{chapter}} chapter.

\begin{syntax}
\cmd{\namenumberand} \cmd{\namenumbercomma} \cmd{\tensunitsep} \\
\end{syntax}
By default some punctuation and conjunctions are used in the representation
of named numbers. According to both American and English practice, a hyphen
should be inserted between a `tens' name (e.g., fifty) and a following 
unit name (e.g., two). This is represented by the value of \cmd{\tensunitsep}.
English practice, but not American, is to include commas (the value of
\cmd{\namenumbercomma}) and conjunctions (the value of \cmd{\namenumberand})
in strategic positions in the typesetting. These macros are initially
defined as:
\begin{lcode}
\newcommand*{\namenumberand}{ and }
\newcommand*{\namenumbercomma}{, }
\newcommand*{\tensunitsep}{-}
\end{lcode}
The next example shows how to achieve American typesetting.
\begin{lcode}
\renewcommand*{\namenumberand}{ }
\renewcommand*{\namenumbercomma}{ }
The maximum number in TeX is \numtoname{2147483647} (i.e., \cardinal{2147483647}). ->
\end{lcode}
\renewcommand*{\namenumberand}{ }\renewcommand*{\namenumbercomma}{ }%
The maximum number in TeX is \numtoname{2147483647} (i.e., \cardinal{2147483647}). 
\renewcommand*{\namenumberand}{ and }
\renewcommand*{\namenumbercomma}{, }

\begin{syntax}
\cmd{\minusname} \cmd{\lcminusname} \cmd{\ucminusname} \\
\end{syntax}
    When a named number is negative, \cmd{\minusname} is put before the
spelled out number. The definitions of the above three comands are:
\begin{lcode}
\newcommand*{\lcminusname}{minus }
\newcommand*{\ucminusname}{Minus }
\let\minusname\lcminusname
\end{lcode}
which means that `minus' is normally all lowercase. To get `minus'
typeset with an initial uppercase letter simply:
\begin{lcode}
\let\minusname\ucminusname
\end{lcode}

    Only one version of \cmd{\namenumberand} is supplied as I consider that
it is unlikely that `and' would ever be typeset as `And'. If the initial
uppercase is required, renew the macro when appropriate.

    There is a group of macros that hold the names for the numbers. To
typeset named numbers in a language other than English these will have to be
changed as appropriate. You can find them in the class and patch code. 
The actual picking and ordering of the names is done by an internal macro
called \cmd{\n@me@number}. If the ordering is not appropriate for a
particular language, that is the macro to peruse and modify. Be prepared,
though, for the default definitions to be changed in a later release in case
there is a more efficient way of implementing their functions.

\subsection{Fractions}

    When typesetting a simple fraction in text there is usually a choice
of two styles, like 3/4 or $\frac{3}{4}$, which do not necessarily look 
as though they fit in with their surroundings. These fractions were
typeset via:
\begin{lcode}
... like 3/4 or $\frac{3}{4}$ ...
\end{lcode}

\begin{syntax}
\cmd{\slashfrac}\marg{top}\marg{bottom} \\
\cmd{\slashfracstyle}\marg{num} \\
\end{syntax}
The class provides the \cmd{\slashfrac} command which typesets its
arguments like \slashfrac{3}{4}. Unlike the \cmd{\frac} command which
can only be used in math mode, the \cmd{\slashfrac} command can be
used in text and math modes.

    The \cmd{\slashfrac} macro calls the \cmd{\slashfracstyle} command to
reduce the size of the numbers in the fraction. You can also use
\cmd{\slashfracstyle} by itself.
\begin{lcode}
In summary, fractions can be typeset like 3/4 or $\frac{3}{4}%
or \slashfrac{3}{4} or \slashfracstyle{3/4} because several choices
are provided.
\end{lcode}
In summary, fractions can be typeset like 3/4 or $\frac{3}{4}$
or \slashfrac{3}{4} or \slashfracstyle{3/4} because several choices
are provided.

\begin{syntax}
\cmd{\textsuperscript}\marg{super} \\
\cmd{\textsubscript}\marg{sub} \\
\end{syntax}
While on the subject of moving numbers up and down, the kernel provides
the \cmd{\textsuperscript} macro for typesetting its argument as though it
is a superscript. The class also provides the \cmd{\textsubscript} macro
for typesetting its argument like a subscript.
\begin{lcode}
You can typeset superscripts like \textsuperscript{3}/4 and 
subscripts like 3/\textsubscript{4}, 
or both like \textsuperscript{3}/\textsubscript{4}.
\end{lcode}
You can typeset superscripts like \textsuperscript{3}/4 and 
subscripts like 3/\textsubscript{4}, 
or both like \textsuperscript{3}/\textsubscript{4}.


\section{An array data structure}

   The class includes some macros for supporting the \Ie{patverse}
environment which may be more generally useful. 

\begin{syntax}
\cmd{\newarray}\marg{arrayname}\marg{low}\marg{high} \\
\end{syntax}
\cmd{\newarray} defines
the \meta{arrayname} array, where \meta{arrayname} is a name like
|MyArray|. The lowest and highest array indices are set to
\meta{low} and \meta{high} respectively, where both are integer numbers.

\begin{syntax}
\cmd{\setarrayelement}\marg{arrayname}\marg{index}\marg{text} \\
\cmd{\getarrayelement}\marg{arrayname}\marg{index}\marg{result} \\
\end{syntax}
The \cmd{\setarrayelement} macro
sets the \meta{index} location in the \meta{arrayname} array to be 
\meta{text}. Conversely, \cmd{\getarrayelement} sets the parameterless
macro \meta{result} to the contents of
the \meta{index} location in the \meta{arrayname} array. 
For example: 
\begin{lcode}
\setarrayelement{MyArray}{23}{$2^{23}$}
\getarrayelement{MyArray}{23}{\result}
\end{lcode}
will result in |\result| being defined as |\def\result{$2^{23}$}|.

\begin{syntax}
\cmd{\checkarrayindex}\marg{arrayname}\marg{index} \\
\piif{ifbounderror} \\
\end{syntax}
\cmd{\checkarrayindex} checks if
\meta{arrayname} is an array and if \meta{index} is a valid index for
the array. If both conditions hold then \piif{ifbounderror} is set
FALSE, but if either \meta{arrayname} is not an array or, if it is,
\meta{index} is out of range then \piif{ifbounderror} is set TRUE.

\begin{syntax}
\cmd{\stringtoarray}\marg{arrayname}\marg{string} \\
\cmd{\arraytostring}\marg{arrayname}\marg{result} \\
\end{syntax}
The macro \cmd{\stringtoarray} puts each character
from \meta{string} sequentially into the \meta{arrayname} array, starting
at index 1.
The macro \cmd{\arraytostring} assumes
that \meta{arrayname} is an array of characters, and defines the macro
\meta{result} to be that sequence of characters. For example: \\
\begin{lcode}
\stringtoarray{MyArray}{Chars}
\arraytostring{MyArray}{\MyString}
\end{lcode}
is equivalent to |\def\MyString{Chars}|.

\begin{syntax}
\cmd{\checkifinteger}\marg{num} \\
\piif{ifinteger} \\
\end{syntax}
The command \cmd{\checkifinteger} ckecks if \meta{num} is an integer 
(not less than zero). If it is then \piif{ifinteger} is set TRUE, 
otherwise it is set FALSE.


\section{pdfLaTeX}

    Both LaTeX and pdfLaTeX can be run on the same document. LaTeX produces
a \file{.dvi} file as output, while pdfLaTeX can produce either a
\file{.dvi} or a \file{.pdf} file. 

    By default pdfLaTeX produces a \file{.dvi} file but there may be a
configuration file on your system that makes pdfLaTeX default to
producing a \file{.pdf} instead. To ensure a \file{.pdf} output file 
you have to put the incantation \cmd{\pdfoutput}\texttt{=1} near the start 
of the preamble\index{preamble}. Conversely, to ensure a \file{.dvi}
file put \cmd{\pdfoutput}\texttt{=0} instead.
Unfortunately this command is unknown to LaTeX so if you used LaTeX it 
would hiccup.

\begin{syntax}
\piif{ifpdf} ... |\fi| \\
\end{syntax}
The class provides \piif{ifpdf} which is true when the document is
being processed by pdfLaTeX and false otherwise. You can use it like this:
\begin{lcode}
\ifpdf
  code for pdfLaTeX only e.g., \pdfoutput=1 or \pdfoutput=0
\else
  code for LaTeX only
\fi
\end{lcode}
If there is no LaTeX specific code then don't write the |\else| part.


\section{Leading}

    LaTeX automatically uses different leading\index{leading} for different
font sizes. 
\begin{syntax}
\lnc{\baselineskip} \lnc{\onelineskip} \\
\end{syntax}

At any point in a document the standard LaTeX \lnc{\baselineskip} length
contains the current value of the leading\footnote{This statement ignores
any attempts to stretch the baseline.}. The class provides the length
\lnc{\onelineskip} which contains the initial leading for the normal
font. This value is useful if you are wanting to specify length values
in terms of numbers of lines of text.

\section{Minor space adjustment}

    The kernel provides the \cmd{\,} macro for inserting a thin space in both
text and math mode. There are
other space adjustment commands, such as \pixabang{} for negative thin space, and
\cmd{\:} and \cmd{\;} for medium
and thick spaces, which can only be used in math mode.

\begin{syntax}
\cmd{\thinspace} \cmd{\medspace} \cmd{\:} \pixabang \\
\end{syntax}
On occasions I have found it useful to be able to tweak spaces in text by some
fixed amount, just as in math mode. The kernel macro \cmd{\thinspace}
specifies a thin space, which is 3/18\,em. 
The class \cmd{\medspace} specifies a medium space of 4/18\,em. 
As mentioned, the kernel macro \cmd{\:} inserts
a medium space in math mode. The class version can be used in both math and
text mode to insert a medium space. Similarly, the class version of 
\pixabang{}
can be used to insert a negative thin space in both text and math mode.

    The math thick space is 5/18\,em. 
To specify this amount of space
in text mode you can combine spacing commands as:
\begin{lcode}
\:\:\!
\end{lcode}
which will result in an overall space of 5/18\,em 
(from $(4 + 4 - 3)/18$).


\section{Cross references}\label{sec:xrefthis}  \label{sec:xref}

    LaTeX supplies the \cmd{\ref} and \cmd{\pageref} commands for cross
referencing to a label or a page which has a label on it.

\begin{syntax}
\cmd{\fref}\marg{label} \cmd{\figurerefname} \\
\cmd{\tref}\marg{label} \cmd{\tablerefname} \\
\cmd{\pref}\marg{label} \cmd{\pagerefname} \\
\end{syntax}
 
    The class provides these more particular named references to a figure\index{figure!reference},
table\index{table!reference} or page\index{page!reference}. For example the default definitions of \cmd{\fref} and 
\cmd{\pref} are
\begin{lcode}
\newcommand{\fref}[1]{\figurerefname~\ref{#1}}
\newcommand{\pref}[1]{\pagerefname~\pageref{#1}}
\end{lcode}
and can be used as 
\begin{lcode}
\ldots footnote parameters are shown in~\fref{fig:fn} on~\pref{fig:fn}.
\end{lcode}
which in this document prints as: 
\begin{syntax}
\ldots footnote parameters are shown in~\fref{fig:fn} on~\pref{fig:fn}. \\
\end{syntax}

\begin{syntax}
\cmd{\Pref}\marg{label} \cmd{\partrefname} \\
\cmd{\Cref}\marg{label} \cmd{\chapterrefname} \\
\cmd{\Sref}\marg{label} \cmd{\sectionrefname} \\
\end{syntax}

    Also provided are named references to labelled 
Part (\cmd{\Pref}), 
Chapter (\cmd{\Cref}) and 
sectional (\cmd{\Sref}) divisions.
These are all defined like
\begin{lcode}
\newcommand{\Sref}[1]{\sectionrefname\ref{#1}}
\end{lcode}
with no tie between the name and the \cmd{\ref}. 

    In this document
\begin{lcode}
`In \Cref{chap:misc} there is a section 
(\Sref{sec:xrefthis}) about cross references.' 
\end{lcode}
is typeset as: 
\begin{syntax}
`In \Cref{chap:misc} there is a section 
(\Sref{sec:xrefthis}) about cross references.' \\
\end{syntax}
 
    It can be useful to refer to parts of a document by name rather than
number, as in
\begin{lcode}
The chapter \textit{\titleref{chap:bringhurst}} describes \ldots
\end{lcode}
The chapter \textit{\titleref{chap:bringhurst}} describes \ldots

    There are two packages, \Lpack{nameref}~\cite{NAMEREF} and 
\Lpack{titleref}~\cite{TITLEREF},
 that let you refer to things by name instead of number.

    Name references were added to the class as a consequence of adding
a second optional argument to the sectioning commands. I found
that this broke the \Lpack{nameref} package, and hence the
\Lpack{hyperref} package as well, so they had to be fixed. The change 
also broke Donald Arseneau's \Lpack{titleref} package, and it turned out
that \Lpack{nameref} also clobbered \Lpack{titleref}. The class also
provides titles, like \cmd{\poemtitle}, that are not recognised by
either of the packages. From my viewpoint the most efficient
thing to do was to enable the class itself to provide name 
referencing.

\begin{syntax}
\cmd{\label}\marg{key} \cmd{\ref}\marg{key} \cmd{\pageref}\marg{key} \\
\cmd{\titleref}\marg{key} \\
\cmd{\headnamereftrue} \cmd{\headnamereffalse} \\
\end{syntax}
The macro \cmd{\titleref} is an addition to the usual set of cross referencing
commands. Instead of typesetting a number it typesets the title associated
with the labelled number. This is, of course, only useful if there is an
associated title, such as from a \cmd{\caption} or \cmd{\section} command.
As a bad example:
\begin{lcode}
Labelling for \verb?\titleref? may be applied to:
\begin{enumerate}
\item Chapters, sections, etc.       \label{sec:xref:item1}
...
\item Items in numbered lists, etc. \ldots \label{sec:xref:item3}
\end{enumerate}
Item \ref{sec:xref:item2} in section~\ref{sec:xref} mentions captions
while item \titleref{sec:xref:item3} in the same section 
\textit{\titleref{sec:xref}} lists other things.
\end{lcode}
Labelling for \verb?\titleref? may be applied to:
\begin{enumerate}
\item Chapters, sections, etc.       \label{sec:xref:item1}
\item Captions                       \label{sec:xref:item2}
\item Legends
\item Poem titles
\item Items in numbered lists, etc.  \label{sec:xref:item3}
\end{enumerate}
Item \ref{sec:xref:item2} in section~\ref{sec:xref} mentions captions
while item \titleref{sec:xref:item3} in the same section 
\textit{\titleref{sec:xref}} lists other things.


    As the above example shows, you have to be a little careful in using
\cmd{\titleref}.
Generally speaking, \cmd{\titleref}\marg{key} produces the last named 
thing before the \cmd{\label} that defines the \meta{key}. 

    Chapters, and the lower level sectional divisions, may have three
different title texts --- the main title, the title in the ToC, and a third
in the page header. By default (\cmd{\headnamereffalse}) the ToC title
is produced by \cmd{\titleref}. Following the declaration
\cmd{\headnamereftrue} the text intended for page headers will be produced.

\Note{} Specifically with the \Lclass{memoir} class, 
do not put a \cmd{\label} command inside an
argument to a \cmd{\chapter} or \cmd{\section} or \cmd{\caption}, etc.,
command. Most likely it will either be ignored or referencing it will
produce incorrect values. This restriction does not apply to the standard
classes, but in any case I think it is good practice not to embed any 
\cmd{\label} commands.

\begin{syntax}
\cmd{\currenttitle} \\
\end{syntax}
    If you just want to refer to the current title you can do so with
\cmd{\currenttitle}. This acts as though there had been a label associated
with the title and then \cmd{\titleref} had been used to refer to that label.
For example:
\begin{lcode}
This sentence in the section titled `\currenttitle' is an example of the
use of the command \verb?\currenttitle?.
\end{lcode}
This sentence in the section titled `\currenttitle' is an example of the
use of the command \verb?\currenttitle?.


\begin{syntax}
\cmd{\theTitleReference}\marg{num}\marg{text} \\
\end{syntax}
Both \cmd{\titleref} and \cmd{\currenttitle} use the \cmd{\theTitleReference}
to typeset the title. This is called with two arguments --- 
the number, \meta{num}, and the text, \meta{text}, of the title. The
default definition is:
\begin{lcode}
\newcommand{\theTitleReference}[2]{#2}
\end{lcode}
so that only the \meta{text} argument is printed. You could, for example,
change the definition to
\begin{lcode}
\renewcommand{\theTitleReference}[2]{#1\space \textit{#2}}
\end{lcode}
to print the number followed by the title in italics. If you do this, only use
\cmd{\titleref} for numbered titles, as a printed number for an 
unnumbered title (a) makes no sense, and (b) will in any case be 
incorrect.

    The commands \cmd{\titleref}, \cmd{\theTitleReference} and 
\cmd{\currenttitle} are direct equivalents of those in the \Lpack{titleref}
package~\cite{TITLEREF}.

\begin{syntax}
\cmd{\namerefon} \cmd{\namerefoff} \\
\end{syntax}
   Implementing name referencing has had an unfortunate side effect of
turning some arguments into moving ones; the argument to the \cmd{\legend}
command is one example. If you don't need name referencing you can turn
it off by the \cmd{\namerefoff} declaration; the \cmd{\namerefon}
declaration enables name referencing.



\section{Words and phrases}

    The class provides several macros that expand into English words or 
phrases. To typeset in another language these need to be changed, or an
author or publisher may want some changes made to the English versions. 
The following lists the macros and their default values.
\begin{itemize}
\item[\cmd{\contentsname}]     \contentsname
\item[\cmd{\listfigurename}]   \listfigurename
\item[\cmd{\listtablename}]    \listtablename
\item[\cmd{\abstractname}]     \abstractname
\item[\cmd{\partname}]         \partname
\item[\cmd{\chaptername}]      \chaptername
\item[\cmd{\appendixname}]     \appendixname
\item[\cmd{\appendixtocname}]  \appendixtocname
\item[\cmd{\appendixpagename}] \appendixpagename
\item[\cmd{\bibname}]          \bibname
\item[\cmd{\indexname}]        \indexname
\item[\cmd{\figurename}]       \figurename
\item[\cmd{\tablename}]        \tablename
\item[\cmd{\figurerefname}]    \figurerefname
\item[\cmd{\tablerefname}]     \tablerefname
\item[\cmd{\pagename}]         \pagename
\item[\cmd{\pagerefname}]      \pagerefname
\item[\cmd{\partrefname}]      \partrefname 
     (defined as |\newcommand{\partrefname}{Part~}|)
\item[\cmd{\chapterrefname}]    \chapterrefname 
     (defined as |\newcommand{\chapterrefname}{Chapter~}|)
\item[\cmd{\sectionrefname}]    \sectionrefname 
     (defined as |\newcommand{\sectionrefname}{\S}|)
\end{itemize}
The above definitions are simple --- for example
\begin{lcode}
\newcommand{\partname}{Part}
\end{lcode} 
and so can be also changed simply.

 The definitions of the macros for the names of numbers are more complex 
--- for example for the number 11 (eleven) 
\begin{lcode}
\newcommand*{\nNamexi}{\iflowertonumname e\else E\fi leven}
\end{lcode}
That is, each definition includes both a lowercase and an uppercase initial
letter, so a bit more care has to be taken when changing these. For specifics
read the documentation of the class code.

\section{Symbols}

    LaTeX lets you typeset an enormous variety of symbols.\index{symbol}
The class adds
nothing to the standard LaTeX capabilities in this respect.
If you want to see what symbols are available then get a copy
of Scott Pakin's 
\textit{The Comprehensive LaTeX Symbol List}~\cite{SYMBOLS}.
You may have to do a little experimentation to get what you want, though.

    For example, the \cmd{\texttrademark} command 
produces the trademark\index{trademark} symbol\texttrademark,
but the \cmd{\textregistered} command produces\textregistered.
When I wanted to use the registered trademark\index{registered trademark}
 symbol it needed to be 
typeset like a superscript
instead of on the baseline. The \cmd{\textsuperscipt} macro typesets
its argument like a superscript\index{superscript}, so using
\begin{lcode}
\textsuperscript{\textregistered}
\end{lcode}
gave the required result\textsuperscript{\textregistered}.

\clearpage

\makeatletter
%% Bringhurst chapter style
\makechapterstyle{bringhurst}{%
  \renewcommand{\chapterheadstart}{}
  \renewcommand{\printchaptername}{}
  \renewcommand{\chapternamenum}{}
  \renewcommand{\printchapternum}{}
  \renewcommand{\afterchapternum}{}
  \renewcommand{\printchaptertitle}[1]{%
    \raggedright\Large\scshape\MakeLowercase{##1}}
  \renewcommand{\afterchaptertitle}{%
    \vskip\onelineskip \hrule\vskip\onelineskip}
}

\setsecheadstyle{\raggedright\scshape\MakeLowercase}
  \setbeforesecskip{-\onelineskip}
  \setaftersecskip{\onelineskip}

%%\setsubsecheadstyle{\renewcommmand\@hangfrom[1]{\noindent ##1}\raggedright\itshape}
\setsubsecheadstyle{\sethangfrom{\noindent ##1}\raggedright\itshape}
  \setbeforesubsecskip{-\onelineskip}
  \setaftersubsecskip{\onelineskip}

%% Bringhurst page style
\makepagestyle{bringhurst}
\makeevenfoot{bringhurst}{\thepage}{}{}
\makeoddfoot{bringhurst}{}{}{\thepage}
\setlength{\pwlayi}{\headsep} % \verb?\setlength{\pwlayi}{\headsep}? headsep=\printlength{\headsep}, pwlayi = \printlength{\pwlayi}.
\addtolength{\pwlayi}{\topskip} % \verb?\addtolength{\pwlayi}{\topskip}? topskip=\printlength{\topskip}, pwlayi=\printlength{\pwlayi}.
\addtolength{\pwlayi}{7.3\onelineskip} % \verb?\addtolength{\pwlayi}{7.3\onelineskip}? onelineskip=\printlength{\onelineskip}, pwlayi=\printlength{\pwlayi}.
\newcommand{\bringpicr}[1]{%
  \setlength{\unitlength}{1pt}
  \begin{picture}(0,0)
    \put(\strip@pt\marginparsep, -\strip@pt\pwlayi){%
      \begin{minipage}[t]{\marginparwidth}
         \raggedright\itshape #1
      \end{minipage}}
  \end{picture}
}

\setlength{\pwlayii}{\marginparsep} % \verb?\setlength{\pwlayii}{\marginparsep}? marginparsep=\printlength{\marginparsep}, pwlayii=\printlength{\pwlayii}.
\addtolength{\pwlayii}{\marginparwidth} % \verb?\addtolength{\pwlayii}{\marginparwidth}?  marginparwidth=\printlength{\marginparwidth}, pwlayii=\printlength{\pwlayii}.
\newcommand{\bringpicl}[1]{%
  \setlength{\unitlength}{1pt}
  \begin{picture}(0,0)
    \put(-\strip@pt\pwlayii, -\strip@pt\pwlayi){%
      \begin{minipage}[t]{\marginparwidth}
        \raggedleft\itshape #1
      \end{minipage}}
  \end{picture}
}

\makepsmarks{bringhurst}{%
  \let\@mkboth\markboth
  \def\chaptermark##1{\markboth{##1}{##1}}
  \def\sectionmark##1{\markright{##1}}
}
\makeevenhead{bringhurst}{\bringpicl{\rightmark}}{}{}
\makeoddhead{bringhurst}{}{}{\bringpicr{\leftmark}}


\renewcommand{\cftchapterfont}{\normalfont}
\renewcommand{\cftchapterpagefont}{\normalfont}
\renewcommand{\cftchapterpresnum}{\bfseries}
\renewcommand{\cftchapterleader}{}
\renewcommand{\cftchapterafterpnum}{\cftparfillskip}
%%%\settocdepth{chapter}

\makeatother

%%%%%%%%%%%%%%%%%%%%%%%%%%%%%%%%
\cleardoublepage
\pagestyle{bringhurst}
\chapterstyle{bringhurst}
%%%%%%%%%%%%%%%%%%%%%%%%%%%%%%%%
\chapter{An example design} \label{chap:bringhurst}

%unitlength=\printlength{\unitlength}, \\
%pwlayi=\printlength{\pwlayi}, \\
%pwlayii=\printlength{\pwlayii}.

\section{Introduction}

    In this chapter I will work through a reasonably complete design
exercise. The chapter is typeset using the results of the exercise.

    Rather than trying to invent something myself I am taking the design
of Bringhurst's 
\textit{The Elements of Typographic Style}~\cite{BRINGHURST92}
as the basis of the
exercise. This is sufficiently different from the normal LaTeX appearance
to demonstrate most of the class capabilities, and also it is a design by
a leading proponent of good typography.

    As much as possible, this chapter is typeset according to the results
of the exercise to provide both a coding and a graphic example.

\section{Design requirements}

    The \textit{Elements of Typographic Style} is typeset using Minion as the
text font and Syntax (a sans font) for the captions. The page layout has been
 shown diagramatically in \fref{fb:1} on \pref{fb:1}, but further details need
to be described for those not fortunate enough to have a copy of their own.

    The trimmed size of a page is 23 by 13.3cm. The foredge is 3.1cm and the
top margin\index{margin} is 1.9cm.

    As already noted, the font for the main text is Minion, with 12pt leading
on a 21pc measure with 42 lines per page. For the purposes of this exercise
I will assume that Minion can be replaced by Computer Modern Roman at 10pt
(like this manual). The captions to figures\index{figure} and tables\index{table} are unnamed and 
unnumbered and typeset in Syntax. The captions give the appearance of being
in a smaller font size than the main text, which is often the case. I'll
assume that the |\small| |\sfseries| CMR font will reasonably do for the
captions. 

    The footer\index{footer} is the same width as the typeblock\index{typeblock} and the folio\index{folio} is placed 
in the footer\index{footer} at the foredge. There are two blank lines between the bottom
of the typeblock\index{typeblock} and the folio\index{folio}.

    There is no header\index{header} in the usually accepted sense of the term but the
chapter title is put on recto pages and section titles are on verso pages.
The running titles are placed in the foredge margin\index{margin} level with the seventh
line of the text in the typeblock\index{typeblock}. The recto headers\index{header} are typeset raggedright
and the verso ones raggedleft.

Bringhurst also uses many marginal\index{marginalia} notes,
their maximum width being about 51pt, and typeset raggedright in a smaller
version of the textfont.

    Chapter titles are in small caps, lowercase, in a larger font than for 
the main text, and a rule is placed between the title and the typeblock\index{typeblock}.
The total vertical space used by a chapter title is three text lines.
Chapters are not numbered in the text but are in the Table of Contents.

    Section titles are again in small caps, lowercase, in the same size as the
text font. The titles are numbered, with both the chapter and section number.

    A subsection title, which is the lowest subdivision in the book, is in
the italic form of the textfont and is typeset as a numbered non-indented
paragraph\index{paragraph}. These are usually multiline as Bringhurst sometimes uses them
like an enumerated list, so on occasion there is a 
subsection title with no following text.

    Only chapter titles are put into the \toc, and these are set raggedright 
with the page numbers immediately after the titles. There is no \lof{} or
\lot.

    Note that unlike the normal LaTeX use of fonts, essentially only three
sizes of fonts are used --- the textfont size, one a bit larger for the
chapter titles, and one a bit smaller for marginal\index{marginalia} notes and captions.
Also, bold fonts are not used except on special occasions, such as when he
is comparing font families and uses large bold
versions to make the differences easier to see.

\section{Specifying the page and typeblock}

    The first and second things to do are to specify the sizes of the page
after trimmming and the typeblock\index{typeblock}. The trimmed size is easy as we have
the dimensions.
\begin{lcode}
\settrimmedsize{23cm}{13.3cm}{*}
\end{lcode}
However, there is a trick to setting the height of the typeblock\index{typeblock} in terms
of lines of text. The
height calculation ensures that an integral number of lines can fit in
the typeblock\index{typeblock}, and as well as the specified height for the block some more
height is added so that the final height is approximately measured from 
the base of the bottom line of text to the top of the first line of text.
This is a complicated way of saying that if you want $N$ lines of text, only
ask for $N-1$. We want 42 lines so we give the height of the typeblock\index{typeblock}
as 41 times the distance between two normal text lines 
(i.e., times \lnc{\onelineskip})
\begin{lcode}
\settypeblocksize{41\onelineskip}{21pc}{*}
\end{lcode}

    To make life easier, we'll do no trimming of the top of the stock\index{stock}
\begin{lcode}
\setlength{\trimtop}{0pt}
\end{lcode}
but will trim the foredge. The next set of calculations first sets the 
value of the \lnc{\trimedge} to be the \lnc{\stockwidth}; subtracting the
trimmed \lnc{\paperwidth} then results in \lnc{\trimedge} being the amount
to trim off the foredge.
\begin{lcode}
\setlength{\trimedge}{\stockwidth}
\addtolength{\trimedge}{-\paperwidth}
\end{lcode}

    The sizes of the trimmed page and the typeblock\index{typeblock} have now been specified.
The typeblock\index{typeblock} is now positioned on the page. The sideways positioning is
easy as we know the foredge margin\index{margin} to be 3.1cm.
\begin{lcode}
\setlrmargins{*}{3.1cm}{*}
\end{lcode}
The top margin\index{margin} is specified as 1.9cm, which is very close to four and a half
lines of text. Just in case someone might want to use a different font size,
I'll specify the top margin\index{margin} so that it is dependent on the font size. The
\lnc{\footskip} can be specified now as well (it doesn't particularly matter
what we do about the header-related lengths as there isn't anything above
the typeblock\index{typeblock}).
\begin{lcode}
\setulmargins{4.5\onelineskip}{*}{*}
\setheadfoot{\onelineskip}{3\onelineskip}
\setheaderspaces{\onelineskip}{*}{*}
\end{lcode}

    Lastly define the dimensions for any marginal\index{marginalia} notes.
\begin{lcode}
\setmarginnotes{17pt}{51pt}{\onelineskip}
\end{lcode}

    If this was for real, the page layout would have to be checked and
implemented.
\begin{lcode}
\checkandfixthelayout
\end{lcode}

    It is possible to implement this layout just for this chapter but
I'm not going to tell you either how to do it, or demonstrate it. Except
under exceptional circumstances it is not good to make such drastic changes
to the page layout in the middle of a document. However, the picture on
\pref{fig:bplayout} illustrates
how this layout would look on US letterpaper\index{paper!size!letterpaper} stock\index{stock}. Looking at the illustration
suggests that the layout would look rather odd unless the stock\index{stock} was trimmed down
to the page size --- another reason for not switching the layout here.

\begin{figure}
\captiontitlefont{\small\sffamily}
\captionstyle{\centerlastline}
\setstocksize{11in}{8.5in}
\settrimmedsize{23cm}{13.3cm}{*}
\settypeblocksize{41\onelineskip}{21pc}{*}
\setlength{\trimtop}{0pt}
\setlength{\trimedge}{\stockwidth}
\addtolength{\trimedge}{-\paperwidth}
\setlrmargins{*}{3.1cm}{*}
\setulmargins{4.5\onelineskip}{*}{*}
\setheadfoot{\onelineskip}{3\onelineskip}
\setheaderspaces{\onelineskip}{*}{*}
\setmarginnotes{17pt}{51pt}{\onelineskip}
\checkandfixthelayout
\currentstock
\oddpagelayouttrue
\twocolumnlayoutfalse
\drawmarginparstrue
\drawparametersfalse
\drawstock
\legend{An illustration of Bringhurst's page layout style when printed
on US letter paper stock. Also shown are the values used for the
page layout parameters for this design.} \label{fig:bplayout}
\end{figure}

\section{Specifying the sectional titling styles}

\subsection{The chapter style}

    Recapping, 
    chapter titles are in small caps, lowercase, in a larger font than for 
the main text, and a rule is placed between the title and the typeblock\index{typeblock}.
The total vertical space used by a chapter title is three text lines.
Chapters are not numbered in the text but are in the Table of Contents.
Titles in the \toc{} are in mixed case.

    The definition of the chapterstyle is remarkably simple, as shown below.
\begin{lcode}
%% Bringhurst chapter style
\makechapterstyle{bringhurst}{%
  \renewcommand{\chapterheadstart}{}
  \renewcommand{\printchaptername}{}
  \renewcommand{\chapternamenum}{}
  \renewcommand{\printchapternum}{}
  \renewcommand{\afterchapternum}{}
  \renewcommand{\printchaptertitle}[1]{%
    \raggedright\Large\scshape\MakeLowercase{##1}}
  \renewcommand{\afterchaptertitle}{%
    \vskip\onelineskip \hrule\vskip\onelineskip}
}
\end{lcode}

    Most of the specification consists of nulling the majority of the normal
LaTeX specification, and modifying just two elements. 

The chapter title (via \cmd{\printchaptertitle}) 
is typeset raggedright using the \cmd{\Large} smallcaps fonts. The 
\cmd{\MakeLowercase} macro is used to ensure that the entire title is 
lowercase before typesetting it. Titles are input in mixed case.

    After the title is typeset the \cmd{\afterchaptertitle} macro
specifies that one line is skipped, a horizontal rule
is drawn and then another line is skipped.

\subsection{Lower level divisions}

    Section titles are in small caps, lowercase, in the same size as the
text font. The titles are numbered, with both the chapter and section number.

The specification is:
\begin{lcode}
\setsecheadstyle{\raggedright\scshape\MakeLowercase}
  \setbeforesecskip{-\onelineskip}
  \setaftersecskip{\onelineskip}
\end{lcode}

    The macro \cmd{\setsecheadstyle} lowercases the title and typesets it
small caps. 

The default skips before and after titles are rubber lengths but this does
not bode well if we are trying to line something up with a particular line
of text --- the presence of section titles may make slight vertical 
adjustments to the text lines because of the flexible spacing. So, we have
to try and have fixed spacings.
A single blank line is used before (\cmd{\setbeforesecskip)}
and after (\cmd{\setaftersecskip}) the title text. 

    A subsection title, which is the lowest subdivision in the book, is in
the italic form of the textfont and is typeset as a numbered non-indented
paragraph\index{paragraph}. The code for this is below.

\begin{lcode}
\setsubsecheadstyle{\sethangfrom{\noindent ##1}\raggedright\itshape}
  \setbeforesubsecskip{-\onelineskip}
  \setaftersubsecskip{\onelineskip}
\end{lcode}

    As in the redefinition of the \cmd{\section} style, there are fixed
spaces before and after the title text. The title is typeset 
(\cmd{\setsubsecheadstyle}) raggedright in a normal sized italic font.
The macro \cmd{\sethangfrom} is used to to redefine the internal
\cmd{\@hangfrom} macro so that the title and number are typeset as a block 
paragraph\index{paragraph!block} instead of the default hanging 
paragraph\index{paragraph!hanging} style. Note the use of
the double |##| mark for denoting the position of the argument to 
\cmd{\@hangfrom}.

\section{Specifying the pagestyle}

    The pagestyle is perhaps the most interesting aspect of the exercise.
Instead of the chapter and section titles being put at the top of the
pages they are put in the margin\index{margin} starting about seven lines below the
top of the typeblock\index{typeblock}. The folios\index{folio} are put at the bottom of the page 
aligned with the outside of the typeblock\index{typeblock}.

    As the folios\index{folio} are easy, we'll deal with those first.
\begin{lcode}
%% Bringhurst page style
\makepagestyle{bringhurst}
\makeevenfoot{bringhurst}{\thepage}{}{}
\makeoddfoot{bringhurst}{}{}{\thepage}
\end{lcode}

    Putting text at a fixed point on a page is typically done by
first putting the text into a zero width picture (which as far as LaTeX
is concerned takes up zero space) and then placing the picture at the
required point on the page. This can be done by hanging it from the
header\index{header}.

    We might as well treat the titles so that they will align with any 
marginal\index{marginalia} notes, which are \lnc{\marginparsep} (17pt) into the margin\index{margin} 
and \lnc{\marginparwidth} (51pt) wide. Earlier in the manual I defined
two lengths called |\pwlayi| and |\pwlayii| which are no longer used.
I will use these as scratch lengths in 
performing some of the necessary calculations.

    For the recto page headers\index{header} the picture will be the \meta{right} part of
the header\index{header} and for the verso pages the picture will be the \meta{left}
part of the header\index{header}, all other parts being empty. 

    For the picture on the \meta{right} the text must be 17pt to
the right of the origin, and some distance below the origin.
From some experiments, this distance turns out to be the \lnc{\headsep}
plus the \cmd{\topskip} plus 7.3 lines, which is calculated as follows:
\begin{lcode}
\setlength{\pwlayi}{\headsep}
\addtolength{\pwlayi}{\topskip}
\addtolength{\pwlayi}{7.3\onelineskip}
\end{lcode}

    There is a nifty internal LaTeX macro called \cmd{\strip@pt} which you
probably haven't heard about, and I have only recently come across. What it
does is strip the `pt' from a following length, reducing it to a plain 
real number. Remembering that the default \lnc{\unitlength} is 1pt we can
do the following, while making sure that the current \lnc{\unitlength}
\emph{is} 1pt:
\begin{lcode}
\makeatletter
\newcommand{\bringpicr}[1]{%
  \setlength{\unitlength}{1pt}
  \begin{picture}(0,0)
    \put(\strip@pt\marginparsep, -\strip@pt\pwlayi){%
      \begin{minipage}[t]{\marginparwidth}
        \raggedright\itshape #1
      \end{minipage}}
  \end{picture}
}
\makeatother
\end{lcode}
The new macro \cmd{\bringpicr}\marg{text} puts \meta{text} 
into a \Ie{minipage} of width \lnc{\marginparwidth}, 
typeset raggedright in an italic font, and puts the top
left of the minipage at the position (|\marginparsep|, -|\pwlayi|) 
in a zero width picture.

    We need a different picture for the \meta{left} as the text needs to be
typeset raggedleft with the right of the text 17pt from the left of the
typeblock\index{typeblock}. I will use the length |\pwlayii| 
to calculate the sum of \lnc{\marginparsep}
and \lnc{\marginparwidth}. Hence:
\begin{lcode}
\makeatletter
\setlength{\pwlayii}{\marginparsep}
\addtolength{\pwlayii}{\marginparwidth}
\newcommand{\bringpicl}[1]{%
  \setlength{\unitlength}{1pt}
  \begin{picture}(0,0)
    \put(-\strip@pt\pwlayii, -\strip@pt\pwlayi){%
      \begin{minipage}[t]{\marginparwidth}
        \raggedleft\itshape #1
      \end{minipage}}
  \end{picture}
}
\makeatother
\end{lcode}
The new macro \cmd{\bringpicl}\marg{text} puts \meta{text} 
into a \Ie{minipage} of width \lnc{\marginparwidth}, 
typeset raggedleft in an italic font, and puts the top
left of the minipage at the position 
(-|(\marginparsep+\marginparwidth)|, -|\pwlayi|) 
in a zero width picture.


    Now we can proceed with the remainder of the pagestyle specification.
The next bit puts the chapter and section titles into the |\...mark| macros.
\begin{lcode}
\makeatletter
\makepsmarks{bringhurst}{%
  \let\@mkboth\markboth
  \def\chaptermark##1{\markboth{##1}{##1}}
  \def\sectionmark##1{\markright{##1}}
}
\makeatother
\end{lcode}

    Finally, specify the evenhead using \cmd{\bringpicl} with the section
title as its argument, and the oddhead using \cmd{\bringpicr} with the
chapter title as its argument.
\begin{lcode}
\makeevenhead{bringhurst}{\bringpicl{\rightmark}}{}{}
\makeoddhead{bringhurst}{}{}{\bringpicr{\leftmark}}
\end{lcode}


\section{Captions and the \prtoc}

    The captions to figures\index{figure} and tables\index{table} are set in a small sans font and
are neither named nor numbered, and there is no \lof{} or \lot.
Setting the caption titles in the desired font is simple:
\begin{lcode}
\captiontitlefont{\small\sffamily}
\end{lcode}

    There are two options regarding table\index{table} and figure\index{figure} captioning: either
use the \cmd{\legend} command (which produces an anonymous unnumbered title)
instead of the \cmd{\caption} command, or
use the \cmd{\caption} command with a modified definition. Just in case
the design might change at a later date to required numbered captions, 
it's probably best to use 
a modified version of \cmd{\caption}. In this case this is simple, just give
the \cmd{\caption} command the same definition as the \cmd{\legend} command.
\begin{lcode}
\let\caption\legend
\end{lcode}


    An aside: I initially used the default caption style (block paragraph) for
the diagram on \pref{fig:bplayout}, but this looked unbalanced so now it
has the last line centered. As a float\index{float} environment, like any other environment,
forms a group, you can make local changes within the float\index{float}. I actually did it
like this:
\begin{lcode}
\begin{figure}
\captiontitlefont{\small\sffamily}
\captionstyle{\centerlastline}
...
\legend{...} \label{...}
\end{figure} 
\end{lcode}
For fine typesetting you may wish to change the style of particular captions.
The default style for a single line caption works well, but for a caption with
two or three lines either the centering or centerlastline style might look
better. A very long caption is again probably best done in a block paragraph style.

    Only chapter titles are included in the \toc. To specify this we
use the \cmd{\settocdepth} command.
\begin{lcode}
\settocdepth{chapter}
\end{lcode}

    The \toc{} is typeset raggedright with no leaders and the page numbers 
coming immediately after the chapter title. This is specified via:
\begin{lcode}
\renewcommand{\cftchapterfont}{\normalfont}
\renewcommand{\cftchapterpagefont}{\normalfont}
\renewcommand{\cftchapterpresnum}{\bfseries}
\renewcommand{\cftchapterleader}{}
\renewcommand{\cftchapterafterpnum}{\cftparfillskip}
\end{lcode}

\section{Preamble or package?}


    When making changes to the document style, or just defining a new
macro or two, there is the question of where to put the changes --- in
the preamble\index{preamble} of the particular document or into a separate package\index{package}?

    If the same changes/macros are likely to be used in more than one
document then I suggest that they be put into a package. 
If just for the single document then the choice remains open.

    I have presented the code in this chapter as though it would be  put
into the preamble\index{preamble}, hence the use of \cmd{\makeatletter} and 
\cmd{\makeatother} to surround macros that include the |@| character. The
code could just as easily be put into a package\index{package} called, say, 
\Lpack{bringhurst}. That is, by putting all the code, except for the
\cmd{\makeatletter} and \cmd{\makeatother} commands, into a file called
\file{bringhurst.sty}. It is a good idea also to end the code in the file
with \cmd{\endinput}; LaTeX stops reading the file at that point and 
will ignore any possible garbage after \cmd{\endinput}.

    You then use the \Lpack{bringhurst} package just like any other by
putting
\begin{lcode}
\usepackage{bringhurst}
\end{lcode}
in your document's preamble\index{preamble}.

\appendix
\appendixpage*

\chapterstyle{normal}
\chapterstyle{section}

\chapter{Packages and macros}
%%%%\pagestyle{ruled}
This chapter is typeset with the \cstyle{section} chapterstyle, otherwise
it uses the layout defined in \Cref{chap:bringhurst}.

\section{Introduction}

    The \Lclass{memoir} class does not provide for everything that you
have seen in the manual. I have used some packages that you are very likely
to have in your LaTeX distribution, and have supplemented these with some
additional macros, some of which I will show you.

\section{Packages}

    The packages that I have used that you are likely to have, and if
you do  not have them please consider getting them, are:
\begin{itemize}
\item \Lpack{url}~\cite{URL} is for typesetting URL's without worrying
  about special characters or line breaking.
\item \Lpack{fixltx2e}~\cite{FIXLTX2E} eliminates some infelicities
      of the original LaTeX kernel. In particular it maintains the order
      of floats\index{float} on a twocolumn\index{column!double} page and ensures the correct marking
      on a twocolumn\index{column!double} page.
\item \Lpack{alltt}~\cite{ALLTT} is a basic package which provides a 
      verbatim-like environment but |\|, |{|, and |}| have their
      usual meanings (i.e., LaTeX commands are not disabled).
\item \Lpack{graphicx}~\cite{GRAPHICX} is a required package for
      performing various kinds of graphical functions. 
\end{itemize}

    The package that I used and you most likely do not have is 
\Lpack{layouts}~\cite{LAYOUTS}. I used it for all the layout diagrams. 
For example, \fref{fig:displaysechead} and \fref{fig:runsechead} 
were drawn simply by:
\begin{lcode}
\begin{figure}
\centering
\setlayoutscale{1}
\drawparameterstrue
\drawheading{}
\caption{Displayed sectional headings} \label{fig:displaysechead}
\end{figure}

\begin{figure}
\centering
\setlayoutscale{1}
\drawparameterstrue
\runinheadtrue
\drawheading{}
\caption{Runin sectional headings} \label{fig:runsechead}
\end{figure}
\end{lcode}
The package also lets you try experimenting with different layout 
parameters and draw diagrams showing what the results would be in a document.

    The version of \Lpack{layouts} used for this manual is 
v2.4 dated 2001/04/30. Earlier versions will fail when attempting
to draw some figures 
( e.g., to draw \fref{fig:oddstock}).

\section{Macros}

    The preamble\index{preamble} of the manual contains many macro definitions, probably
more than most documents would because:
\begin{itemize}
\item I am having to typeset many LaTeX commands, which require
      some sort of special processing;
\item I have tried to minimize the number of external packages needed
      to LaTeX this manual satisfactorily, and so have copied various
      macros from elsewhere;
\item I wanted to do some automatic indexing\index{index};
\item I wanted to set off the syntax specifications and the code examples
      from the main text.
\end{itemize}
To get the whole glory you will have to read the preamble\index{preamble}, but I show a
few of the macros below as they may be of more general interest.

\begin{syntax}
\cmd{\pstyle}\marg{style} \\
\end{syntax}
The command \cmd{\pstyle} prints its argument in the slanted 
font used for pagestyles
and also makes a pagestyle entry in the index\index{index}. Its definition is
\begin{lcode}
\newcommand{\pstyle}[1]{\textsl{#1}\index{#1pages@\textsl{#1} (pagestyle)}}
\end{lcode}
The first part prints the argument in the text and the second adds an
entry to the \file{.idx} file. The fragment |#1pages| is what 
the \textsc{makeindex} program will use for sorting entries, and the fragment following the |@| character is what will be put into the index\index{index}.

\begin{syntax}
\cmd{\cstyle}\marg{style} \\
\end{syntax}
The command \cmd{\cstyle} prints its argument in the slanted 
font used for chapterstyles 
and also makes a chapterstyle entry in the index\index{index}. Its definition is
\begin{lcode}
\newcommand{\cstyle}[1]{\textsl{#1}\index{#1chaps@\textsl{#1} (chapterstyle)}}
\end{lcode}
which is almost identical to \cmd{\pstyle}. 

    There is both a \cstyle{companion} chapterstyle and a \pstyle{companion}
pagestyle. The strings used for sorting the index\index{index} entries for these are
\texttt{companionchaps} and \texttt{companionpages} respectively, so 
the chapterstyle will come before the pagestyle in the index\index{index}. The reason for distinguishing between the string used for sorting and the actual entry is
partly to distinguish between different kinds of entries for a single name
and partly to avoid any formatting commands messing up the sorted order.



\begin{syntax}
\senv{syntax} syntax \eenv{syntax} \\
\end{syntax}
The \Ie{syntax} environment is for specifying command and environment
syntax. Its definition is
\begin{lcode}
\newenvironment{syntax}{\begin{center}
                        \begin{tabular}{|p{0.9\linewidth}|} \hline}%
                       {\hline
                        \end{tabular}
                        \end{center}}
\end{lcode}
It is implemented in terms of the \Ie{tabular} environment, which forms
a box that will not be broken across a pagebreak. The box frame
is just the normal horizontal and vertical lines that you can use with
a \Ie{tabular}. The width is fixed at 90\% of the text width. As it
is a \Ie{tabular} environment, each line of syntax must be ended with
\cmd{\\}. Note that normal LaTeX processing occurs within the \Ie{syntax}
environment, so you can effectively put what you like inside it.

\begin{syntax}
\senv{lcode} LaTeX code \eenv{lcode} \\
\end{syntax}
I use the \Ie{lcode} environment for showing examples of LaTeX code. It
is a special kind of \Ie{verbatim} environment where the font size is
\cmd{\small} but the normal \lnc{\baselineskip} is used, and each line
is indented. It is defined with the help of the \Lpack{verbatim} package.

    At the bottom the environment is defined in terms of a \Ie{list}, 
although that is not obvious from the code; for details see the 
\Lpack{verbatim} package documentation~\cite{VERBATIM}. I wanted
the environment to be a tight list so started off by defining some
two helper items.
\begin{lcode}
% \@zeroseps sets list before/after skips to minimum values
\newcommand{\@zeroseps}{\setlength{\topsep}{\z@}
                        \setlength{\partopsep}{\z@}
                        \setlength{\parskip}{\z@}}
% \gparindent is the \parindent for the body text
\newlength{\gparindent} \setlength{\gparindent}{\parindent}
\end{lcode}
The macro \cmd{\@zeroseps} sets the before, after and middle skips in
a list to 0pt (\cmd{\z@} is shorthand for 0pt). The value of \lnc{\parindent}
is saved in \lnc{\gparindent}, because this will be the line indentation
in the environment.
\begin{lcode}
% Now we can do the new lcode verbatim environment. 
% This has no extra before/after spacing.
\newenvironment{lcode}{\@zeroseps
  \renewcommand{\verbatim@startline}%
    {\verbatim@line{\hskip\gparindent}}
  \small\setlength{\baselineskip}{\onelineskip}\verbatim}%
  {\endverbatim
   \vspace{-\baselineskip}\noindent
}
\end{lcode}

    The fragment |{\hskip\gparindent}| puts \lnc{\gparindent} space at 
the start of each line.

    The fragment |\small\setlength{\baselineskip}{\onelineskip}| sets the
font size to be \cmd{\small}, which has a smaller \lnc{\baselineskip}
than the normal font, but this is corrected for by changing the local
\lnc{\baselineskip} to the normal skip, \lnc{\onelineskip}. At the end
of the environment there is a negative space of one line to compensate
for a one line space that LaTeX inserts.

    The two versals in \S\ref{sec:versal} were typeset with macros defined
in the preamble\index{preamble}. The first and poorer of the two used the \cmd{\versal}
macro. The second used the \cmd{\drop} macro which was written for
LaTeX v2.09 by David Cantor and Dominik Wujastyk in 1998. Now, if you want
to try your hand at this sort of thing there are some more packages
on CTAN. I have found that the \Lpack{lettrine} package~\cite{LETTRINE} 
serves my needs.



% back end
\backmatter
\clearpage
\pagestyle{ruled}
\chapterstyle{section}

\renewcommand{\prebibhook}{%
CTAN is the Comprehensive TeX Archive Network. Information on how to
access CTAN is available at \url{http://www.tug.org}.
\par\vspace{\onelineskip}}

\begin{thebibliography}{GMS94A}

\bibitem[Ado01]{ADOBEBOOK}
  \emph{How to Create Adobe PDF eBooks}.
  \newblock Adobe Systems Inc.,
  \newblock 2001.
  \newblock (Available from 
             \url{http://www.adobe.com/epaper/tips/acr5ebook/pdfs/eBook.pdf})

\bibitem[Ars99]{URL}
  Donald Arseneau.
  \newblock \emph{The url package}.
  \newblock February, 1999.
  \newblock (Available from CTAN as 
             \url{/macros/latex/contrib/misc/url.sty})

\bibitem[Ars01a]{TITLEREF}
  Donald Arseneau.
  \newblock \emph{The titleref package}.
  \newblock April, 2001.
  \newblock (Available from CTAN as 
             \url{/macros/latex/contrib/misc/titleref.sty})

\bibitem[Ars01b]{FRAMED}
  Donald Arseneau.
  \newblock \emph{The framed package}.
  \newblock July, 2001.
  \newblock (Available from CTAN as 
             \url{/macros/latex/contrib/misc/framed.sty})

\bibitem[Ars01b]{CHAPTERBIB}
  Donald Arseneau.
  \newblock \emph{The chapterbib package}.
  \newblock September, 2001.
  \newblock (Available from CTAN as 
             \url{/macros/latex/contrib/cite/chapterbib.sty})

\bibitem[Bar92]{BAROLINI92}
  Helen Barolini.
  \newblock \emph{Aldus and his Dream Book}.
  \newblock Italica Press
  \newblock (ISBN 0--934977--22--4), 1992.

\bibitem[BDG89]{BIGELOW89}
  Charles Bigelow, Paul Hayden Duensing and Linnea Gentry (Eds).
  \newblock \emph{Fine Print on Type}. 1989.
  \newblock Fine Print, CA (ISBN 0--9607290-X) or
  \newblock Bedford Arts, CA (ISBN 0--938491--17--2).

\bibitem[Boh90]{BOHLE90}
  Robert Bohle.
  \newblock \emph{Publication Design for Editors}.
  \newblock Prentice-Hall,
  \newblock 1990.

\bibitem[Ber02]{JURABIB}
  Jens Berger.
  \newblock \emph{The titlesec and titletoc packages}.
  \newblock September, 2002.
  \newblock (Available from CTAN in 
             \url{/macros/latex/contrib/titlesec})

\bibitem[Bez99]{TITLESEC}
  Javier Bezos.
  \newblock \emph{The titlesec and titletoc packages}.
  \newblock February, 1999.
  \newblock (Available from CTAN in 
             \url{/macros/latex/contrib/titlesec})

\bibitem[Bra94]{MAKEIDX}
  Johannes Braams \textit{et al}.
  \newblock \emph{Standard LaTeX2e packages makeidx and showidx}.
  \newblock November, 1994.
  \newblock (Available from CTAN as 
             \url{/macros/latex/base/makeindx.dtx(ins)})

\bibitem[Bra97]{ALLTT}
  Johannes Braams.
  \newblock \emph{The alltt environment}.
  \newblock June, 1997.
  \newblock (Available from CTAN as 
             \url{/macros/latex/base/alltt.dtx(ins)})

\bibitem[Bri92]{BRINGHURST92}
  Robert Bringhurst.
  \newblock \emph{The Elements of Typographic Style}.
  \newblock Hartley \& Marks
  \newblock (ISBN 0--88179--033--8), 1992.

\bibitem[Bur59]{BURT59}
  C.~L.~Burt.
  \newblock \emph{A Psychological Study of Typography}.
  \newblock Cambridge University Press,
  \newblock 1959.

\bibitem[Car94]{DELARRAY}
  David Carlisle.
  \newblock \emph{The delarray package}.
  \newblock March, 1994.
  \newblock (Available from CTAN in
             \url{/macros/latex/required/tools})

\bibitem[Car95]{AFTERPAGE}
  David Carlisle.
  \newblock \emph{The afterpage package}.
  \newblock October, 1995.
  \newblock (Available from CTAN in
             \url{/macros/latex/required/tools})

\bibitem[Car98a]{COLOR}
  David Carlisle.
  \newblock \emph{The color package}.
  \newblock May, 1998.
  \newblock (Available from CTAN in
             \url{/macros/latex/required/tools})

\bibitem[Car98b]{LONGTABLE}
  David Carlisle.
  \newblock \emph{The longtable package}.
  \newblock May, 1998.
  \newblock (Available from CTAN in
             \url{/macros/latex/required/tools})

\bibitem[Car98c]{ENUMERATE}
  David Carlisle.
  \newblock \emph{The enumerate package}.
  \newblock August, 1998.
  \newblock (Available from CTAN in
             \url{/macros/latex/required/tools})

\bibitem[Car98d]{REMRESET}
  David Carlisle.
  \newblock \emph{The remreset package}.
  \newblock August, 1998.
  \newblock (Available from CTAN in
             \url{/macros/latex/contrib/carlisle})

\bibitem[Car99a]{TABULARX}
  David Carlisle.
  \newblock \emph{The tabularx package}.
  \newblock January, 1999.
  \newblock (Available from CTAN in
             \url{/macros/latex/required/tools})


\bibitem[Car01]{DCOLUMN}
  David Carlisle.
  \newblock \emph{The dcolumn package}.
  \newblock May, 2001.
  \newblock (Available from CTAN in
             \url{/macros/latex/required/tools})

\bibitem[CR99]{GRAPHICX}
  David Carlisle and Sebastian Rahtz.
  \newblock \emph{The graphicx package}.
  \newblock February, 1999.
  \newblock (Available from CTAN in
             \url{/macros/latex/required/graphics})

\bibitem[CB99]{CHAPPELL99}
  Warren Chappell and Robert Bringhurst.
  \newblock \emph{A Short History of the Printed Word}.
  \newblock Hartley \& Marks, 1999.
  \newblock ISBN 0--88179--154--7.

\bibitem[CH88]{CHEN88}
  Pehong Chen and Michael A.~Harrison.
  \newblock `Index Preparation and Processing'.
  \newblock \emph{Software: Practice and Experience}, 19:8, pp. 897--915,
            September, 1988.
  \newblock (Available from CTAN in
             \url{/indexing/makeindex/paper})

\bibitem[Chi93]{CMS93}
  \newblock \emph{The Chicago Manual of Style}, Fourteenth Edition.
  \newblock The University of Chicago
  \newblock (ISBN 0--226--10389--7) 1993.


\bibitem[Coc02]{SUBFIGURE}
  Steven Douglas Cochran.
  \newblock \emph{The subfigure package}.
  \newblock March, 2002.
  \newblock (Available from CTAN in
             \url{/macros/latex/contrib/subfigure})

\bibitem[CG96]{CONWAY96}
  John H.~Conway and Richard K.~Guy.
  \newblock \emph{The Book of Numbers}.
  \newblock Copernicus, Springer-Verlag
  \newblock (ISBN 0--387--97993--X), 1996.

\bibitem[Cra92]{CRAIG92}
  James Craig.
  \newblock \emph{Designing with Type: A Basic Course in Typography}.
  \newblock Watson-Guptill, NY,
  \newblock 1992.

\bibitem[Dal99]{NATBIB}
  Patrick W.~Daly.
  \newblock \emph{Natural Sciences Citations and References}.
  \newblock May, 1999.
  \newblock (Available from CTAN in
             \url{/macros/latex/contrib/natbib})

\bibitem[Deg92]{DEGANI92}
  Asaf Degani.
  \newblock \emph{On the Typography of Flight-Deck Documentation}.
  \newblock NASA Contractor Report \# 177605.
  \newblock December, 1992.
  \newblock (Available from 
             \url{http://members.aol.com/willadams/typgrphy.htm#NASA})


\bibitem[Dow96]{DOWDING96}
  Geoffrey Dowding.
  \newblock \emph{Finer Points in the Spacing \& Arrangement of Type}.
  \newblock Hartley \& Marks
  \newblock (ISBN 0--88179--119--9), 1996.

\bibitem[Dow98]{DOWDING98}
  Geoffrey Dowding.
  \newblock \emph{An Introduction to the History of Printing Types}.
  \newblock The British Library and Oak Knoll Press
  \newblock (ISBN 0--7123--4563--9 \textsc{uk},
                  1--884718--44--2 \textsc{usa}), 1998.

\bibitem[Dow00]{PATCHCMD}
  Michael J.~Downes.
  \newblock \emph{The patchcmd package}.
  \newblock July, 2000.
  \newblock (Available from CTAN in
             \url{/macros/latex/contrib/patchcmd})

\bibitem[Eij92]{EIJKHOUT92}
  Victor Eijkhout.
  \newblock \emph{TeX by Topic}.
  \newblock Addison-Wesley, 1992.
  \newblock ISBN 0--201--56882--9.
  \newblock (Available from \url{http://www.eijkhout.net/tbt/}).

\bibitem[Fai00]{FOOTMISC}
  Robin Fairbairns.
  \newblock \emph{footmisc --- a portmanteau package for customising
                  footnotes in LaTeX2e}.
  \newblock March, 2000.
  \newblock (Available from CTAN in
            \url{/macros/latex/contrib/footmisc})

\bibitem[Fai98]{MOREVERB}
  Robin Fairbairns.
  \newblock \emph{The moreverb package}.
  \newblock December, 1998.
  \newblock (Available from CTAN in
            \url{/macros/latex/contrib/moreverb})

\bibitem[FAQ]{FAQ}
  Robin Fairbairns.
  \newblock \emph{The UK TeX FAQ}.
  \newblock (Available from CTAN in
            \url{/help/uk-tex-faq})

\bibitem[Fea03]{BOOKTABS}
  Simon Fear.
  \newblock \emph{Publication quality tables in LaTeX}.
  \newblock March, 2003.
  \newblock (Available from CTAN in
            \url{/macros/latex/contrib/booktabs})

\bibitem[Fli98]{LETTRINE}
  Daniel Flipo.
  \newblock \emph{Typesetting `lettrines' in LaTeX2e documents}.
  \newblock March, 1998.
  \newblock (Available from CTAN in 
             \url{/macros/latex/contrib/lettrine})

\bibitem[Fra00]{CROP}
  Melchior Franz.
  \newblock \emph{The crop package}.
  \newblock February, 2000.
  \newblock (Available from CTAN in 
             \url{/macros/latex/contrib/crop})

\bibitem[FOS98]{FRIEDL98}
  Friedrich Friedl, Nicolaus Ott and Bernard Stein.
  \newblock \emph{Typography: An Encyclopedic Survey of Type Designs and
                  Techniques throughout History}.
  \newblock Black Dog \& Leventhal Publishers Inc.
  \newblock (ISBN 1--57912--023--7), 1998.

\bibitem[Gar66]{GARDNER66}
  Martin Gardner.
  \newblock \emph{More Mathematical Puzzles and Diversions}.
  \newblock Penguin Books
  \newblock (ISBN 0--14--020748--1), 1966.

\bibitem[GMS94]{GOOSSENS94}
  Michel Goossens, Frank Mittelbach and Alexander Samarin.
  \newblock \emph{The LaTeX Companion}.
  \newblock Addison-Wesley Publishing Company
  \newblock (ISBN 0--201--54199--8), 1994.

\bibitem[GRM97]{GOOSSENS97}
  Michel Goossens, Sebastian Rahtz and Frank Mittelbach.
  \newblock \emph{The LaTeX Graphics Companion: Illustrating Documents
                  with TeX and PostScript}.
  \newblock Addison-Wesley Publishing Company
  \newblock (ISBN 0--201--85469--4), 1997.

\bibitem[GR99]{GOOSSENS99}
  Michel Goossens and Sebastian Rahtz (with Eitan Gurari,
  Ross Moore and Robert Sutor).
  \newblock \emph{The LaTeX Web Companion: Integrating TeX, HTML and XML}.
  \newblock Addison-Wesley Publishing Company
  \newblock (ISBN 0--201--43311--7), 1999.

\bibitem[Gou87]{GOULD87}
  J.~D.~Gould \textit{et al}.
  \newblock `Reading from CRT displays can be as fast as reading from paper'.
  \newblock \emph{Human Factors}, pp 497--517, 29:5, 1987.

\bibitem[HR83]{HARTLEY83}
  J.~Hartley and D.~Rooum.
  \newblock `Sir Cyril Burt and typography'.
  \newblock \emph{British Journal of Psychology}, pp 203--212, 74:2, 1983.

\bibitem[HM01]{HELLER01}
  Steven Heller and Philip B.~Maggs (Eds).
  \newblock \emph{Texts on Type: Critical Writings on Typography}.
  \newblock Allworth Press
  \newblock (ISBN 1--58115--082--2), 2001.

\bibitem[Hoe98]{HOENIG98}
  Alan Hoenig.
  \newblock \emph{TeX Unbound: LaTeX and TeX strategies for fonts,
                  graphics, and more}.
  \newblock Oxford University Press
  \newblock (ISBN 0--19--509686--X), 1998.

\bibitem[HK75]{HVISTENDAHL75}
  J.~K.~Hvistendahl and M.~R.~Kahl.
  \newblock `Roman vs. sans serif body type: Readability and reader prference'.
  \newblock \emph{AANPA News Research Bulletin}, pp 3--11, 17 Jan., 1975.


\bibitem[Jon95]{INDEX}
  David M.~Jones.
  \newblock \emph{A new implementation of LaTeX's indexing commands}.
  \newblock September, 1995.
  \newblock (Available from CTAN in \url{/macros/latex/contrib/camel})

\bibitem[Knu84]{KNUTH84a}
  Donald E.~Knuth.
  \newblock \emph{The TeXbook}.
  \newblock Addison-Wesley Publishing Company
  \newblock (ISBN 0--201--13448--9), 1984.

\bibitem[Lam94]{LAMPORT94}
  Leslie Lamport.
  \newblock \emph{LaTeX: A Document Preparation System}.
  \newblock Addison-Wesley Publishing Company
  \newblock (ISBN 0--201--52983--1), 1994.

\bibitem[LMB99]{CLASSES}
  Leslie Lamport, Frank Mittelbach and Johannes Braams.
  \newblock \emph{Standard document classes for LaTeX version 2e}.
  \newblock September, 1999.
  \newblock (Available from CTAN as \url{/macros/latex/base/classes.dtx})

\bibitem[Law90]{LAWSON90}
  Alexander Lawson.
  \newblock \emph{Anatomy of a Typeface}.
  \newblock David R.~Godine 
  \newblock (ISBN 0--87923--333--8), 1990.

\bibitem[Leu92]{LEUNEN92}
  Mary-Claire van Leunen.
  \newblock \emph{A Handbook for Scholars}.
  \newblock Oxford University Press
  \newblock (ISBN 0--19--506954--4), 1992.

\bibitem[Lon91]{MULTIND}
  F.~W.~Long.
  \newblock \emph{multind}.
  \newblock August, 1991.
  \newblock (Available from CTAN as \url{/macros/latex209/contrib/misc/multind.sty})

\bibitem[McD98]{SECTSTY}
  Rowland McDonnell.
  \newblock \emph{The sectsty package}.
  \newblock November, 1998.
  \newblock (Available from CTAN in 
             \url{/macros/latex/contrib/secsty})

\bibitem[McL75]{MCLEAN75}
  Ruari McLean.
  \newblock \emph{Jan Tschichold: Typographer}.
  \newblock David R.~Godine
  \newblock (ISBN 0--87923--841--0), 1975.

\bibitem[McL80]{MCLEAN80}
  Ruari McLean.
  \newblock \emph{The Thames \& Hudson Manual of Typography}.
  \newblock Thames \& Hudson
  \newblock (ISBN 0--500--68022--1), 1980.

\bibitem[McL95]{MCLEAN95}
  Ruari McLean (Ed).
  \newblock \emph{Typographers on Type}.
  \newblock W.~W.~Norton \& Co.
  \newblock (ISBN 0--393--70201--4), 1995.

\bibitem[Mit95]{DOC}
  Frank Mittelbach.
  \newblock \emph{The doc and shortvrb packages}.
  \newblock May, 1995.
  \newblock (Available from CTAN in 
            \url{/macros/latex/base})

\bibitem[Mit95]{SHORTVRB}
  Frank Mittelbach.
  \newblock \emph{The doc and shortvrb packages}.
  \newblock May, 1995.
  \newblock (Available from CTAN in 
            \url{/macros/latex/base})

\bibitem[Mit98]{MULTICOL}
  Frank Mittelbach.
  \newblock \emph{An environment for multicolumn output}.
  \newblock January, 1998.
  \newblock (Available from CTAN in 
            \url{/macros/latex/required/tools})

\bibitem[MC98]{ARRAY}
  Frank Mittelbach and David Carlisle.
  \newblock \emph{A new implementation of LaTeX's tabular and array environment}.  \newblock May, 1998.
  \newblock (Available from CTAN in 
            \url{/macros/latex/required/tools})

\bibitem[MC00]{FIXLTX2E}
  Frank Mittelbach and David Carlisle.
  \newblock \emph{The fixltx2e package}.
  \newblock September, 2000.
  \newblock (Available from CTAN in 
            \url{/macros/latex/base})

\bibitem[Mor99]{MORISON99}
  Stanley Morison.
  \newblock \emph{A Tally of Types}.
  \newblock David R. Godine
  \newblock (ISBN 1--56792--004--7), 1999.

\bibitem[NG98]{SIDECAP}
  Rolf Niespraschk and Hubert G\"{a}\ss{}lein. 
  \newblock \emph{The sidecap package}.
  \newblock June, 1998.
  \newblock (Available from CTAN in 
            \url{/macros/latex/contrib/sidecap})

\bibitem[Oet]{LSHORT}
  Tobias Oetiker.
  \newblock \emph{The Not So Short Introduction to LaTeX2e}.
  \newblock (Available from CTAN in 
            \url{/info/lshort/english})

\bibitem[Oos96]{FANCYHDR}
  Piet van Oostrum.
  \newblock \emph{Page Layout in LaTeX}.
  \newblock June, 1996.
  \newblock (Available from CTAN in 
            \url{/macros/latex/contrib/fancyhdr})

\bibitem[Pak01]{SYMBOLS}
  Scott Pakin.
  \newblock \emph{The Comprehensive LaTeX Symbol List}.
  \newblock July, 2001.
  \newblock (Available from CTAN in 
            \url{/info/symbols/comprehensive})

\bibitem[Pug02]{MATHPAZO}
  Diego Puga.
  \newblock \emph{The Pazo Math fonts for mathematical typesetting
                  with the Palatino fonts}.
  \newblock May, 2002.
  \newblock (Available from CTAN in 
            \url{/fonts/mathpazo})

\bibitem[Rahtz01]{NAMEREF}
  Sebastian Rahtz.
  \newblock \emph{Section name references in LaTeX}.
  \newblock January, 2001.
  \newblock (Available from CTAN in 
            \url{/macros/latex/contrib/hyperref})

\bibitem[Rahtz02]{HYPERREF}
  Sebastian Rahtz.
  \newblock \emph{Hypertext marks in LaTeX}.
  \newblock May, 2002.
  \newblock (Available from CTAN in 
            \url{/macros/latex/contrib/hyperref})

\bibitem[Rec97]{EPSLATEX}
  Keith Reckdahl.
  \newblock \emph{Using Imported Graphics in LaTeX2e}.
  \newblock December, 1997.
  \newblock (Available from CTAN as
             \url{/info/epspatex.ps} or \url{/info/epslatex.pdf})

\bibitem[Reh72]{REHE72}
  Rolf Rehe.
  \newblock `Type and how to make it most legible'.
  \newblock \emph{Design Research International}, 1972.

\bibitem[RAE71]{ROBINSON71}
  D.~O.~Robinson, M.~Abbamonte and S.~H.~Evans.
  \newblock `Why serifs are important: The perception of small print'.
  \newblock \emph{Visible Language}, pp 353--359, 4, 1971.

\bibitem[Rog43]{ROGERS43}
  Bruce Rogers.
  \newblock \emph{Paragraphs on Printing}.
  \newblock William E. Rudge's Sons, Inc.
  \newblock (no ISBN), 1943.
  \newblock (Reissued by Dover, 1979, ISBN 0--486--23817--2)

\bibitem[Rog49]{ROGERS49}
  Bruce Rogers.
  \newblock \emph{Centaur Types}.
  \newblock October House
  \newblock (no ISBN), 1949.

\bibitem[SW94]{EBOOK}
  Douglas Schenck and Peter Wilson.
  \newblock \emph{Information Modeling the EXPRESS Way}.
  \newblock Oxford University Press
  \newblock (ISBN 0--19--508714--3), 1994.

\bibitem[SRR99]{VERBATIM}
  Rainer Sch\"{o}pf, Bernd Raichle and Chris Rowley.
  \newblock \emph{A New Implementation of LaTeX's verbatim
                  and verbatim* Environments}.
  \newblock December, 1999.
  \newblock (Available from CTAN in
            \url{/macros/latex/required/tools})

\bibitem[Sch97]{SCHRIVER97}
  Karen A.~Schriver.
  \newblock \emph{Dynamics in Document Design}.
  \newblock Wiley \& Sons, 1997.

\bibitem[Thi99]{TTC199}
  Christina Thiele.
  \newblock `The Treasure Chest: Package tours from CTAN', 
  \newblock \emph{TUGboat}, 
  \newblock vol. 20, no. 1, pp 53--58, March 1999.

\bibitem[Tin63]{TINKER63}
  Miles A.~Tinker.
  \newblock \emph{Legibility of Print}.
  \newblock Books on Demand (University Microfilms International), 1963.

\bibitem[Tsc91]{TSCHICHOLD91}
  Jan Tschichold.
  \newblock \emph{The Form of the Book}.
  \newblock Lund Humphries
  \newblock (ISBN 0--85331--623--6), 1991.

\bibitem[Ume99]{GEOMETRY}
  Hideo Umeki.
  \newblock \emph{The geometry package}.
  \newblock November, 1999.
  \newblock (Available from CTAN in
            \url{/macros/latex/contrib/geometry})


\bibitem[Whe95]{WHEILDON95}
  Colin Wheildon.
  \newblock \emph{Type \& Layout}.
  \newblock Strathmore Press
  \newblock (ISBN 0--9624891--5--8), 1995.

\bibitem[Wil93]{ADRIANWILSON93}
  Adrian Wilson.
  \newblock \emph{The Design of Books}.
  \newblock Chronicle Books
  \newblock (ISBN 0--8118--0304--X), 1993.



\bibitem[Wil99a]{LAYOUTS}
  Peter Wilson.
  \newblock \emph{The layouts package}.
  \newblock January, 1999.
  \newblock (Available from CTAN in 
            \url{/macros/latex/contrib/layouts})

\bibitem[Wil99b]{TOCVSEC2}
  Peter Wilson.
  \newblock \emph{The tocvsec2 package}.
  \newblock January, 1999.
  \newblock (Available from CTAN in 
            \url{/macros/latex/contrib/tocvsec2})

\bibitem[Wil00a]{EPIGRAPH}
  Peter Wilson.
  \newblock \emph{The epigraph package}.
  \newblock February, 2000.
  \newblock (Available from CTAN in 
            \url{/macros/latex/contrib/epigraph})

\bibitem[Wil00b]{ISOCLASS}
  Peter Wilson.
  \newblock \emph{LaTeX files for typesetting ISO standards}.
  \newblock February, 2000.
  \newblock (Available from CTAN in 
            \url{/macros/latex/contrib/isostds/iso})

\bibitem[Wil00c]{NEXTPAGE}
  Peter Wilson.
  \newblock \emph{The nextpage package}.
  \newblock February, 2000.
  \newblock (Available from CTAN as 
            \url{/macros/latex/contrib/misc/nextpage.sty})

\bibitem[Wil00d]{NEEDSPACE}
  Peter Wilson.
  \newblock \emph{The needspace package}.
  \newblock March, 2000.
  \newblock (Available from CTAN as 
            \url{/macros/latex/contrib/misc/needspace.sty})

\bibitem[Wil00e]{XTAB}
  Peter Wilson.
  \newblock \emph{The xtab package}.
  \newblock April 2000.
  \newblock (Available from CTAN in 
             \texttt{macros/latex/contrib/xtab})

\bibitem[Wil01a]{ABSTRACT}
  Peter Wilson.
  \newblock \emph{The abstract package}.
  \newblock February, 2001.
  \newblock (Available from CTAN in 
            \url{/macros/latex/contrib/abstract})

\bibitem[Wil01b]{CHNGPAGE}
  Peter Wilson.
  \newblock \emph{The chngpage package}.
  \newblock February, 2001.
  \newblock (Available from CTAN as 
            \url{/macros/latex/contrib/misc/chngpage.sty})

\bibitem[Wil01c]{APPENDIX}
  Peter Wilson.
  \newblock \emph{The appendix package}.
  \newblock March, 2001.
  \newblock (Available from CTAN in 
            \url{/macros/latex/contrib/appendix})

\bibitem[Wil01d]{CCAPTION}
  Peter Wilson.
  \newblock \emph{The ccaption package}.
  \newblock March, 2001.
  \newblock (Available from CTAN in 
            \url{/macros/latex/contrib/ccaption})

\bibitem[Wil01e]{CHNGCNTR}
  Peter Wilson.
  \newblock \emph{The chngcntr package}.
  \newblock April, 2001.
  \newblock (Available from CTAN as 
            \url{/macros/latex/contrib/misc/chngcntr.sty})

\bibitem[Wil01f]{HANGING}
  Peter Wilson.
  \newblock \emph{The hanging package}.
  \newblock March, 2001.
  \newblock (Available from CTAN in 
            \url{/macros/latex/contrib/hanging})

\bibitem[Wil01g]{TITLING}
  Peter Wilson.
  \newblock \emph{The titling package}.
  \newblock March, 2001.
  \newblock (Available from CTAN in 
            \url{/macros/latex/contrib/titling})

\bibitem[Wil01h]{TOCBIBIND}
  Peter Wilson.
  \newblock \emph{The tocbibind package}.
  \newblock April, 2001.
  \newblock (Available from CTAN in 
            \url{/macros/latex/contrib/tocbibind})


\bibitem[Wil01i]{TOCLOFT}
  Peter Wilson.
  \newblock \emph{The tocloft package}.
  \newblock April, 2001.
  \newblock (Available from CTAN in 
            \url{/macros/latex/contrib/tocloft})


\bibitem[Wil01j]{MEMOIR}
  Peter Wilson.
  \newblock \emph{The LaTeX memoir class for configurable book 
                  typesetting: Source code}.
  \newblock July, 2001.
  \newblock (Available from CTAN in 
            \url{/macros/latex/contrib/memoir})

\bibitem[Wil01k]{VERSE}
  Peter Wilson.
  \newblock \emph{Typesetting simple verse with LaTeX}
  \newblock July, 2001.
  \newblock (Available from CTAN in 
            \url{/macros/latex/contrib/verse})

\bibitem[Wil01l]{BOOKLET}
  Peter Wilson.
  \newblock \emph{Printing booklets with LaTeX}
  \newblock August, 2001.
  \newblock (Available from CTAN in 
            \url{/macros/latex/contrib/booklet})

\bibitem[Wil03]{LEDMAC}
  Peter Wilson.
  \newblock \emph{ledmac: A presumptuous attaempt to port EDMAC and TABMAC
                  to LaTeX}
  \newblock November, 2003.
  \newblock (Available from CTAN in 
            \url{/macros/latex/contrib/ledmac})

\bibitem[Wil??]{RUMOUR}
Peter Wilson.
\newblock \emph{A Rumour of Humour: A scientist's commonplace book}.
\newblock To be published?


\bibitem[Zac69]{ZACHRISSOM69}
  B.~Zachrissom.
  \newblock \emph{Studies in the Legibility of Printed Text}.
  \newblock Almqvist \& Wiksell, Stockholm, 1969.

\bibitem[Zap00]{ZAPF00}
  Hermann Zapf.
  \newblock \emph{The Fine Art of Letters}.
  \newblock The Grolier Club
  \newblock (ISBN 0--910672--35--0), 2000.



\end{thebibliography}

\clearpage
\pagestyle{index}
%\renewcommand{\chaptermark}[1]{}
\renewcommand{\preindexhook}{%
The first page number is usually, but not always, the primary reference to
the indexed topic.\vskip\onelineskip}
\indexintoc
\printindex

\cleardoublepage
\pagestyle{empty}
\null\vfil

\begin{adjustwidth}{1in}{1in}
\begin{center}
{\Large\textsf{Colophon}}
\end{center}
\begin{center}
This manual was typeset using the LaTeX typesetting system
created by Leslie Lamport and the memoir class. 
The body text is set 10/12pt on a
33pc measure with Computer
Modern Roman designed by Donald Knuth. Other fonts include
Sans, Smallcaps, Italic, Slanted and Typewriter, all from Knuth's 
Computer Modern family.

\end{center}

\end{adjustwidth}

\vfil

\end{document}

\endinput

%%%%%%%%%%%%%%%%%%%%%%%%%%%%%%%%%%%%%%%%%%%%%%%%%%%%%%%%%%%%%%%%%%%%%%%%%%%%%%%%%%%

%%%%%%%%%%%%
patchit  The addendum
%%%%%%%%%%%

% memmanadd.tex     Addendum to the Memoir class user manual
%                   Author: Peter Wilson
%                   Copyright 2002, 2003 Peter R. Wilson
%                             All rights reserved



%\AtBeginDocument{%
%  \ifpdf
%    \ifnum\pdfoutput>0\relax
%      \pdfpageheight=\the\stockheight
%      \pdfpagewidth=\the\stockwidth
%      \pdfvorigin=1in
%      \pdfhorigin=1in
%    \fi
%  \fi}

%%%%%%%%%%%%%%%%%% change fonts
%\renewcommand{\rmdefault}{ppl}   % palatino
\usepackage{mempalatino}
%%\usepackage{pifont}

%%%%%%\usepackage{amsmath}

\ifpdf
  \pdfoutput=1
  \usepackage[pdftex,
              plainpages=false,
              pdfpagelabels,
  hyperfootnotes=false,
              bookmarksnumbered
             ]{hyperref} 
  \usepackage{memhfixc}
\else
  \usepackage[plainpages=false,
              pdfpagelabels,
  hyperfootnotes=false,
              bookmarksnumbered
             ]{hyperref} 
  \usepackage{memhfixc}
\fi


%\showindexmarktrue

\newcommand{\theclass}{memoir}

\providecommand{\tx}{TeX}
\providecommand{\ltx}{LaTeX}

\newsubfloat{table}


\makeatletter

%%% Print and Index macros
\newcommand{\Ppstyle}[1]{\textsl{#1}}
\newcommand{\pstyle}[1]{\Ppstyle{#1}%
            \index{#1 pages?\Ppstyle{#1} (pagestyle)}%
            \index{pagestyle!#1?\Ppstyle{#1}}}            % pagestyle
\newcommand{\Pcstyle}[1]{\textsl{#1}}
\newcommand{\cstyle}[1]{\Pcstyle{#1}%
            \index{#1 chaps?\Pcstyle{#1} (chapterstyle)}%
            \index{chapterstyle!#1?\Pcstyle{#1}}}          % chapterstyle
\newcommand{\Pclass}[1]{\textsf{#1}}
\newcommand{\Lclass}[1]{\Pclass{#1}%
            \index{#1 class?\Pclass{#1} (class)}%
            \index{class!#1?\Pclass{#1}}}                % class name
\newcommand{\Ppack}[1]{\textsf{#1}}
\newcommand{\Lpack}[1]{\Ppack{#1}%
            \index{#1 pack?\Ppack{#1} (package)}%
            \index{package!#1?\Ppack{#1}}}              % package name
\newcommand{\Popt}[1]{\textsf{#1}}
\newcommand{\Lopt}[1]{\Popt{#1}%
            \index{#1 opt?\Popt{#1} (option)}%
            \index{option!#1?\Popt{#1}}}               % option name
\newcommand{\Ie}[1]{\texttt{#1}\index{#1 env?\texttt{#1} (environment)}%
                               \index{environment!#1?\texttt{#1}}}
\newcommand{\Icn}[1]{\texttt{#1}\index{#1 cou?\texttt{#1} (counter)}%
                                \index{counter!#1?\texttt{#1}}}
\newcommand{\Itt}[1]{\texttt{#1}\index{#1tt?\texttt{#1}}}

% print and index an \if... \piif{if...}
\newcommand{\piif}[1]{\cs{#1}\index{#1?\cs{#1}}}

% index command \!
\newcommand{\iexcl}{\index{"!?\cs{!}}}

%%%\newcommand{\seealso}[2]{\emph{see also} #1}

\newcommand{\listofx}{`List of\ldots'}

% print a begin environment
\newcommand{\senv}[1]{\texttt{\bs begin\{#1\}}}

% print an end environment
\newcommand{\eenv}[1]{\texttt{\bs end\{#1\}}}

% print a file name
\newcommand{\file}[1]{\texttt{#1}}

% print and index a file name
\newcommand{\ixfile}[1]{\file{#1}%
            \index{#1 file?\file{#1} (file)}%
            \index{file!#1?\file{#1}}}

% print CTT
\newcommand{\ctt}{\textsc{ctt}}
% print & index CTT
\newcommand{\pictt}{\ctt\index{CTT?\ctt}}

% index marking
\newcommand{\idxmark}[1]{#1\markboth{#1}{#1}}

\newcommand{\foredge}{foredge}
\newlength{\pwlayi}\setlength{\pwlayi}{0.45\textwidth} % 
\newlength{\pwlayii}\setlength{\pwlayii}{0.45\pwlayi}

%% definition of \meta is taken from doc.dtx
\begingroup
\obeyspaces%
\catcode`\^^M\active%
\gdef\meta{\begingroup\obeyspaces\catcode`\^^M\active%
\let^^M\do@space\let \do@space%
\def\-{\egroup\discretionary{-}{}{}\hbox\bgroup\itshape}%
\m@ta}%
\endgroup
\def\m@ta#1{\leavevmode\hbox\bgroup$\langle$\itshape#1\/$\rangle$\egroup
    \endgroup}
\def\do@space{\egroup\space
    \hbox\bgroup\itshape\futurelet\next\sp@ce}
\def\sp@ce{\ifx\next\do@space\expandafter\sp@@ce\fi}
\def\sp@@ce#1{\futurelet\next\sp@ce}
%% end of \meta and supports


%% The definition of \MakeShortVerb & \DeleteShortVerb
%% are now in the memoir class.

%% The macros \cmd, \cs, \marg, \oarg, \parg are taken from ltxdoc.dtx
% \cmd{\fred} typesets \fred
\def\cmd#1{\cs{\expandafter\cmd@to@cs\string#1}}
\def\cmd@to@cs#1#2{\char\number`#2\relax}
% \cs{fred} typesets \fred
\DeclareRobustCommand{\cs}[1]{\texttt{\char`\\#1}}
% \marg{arg} typesets {<arg>}
\providecommand{\marg}[1]{%
  {\ttfamily\char`\{}\meta{#1}{\ttfamily\char`\}}}
% \oarg{arg} typesets [<arg>]
\providecommand{\oarg}[1]{%
  {\ttfamily\char`\[}\meta{#1}{\ttfamily\char`\]}}
% \parg{x,y} typesets (<x,y>)
\providecommand{\parg}[1]{%
  {\ttfamily\char`\(}\meta{#1}{\ttfamily\char`\)}}

\def\bs{\texttt{\char`\\}}

%%% macro \cmd from Heiko Oberdiek CTT 2001/05/26 (prints and indexes a command)
\newcommand*{\cmdprint}[1]{\texttt{\string#1}}
\def\cmd#1{\cmdprint{#1}%
  \index{\expandafter\@gobble\string#1?\string\cmdprint{\string#1}}}

% print and index \\!
\newcommand{\pixslashbang}{\cmdprint{\\!}\index{"\"\"!?\string\cmdprint{\\}\texttt{"!}}}

% print and index \!
\newcommand{\pixabang}{\cmdprint{\!}\index{"!?\string\cs{}\texttt{"!}}}

% print and index a length
\newcommand{\lnc}[1]{\cmdprint{#1}%
  \index{\expandafter\@gobble\string#1len?\string\cmdprint{\string#1} (length)}%
  \index{length!\expandafter\@gobble\string#1len?\string\cmdprint{\string#1}}}

%%%%%% stuff for new LaTeX code environment

% \zeroseps sets list before/after skips to minimum values
\newcommand{\@zeroseps}{\setlength{\topsep}{\z@}
                        \setlength{\partopsep}{\z@}
                        \setlength{\parskip}{\z@}}
\newlength{\gparindent} \setlength{\gparindent}{\parindent}
\setlength{\gparindent}{0.5\parindent}
% now we can do the new environment. This has no extra before/after spacing.
\newenvironment{lcode}{\@zeroseps
  \renewcommand{\verbatim@startline}{\verbatim@line{\hskip\gparindent}}
  \small\setlength{\baselineskip}{\onelineskip}\verbatim}%
  {\endverbatim
   \vspace{-\baselineskip}%
   \noindent
}

%%%% this is the code environment for the book
\newenvironment{bcode}{\@zeroseps
  \renewcommand{\verbatim@processline}{%
    \hskip\gparindent\the\verbatim@line\par}%
  \small\setlength{\baselineskip}{\onelineskip}\verbatim}%
  {\endverbatim\noindent}

%%%%%%%%%%%% for experiments

%\renewenvironment{lcode}{\@zeroseps
%  \renewcommand{\verbatim@startline}{\verbatim@line{ * }}
%  \small\setlength{\baselineskip}{\onelineskip}\verbatim}%
%  {\endverbatim
%   \vspace{-\baselineskip}  
%   \noindent % need the \noindent
%}

%\renewenvironment{bcode}{\@zeroseps
%  \renewcommand{\verbatim@processline}{%
%    * \the\verbatim@line\par}%
%  \small\setlength{\baselineskip}{\onelineskip}\verbatim}%
%  {\endverbatim\noindent}  % need the \noindent



%%%%%%%%%%%%%%%%%%%%%%%%%%%%%%%%%%%%%%%%%%%%%%%%%%%%%

%%%%% LaTeX syntax
\newenvironment{syntax}{\begin{center}
                        \begin{tabular}{|p{0.9\linewidth}|} \hline}%
                       {\hline
                        \end{tabular}
                        \end{center}}

%%%%%%%%%%%%%%%%%% index pagestyle
\makepagestyle{index}
  \makeheadrule{index}{\textwidth}{\normalrulethickness}
  \makeevenhead{index}{\rightmark}{}{\leftmark}
  \makeoddhead{index}{\rightmark}{}{\leftmark}
  \makeevenfoot{index}{\thepage}{}{}
  \makeoddfoot{index}{}{}{\thepage}


%%%%%%%%%%%%%%%%%%%%%%%%%%%%%%%%%%%%%%
%Date: Tue, 22 Jul 2003 19:27:13 +0200
%From: Bastiaan Veelo <Bastiaan.N.Veelo@immtek.ntnu.no>
%Newsgroups: comp.text.tex
%CC: "Wilson, Peter R" <peter.r.wilson@boeing.com>
%Subject: [memoir] [contrib] New chapter style
%
%Hello all,
%
%In case anyone is interested, I thought I'd share a new chapter style to 
%be used with the memoir class by Peter Wilson. The style is tailored at 
%documents that are to be trimmed to a smaller width. When the binded 
%document is bent, black tabs will appear on the fore side at the places 
%where new chapters start, as a navigational aid. Sample attached.
%
%Regards,
%Bastiaan.

%% A new chapter style, that suits well for trimmed documents.
%% We are scaling the chapter number, which most DVI viewers
%% will not display accurately.
\makeatletter
   \newlength{\numberheight}
   \newlength{\barlength}
%\makechapterstyle{trim}{%
\makechapterstyle{veelo}{%
   \setlength{\beforechapskip}{40pt}
   \setlength{\midchapskip}{25pt}
   \setlength{\afterchapskip}{40pt}
   \renewcommand{\chapnamefont}{\normalfont\LARGE\flushright}
   \renewcommand{\chapnumfont}{\normalfont\HUGE}
   \renewcommand{\chaptitlefont}{\normalfont\HUGE\bfseries\flushright}
   \renewcommand{\printchaptername}{%
     \chapnamefont\MakeUppercase{\@chapapp}}
   \renewcommand{\chapternamenum}{}
%   \newlength{\numberheight}
%   \newlength{\barlength}
   \setlength{\numberheight}{18mm}
   \setlength{\barlength}{\paperwidth}
   \addtolength{\barlength}{-\textwidth}
   \addtolength{\barlength}{-\spinemargin}
   \renewcommand{\printchapternum}{%
     \makebox[0pt][l]{%
       \hspace{.8em}%
       \resizebox{!}{\numberheight}{\chapnumfont \thechapter}%
       \hspace{.8em}%
       \rule{\barlength}{\numberheight}
     }
   }
   \makeoddfoot{plain}{}{}{\thepage}
}
\makeatother

%%\chapterstyle{veelo}


\MakeShortVerb{\|}
% \DeleteShortVerb{\|}


%%% need more space for ToC page numbers
\setpnumwidth{2.55em}
\setrmarg{3.55em}

%%% need more space for ToC section numbers
\cftsetindents{section}{1.5em}{3.0em}
\cftsetindents{subsection}{4.5em}{3.9em}
\cftsetindents{subsubsection}{8.4em}{4.8em}
\cftsetindents{paragraph}{10.7em}{5.7em}
\cftsetindents{subparagraph}{12.7em}{6.7em}

%%% need more space for LoF numbers
\cftsetindents{figure}{0em}{3.0em}

%%% and do the same for the LoT
\cftsetindents{table}{0em}{3.0em}


%%% set up the page layout
\settrimmedsize{11in}{210mm}{*}
\setlength{\trimtop}{0pt}
\setlength{\trimedge}{\stockwidth}
\addtolength{\trimedge}{-\paperwidth}
\settypeblocksize{7.75in}{33pc}{*}
\setulmargins{4cm}{*}{*}
\setlrmargins{1.25in}{*}{*}
\setmarginnotes{17pt}{51pt}{\onelineskip}
\setheadfoot{\onelineskip}{2\onelineskip}
\setheaderspaces{*}{2\onelineskip}{*}
\checkandfixthelayout


\bibliographystyle{alpha}

\newcommand{\prtoc}{ToC}             % print ToC
\newcommand{\prlof}{LoF}
\newcommand{\prlot}{LoT}
\newcommand{\ixtoc}{\index{ToC}}     % index ToC
\newcommand{\ixlof}{\index{LoF}}
\newcommand{\ixlot}{\index{LoT}}
\newcommand{\toc}{\prtoc\ixtoc}      % print & index ToC
\newcommand{\lof}{\prlof\index{LoF}}
\newcommand{\lot}{\prlot\index{LoT}}

\newcommand{\noidxnum}[1]{}

\newcommand{\Note}{\textbf{NOTE:}}

%%% subfigures
\newsubfloat{figure}
\tightlists

%% end preamble
\typeout{The end of the preamble, begin document is next}
%%%%%%%%%%%%%%%%%%%%%%%%%%%%%%%%%%%%%%%%%%%%%%%%%%%%%%%
\begin{document}
%%%%%%%%%%%%%%%%%%%%%%%%%%%%%%%%%%%%%%%%%%%%%%%%%%%%%%%

%% list subfigures
\setcounter{lofdepth}{2}

\index{counter|noidxnum}
\index{file|noidxnum}
\index{environment|noidxnum}
\index{length|noidxnum}
\index{package|noidxnum}
\index{option|noidxnum}

\index{pagestyle|noidxnum}
\index{chapterstyle|noidxnum}

\begin{comment}
\index{table of contents|see{ToC}}
\index{list!of figures|see{LoF}}
\index{figure!list of|see{LoF}}
\index{list!of tables|see{LoT}}
\index{table!list of|see{LoT}}
\index{marginal note|see{marginalia}}
\index{footnote!in title|see{thanks}}
\index{counter|noidxnum}
\index{length|noidxnum}
\index{list|noidxnum}
\index{illustration|seealso{float, figure}}
\index{figure|seealso{float}}
\index{table|seealso{float}}
\index{chapter!style|see{chapterstyle}}
\index{chapter!heading|see{heading}}
\index{page!style|see{pagestyle}}
\index{part!heading|see{heading}}
\index{subfloat|noidxnum}
\end{comment}

%%% ToC down to subsections
\settocdepth{subsection}
\frontmatter
\pagestyle{empty}


% ToC, etc
\pagenumbering{roman}
%%\pagestyle{headings}
\pagestyle{ruled}
\chapterstyle{demo}
%%\chapterstyle{veelo}

% change sections and subsections
\setsecheadstyle{\bfseries\raggedright}
\setsubsecheadstyle{\scshape\raggedright}
\setsubsubsecheadstyle{\itshape\raggedright}


%% Short contents and Different ToC style
\renewcommand{\contentsname}{Short contents}
\let\oldchangetocdepth\changetocdepth
\renewcommand{\changetocdepth}[1]{}
\setcounter{tocdepth}{0} % chapters
\renewcommand{\cftchapterfont}{\hfill\sffamily}
\renewcommand{\cftchapterleader}{ \textperiodcentered\space}
\renewcommand{\cftchapterafterpnum}{\cftparfillskip}
\setpnumwidth{0pt}
\setrmarg{0.3\textwidth}
\tableofcontents
\clearpage

%%% have subsections as a paragraph in the ToC
\makeatletter
\let\oldnumberline\numberline
\renewcommand{\cftsubsectionfont}{\itshape}
\renewcommand{\cftsubsectionpagefont}{\itshape}
\renewcommand{\l@subsection}[2]{\relax
  \def\numberline##1{\textit{##1}~}%
  \leftskip=\cftsubsectionindent
  \rightskip=\@tocrmarg
%  \advance\rightskip \z@ plus \hsize % uncomment this for raggedright
%  \advance\rightskip \z@ plus 2em % uncomment this for semi-raggedright
  \parfillskip=\fill
  \ifhmode ,\ \else\noindent\fi
  \ignorespaces {\cftsubsectionfont #1}~{\cftsubsectionpagefont #2}%\ignorespaces
  \let\numberline\oldnumberline\ignorespaces
}
\AtEndDocument{\addtocontents{toc}{\par}}
\makeatother



%% Default contents
\renewcommand{\contentsname}{Contents}
\let\changetocdepth\oldchangetocdepth
\renewcommand{\cftchapterfont}{\normalfont\sffamily}
\renewcommand{\cftchapterleader}{\sffamily\cftdotfill{\cftchapterdotsep}}
\renewcommand{\cftchapterafterpnum}{}
\makeatletter
\renewcommand{\cftchapterbreak}{\par\addpenalty{-\@highpenalty}}
\makeatother
\setpnumwidth{2.55em}
\setrmarg{3.55em}
\setcounter{tocdepth}{2}
\tableofcontents
%\clearpage
%\listoffigures
%\clearpage
%\listoftables


\cleardoublepage
\pagenumbering{arabic}
\mainmatter




%%%%%%%%%%%%%%%%%%%%%%%%%%%%%%%%%%%%%%
\DeleteShortVerb{\|}
\MakeShortVerb{\=}
\input{tabmanbody}  %% rows & columns
\DeleteShortVerb{\=}
\MakeShortVerb{\|}
%%%%%%%%%%%%%%%%%%%%%%%%%%%%%%%%%%%%%%%%


\chapterstyle{veelo}
\chapter{Miscellaneous} \label{chap:misc}

\section{Chapter style}

    Bastiaan Veelo\footnote{\texttt{Bastiaan.N.Veelo@immtek.ntnu.no}} 
posted the code for a new chapter style to \pictt{} on 2003/07/22 under the
title \textit{[memoir] [contrib] New chapter style}. His code, which
I have slightly modified and changed the name to \cstyle{veelo},
is below. I have also exercised editorial privilege on his comments.

\begin{quote}
 I thought I'd share a new chapter style to be used with the memoir class 
 The style is tailored for documents that are to be trimmed to a smaller 
 width. When the bound document is bent, black tabs will appear on the 
 fore side at the places where new chapters start as a navigational aid.
 We are scaling the chapter number, which most DVI viewers
 will not display accurately. \\[0.5\onelineskip]
Bastiaan.
\end{quote}

\begin{lcode}
\makeatletter
\newlength{\numberheight}
\newlength{\barlength}
\makechapterstyle{veelo}{%
   \setlength{\beforechapskip}{40pt}
   \setlength{\midchapskip}{25pt}
   \setlength{\afterchapskip}{40pt}
   \renewcommand{\chapnamefont}{\normalfont\LARGE\flushright}
   \renewcommand{\chapnumfont}{\normalfont\HUGE}
   \renewcommand{\chaptitlefont}{\normalfont\HUGE\bfseries\flushright}
   \renewcommand{\printchaptername}{%
     \chapnamefont\MakeUppercase{\@chapapp}}
   \renewcommand{\chapternamenum}{}
%   \newlength{\numberheight}
%   \newlength{\barlength}
   \setlength{\numberheight}{18mm}
   \setlength{\barlength}{\paperwidth}
   \addtolength{\barlength}{-\textwidth}
   \addtolength{\barlength}{-\spinemargin}
   \renewcommand{\printchapternum}{%
     \makebox[0pt][l]{%
       \hspace{.8em}%
       \resizebox{!}{\numberheight}{\chapnumfont \thechapter}%
       \hspace{.8em}%
       \rule{\barlength}{\numberheight}
     }
   }
   \makeoddfoot{plain}{}{}{\thepage}
}
\makeatother
\end{lcode}

    The style requires the \Lpack{graphicx} package because of the 
\cmd{\resizebox} macro. I have removed the two \cmd{\newlength} macros to
outside the \cmd{\makechapterstyle} code just in case the style is called more
than once in a document (otherwise there will be `command already defined' 
complaints).

    As a demonstration, this chapter uses the \cstyle{veelo} chapterstyle.
The style works best for chapters that start on recto pages.


\section{Cross referencing} \label{sec:xref}

%This is section \currenttitle.

    The class provides the normal \cmd{\label} and \cmd{\ref} macros
for numeric cross-referencing. For example, the following code and typeset
result
\begin{lcode}
Chapter~\ref{chap:mempack} starts on page~\pageref{chap:mempack}.
\end{lcode}
Chapter~\ref{chap:mempack} starts on page~\pageref{chap:mempack}.


    It can be useful to refer to parts of a document by name rather than
number, as in
\begin{lcode}
The chapter \textit{\titleref{chap:mempack}} describes \ldots
\end{lcode}
The chapter \textit{\titleref{chap:mempack}} describes \ldots

    There are two packages, \Lpack{nameref} and \Lpack{titleref},
 that let you refer to things by name instead of number.

    Name references were added to the class as a consequence of adding
a second optional argument to the sectioning commands. I found
that this broke the \Lpack{nameref} package, and hence the
\Lpack{hyperref} package as well, so they had to be fixed. The change 
also broke Donald Arseneau's \Lpack{titleref} package, and it turned out
that \Lpack{nameref} also clobbered \Lpack{titleref}. The class also
provides titles, like \cmd{\poemtitle}, that are not recognised by
either of the packages. From my viewpoint the most efficient
thing to do was to enable the class itself to provide name 
referencing.

\begin{syntax}
\cmd{\label}\marg{key} \cmd{\ref}\marg{key} \cmd{\pageref}\marg{key} \\
\cmd{\titleref}\marg{key} \\
\cmd{\headnamereftrue} \cmd{\headnamereffalse} \\
\end{syntax}
The macro \cmd{\titleref} is an addition to the usual set of cross referencing
commands. Instead of typesetting a number it typesets the title associated
with the labelled number. This is, of course, only useful if there is an
associated title, such as from a \cmd{\caption} or \cmd{\section} command.
As a bad example:
\begin{lcode}
Labelling for \verb?\titleref? may be applied to:
\begin{enumerate}
\item Chapters, sections, etc.       \label{sec:xref:item1}
...
\item Items in numbered lists, etc. \ldots \label{sec:xref:item3}
\end{enumerate}
Item \ref{sec:xref:item2} in section~\ref{sec:xref} mentions captions
while item \titleref{sec:xref:item3} in the same section 
\textit{\titleref{sec:xref}} lists other things.
\end{lcode}
Labelling for \verb?\titleref? may be applied to:
\begin{enumerate}
\item Chapters, sections, etc.       \label{sec:xref:item1}
\item Captions                       \label{sec:xref:item2}
\item Legends
\item Poem titles
\item Items in numbered lists, etc.  \label{sec:xref:item3}
\end{enumerate}
Item \ref{sec:xref:item2} in section~\ref{sec:xref} mentions captions
while item \titleref{sec:xref:item3} in the same section 
\textit{\titleref{sec:xref}} lists other things.


    As the above example shows, you have to be a little careful in using
\cmd{\titleref}.
Generally speaking, \cmd{\titleref}\marg{key} produces the last named 
thing before the \cmd{\label} that defines the \meta{key}. 

    Chapters, and the lower level sectional divisions, may have three
different title texts --- the main title, the title in the ToC, and a third
in the page header. By default (\cmd{\headnamereffalse}) the ToC title
is produced by \cmd{\titleref}. Following the declaration
\cmd{\headnamereftrue} the text intended for page headers will be produced.

\Note{} Specifically with the \Lclass{memoir} class, 
do not put a \cmd{\label} command inside an
argument to a \cmd{\chapter} or \cmd{\section} or \cmd{\caption}, etc.,
command. Most likely it will either be ignored or referencing it will
produce incorrect values. This restriction does not apply to the standard
classes, but in any case I think it is good practice not to embed any 
\cmd{\label} commands.

\begin{syntax}
\cmd{\currenttitle} \\
\end{syntax}
    If you just want to refer to the current title you can do so with
\cmd{\currenttitle}. This acts as though there had been a label associated
with the title and then \cmd{\titleref} had been used to refer to that label.
For example:
\begin{lcode}
This sentence in the section titled `\currenttitle' is an example of the
use of the command \verb?\currenttitle?.
\end{lcode}
This sentence in the section titled `\currenttitle' is an example of the
use of the command \verb?\currenttitle?.


\begin{syntax}
\cmd{\theTitleReference}\marg{num}\marg{text} \\
\end{syntax}
Both \cmd{\titleref} and \cmd{\currenttitle} use the \cmd{\theTitleReference}
to typeset the title. This is called with two arguments --- 
the number, \meta{num}, and the text, \meta{text}, of the title. The
default definition is:
\begin{lcode}
\newcommand{\theTitleReference}[2]{#2}
\end{lcode}
so that only the \meta{text} argument is printed. You could, for example,
change the definition to
\begin{lcode}
\renewcommand{\theTitleReference}[2]{#1\space \textit{#2}}
\end{lcode}
to print the number followed by the title in italics. If you do this, only use
\cmd{\titleref} for numbered titles, as a printed number for an 
unnumbered title (a) makes no sense, and (b) will in any case be 
incorrect.

    The commands \cmd{\titleref}, \cmd{\theTitleReference} and 
\cmd{\currenttitle} are direct equivalents of those in the \Lpack{titleref}
package.

\begin{syntax}
\cmd{\namerefon} \cmd{\namerefoff} \\
\end{syntax}
   Implementing name referencing has had an unfortunate side effect of
turning some arguments into moving ones; the argument to the \cmd{\legend}
command is one example. If you don't need name referencing you can turn
it off by the \cmd{\namerefoff} declaration; the \cmd{\namerefon}
declaration enables name referencing.

\section{Needing space}

    There are two new macros in addition to the original \cmd{\needspace}
for reserving space at the bottom of a page. The \cmd{\needspace} macro 
depends on penalties for deciding what to do which means that the reserved
space is an approximation. However, except for the odd occasion, the
macro gives adequate results. 

\begin{syntax}
\cmd{\Needspace}\marg{length} \\
\cmd{\Needspace*}\marg{length} \\
\end{syntax}
    Like \cmd{\needspace}, the \cmd{\Needspace} macro checks if there is
\meta{length} space at the bottom of the current page and if there is not
it starts a new page. The command should only be used between paragraphs;
indeed, the first thing it does is to call \cs{par}. The \cmd{\Needspace}
command checks for the actual space left on the page and is more exacting
than \cmd{\needspace}.

    If either \cmd{\needspace} or \cmd{\Needspace} produce a short page it
will be ragged bottom even if \cmd{\flushbottom} is in effect. With
the starred \cmd{\Needspace*} version, short pages will be flush bottom
if \cmd{\flushbottom} is in effect and will be ragged bottom
when \cmd{\raggedbottom} is in effect.

    Generally speaking, use \cmd{\needspace} in preference to \cmd{\Needspace}
unless it gives a bad break or the pages must be flush bottom.

\section{Minor space adjustment}

    The kernel provides the \cmd{\,} macro for inserting a thin space in both
text and math mode. There are
other space adjustment commands, such as \pixabang{} for negative thin space, and
\cmd{\:} and \cmd{\;} for medium
and thick spaces, which can only be used in math mode.

\begin{syntax}
\cmd{\thinspace} \cmd{\medspace} \cmd{\:} \pixabang \\
\end{syntax}
On occasions I have found it useful to be able to tweak spaces in text by some
fixed amount, just as in math mode. The kernel macro \cmd{\thinspace}
specifies a thin space, which is 3/18\,em. 
The class \cmd{\medspace} specifies a medium space of 4/18\,em. 
As mentioned, the kernel macro \cmd{\:} inserts
a medium space in math mode. The class version can be used in both math and
text mode to insert a medium space. Similarly, the class version of 
\pixabang{}
can be used to insert a negative thin space in both text and math mode.

    The math thick space is 5/18\,em. 
To specify this amount of space
in text mode you can combine spacing commands as:
\begin{lcode}
\:\:\!
\end{lcode}
which will result in an overall space of 5/18\,em 
(from $(4 + 4 - 3)/18$).

\section{Fractions}

    When typesetting a simple fraction in text there is usually a choice
of two styles, like 3/4 or $\frac{3}{4}$, which do not necessarily look 
as though they fit in with their surroundings. These fractions were
typeset via:
\begin{lcode}
... like 3/4 or $\frac{3}{4}$ ...
\end{lcode}

\begin{syntax}
\cmd{\slashfrac}\marg{top}\marg{bottom} \\
\cmd{\slashfracstyle}\marg{num} \\
\end{syntax}
The class provides the \cmd{\slashfrac} command which typesets its
arguments like \slashfrac{3}{4}. Unlike the \cmd{\frac} command which
can only be used in math mode, the \cmd{\slashfrac} command can be
used in text and math modes.

    The \cmd{\slashfrac} macro calls the \cmd{\slashfracstyle} command to
reduce the size of the numbers in the fraction. You can also use
\cmd{\slashfracstyle} by itself.
\begin{lcode}
In summary, fractions can be typeset like 3/4 or $\frac{3}{4}%
or \slashfrac{3}{4} or \slashfracstyle{3/4} because several choices
are provided.
\end{lcode}
In summary, fractions can be typeset like 3/4 or $\frac{3}{4}$
or \slashfrac{3}{4} or \slashfracstyle{3/4} because several choices
are provided.

\begin{syntax}
\cmd{\textsuperscript}\marg{super} \\
\cmd{\textsubscript}\marg{sub} \\
\end{syntax}
While on the subject of moving numbers up and down, the kernel provides
the \cmd{\textsuperscript} macro for typesetting its argument as though it
is a superscript. The class also provides the \cmd{\textsubscript} macro
for typesetting its argument like a subscript.
\begin{lcode}
You can typeset superscripts like \textsuperscript{3}/4 and 
subscripts like 3/\textsubscript{4}, 
or both like \textsuperscript{3}/\textsubscript{4}.
\end{lcode}
You can typeset superscripts like \textsuperscript{3}/4 and 
subscripts like 3/\textsubscript{4}, 
or both like \textsuperscript{3}/\textsubscript{4}.


\section{Framed boxes}

    Donald Arseneau's \Lpack{framed} package is currently at or beyond
v0.7 while
the original copy used in the class is from an earlier version. The class
version of the \Lpack{framed} functions has been updated to v0.7.

\begin{syntax}
\senv{framed} text \eenv{framed} \\
\senv{shaded} text \eenv{shaded} \\
\senv{leftbar} text \eenv{leftbar} \\
\end{syntax}
The \Ie{framed} environment puts the text into an \cmd{\fbox}-like 
framed box, the
\Ie{shaded} environment puts the text into a coloured box, and the
\Ie{leftbar} environment draws a vertical line at the left of the text.
In all cases the text and boxes can include page breaks.

\begin{syntax}
\lnc{\FrameRule} \lnc{\FrameSep} \cmd{\FrameHeightAdjust} \\
\Itt{shadecolor} \\
\end{syntax}
The thickness of the rules is the length \lnc{\FrameRule} and the separation
between the text and the box is given by the length \lnc{\FrameSep}.
The height of the frame above the baseline at the top of a page is specified
by the macro \cmd{\FrameHeightAdjust}. The default definitions being:
\begin{lcode}
\setlength{\FrameRule}{\fboxrule}
\setlength{\FrameSep}{3\fboxsep}
\newcommand{\FrameHeightAdjust}{0.6em}
\end{lcode}
The background color in the \Ie{shaded} environment is specified by
\Itt{shadecolor} which you have to specify using the \Lpack{color}
package. For example:
\begin{lcode}
\usepackage{color}
\definecolor{shadecolor}{gray}{0.75}
\end{lcode}

\begin{syntax}
\cmd{\frameasnormaltrue} \cmd{\frameasnormalfalse} \\
\end{syntax}
Following the declaration \cmd{\frameasnormaltrue} paragraphing within
the environments will be as specified for the main text. After the declaration
\cmd{\frameasnormalfalse} paragraphing will be as though the environments
were like a \Ie{minipage}. The initial declaration is 
\cmd{\frameasnormaltrue}.

    There is one known problem with framing: when framing is used on 
a page where the page header is in a smaller type than the body, the 
header may be moved slightly up or down. This can be avoided by putting
the font size change in a group, but it seems that a larger font does not need
to be grouped. For example:
\begin{lcode}
\makeoddhead{myheadings}{{\tiny Tiny header}}{}{\LARGE Large header}
\end{lcode}

    You can use the \Lpack{framed} package with the \Lclass{memoir} class, in
which case you will get whatever functionality is provided by the package as
it will override the class' code.

\section{Captions}

    The \cmd{\captionstyle} macro has been extended so that it is now 
possible to separately specify the style for short and long
captions.

\begin{syntax}
\cmd{\captionstyle}\oarg{short}\marg{normal} \\
\cmd{\raggedleft} \cmd{\centering} \cmd{\raggedright} \cmd{centerlastline} \\
\end{syntax}
Caption styles are set according to the \cmd{\captionstyle} declaration.
Unless the optional \meta{short} argument is given all captions are typeset
according to \meta{normal}. If the optional \meta{short} argument
is specififed, captions that are less than one line in length are typeset
according to \meta{short}. 

    Permissable values for the arguments include, but are not limited to,
\cmd{\raggedleft}, \cmd{\centering}, \cmd{\raggedright}, and
 \cmd{centerlastline}. The class initially specifies
\begin{lcode}
\captionstyle{}
\end{lcode}
which gives the normal block paragraph style.

\section{Hung paragraphs}

    As noted eleswhere the sectioning commands use the internal 
\cmd{\@hangfrom} as part of the formatting of the titles.

\begin{syntax}
\cmd{\hangfrom}\marg{text} \\
\end{syntax}

\hangfrom{Simple hung paragraphs }(like this one) can be specified
using the \cmd{\hangfrom} macro. The macro puts \meta{text} in a box
and then makes a hanging paragraph of the following material. This
paragraph commenced with \\
\verb?\hangfrom{Simple hung paragraphs }(like ...? \\
and you are now reading the result.

\chapterstyle{demo}
\chapter{Memoir and packages} \label{chap:mempack}

    The \Lclass{memoir} class does some things differently from the standard
classes. Some packages that might be used with \Lclass{memoir} rely on
the standard methods, and change them to suit their own purposes. Some such
changes may not work with \Lclass{memoir} and the package may not recognize
that it is being used with \Lclass{memoir} and not with a standard class. 

    From my viewpoint, the ideal solution is for the packages to be changed
so that they cooperate with \Lclass{memoir}. However, until that happy
day arrives I have provided the \Lpack{memhfixc} package that attempts to 
make these packages cooperate with the class. 

    Currently, if you use either the \Lpack{hyperref} or the \Lpack{nameref}
package you will also need to use the \Lpack{memhfixc} package. 
The ordering of the \Lpack{memhfixc} and other packages can be important:
\begin{itemize}
\item \Lpack{memhfixc} must be used \emph{after} the \Lpack{hyperref} package.
\item The ordering of \Lpack{memhfixc} and \Lpack{nameref} is immaterial.
\end{itemize}

    There is a basic incompatability between the \Lpack{hyperref} package
and sequential footnotes --- the \cmd{\multfootsep} macro is essentially 
ignored. If that is important to you and you don't mind not having 
hyperreferences to footnotes, call \Lpack{hyperref} like:
\begin{lcode}
\usepackage[hyperfootnotes=false, ...]{hyperref}
\end{lcode}
A similar \Lpack{hyperref} incompatibily also occurs with at least 
the \Lpack{footmisc} package.

    Thinking of the \Lpack{footmisc} package, for some of its options it
changes the kernel output routine. The class itself changes the output routine
in order to add sidebars. Packages, like the version of \Lpack{footmisc}
current at the date of writing (2003/04/28), which change the output 
routine without specifically catering for the \Lclass{memoir} class
are likely to cause problems. As an alternative to the \Lpack{footmisc} 
package, the \Lpack{ledmac} package understands \Lclass{memoir} and provides
further multiple classes of footnotes.



% back end
\backmatter
\clearpage
%\pagestyle{ruled}
%\chapterstyle{section}



 \bibliographystyle{alpha}
 \begin{thebibliography}{WWW99}
    \bibitem[ABH90]{bk:Impatient}
      Paul W.~Abrahams, Karl Berry and Kathryn A.~Hargreaves.
      \newblock \emph{TeX{} for the Impatient}.
      \newblock
       Addison-Wesley, Reading, Massachusetts, 1990.
      \newblock (Available from CTAN in \texttt{info/impatient})
    \bibitem[Car94]{DELARRAY}
      David Carlisle.
      \newblock \emph{The delarray package}.
       \newblock March 1994.
      \newblock (Available from CTAN in 
                 \texttt{macros/latex/required/tools})
    \bibitem[Car98]{LONGTABLE}
      David Carlisle.
      \newblock \emph{The longtable package}.
       \newblock May 1998.
      \newblock (Available from CTAN in 
                 \texttt{macros/latex/required/tools})
    \bibitem[Car99]{TABULARX}
      David Carlisle.
      \newblock \emph{The tabularx package}.
       \newblock January 1999.
      \newblock (Available from CTAN in 
                 \texttt{macros/latex/required/tools})
    \bibitem[Car01]{DCOLUMN}
      David Carlisle.
      \newblock \emph{The dcolumn package}.
       \newblock May 2001.
      \newblock (Available from CTAN in 
                 \texttt{macros/latex/required/tools})
    \bibitem[Fea03]{BOOKTABS}
      Simon Fear.
      \newblock \emph{Publication quality tables in \ltx}.
       \newblock March 2003.
      \newblock (Available from CTAN in 
                 \texttt{macros/latex/contrib/booktabs})
    \bibitem[GMS94]{bk:GMS94} 
      M.~Goossens, F.~Mittelbach and A.~Samarin.
       \newblock \emph{The \ltx{} Companion}.
       \newblock
       Addison-Wesley, Reading, Massachusetts, 1994.
    \bibitem[Knu84]{bk:knuth}  
       Donald E. Knuth.
       \newblock  \emph{The \TeX{}book}.
       \newblock
       Addison-Wesley, Reading, Massachusetts, 1984.
    \bibitem[Lam94]{bk:lamport} 
       Leslie Lamport.
       \newblock  \emph{\ltx\ --- A Document Preparation System}.
       \newblock
       Addison-Wesley, Reading, Massachusetts, 1994.
    \bibitem[MC98]{ARRAY}
      Frank Mittelbach and David Carlisle.
      \newblock \emph{A new implementation of LaTeX's tabular and array
                      environment}.
       \newblock May 1998.
      \newblock (Available from CTAN in 
                 \texttt{macros/latex/required/tools})
    \bibitem[Wil00]{XTAB}
      Peter Wilson.
      \newblock \emph{The xtab package}.
       \newblock April 2000.
      \newblock (Available from CTAN in 
                 \texttt{macros/latex/contrib/xtab})
 \end{thebibliography}


\clearpage
\pagestyle{index}
%\renewcommand{\chaptermark}[1]{}
\renewcommand{\preindexhook}{%
The first page number is usually, but not always, the primary reference to
the indexed topic.\vskip\onelineskip}
\indexintoc
\printindex


\end{document}






